\documentclass[11pt,spanish,a4paper,twoside,openany]{book}
\raggedbottom
% descomentar para debuggear overfull warnings:
% \overfullrule=1mm

\usepackage[
  active=true,
  header=true,
  generate=ejercicios.tex,
  copydocumentclass=false,
  extract-env=ejercicio,
  ejercicio-labels={guia},
  extract-cmd=chapter,
  chapter-nrs={1-9,11,14-18,20},
]{extract}

% Document class para ejercicios.tex.
\begin{extract}
\documentclass[11pt,a4paper]{article}
\end{extract}

% Preámbulo compartido entre apunte y ejercicios.
\begin{extract*}
\tolerance=5000

% Entrada de texto
\usepackage{polyglossia}
\setmainlanguage{spanish}

% Cuestiones de estilo
\usepackage{listings}         % Permite mostrar codigo de forma mas linda
\usepackage{verbatim}         % Permite incluir archivos de texto verbatim

\usepackage{amsmath, amsthm, amssymb} % Se usan para theoremstyle

\usepackage{relsize}
\newcommand{\dificil}{\mathlarger{\mathlarger{\star}}}

\usepackage[table]{xcolor}
\definecolor{light-gray}{gray}{0.96}
\definecolor{med-gray}{gray}{0.8}

\definecolor{keyword}{rgb}{0,0,0.5}
\definecolor{string}{rgb}{0.5,0,0.5}
\definecolor{error}{rgb}{0.5,0,0}

% para poder usar upquote=true (mostrar bien las comillas simples)
\usepackage{textcomp}

\usepackage[labelfont=bf]{caption}

% Parámetros para los listings de python
\lstset{
    language=Python,
    columns=fixed,
    keepspaces=true,
    numbers=left,
    tabsize=4,
    numberstyle=\color{gray}\tiny\ttfamily,
    numbersep=5pt,
    showstringspaces=false,
    basicstyle=\small\ttfamily,
    keywordstyle=\color{keyword},
    backgroundcolor=\color{light-gray},
    commentstyle=\color{gray},
    stringstyle=\color{string},
    morecomment=[s]{>>}{>}, % hack para pintar los prompts >>>
    morecomment=[s]{..}{.}, % y ...
    xleftmargin=\dimexpr\parindent+10pt,
    breaklines=true,
    postbreak=\lst@ifdisplaystyle{\raisebox{0ex}[0ex][0ex]{\ensuremath{\hookrightarrow\space}}}\fi,
}

\lstdefinestyle{inlinecode}{
    backgroundcolor=\color{light-gray}
}

% Para poner partes dentro de los ejercicios
\newcounter{partesi}
\newenvironment{partes}
{   \begin{list}{\alph{partesi})}{
        \usecounter{partesi}
        \setlength{\topsep}{0pt}
        \setlength{\itemsep}{0pt}}}
{   \end{list} }

\usepackage{unicode-math}

\setmainfont
     [ BoldFont       = texgyrepagella-bold.otf ,
       ItalicFont     = texgyrepagella-italic.otf ,
       BoldItalicFont = texgyrepagella-bolditalic.otf ]
     {texgyrepagella-regular.otf}
\setmathfont[Scale=MatchLowercase]{texgyrepagella-math.otf}
\usepackage{DejaVuSansMono}
\setmonofont[Scale=0.9]{DejaVuSansMono}

\usepackage{array}            % Permite alinear los párrafos en las celdas verticalmente
\usepackage{enumitem}
\setlist{labelindent=\parindent, leftmargin=*, align=left, topsep=1.25\parsep,
itemsep=0.5\parsep}

\end{extract*}

% Resto del preámbulo de ejercicios.tex
\begin{extract}

\renewcommand{\baselinestretch}{1.1}
\oddsidemargin 0.0cm \evensidemargin 0.0cm
\setlength{\textwidth}{16cm}

\theoremstyle{definition}
\newtheorem{ejercicio}{Ejercicio}[section]

% Para poder usar los nombres de capítulos como secciones.
% TODO: al omitir capítuos en los ejercicios, las secciones en los ejercicios
% se siguen numerando consecutivamente, y se desacoplan del número de capítulo.
\newcommand{\chapter[2]}{\section{#2}}

\title{75.40 Algoritmos y Programación I \\
    \textbf{Guía de Ejercicios}}
\date{2.\textsuperscript{do} cuatrimestre 2019}

\usepackage[xetex,setpagesize=false,colorlinks]{hyperref}
\hypersetup{%
	linkcolor=black,
	urlcolor=blue,
	pdftitle={Algoritmos y Programación I - Guía de ejercicios},
	bookmarksnumbered,
}%

\usepackage{fancyhdr}
\pagestyle{fancy}
\fancyhead{}
\fancyfoot{}
\fancyfoot[R]{\thepage}
\renewcommand{\headrulewidth}{0pt}
\renewcommand{\footrulewidth}{0pt}

\end{extract}

%% Resto del preámbulo solo definido para el apunte. %%

\usepackage[title,titletoc]{appendix}

\usepackage{graphicx}         % Permite insertar imagenes.

\usepackage{longtable}        % tablas largas y flexibles
\usepackage{nonfloat}         % Hace que las figuras no floten
\usepackage{eso-pic}

% Margenes y largo del texto
%
% medidas horizontales
% 1in(fijo) + \hoffset + \(odd|even)sidemargin + \textwidth + \marginparsep +
% \marginparwidth + \marginparpush
%
% medidas verticales
% 1in(fijo) + \voffset + \topmargin + \headheight + \headsep + \textheight +
% \footskip

\setlength\oddsidemargin{-0.04cm}
\setlength\evensidemargin{-0.04cm}
\setlength\topmargin{0cm}
\setlength\headheight{0.5cm}
\setlength\headsep{0.3cm}
\setlength\footskip{0.8cm}
\setlength\textwidth{16cm}  % ancho para apunte
\setlength\textheight{23cm} % largo para apunte

\usepackage{fancyhdr}         %
\pagestyle{fancy}
\fancyhf{} % clear all header and footer fields
\fancyhead[RO]{\slshape \rightmark \hspace{1cm} \normalfont \bfseries \thepage}
\fancyhead[LE]{\bfseries \thepage \hspace{1cm} \normalfont \leftmark}
\renewcommand{\headrulewidth}{0.3pt}
\renewcommand{\footrulewidth}{0pt}

\fancypagestyle{plain}{%
\fancyhf{} % clear all header and footer fields
\renewcommand{\headrulewidth}{0pt}
\renewcommand{\footrulewidth}{0pt}}

\renewcommand{\sectionmark}[1]{\markright{\thesection.\ #1}}
\renewcommand{\chaptermark}[1]{\markboth{\chaptername\ \thechapter.\ #1}{}}

\usepackage{color}            % Permite definir colores
\definecolor{rosa}{rgb}{1,0.90,0.90}
\definecolor{celeste}{rgb}{0.90,0.90,0.95}
\definecolor{verde}{rgb}{0.85,1.00,0.85}
\definecolor{amarillo}{rgb}{1.00,1.00,0.75}

\usepackage[framemethod=tikz]{mdframed}
\usepackage{pgfplots}

\newmdenv[
    font=\small,
    linewidth=0,
    backgroundcolor=celeste,
    nobreak=true
]{observacion}

\newlength{\currentparskip}
\newlength{\currentparindent}

\newenvironment{sabias_que}
{\setlength{\currentparskip}{\parskip}%
\setlength{\currentparindent}{\parindent}%
\begin{figure*}[h!t]\begin{mdframed}[
    font=\small,
    linecolor=black,
    backgroundcolor=verde,
    frametitlefont=\normalsize\bfseries,
    frametitle={$\vcenter{\hbox{\includegraphics[height=4ex]{graficos/lamparita}}}$ Sabías que\ldots},
    frametitlealignment={\hspace*{-\parindent}},
    startinnercode={\setlength{\parskip}{\currentparskip}\setlength{\parindent}{\currentparindent}}
]}{\end{mdframed}\end{figure*}}

\newmdenv[
    font=\small,
    linecolor=black,
    backgroundcolor=amarillo,
    frametitlefont=\normalsize\bfseries,
    frametitle={$\vcenter{\hbox{\includegraphics[height=3ex]{graficos/atencion}}}$ Atención},
    frametitlealignment={\hspace*{-\parindent}},
    nobreak=true
]{atencion}

\newmdenv[
    font=\small,
    backgroundcolor=verde,
    hidealllines=true,
    frametitlerule=true,
    frametitlefont=\normalsize\bfseries,
    frametitle={Referencia Python\hfill$\vcenter{\hbox{\includegraphics[height=4ex]{graficos/logo-python}}}$},
    frametitlealignment={\hspace*{-\parindent}},
	needspace=10\baselineskip
]{referencia_python}

\newenvironment{sintaxis}[1]
  {\needspace{2\baselineskip}\mdfsubtitle[
    subtitlebackgroundcolor=verde,
    subtitleaboveskip=4pt,
    subtitlebelowskip=8pt,
    subtitleinneraboveskip=8pt,
    subtitleinnerbelowskip=0pt
  ]{#1}\parskip=0pt}
  {}

\usepackage{float}            % Agrega estilos a los floats

\floatstyle{ruled}
\newfloat{codigo-float}{htbp}{lop}[chapter]
\floatname{codigo-float}{Código}

\newcommand{\lsthighlight}{\lstset{
    moredelim=**[is][\bfseries]{(@}{@)},
    moredelim=[is][\color{error}]{(^}{^)},
    escapeinside={(~}{~)}
}}
\newcommand{\lstnonumbers}{\lstset{
    numbers=none,
    xleftmargin=\parindent
}}

\newenvironment{codigo}[2]{\begin{codigo-float}
    \caption{{\small\texttt{#1}}\textbf{:}\ \small #2}
    }{\end{codigo-float}}

\lstnewenvironment{codigo-python}{\lsthighlight}{}
\lstnewenvironment{codigo-python-sn}{\lsthighlight\lstnonumbers}{}
\lstnewenvironment{codigo-python-sn-small}{\lsthighlight\lstnonumbers\lstset{basicstyle=\footnotesize\ttfamily}}{}
\lstnewenvironment{codigo-nohl}{\lstset{language={}}\lsthighlight}{}
\lstnewenvironment{codigo-nohl-sn}{\lstset{language={}}\lsthighlight\lstnonumbers}{}

\newcommand{\tituloCodigo}[1]{
\bigskip
\needspace{4cm}
\hrule
\smallskip
{\noindent #1}
\smallskip
\hrule
}

% Numeracion para secciones y subsecciones - Hace falta?
\renewcommand{\thesection}{\thechapter.\arabic{section}}
\renewcommand{\thesubsection}{\thesection.\arabic{subsection}}

% Estilos usados en el texto
\theoremstyle{definition}
\newtheorem{definicion}{Definici\'on}[section]
\newtheorem{ejemplo}{Ejemplo}[section]
\newtheorem{problema}{Problema}[section]
\newtheorem{ejercicio}{Ejercicio}[section]
\newtheorem{ejerciciof}{}[section]
\newtheorem{problemac}{Problema}[chapter]
\newtheorem{ejercicioc}{Ejercicio}[chapter]
\renewcommand{\qedsymbol}{} % El proof inserta dos qed al final
\newenvironment{solucion}[1][Solución]{\begin{proof}[#1]}{\end{proof}}

\theoremstyle{remark}
\newtheorem*{nota}{Nota}

% Traducciones propias
\addto\captionsspanish{
\renewcommand{\chaptername}{Unidad}
\renewcommand{\contentsname}{Contenidos}
}

\def\ra{\rightarrow}

% Para tener hiperlinks, en el panel lateral y en la toc.
\usepackage[xetex,unicode=false,setpagesize=false,colorlinks]{hyperref}
\hypersetup{%
	linkcolor=black,
	urlcolor=blue,
	pdftitle={Algoritmos y Programación I - Aprendiendo a programar usando Python como herramienta},
	bookmarksnumbered,
}%
\usepackage{hypcap}

\usepackage{menukeys}

\usepackage{tikz}
\usetikzlibrary{calc,shapes,arrows,babel,trees,fit,positioning}
\newcommand{\tikzmark}[1]{\tikz[overlay,remember picture] \node (#1) {};}

\tikzset{
    umlattr/.style={right,node font=\ttfamily,draw=none,fill=none,node distance=0pt},
    umltitle/.style={node font=\ttfamily,draw=none,fill=none,node distance=0pt},
    umlattrs/.style={draw=none,fill=none,inner sep=0pt},
    umlclass/.style={draw,fill=none,inner sep=0pt},
    flecha/.style={draw,->,>=latex},
    decision/.style={diamond,draw,aspect=2},
    bloque/.style={rectangle,draw},
    start/.style={circle,draw},
    end/.style={circle,draw,fill},
}

\newcommand*\circled[1]{\tikz[]{
  \node[shape=circle,font=\ttfamily,draw,fill=black,text=white,inner sep=0.5pt] (char) {\tiny{#1}};}}

\usepackage{multirow}

\begin{document}

%\AddToShipoutPictureBG*{%
%  \AtPageLowerLeft{%
%    \hspace{\baselineskip}%
%    \raisebox{\baselineskip}{%
%      \makebox[0pt][l]{Última revisión: \today}}}}

\begin{extract*}
\lstMakeShortInline[style=inlinecode]|
\end{extract*}

\begin{extract} % La guía de estilo para los ejercicios
\maketitle
\thispagestyle{empty}

\newpage

\section*{Recomendaciones al realizar las guías.}

\begin{itemize}
	\item Probá el código que escribas ejecutándolo con el intérprete Python,
        para verificar que tu solución sea correcta.
	\item Prestá atención al leer el enunciado del ejercicio. En particular:
    \begin{itemize}
        \item Revisá si lo que se pide es una \emph{función} o un
            \emph{programa}. Si se pide un programa, se espera que produzcas
            un archivo \verb|.py|, que puede contener varias
            funciones y ser ejecutado con el comando \verb|python3 archivo.py|.
        \item Si se pide que una función \emph{devuelva} o \emph{calcule}
            un valor, la función debe tener una instrucción
            \verb|return| y {\bf no} utilizar las funciones \verb|input| ni
            \verb|print|.
        \item Si se pide que una función o programa \emph{imprima} o
            \emph{muestre} un valor, la función debe llamar a \verb|print|.
        \item Si se pide que una función o programa \emph{pida} o
            \emph{pregunte} un valor al usuaio, debe haber al menos una
            invocación de la función \verb|input|.
    \end{itemize}
    \item Cada ejercicio puede tener muchas soluciones posibles. Una vez que
        encuentres una solución, en lugar de pasar al siguiente ejercicio pensá
        si se te ocurre una solución cuya codificación sea más simple.
	\item Es muy importante que el código que escribas sea lo más claro y legible posible.
    \begin{itemize}
	    \item En particular, los nombres de funciones y variables deben ser descriptivos.
	    \item También prestá atención a los espacios. Por ejemplo, una línea en
            blanco entre las definiciones de función simplifica la lectura del programa.
    \end{itemize}
	\item Los ejercicios marcados con el símbolo $\dificil$ son más difíciles y no son de resolución obligatoria.
	\item No documentes en exceso, pero tampoco ahorres documentación necesaria. La documentación
        debe ser breve y concisa.
    \begin{itemize}
        \item Documentá con \verb|"""| \emph{qué} hace cada una de las funciones
            que desarrolles. ¿Qué parámetros recibe? ¿Qué devuelve? ¿Imprime algo?
            ¿Recibe información del usuario?
        \item En caso de ser necesario, documentá con comentarios (\verb|#|) \emph{cómo} funciona
            el código. Sólo debería ser necesario en secciones de código complejas, donde sea difícil
            entender la intención del código.
    \end{itemize}
\end{itemize}

\newpage

\end{extract}

\title{{\bf Algoritmos y Programación I} \\ Aprendiendo a programar usando Python como herramienta}
\date{2\textsuperscript{da} edición}
\maketitle

\chapter*{Prólogo}
\addcontentsline{toc}{chapter}{Prólogo}

Durante mucho tiempo nos preguntamos cómo diseñar un curso de Algoritmos y Programación I
(primera materia de programación de las carreras de Ingeniería en Informática y Licenciatura
en Análisis de Sistemas de la Facultad de Ingeniería de la UBA) que al mismo tiempo fuera
atractivo para los futuros profesionales de la informática, les permitiera aprender a resolver
problemas, y atacara muy directamente el problema de la deserción temprana de estudiantes.


Muchos de los estudiantes que ingresan a primer año de las carreras de informática en nuestro
país lo hacen sin saber programar pese a ser nativos digitales, para los cuales las computadoras
y muchos programas forman parte de su vida cotidiana. El primer curso de programación plantea
entonces varios desafíos: enseñar una metodología para la resolución de problemas, un lenguaje
formal para escribir los programas, y al mismo tiempo hacer que los estudiantes no se sientan
abrumados, tengan éxito en este primer esfuerzo y se sientan atraídos por la posibilidad de
escribir sus propios programas.


En este sentido, la elección del lenguaje de programación para dicho curso no es un tema menor:
el problema radica en elegir un lenguaje que al mismo tiempo sea suficientemente expresivo, tenga
una semántica clara y cuyas complicaciones sintácticas sean mínimas.


Hacía tiempo que buscábamos un lenguaje con todas estas características cuando, durante el debate
posterior a un panel sobre programación en el nivel que tuvo lugar en la conferencia Frontiers in
Engineering Education 2007 organizada por la IEEE, el Dr. John Impagliazzo, de Hofstra University,
relató cómo sus cursos habían pasado de duras experiencias, con altas tasas de deserción, a una
situación muy exitosa, por el solo hecho de haber cambiado el lenguaje de programación de ese curso
de Java a Python. Después de ese estimulante encuentro con Impagliazzo nos pusimos manos a la obra
para diseñar este curso. Fue una grata sorpresa enterarnos también de que el venerable curso de
programación de MIT también había migrado a Python: todo hacía pensar que nuestra elección de lenguaje
no era tan descabellada como se podía pensar.


Este libro pretende entonces ser una muy modesta contribución a la discusión sobre cómo enseñar a
programar en primer año de una carrera de Informática a través de un lenguaje que tenga una suave
curva de aprendizaje, de modo tal que este primer encuentro le resulte a los estudiantes placentero
y exitoso, sin que los detalles del lenguaje los distraiga del verdadero objetivo del curso: la resolución
de problemas mediante computadoras.


Queremos agradecer a la Facultad de Ingeniería (en particular a su Comisión de Publicaciones)
por la publicación de este libro. Y también a todos los que apoyaron desde un primer momento
la escritura y mejora continua de lo que fueron durante varios años las notas de nuestro curso de
Algoritmos y Programación I: a la Comisión Curricular de la Licenciatura en Análisis de Sistemas y al
Departamento de Computación (y a su director, Gustavo López) por apoyar la iniciativa de dar este curso
como piloto, a quienes leyeron y discutieron los manuscritos desde sus primeras versiones y colaboraron
con el curso (Melisa Halsband, Alberto Bertogli, Sebastián Santisi, Pablo Antonio, Pablo Najt, Diego Essaya,
Leandro Ferrigno, Martín Albarracín, Gastón Kleiman, Matías Gavinowich), a quienes fueron primero alumnos y
luego colaboradores de esta experiencia educativa (Bruno Merlo Schurmann, Damian Schenkelman, Débora Martín,
Ezequiel Genender, Fabricio Pontes Harsich, Fabrizio Graffe, Federico Barrios, Federico López, Gaston
Martinez, Gaston Goncalves, Ignacio Garay, Javier Choque, Jennifer Woites, Manuel Soldini, Martín Buchwald,
Pablo Musumeci), a los amigos que, como Pablo Jacovkis, Elena García y Alejandro Tiraboschi se ofrecieron a
revisar manuscritos y a hacer sugerencias, y a todos los alumnos de nuestro curso de Algoritmos y
Programación I de la FIUBA que, desde 2009, han usado este texto y nos han ayudado a mejorarlo permanentemente.
De todos modos, por supuesto, los errores son nuestra responsabilidad.

\vspace{1cm}
\hfill Buenos Aires, diciembre de 2012.

\section*{Segunda edición}

Habiendo dictado el curso de Algoritmos y Programación I durante
varios años ya, podemos felizmente decir que el curso ya no es más \enquote{piloto},
y conseguimos en el camino una buena cantidad de logros.

Al final de cada cuatrimestre presentamos a los alumnos una encuesta de
satisfacción, en la que tanto el curso como el apunte recibieron
mayoritariamente buenas
críticas\footnote{Los resultados de las encuestas están publicados en
\url{https://algoritmos1rw.ddns.net/encuestas}}. A continuación se muestra el
resultado de dos de las preguntas incluidas en las encuestas.

\begin{center}
\begin{tikzpicture}
\begin{axis}[
ybar,
axis x line={bottom},
axis y line={left},
axis line style={-,gray},
enlarge x limits=0.15,
enlarge y limits={0.15,upper},
legend style={draw=none,at={(0.3,0.8)},font=\scriptsize,anchor={west}},
legend cell align={left},
ylabel={Cantidad de respuestas},
ylabel style={font=\scriptsize},
symbolic x coords={Muy bueno,Bueno,Regular,Malo,Muy malo},
xtick=data,
x tick label style={rotate=30,anchor=east,font=\scriptsize},
y tick label style={font=\scriptsize},
width=0.6\textwidth,
height=5cm,
]
\addplot[black,fill=black] coordinates {(Muy bueno,295) (Bueno,110) (Regular,8) (Malo,0) (Muy malo,0)};
\addplot[black,fill=black!25] coordinates {(Muy bueno,293) (Bueno,81) (Regular,12) (Malo,0) (Muy malo,1)};
\legend{¿Qué te pareció el curso en general?,¿Qué opinás del apunte provisto por la cátedra?}
\end{axis}
\end{tikzpicture}
\end{center}

Debido a la buena aceptación que tuvo el curso, en 2014 se dio el reconocimiento
de que la Comisión Curricular de la Licenciatura en Sistemas adoptara el temario
desarrollado por el curso como programa del plan de estudios oficial,
actualización que aprobó el Consejo Superior ese mismo año.  También en 2014
se comenzaron a dictar cursos de Introducción a la Programación en Python que
utilizan la planificación del curso, aprobados por el Consejo Directivo.

En paralelo con todo esto fuimos mejorando el apunte: arreglando erratas,
mejorando redacciones, y agregando y actualizando temas. El cambio más
importante fue la actualización a la versión 3 de Python, que decidimos hacer
entre otras razones porque Python 2 será discontinuado en 2020. Entre los
temas adicionados figuran: diseño de funciones recursivas, composición y mutabilidad
de objetos, funciones de orden superior, listas por comprensión, la instrucción
|import|, conjuntos. Además se mejoró la redacción general de todos los capítulos,
y se reorganizó la guía de ejercicios. Todas estas mejoras se incluyen en esta
Segunda Edición.

Por último, extendemos la lista de agradecimientos a todos aquellos que
fueron alumnos y luego colaboradores del curso a lo largo de los últimos años:
Agustín Santiago,
Agustina Mendez,
Alan Rinaldi,
Alejandro Levinas,
Ana Czarnitzki,
Ariel Vergara,
Ayelén Bibiloni Lombardi,
Carlos Talavara,
Constanza Gonzalez,
Daniela Riesgo,
Daniela Soto,
Diego Alfonso,
Emiliano Sorbello,
Eugenia Mariotti,
Federico Esteban,
Florencia Álvarez Etcheverry,
Florencia Rodriguez,
Franco Di María,
Gianmarco Cafferatta,
Ignacio Sueiro,
Joel Saidman,
Juan Costamagna,
Juan Ignacio Kristal,
Juan Patricio Marshall,
Julián Crespo,
Klaus Lungwitz,
Lucas Perea,
Luciano Sportelli Castro,
Manuel Battan,
Manuel Porto,
Manuel Sturla,
Martín Coll,
Martín Dabat,
Matías Scacosky,
Maximiliano Suppes,
Maximiliano Yung,
Miguel Alfaro,
Milena Farotto,
Nicolás Poncet,
Ramiro Santos,
Robinson Fang,
Rodrigo Velez,
Sebastián Gonzalez,
Sofía Morseletto,
Tomás Rocchi.

\vspace{1cm}
\hfill Buenos Aires, agosto de 2018

\tableofcontents
\include{1_conceptos}
\chapter[Programas sencillos]{Programas sencillos}

En esta unidad empezaremos a resolver problemas sencillos, y a
programarlos en Python.

\section{Construcción de programas}

Cuando nos disponemos a escribir un programa debemos seguir una cierta cantidad
de pasos para asegurarnos de que tendremos éxito en la tarea. La acción
irreflexiva (me siento frente a la computadora y escribo rápidamente y sin
pensar lo que me parece que es la solución) no constituye una actitud
profesional (e ingenieril) de resolución de problemas. Toda construcción tiene
que seguir una metodología, un protocolo de desarrollo.

Existen muchas metodologías para construir programas, pero en este
curso aplicaremos una sencilla, que es adecuada para
la construcción de programas pequeños, y que se puede resumir en
los siguientes pasos:

\begin{enumerate}
\item {\bf Analizar el problema.} Entender profundamente \emph{cuál} es
el problema que se trata de resolver, incluyendo el contexto en el
cual se usará.

\begin{observacion}
Una vez analizado el problema, asentar el análisis por escrito.
\end{observacion}

\item {\bf Especificar la solución.} Éste es el punto en el cual
se describe \emph{qué} debe hacer el programa, sin importar
el cómo. En el caso de los problemas sencillos que abordaremos,
deberemos decidir cuáles son los datos de entrada que se nos
proveen, cuáles son las salidas que debemos producir, y cuál es la
relación entre todos ellos.

\begin{observacion}
Al especificar el problema a resolver, documentar la especificación por
escrito.
\end{observacion}

\item {\bf Diseñar la solución.} Éste es el punto en el cuál
atacamos el \emph{cómo} vamos a resolver el problema, cuáles son
los algoritmos y las estructuras de datos que usaremos. Analizamos
posibles variantes, y las decisiones las tomamos usando como dato
de la realidad el contexto en el que se aplicará la solución, y
los costos asociados a cada diseño.

\begin{observacion}
Luego de diseñar la solución, asentar por escrito el diseño, asegurándonos de
que esté completo.
\end{observacion}

\item {\bf Implementar el diseño.} Traducir a un lenguaje de
programación (en nuestro caso, y por el momento, Python) el diseño
que elegimos en el punto anterior.

\begin{observacion}
La implementación también se debe documentar, con comentarios
dentro y fuera del código, al respecto de qué hace el programa, cómo lo hace y
por qué lo hace de esa forma.
\end{observacion}

\item {\bf Probar el programa.} Diseñar un conjunto de pruebas
para probar cada una de sus partes por separado, y también la
correcta integración entre ellas. Utilizar la \emph{depuración} como
instrumento para descubir dónde se producen ciertos errores.

\begin{observacion}
Al ejecutar las pruebas, documentar los resultados obtenidos.
\end{observacion}

\item {\bf Mantener el programa.} Realizar los cambios en
respuesta a nuevas demandas.

\begin{observacion}
Cuando se realicen cambios, es necesario documentar el análisis,
la especificación, el diseño, la implementación y las pruebas que surjan para
llevar estos cambios a cabo.
\end{observacion}

\end{enumerate}

\section{Realizando un programa sencillo}

Al leer un artículo en una revista norteamericana que contiene información de
longitudes expresadas en millas, pies y pulgadas, queremos poder convertir esas
distancias de modo que sean fáciles de entender.  Para ello, decidimos escribir
un programa que convierta las longitudes del sistema inglés al sistema métrico
decimal.

Antes de comenzar a programar, utilizamos la guía de la sección anterior, para
analizar, especificar, diseñar, implementar y probar el problema.

\begin{enumerate}
\item {\bf Análisis del problema.} En este caso el problema es
sencillo: nos dan un valor expresado en millas, pies y pulgadas y
queremos transformarlo en un valor en el sistema métrico decimal.
Sin embargo hay varias respuestas posibles, porque no hemos fijado
en qué unidad queremos el resultado. Supongamos que decidimos que
queremos expresar todo en metros.

\item {\bf Especificación.} Debemos establecer la relación entre
los datos de entrada y los datos de salida. Ante todo debemos
averiguar los valores para la conversión de las unidades básicas.
Buscando en Internet encontramos la siguiente tabla:

\begin{itemize}
\item 1 milla = 1.609344 km
\item 1 pie = 30.48 cm
\item 1 pulgada = 2.54 cm
\end{itemize}

\begin{atencion}
A lo largo de todo el curso usaremos punto decimal,
en lugar de coma decimal, para representar valores no enteros,
dado que esa es la notación que utiliza Python.
\end{atencion}

La tabla obtenida no traduce las longitudes a metros. La manipulamos para
llevar todo a metros:

\begin{itemize}
\item 1 milla = 1609.344 m
\item 1 pie = 0.3048 m
\item 1 pulgada = 0.0254 m
\end{itemize}

Si una longitud se expresa como $L$ millas, $F$ pies y $P$ pulgadas, su
conversión a metros se calculará como:

$$
M = 1609.344 * L + 0.3048 * F + 0.0254 * P
$$

Hemos especificado el problema. Pasamos entonces a la próxima etapa.

\item {\bf Diseño.} La estructura de este programa es sencilla:
leer los datos de entrada, calcular la solución, mostrar el
resultado, o \emph{Entrada-Cálculo-Salida}.

Antes de escribir el programa, escribiremos en \emph{pseudocódigo}
(un castellano preciso que se usa para describir lo que hace un
programa) una descripción del mismo:

\begin{codigo-nohl-sn}
Leer cuántas millas tiene la longitud dada
 (y referenciarlo con la variable millas)

Leer cuántos pies tiene la longitud dada
 (y referenciarlo con la variable pies)

Leer cuántas pulgadas tiene la longitud dada
 (y referenciarlo con la variable pulgadas)

Calcular metros = 1609.344 * millas +
    0.3048 * pies + 0.0254 * pulgadas

Mostrar por pantalla la variable metros
\end{codigo-nohl-sn}

\item {\bf Implementación.} Ahora estamos en condiciones de
traducir este pseudocódigo a un programa en lenguaje Python:

\begin{codigo}{ametrico.py}{Convierte medidas inglesas a sistema metrico}
\begin{codigo-python}
def main():
    print("Convierte medidas inglesas a sistema metrico")

    millas = int(input("Cuántas millas?: "))
    pies = int(input("Y cuántos pies?: "))
    pulgadas = int(input("Y cuántas pulgadas?: "))

    metros = 1609.344 * millas + 0.3048 * pies + 0.0254 * pulgadas
    print("La longitud es de ", metros, " metros")

main()
\end{codigo-python}
\end{codigo}

\begin{nota}
En nuestra implementación decidimos dar el nombre |main| a la función principal
del programa. Esto no es más que una convención: ``main'' significa
``principal'' en inglés.
\end{nota}

\item {\bf Prueba.} Probaremos el programa con valores para los que conocemos
la solución:

\begin{itemize}
\item 1 milla, 0 pies, 0 pulgadas (el resultado debe ser 1609.344 metros).
\item 0 millas, 1 pie, 0 pulgada (el resultado debe ser 0.3048 metros).
\item 0 millas, 0 pies, 1 pulgada (el resultado debe ser 0.0254 metros).
\end{itemize}

La prueba la documentaremos con la sesión de Python
correspondiente a las tres invocaciones a \lstinline!ametrico.py!.
\end{enumerate}

En la sección anterior hicimos hincapié en la necesidad de
documentar todo el proceso de desarrollo. En este ejemplo la
documentación completa del proceso lo constituye todo lo escrito
en esta sección.

\section {Piezas de un programa Python}
Cuando empezamos a hablar en un idioma extranjero es posible que nos entiendan
pese a que cometamos errores. No sucede lo mismo con los lenguajes de
programación: la computadora no nos entenderá si nos desviamos un poco de
alguna de las reglas.

Por eso es que para poder empezar a programar en Python es necesario conocer los elementos
que constituyen un programa en dicho lenguaje y las reglas para construirlos.


\subsection{Nombres}
Ya hemos visto que se usan nombres para denominar a los programas
(\lstinline!ametrico!) y para denominar a las funciones dentro de un
módulo (\lstinline!main!). Cuando queremos dar nombres a valores usamos
variables (\lstinline!millas!, \lstinline!pies!, \lstinline!pulgadas!,
\lstinline!metros!). Todos esos nombres se llaman \emph{identificadores}
y Python tiene reglas sobre qué es un identificador válido y qué
no lo es.

Un identificador comienza con una letra o con guión bajo (\_) y
luego sigue con una secuencia de letras, números y guiones bajos.
Los espacios no están permitidos dentro de los identificadores.

Los siguientes son todos identificadores válidos de Python:

\begin{itemize}[noitemsep]
\item \lstinline!hola!
\item \lstinline!hola12t!
\item \lstinline!_hola!
\item \lstinline!Hola!
\end{itemize}

Python distingue mayúsculas de minúsculas, así que \lstinline!Hola! es
un identificador y \lstinline!hola! es otro identificador.

\begin{observacion}
Por convención, no usaremos identificadores que empiezan con mayúscula.
\end{observacion}

Los siguientes son todos identificadores inválidos de Python:

\begin{itemize}[noitemsep]
\item \lstinline!hola a12t!
\item \lstinline!8hola!
\item \lstinline!hola\%!
\item \lstinline!Hola*9!
\end{itemize}

Python reserva 31 palabras para describir la estructura del
programa, y no permite que se usen como identificadores. Cuando en
un programa nos encontramos con que un nombre no es admitido pese
a que su formato es válido, seguramente se trata de una de las
palabras de esta lista, a la que llamaremos de \emph{palabras
reservadas}. Esta es la lista completa de las palabras reservadas de
Python:

\begin{codigo-nohl-sn}
False      class      finally    is         return
None       continue   for        lambda     try
True       def        from       nonlocal   while
and        del        global     not        with
as         elif       if         or         yield
assert     else       import     pass
break      except     in         raise
\end{codigo-nohl-sn}

\subsection{Expresiones}
Una \emph{expresión} es una porción de código Python que produce o
calcula un \emph{valor} (resultado).

\begin{itemize}
\item La expresión más sencilla es un valor \emph{literal}.  Por ejemplo,
    la expresión |12345| produce el valor numérico 12345.

\item Una expresión puede ser una \emph{variable}, y el valor que produce es el
    que tiene asociada la variable en el estado. Por ejemplo, si |x| $\ra$ 5 en
    el estado, entonces el resultado de la expresión |x|  es el valor 5.

\item Usamos \emph{operaciones} para combinar expresiones y construir
expresiones más complejas:

\begin{itemize}
\item Si \lstinline!x! es como antes, \lstinline!x + 1! es una expresión cuyo
resultado es 6.

\item Si en el estado \lstinline!millas! $\ra$ 1, \lstinline!pies! $\ra$ 0 y
\lstinline!pulgadas! $\ra$ 0, entonces
\lstinline[breaklines=true]!1609.344 * millas + 0.3048 * pies + 0.0254 * pulgadas! es una
expresión cuyo resultado es 1609.344.

\item La exponenciación se representa con el símbolo \lstinline!**!. Por
ejemplo, \lstinline!x**3! significa $x^3$.

\item Se pueden usar paréntesis para indicar un orden de
evaluación: \lstinline[breaklines=true]!((b * b) - (4 * a * c)) / (2 * a)!.

\item Igual que en la notación matemática, si no hay paréntesis en la
expresión, primero se agrupan las exponenciaciones, luego los
productos y cocientes, y luego las sumas y restas.

\item Hay que prestar atención con lo que sucede con los
cocientes:

\begin{itemize}
\item La expresión |6 / 4| produce el valor |1.5|.
\item La expresión |6 // 4| produce el valor |1|, que es el resultado de la
    \emph{división entera} entre 6 y 4.
\item La expresión |6 % 4| produce el valor |2|, que es el
    \emph{resto de la división entera} entre 6 y 4.
\end{itemize}

Como vimos en la sección \ref{punto-flotante}, los números pueden ser tanto
enteros (|0|, |111|, |-24|, almacenados internamente en forma exacta),
como reales (|0.0|, |12.5|, |-12.5|, representados internamente en
forma aproximada como números \emph{de punto flotante}). Dado que los
números enteros y reales se representan de manera diferente, se
comportan de manera diferente frente a las operaciones. En Python, los
números enteros se denominan |int| (de \emph{integer}), y los números
reales |float| (de \emph{floating point}).

\end{itemize}

\item Una expresión puede ser una \emph{llamada a una función}: si |f| es una
    función que recibe un parámetro, y |x| es una variable, la expresión |f(x)|
    produce el valor que devuelve la función |f| al invocarla pasándole el valor
    de |x| por parámetro.

    Algunos ejemplos:

\begin{itemize}
\item |input()| produce el valor ingresado por teclado tal como se lo digita.
\item |abs(x)| produce el valor absoluto del número pasado por parámetro.
\end{itemize}

\end{itemize}

\ejercicioc{%
Aplicando las reglas matemáticas de asociatividad, decidir
cuáles de las siguientes expresiones son iguales entre sí:

\begin{partes}
\item \lstinline!((b * b) - (4 * a * c)) / (2 * a)!
\item \lstinline!((b * b) - (4 * a * c)) // (2 * a)!
\item \lstinline!(b * b - 4 * a * c) / (2 * a)!
\item \lstinline!b * b - 4 * a * c / 2 * a!
\item \lstinline!(b * b) - (4 * a * c / 2 * a)!
\item \lstinline!1 / 2 * b!
\item \lstinline!b / 2!
\end{partes}}

\ejercicioc{Escribir un programa que le asigne a \lstinline!a!, \lstinline!b! y
\lstinline!c!  los valores 10, 100 y 1000 respectivamente y evalúe las
expresiones del ejercicio anterior.}

\ejercicioc{Escribir un programa que le asigne a \lstinline!a!, \lstinline!b! y
\lstinline!c!  los valores 10.0, 100.0 y 1000.0 respectivamente y evalúe las
expresiones del ejercicio anterior.}

\subsection{No sólo de números viven los programas} \label{nosolo}

No sólo tendremos expresiones numéricas en un programa Python.
Recordemos el programa que se usó para saludar a muchos amigos:

\begin{codigo-python-sn}
>>> def hola(alguien):
...     return "Hola " + alguien + "! Estoy programando en Python."
\end{codigo-python-sn}

Para invocar a ese programa y hacer que saludara a Ana había que
escribir \lstinline!hola("Ana")!.
La variable \lstinline!alguien! en dicha invocación queda ligada a un
valor que es una \emph{cadena de caracteres} (letras, dígitos, símbolos,
etc.), en este caso, \lstinline!"Ana"!.

Como en la sección anterior, veremos las reglas de qué constituyen
expresiones con caracteres:

\begin{itemize}
\item Una expresión puede ser simplemente una cadena de texto.  El resultado de
    la expresión literal \lstinline!'Ana'! es precisamente el valor \lstinline!'Ana'!.

\item Una variable puede estar asociada a una cadena de texto: si
    \lstinline!amiga! $\ra$ \lstinline!'Ana'! en el estado, entonces el
    resultado de la expresión \lstinline!amiga! es el valor \lstinline!'Ana'!.

\item Se puede usar comillas simples o dobles para representar cadenas simples:
    |'Ana'| y |"Ana"| son equivalentes.

\item Se puede usar tres comillas (simples o dobles) para representar cadenas
    que incluyen más de una línea de texto:

\begin{codigo-python-sn}
martin_fierro = """Aquí me pongo a cantar
al compás de la vigüela,
que al hombre que lo desvela
una pena estraordinaria,
como el ave solitaria
con el cantar se consuela."""
\end{codigo-python-sn}

\item Usamos operaciones para combinar expresiones y construir
expresiones más complejas, pero atención con qué operaciones están
permitidas sobre cadenas:

\begin{itemize}
    \item El signo \lstinline!+! no representa la suma sino la
        \emph{concatenación} de cadenas: Si \lstinline!amiga! es como antes,
        \lstinline!amiga + 'Laura'!  es una expresión cuyo valor es
        \lstinline!AnaLaura!.

\begin{atencion}
No se puede sumar cadenas con números.
\begin{codigo-python-sn}
>>> amiga="Ana"
>>> amiga+'Laura'
'AnaLaura'
>>> amiga+3
(^Traceback (most recent call last):
  File "<stdin>", line 1, in <module>
TypeError: cannot concatenate 'str' and 'int' objects^)
>>>
\end{codigo-python-sn}
\end{atencion}

\item El signo \lstinline!*! permite repetir una cadena una cantidad de veces:
    \lstinline!amiga * 3! es una expresión cuyo valor es
    \lstinline!'AnaAnaAna'!.

\begin{atencion}
No se pueden multiplicar cadenas entre sí

\begin{codigo-python-sn}
>>> amiga * 3
'AnaAnaAna'
>>> amiga * amiga
(^Traceback (most recent call last):
  File "<stdin>", line 1, in <module>
TypeError: can't multiply sequence by non-int of type 'str'^)
\end{codigo-python-sn}
\end{atencion}

\end{itemize}

\end{itemize}

\subsection{Instrucciones}

Las \emph{instrucciones} son las órdenes que entiende Python.  En general cada
línea de un programa Python corresponde a una instrucción.  Algunos ejemplos de
instrucciones que ya hemos utilizado:

\begin{itemize}
\item La instrucción de asignación |<nombre> = <valor>|.

\item La instrucción |return <expresión>|, que provoca que una función devuelva
    el valor resultante de evaluar la expresión.

\item \label{instruccion-expresion} La instrucción más simple que hemos utilizado es la que contiene una
    única |<expresión>|, y el efecto de dicha instrucción es que Python evalúa
    la expresión y descarta su resultado. El siguiente es un programa válido
    en el que todas las instrucciones son del tipo |<expresión>|:

\begin{codigo-python-sn}
0
23.9
abs(-10)
"Este programa no hace nada útil :("
\end{codigo-python-sn}

\end{itemize}

\subsection{Ciclos definidos}
Algunas instrucciones son compuestas, como por ejemplo la instrucción |for|,
que indica a Python que queremos inicializar un \emph{ciclo definido}:

\begin{codigo-python-sn}
for x in range(n1, n2):
    print(x * x)
\end{codigo-python-sn}

Un ciclo definido es de la forma
\begin{codigo-python-sn}
for <nombre> in <expresión>:
    <cuerpo>
\end{codigo-python-sn}

El ciclo |for| es una instrucción compuesta ya que incluye una línea de
inicialización y un |<cuerpo>|, que a su vez está formado por una o más
instrucciones.

Decimos que el ciclo es definido porque una vez evaluada la |<expresión>|
(cuyo resultado debe ser una \emph{secuencia de valores}),
se sabe exactamente cuántas veces se ejecutará
el |<cuerpo>| y qué valores tomará la variable |<nombre>|.

En nuestro ejemplo la secuencia de valores resultante de la expresión
|range(n1, n2)| es el intervalo de enteros |[n1, n1+1, ..., n2-1]| y la variable
es |x|.

La secuencia de valores se puede indicar como:

\begin{itemize}
\item \lstinline!range(n)!. Establece como secuencia de valores a
 \lstinline![0, 1, ..., n-1]!.

\item \lstinline!range(n1, n2)!. Establece como secuencia de valores a
\lstinline![n1, n1+1, ..., n2-1]!.

\item Se puede definir a mano una secuencia entre corchetes. Por ejemplo,
\begin{codigo-python-sn}
for x in [1, 3, 9, 27]:
    print(x * x)
\end{codigo-python-sn}
imprimirá los cuadrados de los números 1, 3, 9 y 27.
\end{itemize}

\section{Una guía para el diseño}

En su artículo ``How to program it'', Simon Thompson plantea
algunas preguntas a sus alumnos que son muy útiles para la etapa
de diseño:

\begin{itemize}
\item ¿Has visto este problema antes, aunque sea de manera
ligeramente diferente?

\item ¿Conoces un problema relacionado? ¿Conoces un programa que
pueda ser útil?

\item Observa la especificación. Intenta encontrar un
problema que te resulte familiar y que tenga la misma
especificación o una parecida.

\item Supongamos que hay un problema relacionado, y que ya fue
resuelto. ¿Puedes usarlo? ¿Puedes usar sus resultados?  ¿Puedes usar sus
métodos? ¿Puedes agregarle alguna parte auxiliar a ese programa del que ya
dispones?

\item Si no puedes resolver el problema propuesto, intenta
resolver uno relacionado. ¿Puedes imaginarte uno relacionado que
sea más fácil de resolver? ¿Uno más general? ¿Uno más específico?
¿Un problema análogo?

\item ¿Puedes resolver una parte del problema?  ¿Puedes sacar algo
útil de los datos de entrada? ¿Puedes pensar qué información es
útil para calcular las salidas? ¿De qué manera se puede manipular
las entradas y las salidas de modo tal que estén ``más cerca''
unas de las otras?

\item ¿Utilizaste todos los datos de entrada? ¿Utilizaste las condiciones
especiales sobre los datos de entrada que aparecen en el
enunciado? ¿Has tenido en cuenta todos los requisitos que se
enuncian en la especificación?

\end{itemize}

\section{Calidad de software}

Los programas que hemos construido hasta ahora son pequeños y simples. Existen
proyectos de software profesionales de tamaños muy diversos, yendo desde
programas sencillos desarrollados por una única persona hasta
proyectos gigantescos, con millones de líneas de código y desarrollados
durante años por miles de personas.

\begin{sabias_que}
Uno de los proyectos de código abierto más colosales es el núcleo
del sistema operativo Linux. Fue publicado por primera vez en 1991, y aun
hoy sigue en desarrollo activo. El código fuente es
público\footnote{\url{https://github.com/torvalds/linux}}, y cualquiera
puede contribuir aportando mejoras. Hasta la versión 4.13 publicada en 2017
participaron más de $15\,000$ personas, creando en total más de 24 millones de
líneas de código.
\end{sabias_que}

Cuanto más grande es un proyecto de software, más difícil es su construcción y
mantenimiento, y más tenemos que prestar atención a la \emph{calidad} con la
que está construido. Presentamos aquí una lista no completa de propiedades que
contribuyen a la calidad, y algunas preguntas que podemos hacer para medir
cuánto contribuye cada factor:

\begin{itemize}
        \item {\bf Confiabilidad:} ¿El sistema resuelve el problema inicial en
            forma correcta? ¿Lo resuelve siempre o a veces falla? ¿Cuántas
            veces falla en un período de tiempo?
        \item {\bf Testabilidad:} ¿Qué tan fácil es probar que el sistema
            funciona correctamente? ¿Hay algún proceso de pruebas automáticas o
            manuales?
        \item {\bf Performance:} ¿Cuánto tarda el sistema en producir un
            resultado? ¿Cuántos recursos consume (memoria, espacio en disco,
            etc.)?
        \item {\bf Usabilidad:} ¿Puede un nuevo usuario aprender a utilizar el
            sistema fácilmente? ¿Las operaciones más comunes son fáciles de
            realizar?
        \item {\bf Mantenibilidad:} ¿Qué tan legible y entendible es el código?
            ¿Qué tan fácil es modificar el comportamiento del programa o
            agregar nuevas funcionalidades?
        \item {\bf Escalabilidad:} ¿Cómo se comporta el sistema cuando
            se incrementa la demanda (cantidad de usuarios, cantidad de datos,
            etc.)?
        \item {\bf Portabilidad:} ¿El sistema puede funcionar en diferentes
            plataformas (arquitecturas de procesador, sistemas operativos,
            navegadores web, etc.)?
        \item {\bf Seguridad:} ¿Los datos sensibles están protegidos de ataques
            informáticos? ¿Qué tan difícil es para un atacante tomar el control,
            desestabilizar o dañar el sistema?
\end{itemize}

Por supuesto, cada proyecto es particular y algunos de las propiedades
mencionadas tendrán más o menos prioridad según el caso. En particular en este
curso nos concentraremos más en que nuestros programas sean confiables y
mantenibles, y también prestaremos atención a la performance (sobre todo al
comparar diferentes algoritmos).

\section{Ejercicios}

\begin{extract*}
En los ejercicios a continuación, utilizar los conceptos de análisis,
especificación y diseño antes de realizar la implementación.
\end{extract*}

\begin{ejercicio} Ciclos definidos
\begin{partes}
	\item Escribir un ciclo definido para imprimir por pantalla
todos los números entre 10 y 20.
	\item Escribir un ciclo definido que salude por pantalla a
sus cinco mejores amigos/as.
	\item Escribir un programa que use un ciclo definido con
rango numérico, que pregunte los nombres de sus cinco mejores
amigos/as, y los salude.
	\item Escribir un programa que use un ciclo definido con
rango numérico, que pregunte los nombres de sus seis mejores
amigos/as, y los salude.
	\item Escribir un programa que use un ciclo definido con
rango numérico, que averigue a cuántos amigos quieren saludar, les
pregunte los nombres de esos amigos/as, y los salude.
\end{partes}
\end{ejercicio}

\extractionlabel{guia}
\begin{ejercicio}
Escribir una función que reciba una cantidad de pesos,
una tasa de interés y un número de años y devuelva el monto
final a obtener.  La fórmula a utilizar es:
\begin{displaymath}
C_n = C \times \left(1+\frac{x}{100}\right)^n
\end{displaymath}
Donde $C$ es el capital inicial, $x$ es la tasa de interés y $n$ es el
número de años a calcular.
\end{ejercicio}

\extractionlabel{guia}
\begin{ejercicio}
Utilizando la función del ejercicio anterior, escribir un programa que le
pregunte al usuario la cantidad de pesos inicial, la tasa de interés y el
número de años y muestre el monto final a obtener.
\end{ejercicio}

\extractionlabel{guia}
\begin{ejercicio}
Escribir una función que convierta un valor dado en grados Fahrenheit a
grados Celsius.  Recordar que la fórmula para la conversión es:
$F = \frac{9}{5}C+32$
\end{ejercicio}

\extractionlabel{guia}
\begin{ejercicio}
Escribir un programa que utilice la función anterior para generar una tabla de conversión de
temperaturas, desde 0 °F hasta 120 °F, de 10 en 10.
\end{ejercicio}

\extractionlabel{guia}
\begin{ejercicio}
\begin{partes}
\item Escribir una función que dado un número entero devuelva 1 si el mismo
es impar y 0 si fuera par.
\item Escribir una función que dado un número entero devuelva 0 si el mismo
es impar y 1 si fuera par.
\item Escribir una función que dado un número entero devuelva el dígito de las unidades.
Por ejemplo, para 153 debe devolver 3.
\item Escribir una función que dado un número devuelva el primer número múltiplo
de 10 inferior a él. Por ejemplo, para 153 debe devolver 150.
\end{partes}
\end{ejercicio}

\extractionlabel{guia}
\begin{ejercicio}
Escribir un programa que imprima todos los números pares entre dos números
que se le pidan al usuario.
\end{ejercicio}

\extractionlabel{guia}
\begin{ejercicio}
Escribir un programa que le pregunte al usuario un número $n$
e imprima los primeros $n$ números triangulares, junto con su
índice. Los números triangulares se obtienen mediante la suma de los números
naturales desde $1$ hasta $n$.  Es decir, si se piden los primeros 5
números triangulares, el programa debe imprimir:

\begin{verbatim}
1 - 1
2 - 3
3 - 6
4 - 10
5 - 15
\end{verbatim}

{\bf Nota}: hacerlo usando y sin usar la ecuación $\sum_{i=1}^n i = n\,(n+1)/2$.
¿Cuál realiza más operaciones?
\end{ejercicio}

\extractionlabel{guia}
\begin{ejercicio}
Escribir un programa que tome una cantidad $m$ de valores ingresados
por el usuario, a cada uno le calcule el factorial (utilizando la función escrita en el ejercicio
\ref{ej:factorial}) e imprima el resultado junto con el número de orden correspondiente.
\end{ejercicio}

\extractionlabel{guia}
\begin{ejercicio}
Escribir un programa que imprima por pantalla todas las fichas de dominó, de
una por línea y sin repetir.
\end{ejercicio}

\extractionlabel{guia}
\begin{ejercicio}
Modificar el programa anterior para que pueda generar fichas de un juego
que puede tener números de 0 a $n$.
\end{ejercicio}

\include{3_funciones}
\include{4_decisiones}
\include{5_ciclos}
\include{6_cadenas}
\chapter{Tuplas y listas}

Python cuenta con una gran variedad de tipos de datos que permiten
representar la información según cómo esté estructurada.  En esta unidad se
estudian las tuplas y las listas, que son tipos de datos utilizados cuando
se quiere agrupar elementos.

\section{Tuplas}

Al diseñar algoritmos, es muy común que queramos describir un agrupamiento de
datos de distintos tipos.

Esto es algo que ya hicimos anteriormente: en la conversión de un tiempo a
horas, minutos y segundos (sección~\ref{fun:multiple_return}) y también en el
juego Mastermind (sección \ref{mastermind}), usamos n-uplas, que en Python se
llaman \emph{tuplas}, y sirven para representar agrupaciones de datos ordenados.

Veamos más ejemplos:

\begin{itemize}

\item Una fecha la podemos querer representar como la terna día (un número
entero), mes (una cadena de caracteres), y año (un número entero), y
tendremos por ejemplo la tupla \lstinline!(25, "Mayo", 1810)!.

\item Como datos de los alumnos queremos guardar número de padrón, nombre y
apellido, como por ejemplo \lstinline!(89766, "Alicia", "Hacker")!.

\item {\bf Es posible anidar tuplas:} como datos de los alumnos
queremos guardar número de padrón, nombre, apellido y fecha de nacimiento,
como por ejemplo: \\
\lstinline!(89766, "Alicia", "Hacker", (9, "Julio", 1988))!.
\end{itemize}

\begin{observacion}
En Python el tipo de dato asociado a las tuplas se llama |tuple|:

\begin{codigo-python-sn}
>>> fecha = (25, "Mayo", 1810)
>>> type(fecha)
<class 'tuple'>
\end{codigo-python-sn}
\end{observacion}

\subsection{Elementos y segmentos de tuplas}

Las tuplas son \emph{secuencias}, igual que las cadenas, y se puede utilizar la
misma notación de índices que en las cadenas para obtener cada una de sus
componentes:
\begin{codigo-python-sn}
>>> fecha = (25, "Mayo", 1810)
>>> fecha[0]
25
>>> fecha[1]
'Mayo'
>>> fecha[2]
1810
\end{codigo-python-sn}

\begin{atencion}
Todas las secuencias en Python comienzan a numerarse desde 0.  Es por eso
que se produce un error si se quiere acceder al n-ésimo elemento de un
tupla:
\begin{codigo-python-sn}
>>> fecha[3]
(^Traceback (most recent call last):
  File "<stdin>", line 1, in <module>
IndexError: tuple index out of range^)
\end{codigo-python-sn}
\end{atencion}

También se puede utilizar la notación de rangos, que se vio aplicada a
cadenas para obtener una nueva tupla, con un subconjunto de componentes. Si
en el ejemplo de la fecha queremos quedarnos con un par que sólo contenga
día y mes podremos tomar el rango |[:2]| de la misma:

\begin{codigo-python-sn}
>>> fecha[:2]
(25, 'Mayo')
\end{codigo-python-sn}

\ejercicioc{¿Cuál es el resultado de obtener el cuarto elemento de la tupla
\lstinline!(89766, "Alicia", "Hacker", (9, "Julio", 1988))!?}

\subsection{Las tuplas son inmutables}

Al igual que con las cadenas, las componentes de las tuplas no pueden ser
modificadas:

\begin{codigo-python-sn}
>>> fecha[2] = 2018
(^Traceback (most recent call last):
  File "<stdin>", line 1, in <module>
TypeError: 'tuple' object does not support item assignment^)
\end{codigo-python-sn}

\subsection{Longitud de tuplas}

A las tuplas también se les puede aplicar la función \lstinline+len()+
para calcular su longitud. El valor de esta función aplicada a
una tupla nos indica cuántas componentes tiene esa tupla.

\begin{codigo-python-sn}
>>> len(fecha)
3
\end{codigo-python-sn}

\ejercicioc{¿Cuál es la longitud de la tupla
\lstinline!(89766, "Alicia", "Hacker", (9, "Julio", 1988))!?}

\begin{itemize}
\item Una \emph{tupla vacía} es una tupla con $0$ componentes, y se la
indica como \lstinline+()+.

\begin{codigo-python-sn}
>>> z = ()
>>> len(z)
0
>>> z[0]
(^Traceback (most recent call last):
  File "<stdin>", line 1, in <module>
IndexError: tuple index out of range^)
\end{codigo-python-sn}

\item Una \emph{tupla unitaria} es una tupla con una componente. Para
distinguir la tupla unitaria de la componente que contiene, la sintaxis de Python exige
que a la componente no sólo se la encierre entre paréntesis sino que se le
ponga una coma a continuación del valor de la componente (así,
\lstinline+(1810)+ es un número, pero \lstinline+(1810,)+ es la tupla
unitaria cuya única componente vale 1810).

\begin{codigo-python-sn}
>>> u = (1810)
>>> len(u)
(^Traceback (most recent call last):
  File "<stdin>", line 1, in <module>
TypeError: object of type 'int' has no len()^)
>>> u = (1810,)
>>> len(u)
1
>>> u[0]
1810
\end{codigo-python-sn}
\end{itemize}

\subsection{Empaquetado y desempaquetado de tuplas}
Si a una variable se le asigna una secuencia de valores separados por comas,
el valor de esa variable será la tupla formada por todos los valores asignados.
A esta operación se la denomina \emph{empaquetado de tuplas}.

\begin{codigo-python-sn}
>>> a = 125
>>> b = "#"
>>> c = "Ana"
>>> d = a, b, c
>>> len(d)
3
>>> d
(125, '#', 'Ana')
\end{codigo-python-sn}

Si se tiene una tupla de longitud \lstinline+k+, se puede asignar
la tupla a \lstinline+k+ variables distintas y en cada variable quedará
una de las componentes de la tupla. A esta operación se la denomina
\emph{desempaquetado de tuplas}.

\begin{codigo-python-sn}
>>> x, y, z = d
>>> x
125
>>> y
'#'
>>> z
'Ana'
\end{codigo-python-sn}

\begin{atencion}
Si las variables no son distintas, se pierden valores. Y si las variables
no son exactamente \lstinline+k+ se produce un error.

\begin{codigo-python-sn}
>>> p, p, p = d
>>> p
'Ana'
>>> m, n = d
(^Traceback (most recent call last):
  File "<stdin>", line 1, in <module>
ValueError: too many values to unpack^)
>>> m, n, o, p = d
(^Traceback (most recent call last):
  File "<stdin>", line 1, in <module>
ValueError: need more than 3 values to unpack^)
\end{codigo-python-sn}
\end{atencion}

\begin{sabias_que}
En la implementación de algunos programas suele ser necesario intercambiar el
valor de dos variables. Si queremos intercambiar el valor de |a| y |b|, una
posibilidad es utilizar una variable auxiliar:

\begin{codigo-python-sn}
aux = a
a = b
b = aux
\end{codigo-python-sn}

Pero un ``truco'' que permiten algunos lenguajes, entre ellos Python, es la
posibilidad de empaquetar y desempaquetar una tupla en una única operación:

\begin{codigo-python-sn}
a, b = b, a
\end{codigo-python-sn}
\end{sabias_que}

\subsection{Ejercicios con tuplas}

\ejercicioc{Cartas como tuplas.
\begin{partes}
\item Proponer una representación con tuplas para las cartas de la baraja
francesa.

\item Escribir una función \lstinline!poker! que reciba cinco cartas de la
baraja francesa e informe (devuelva el valor lógico correspondiente) si esas
cartas forman o no un \emph{poker} (es decir que hay 4 cartas con el mismo
número).
\end{partes}
}

\ejercicioc{El tiempo como tuplas.
\begin{partes}
\item Proponer una representación con tuplas para representar el tiempo.
\item Escribir una función \lstinline!sumar_tiempos! que reciba dos tiempos dados y
devuelva su suma.
\end{partes}
}

\ejercicioc{Escribir una función \lstinline!dia_siguiente! que dada una fecha
expresada como la terna |(Día, Mes, Año)| (donde |Día|, |Mes| y |Año| son números
enteros) calcule el día siguiente al dado, en el mismo formato.}

\ejercicioc{Escribir una función \lstinline!dia_siguiente_m! que dada una fecha
expresada como la terna |(Día, Mes, Año)| (donde |Día| y |Año| son números
enteros, y |Mes| es el texto \lstinline!"Ene"!, \lstinline!"Feb"!, $\ldots$,
\lstinline!"Dic"!, según corresponda) calcule el día siguiente al dado, en
el mismo formato.}

\section{Listas}

Presentaremos ahora una nueva estructura de datos: la \emph{lista}.
Usaremos listas para poder modelar datos compuestos, pero cuya cantidad y
valor varían a lo largo del tiempo. Son secuencias \emph{mutables} y vienen
dotadas de una variedad de operaciones muy útiles.

La notación para lista es una secuencia de valores encerrados entre
corchetes y separados por comas.  Por ejemplo, si representamos a los
alumnos mediante su número de padrón, se puede tener una lista de
inscriptos en la materia como la siguiente:
\lstinline![78455, 89211, 66540, 45750]!.
Al abrirse la inscripción, antes de que hubiera inscriptos, la lista de
inscriptos se representará por una lista vacía: \lstinline![]!.

\begin{observacion}
En Python el tipo de dato asociado a las listas se llama |list|:

\begin{codigo-python-sn}
>>> type([78455, 89211, 66540, 45750])
<class 'list'>
\end{codigo-python-sn}
\end{observacion}

\subsection{Longitud de la lista. Elementos y segmentos de listas}

\begin{itemize}

\item Como a las secuencias ya vistas, a las listas también se les puede
aplicar la función \lstinline+len()+ para conocer su longitud.

\item Para acceder a los distintos elementos de la lista se utilizará la
misma notación de índices de cadenas y tuplas, con valores que van de $0$ a la
longitud de la lista $- 1$.

\begin{codigo-python-sn}
>>> padrones = [78455, 89211, 66540, 45750]
>>> padrones[0]
78455
>>> len(padrones)
4
>>> padrones[4]
(^Traceback (most recent call last):
  File "<stdin>", line 1, in <module>
IndexError: list index out of range^)
>>> padrones[3]
45750
\end{codigo-python-sn}

\item Para obtener una sublista a partir de la lista original, se utiliza
la notación de rangos, como en las otras secuencias.

Para obtener la lista que contiene sólo a quién se inscribió en segundo
lugar podemos escribir:

\begin{codigo-python-sn}
>>> padrones[1:2]
[89211]
\end{codigo-python-sn}

Para obtener la lista que contiene al segundo y tercer inscriptos
podemos escribir:

\begin{codigo-python-sn}
>>> padrones[1:3]
[89211, 66540]
\end{codigo-python-sn}

Para obtener la lista que contiene al primero y segundo inscriptos
podemos escribir:

\begin{codigo-python-sn}
>>> padrones[:2]
[78455, 89211]
\end{codigo-python-sn}

\end{itemize}

\subsection{Cómo mutar listas}

Dijimos antes que las listas son secuencias mutables. Para lograr la
mutabilidad Python provee operaciones que nos permiten cambiarle valores,
agregarle valores y quitarle valores.

\begin{itemize}

\item Para cambiar una componente de una lista, se selecciona la componente
mediante su índice y se le asigna el nuevo valor:

\begin{codigo-python-sn}
>>> padrones[1] = 79211
>>> padrones
[78455, 79211, 66540, 45750]
\end{codigo-python-sn}

\item Para agregar un nuevo valor al final de la lista se utiliza la
operación \lstinline+append()+.  Escribimos \lstinline+padrones.append(47890)+ para
agregar el padrón 47890 al final de \lstinline+padrones+.

\begin{codigo-python-sn}
>>> padrones.append(47890)
>>> padrones
[78455, 79211, 66540, 45750, 47890]
\end{codigo-python-sn}

\item Para insertar un nuevo valor en la posición cuyo índice es
\lstinline+k+ (y desplazar un lugar el resto de la lista) se utiliza la
operación \lstinline+insert()+.

Escribimos \lstinline+padrones.insert(2, 54988)+ para insertar el padrón 54988 en
la tercera posición de \lstinline+padrones+.

\begin{codigo-python-sn}
>>> padrones.insert(2, 54988)
>>> padrones
[78455, 79211, 54988, 66540, 45750, 47890]
\end{codigo-python-sn}

\item Las listas no controlan si se insertan elementos repetidos. Si necesitamos
exigir unicidad, debemos hacerlo mediante el código de nuestros programas.
\begin{codigo-python-sn}
>>> padrones.insert(1,78455)
>>> padrones
[78455, 78455, 79211, 54988, 66540, 45750, 47890]
\end{codigo-python-sn}

\item Para eliminar un valor de una lista se utiliza la operación
\lstinline+remove()+.

Escribimos \lstinline+padrones.remove(45750)+ para borrar el padrón 45750 de la
lista de inscriptos:

\begin{codigo-python-sn}
>>> padrones.remove(45750)
>>> padrones
[78455, 78455, 79211, 54988, 66540, 47890]
\end{codigo-python-sn}

Si el valor a borrar está repetido, se borra sólo su primera aparición:

\begin{codigo-python-sn}
>>> padrones.remove(78455)
>>> padrones
[78455, 79211, 54988, 66540, 47890]
\end{codigo-python-sn}

\begin{atencion}
Si el valor a borrar no existe, se produce un error:

\begin{codigo-python-sn}
>>> padrones.remove(78)
(^Traceback (most recent call last):
  File "<stdin>", line 1, in <module>
ValueError: list.remove(x): x not in list^)
\end{codigo-python-sn}
\end{atencion}
\end{itemize}

\subsection{Cómo buscar dentro de las listas}

Queremos poder formular dos preguntas más respecto de la lista de
inscriptos:

\begin{itemize}
\item ¿Está la persona cuyo padrón es $v$ inscripta en esta materia?

\item ¿En qué orden se inscribió la persona cuyo padrón es $v$?.
\end{itemize}

Veamos qué operaciones sobre listas se pueden usar para lograr esos dos
objetivos:

\begin{itemize}
\item Para preguntar si un valor determinado es un elemento de una lista
usaremos la operación \lstinline+in+:

\begin{codigo-python-sn}
>>> padrones
[78455, 79211, 54988, 66540, 47890]
>>> 78 in padrones
False
>>> 66540 in padrones
True
\end{codigo-python-sn}

\begin{observacion}
El operador |in| se puede utilizar para todas las secuencias, incluyendo
tuplas y cadenas.
\end{observacion}

\item Para averiguar la posición de un valor dentro de una lista usaremos
la operación \lstinline+index()+.

\begin{codigo-python-sn}
>>> padrones.index(78455)
0
>>> padrones.index(47890)
4
\end{codigo-python-sn}

\begin{atencion}
Si el valor no se encuentra en la lista, se producirá un error:

\begin{codigo-python-sn}
>>> padrones.index(78)
(^Traceback (most recent call last):
  File "<stdin>", line 1, in <module>
ValueError: list.index(x): x not in list^)
\end{codigo-python-sn}
\end{atencion}

Si el valor está repetido, el índice que devuelve es el de la primera aparición:

\begin{codigo-python-sn}
>>> [10, 20, 10].index(10)
0
\end{codigo-python-sn}

\begin{observacion}
La función |index| también se puede utilizar con cadenas y tuplas.
\end{observacion}

\item Para iterar sobre todos los elementos de una lista usaremos una
construcción \lstinline+for+:

\begin{codigo-python-sn}
>>> for p in padrones:
...     print(p)
...
78455
79211
54988
66540
47890
\end{codigo-python-sn}

\begin{observacion}
El ciclo |for <variable> in <secuencia>:| se puede utilizar sobre cualquier
secuencia, incluyendo tuplas y cadenas.
\end{observacion}

\item Muchas veces, dentro del cuerpo del ciclo |for| es necesario contar con la
posición de cada elemento de la lista. Para esto es posible utilizar la función
|enumerate|: \label{enumerate}

\begin{codigo-python-sn}
>>> for i, p in enumerate(padrones):
...     print(i, p)
...
0 78455
1 79211
2 54988
3 66540
4 47890
\end{codigo-python-sn}
\end{itemize}

\begin{sabias_que}
En Python, las listas, las tuplas y las cadenas son parte del conjunto
de las \emph{secuencias}.  Todas las secuencias cuentan con las siguientes
operaciones:

\vspace{\medskipamount}
\begin{tabular}{l l}
{\bf Operación} & {\bf Resultado} \\
\hline
\lstinline!x in s! & Indica si el valor \lstinline!x! se encuentra en
    \lstinline!s! \\
\lstinline!s + t! & Concantena las secuencias \lstinline!s! y \lstinline!t! \\
\lstinline!s * n! & Concatena \lstinline!n! copias de \lstinline!s! \\
\lstinline!s[i]! & Elemento \lstinline!i! de \lstinline!s!, empezando por 0 \\
\lstinline!s[i:j]! & Porción de la secuencia \lstinline!s! desde
	\lstinline!i! hasta \lstinline!j! (no inclusive) \\
\lstinline!s[i:j:k]! & Porción de la secuencia \lstinline!s! desde
	\lstinline!i! hasta \lstinline!j! (no inclusive), con paso \lstinline!k!  \\
\lstinline!len(s)! & Cantidad de elementos de la secuencia \lstinline!s!  \\
\lstinline!min(s)! & Mínimo elemento de la secuencia \lstinline!s! \\
\lstinline!max(s)! & Máximo elemento de la secuencia \lstinline!s! \\
\lstinline!sum(s)! & Suma de los elementos de la secuencia \lstinline!s! \\
\lstinline!enumerate(s)! & Enumerar los elementos de \lstinline!s! junto con sus posiciones \\
\end{tabular}
\vspace{\medskipamount}

Además, es posible crear una lista o una tupla a partir de cualquier otra
secuencia, utilizando las funciones |list| y |tuple|, respectivamente:

\begin{codigo-python-sn}
>>> list("Hola")
['H', 'o', 'l', 'a']
>>> tuple("Hola")
('H', 'o', 'l', 'a')
>>> list((1, 2, 3, 4))
[1, 2, 3, 4]
\end{codigo-python-sn}
\end{sabias_que}

\problemac{Queremos escribir un programa que nos permita armar la lista de
los inscriptos de una materia.}

\begin{enumerate}

\item {\bf Análisis:} El usuario ingresa datos de padrones que se van
guardando en una lista.

\item {\bf Especificación:} El programa solicitará al usuario que ingrese
uno a uno los padrones de los inscriptos. Con esos números construirá una
lista, que al final se mostrará.

\item {\bf Diseño:}
\begin{itemize}
\item ¿Qué estructura tiene este programa? ¿Se parece a algo conocido?

Es claramente un ciclo en el cual se le pide al usuario que ingrese uno a
uno los padrones de los inscriptos, y estos números se agregan a una lista.
Y en algún momento, cuando se terminaron los inscriptos, el usuario deja de
cargar.

\item ¿El ciclo es definido o indefinido?

Para que fuera un ciclo definido deberíamos contar de antemano cuántos
inscriptos tenemos, y luego cargar exactamente esa cantidad, pero eso no
parece muy útil.

Estamos frente a una situación parecida al problema de la lectura de los
números, en el sentido de que no sabemos cuántos elementos queremos cargar
de antemano. Para ese problema, en la sección~\ref{centinela}, vimos una solución muy
sencilla y cómoda: se le piden datos al usuario y, cuando se cargaron todos
los datos se ingresa un valor arbitrario (que se usa sólo para indicar que
no hay más información). A ese diseño lo hemos llamado ciclo con centinela
y tiene el siguiente esquema:

\begin{codigo-nohl-sn}
Repetir indefinidamente:
    Pedir datos
    Si el dato pedido coincide con el centinela:
        Salir del ciclo
	Realizar cálculos
\end{codigo-nohl-sn}

Como sabemos que los números de padrón son siempre enteros positivos,
podemos considerar que el centinela puede ser cualquier número menor o
igual a cero.  También sabemos que en nuestro caso tenemos que ir armando
una lista que inicialmente no tiene ningún inscripto.

Modificamos el esquema anterior para ajustarnos a nuestra situación:

\begin{codigo-nohl-sn}
La lista de inscriptos es vacía
Repetir indefinidamente:
    Pedir padrón
    Si el padrón no es positivo:
        Salir del ciclo
	Agregar el padrón a la lista
Devolver la lista de inscriptos
\end{codigo-nohl-sn}

\end{itemize}

\item {\bf Implementación:}
De acuerdo a lo diseñado, el programa quedaría como
se muestra en el Código~\ref{ins0}.

\begin{observacion}
Para entender mejor la implementación propuesta, es una buena idea repasar los
conceptos de ciclos, en especial las instrucciones |break| y |continue|,
explicadas en la sección~\ref{ciclos:resumen}.
\end{observacion}

\begin{codigo}{inscripcion.py}{Permite ingresar padrones de alumnos
inscriptos}
\label{ins0}
\lstinputlisting{src/7_tuplas_y_listas/ins0.py}
\end{codigo}

\item {\bf Prueba:}
Para probarlo lo ejecutamos con algunos lotes de prueba (inscripción de
tres alumnos, inscripción de cero alumnos, inscripción de alumnos
repetidos):

\begin{codigo-nohl-sn}
(~\$~) python3 inscripcion.py
Inscripcion en el curso de Algoritmos y Programación I
Ingresa un padrón (<=0 para terminar): 30
Ingresa un padrón (<=0 para terminar): 40
Ingresa un padrón (<=0 para terminar): 50
Ingresa un padrón (<=0 para terminar): 0
La lista de inscriptos es: [30, 40, 50]

(~\$~) python3 inscripcion.py
Inscripcion en el curso de Algoritmos y Programación I
Ingresa un padrón (<=0 para terminar): 0
La lista de inscriptos es: []

(~\$~) python3 inscripcion.py
Inscripcion en el curso de Algoritmos y Programación I
Ingresa un padrón (<=0 para terminar): 30
Ingresa un padrón (<=0 para terminar): 40
Ingresa un padrón (<=0 para terminar): 40
Ingresa un padrón (<=0 para terminar): 30
Ingresa un padrón (<=0 para terminar): 50
Ingresa un padrón (<=0 para terminar): 0
La lista de inscriptos es: [30, 40, 40, 30, 50]
\end{codigo-nohl-sn}

Evidentemente el programa funciona de acuerdo a lo especificado, pero
hay algo que no tuvimos en cuenta: permite inscribir a una misma persona
más de una vez.

\item {\bf Mantenimiento:} No permitir que haya padrones repetidos.

\item {\bf Diseño revisado:} Para no permitir que haya padrones repetidos
debemos revisar que no exista el padrón antes de agregarlo en la lista:

\begin{codigo-nohl-sn}
La lista de inscriptos es vacía
Repetir indefinidamente:
    Pedir padrón
    Si el padrón no es positivo:
        Salir del ciclo
    (@Si el padrón está en la lista:@)
        (@Avisar que el padrón ya está en la lista@)
        (@Saltear a la siguiente iteración del ciclo@)
    Agregar el padrón a la lista
Devolver la lista de inscriptos
\end{codigo-nohl-sn}

\item {\bf Nueva implementación:}
De acuerdo a lo diseñado en el párrafo anterior, el programa ahora quedaría
como se muestra en el Código~\ref{ins1}.

\begin{codigo}{inscripcion.py}{Permite ingresar padrones, sin repetir}
\label{ins1}
\lstinputlisting{src/7_tuplas_y_listas/ins1.py}
\end{codigo}

\item {\bf Nueva prueba:}
Para probarlo lo ejecutamos con los mismos lotes de prueba anteriores
(inscripción de tres alumnos, inscripción de cero alumnos, inscripción de
alumnos repetidos):

\begin{codigo-nohl-sn}
(~\$~) python3 inscripcion.py
Inscripcion en el curso de Algoritmos y Programación I
Ingresa un padrón (<=0 para terminar): 30
Ingresa un padrón (<=0 para terminar): 40
Ingresa un padrón (<=0 para terminar): 50
Ingresa un padrón (<=0 para terminar): 0
La lista de inscriptos es: [30, 40, 50]

(~\$~) python3 inscripcion.py
Inscripcion en el curso de Algoritmos y Programación I
Ingresa un padrón (<=0 para terminar): 0
La lista de inscriptos es: []

(~\$~) python3 inscripcion.py
Inscripcion en el curso de Algoritmos y Programación I
Ingresa un padrón (<=0 para terminar): 30
Ingresa un padrón (<=0 para terminar): 40
(@Ingresa un padrón (<=0 para terminar): 40@)
(@El padrón ya está en la lista de inscriptos.@)
(@Ingresa un padrón (<=0 para terminar): 30@)
(@El padrón ya está en la lista de inscriptos.@)
Ingresa un padrón (<=0 para terminar): 50
Ingresa un padrón (<=0 para terminar): 0
La lista de inscriptos es: [30, 40, 50]
\end{codigo-nohl-sn}

Ahora el resultado es satisfactorio: no tenemos inscriptos repetidos.

\end{enumerate}

\ejercicioc{Permitir que los alumnos se puedan inscribir o borrar.}

\ejercicioc{Inscribir y borrar alumnos como antes, pero registrar también
el nombre y apellido de la persona inscripta, de modo de tener como lista
de inscriptos:
\lstinline![(20, "Ana", "García"), (10, "Juan", "Salas")]!.}

\section{Ordenar listas}
\label{ordenar}

Nos puede interesar que los elementos de una lista estén ordenados: una vez
que finalizó la inscripción en un curso, tener a los padrones de los
alumnos por orden de inscripción puede ser muy incómodo, siempre será
preferible tenerlos ordenados por número para realizar cualquier
comprobación.

Python provee dos operaciones para obtener una lista ordenada a partir de
una lista desordenada.

\begin{itemize}

\item Para dejar la lista original intacta pero obtener una nueva lista
ordenada a partir de ella, se usa la función \lstinline!sorted!.\label{sorted}

\begin{codigo-python-sn}
>>> bs = [5, 2, 4, 2]
>>> cs = sorted(bs)
>>> bs
[5, 2, 4, 2]
>>> cs
[2, 2, 4, 5]
\end{codigo-python-sn}

\item Para modificar directamente la lista original usaremos la operación
\lstinline+sort()+.

\begin{codigo-python-sn}
>>> ds = [5, 3, 4, 5]
>>> ds.sort()
>>> ds
[3, 4, 5, 5]
\end{codigo-python-sn}

\end{itemize}

\section{Listas y cadenas}

A partir de una cadena de caracteres, podemos obtener una lista con sus
componentes usando la función \lstinline!split!.

Si queremos obtener las palabras (separadas entre sí por espacios) que
componen la cadena \lstinline!padrones! escribiremos simplemente
\lstinline!padrones.split()!:

\begin{codigo-python-sn}
>>> c = "   Una    cadena   con    espacios    "
>>> c.split()
['Una', 'cadena', 'con', 'espacios']
\end{codigo-python-sn}

En este caso \lstinline!split! elimina todos los blancos de más, y devuelve
sólo las palabras que conforman la cadena.

Si en cambio el separador es otro carácter (por ejemplo la arroba,
\lstinline!"@"!), se lo debemos pasar como parámetro a la función
\lstinline!split!. En ese caso se considera una componente todo lo que se
encuentra entre dos arrobas consecutivas. En el caso particular de que el
texto contenga dos arrobas una a continuación de la otra, se devolverá una
componente vacía:

\begin{codigo-python-sn}
>>> d="@@Una@@@cadena@@@con@@arrobas@"
>>> d.split("@")
['', '', 'Una', '', '', 'cadena', '', '', 'con', '', 'arrobas', '']
\end{codigo-python-sn}

La ``casi''--inversa de \lstinline!split! es una función \lstinline!join!
que tiene la siguiente sintaxis:
\begin{codigo-python-sn}
<separador>.join(<lista de componentes a unir>)
\end{codigo-python-sn}
y que devuelve la cadena que resulta de unir todas las componentes
separadas entre sí por medio del \emph{separador}:

\begin{codigo-python-sn}
>>> xs = ['aaa', 'bbb', 'cccc']
>>> " ".join(xs)
'aaa bbb cccc'
>>> ", ".join(xs)
'aaa, bbb, cccc'
>>> "@@".join(xs)
'aaa@@bbb@@cccc'
\end{codigo-python-sn}

\subsection{Ejercicios con listas y cadenas}

\ejercicioc{Escribir una función que reciba como parámetro una cadena de
palabras separadas por espacios y devuelva, como resultado, cuántas
palabras de más de cinco letras tiene la cadena dada.}

\ejercicioc{{\bf Procesamiento de telegramas.} Un oficial de correos decide
optimizar el trabajo de su oficina cortando todas las palabras de más de
cinco letras a sólo cinco letras (e indicando que una palabra fue cortada
con el agregado de una arroba). Además elimina todos los espacios en blanco
de más.

Por ejemplo, al texto
\lstinline+"   Llego   mañana   alrededor  del   mediodía   "+
se transcribe como \lstinline+"Llego mañan@ alred@ del medio@"+.

Por otro lado cobra un valor para las palabras cortas y otro valor para las
palabras largas (que deben ser cortadas).

\begin{partes}
\item Escribir una función que reciba un texto, la longitud máxima de las
palabras, el costo de cada palabra corta, el costo de cada palabra larga, y
devuelva como resultado el texto del telegrama y el costo del mismo.

\item Los puntos se reemplazan por la palabra especial "STOP", y el punto
final (que puede faltar en el texto original) se indica como "STOPSTOP".

Al texto: \\
\lstinline+"   Llego   mañana   alrededor  del   mediodía. Voy a almorzar "+ \\
Se lo transcribe como: \\
\lstinline+"Llego mañan@ alred@ del medio@ STOP Voy a almor@ STOPSTOP"+.

Extender la función anterior para agregar el tratamiento de los puntos.
\end{partes}
}

\section{Resumen}

\begin{itemize}

\item Python nos provee con varias estructuras que nos permiten agrupar los
datos que tenemos.  En particular, las {\bf tuplas} son estructuras
{\bf inmutables} que permiten agrupar valores al momento de crearlas, y las {\bf
listas} son estructuras {\bf mutables} que permiten agrupar valores, con la
posibilidad de agregar, quitar o reemplazar sus elementos.

\item Las tuplas se utilizan para modelar situaciones en las cuales al
momento de crearlas ya se sabe cuál va a ser la información a almacenar.
Por ejemplo, para representar una fecha, una carta de la baraja, una ficha
de dominó.

\item Las listas se utilizan en las situaciones en las que los elementos a
agrupar pueden ir variando a lo largo del tiempo.  Por ejemplo, para
representar un las notas de un alumno en diversas materias, los inscriptos
para un evento o la clasificación de los equipos en una competencia.

\item Las cadenas, tuplas y listas son tres tipos diferentes de {\bf
secuencias}. Las secuencias ofrecen un conjunto de operaciones básicas, como
obtener la longitud y recorrer sus elementos, que se aplican de la misma manera
sin importar qué tipo de secuencia es.

\end{itemize}

\begin{referencia_python}

\begin{sintaxis}{\lstinline!(valor1, valor2, valor3)!}
Las tuplas se definen como una sucesión de valores encerrados entre paréntesis
y separados por comas. Una vez definidas, no se pueden modificar los valores
asignados.

Casos particulares:
\begin{codigo-python-sn}
tupla_vacia = ()
tupla_unitaria = (3459,)
\end{codigo-python-sn}
\end{sintaxis}

\begin{sintaxis}{\lstinline![valor1, valor2, valor3]!}
Las listas se definen como una sucesión de valores encerrados entre corchetes y
separados por comas. Se les puede agregar, quitar o cambiar los valores que
contienen.

\begin{codigo-python-sn}
lista = [1, 2, 3]
lista[0] = 5
\end{codigo-python-sn}

Caso particular:
\begin{codigo-python-sn}
lista_vacia = []
\end{codigo-python-sn}
\end{sintaxis}

\begin{sintaxis}{\lstinline!x, y, z = secuencia!}
Es posible \emph{desempaquetar} una secuencia, asignando a la izquierda
tantas variables como elementos tenga la secuencia. Cada variable
tomará el valor del elemento que se encuentra en la misma posición.
\end{sintaxis}

\begin{sintaxis}{\lstinline!len(secuencia)!}
Devuelve la cantidad de elementos que contiene la secuencia, 0 si está vacía.
\end{sintaxis}

\begin{sintaxis}{\lstinline!for elemento in secuencia:!}
Itera uno a uno por los elementos de la secuencia.
\end{sintaxis}

\begin{sintaxis}{\lstinline!elemento in secuencia!}
Indica si el elemento se encuentra o no en la secuencia
\end{sintaxis}

\begin{sintaxis}{\lstinline!secuencia[i]!}
Corresponde al valor de la secuencia en la posición \lstinline!i!, comenzando
desde 0.

Si se utilizan números negativos, se puede acceder a los
elementos desde el último (\lstinline!-1!) hasta el primero
(\lstinline!-len(secuencia)!).

En el caso de las tuplas o cadenas (inmutables) sólo puede usarse para obtener el
valor, mientra que en las listas (mutables) puede usarse también para
modificar su valor.
\end{sintaxis}

\begin{sintaxis}{\lstinline!secuencia[i:j:k]!}
Permite obtener un segmento de la secuencia, desde la posición \lstinline!i!
inclusive, hasta la posición \lstinline!j! exclusive, con paso
\lstinline!k!.

En el caso de que se omita \lstinline!i!, se asume \lstinline!0!.  En el
caso de que se omita \lstinline!j!, se asume \lstinline!len(secuencia)!.
En el caso de que se omita \lstinline!k!, se asume 1. Si
se omiten todos, se obtiene una copia completa de la secuencia.
\end{sintaxis}

\begin{sintaxis}{\lstinline!lista.append(valor)!}
Agrega un elemento al final de la lista.
\end{sintaxis}

\begin{sintaxis}{\lstinline!lista.insert(posicion, valor)!}
Agrega un elemento a la lista, en la posición \lstinline!posicion!.
\end{sintaxis}

\begin{sintaxis}{\lstinline!lista.remove(valor)!}
Quita de la lista la primera aparción de elemento, si se encuentra.  De no
encontrarse en la lista, se produce un error.
\end{sintaxis}

\begin{sintaxis}{\lstinline!lista.pop()!}
Quita el elemento del final de la lista, y lo devuelve. Si la lista está
vacía, se produce un error.
\end{sintaxis}

\begin{sintaxis}{\lstinline!lista.pop(posicion)!}
Quita el elemento que está en la posición indicada, y lo devuelve. Si la
lista tiene menos de |posicion + 1| elementos, se produce un error.
\end{sintaxis}

\begin{sintaxis}{\lstinline!lista.index(valor)!}
Devuelve la posición de la primera aparición de valor.  Si no se encuentra
en la lista, se produce un error.
\end{sintaxis}

\begin{sintaxis}{\lstinline!sorted(secuencia)!}
Devuelve una lista nueva, con los elementos de la secuencia ordenados.
\end{sintaxis}

\begin{sintaxis}{\lstinline!lista.sort()!}
Ordena la misma lista.
\end{sintaxis}

\begin{sintaxis}{\lstinline!cadena.split(separador)!}
Devuelve una lista con los elementos de cadena, utilizando
\lstinline!separador! como separador de elementos.

Si se omite el separador, toma todos los espacios en blanco como
separadores.
\end{sintaxis}

\begin{sintaxis}{\lstinline!separador.join(lista)!}
Genera una cadena a partir de los elementos de \lstinline!lista!,
utilizando \lstinline!separador! como unión entre cada elemento y el
siguiente.
\end{sintaxis}
\end{referencia_python}


\newpage
\section{Ejercicios}

\extractionlabel{guia}
\begin{ejercicio}
Escribir una función que reciba una tupla de elementos e indique si se
encuentran ordenados de menor a mayor o no.
\end{ejercicio}


\extractionlabel{guia}
\begin{ejercicio}
{\bf Dominó.}
\begin{partes}
\item Escribir una función que indique si dos fichas de dominó
\emph{encajan} o no. Las fichas son recibidas en dos tuplas, por ejemplo:
\verb!(3,4)! y \verb!(5,4)!
\item Escribir una función que indique si dos fichas de dominó
\emph{encajan} o no. Las fichas son recibidas en una cadena, por ejemplo:
\verb!3-4 2-5!. {\bf Nota:} utilizar la función \verb!split! de las cadenas.
\end{partes}
\end{ejercicio}


\extractionlabel{guia}
\begin{ejercicio}{\bf Campaña electoral}
\begin{partes}
\item Escribir una función que reciba una tupla con nombres, y para cada
nombre imprima el mensaje \emph{Estimado <nombre>, vote por mí.}
\item Escribir una función que reciba una tupla con nombres, una posición
de origen \verb!p! y una cantidad \verb!n!, e imprima el mensaje anterior
para los \verb!n! nombres que se encuentran a partir de la posición
\verb!p!.
\item Modificar las funciones anteriores para que tengan en cuenta el
género del destinatario, para ello, deberán recibir una tupla de tuplas,
conteniendo el nombre y el género.
\end{partes}
\end{ejercicio}


\extractionlabel{guia}
\begin{ejercicio}
{\bf Vectores}
\begin{partes}
\item Escribir una función que reciba dos vectores y devuelva su producto
escalar.
\item Escribir una función que reciba dos vectores y devuelva si son o no
ortogonales.
\item Escribir una función que reciba dos vectores y devuelva si son
paralelos o no.
\item Escribir una función que reciba un vector y devuelva su norma.
\end{partes}
\end{ejercicio}


\extractionlabel{guia}
\begin{ejercicio}
Dada una lista de números enteros, escribir una función que:
\begin{partes}
\item Devuelva una lista con todos los que sean primos.
\item Devuelva la sumatoria y el promedio de los valores.
\item Devuelva una lista con el factorial de cada uno de esos números.
\end{partes}
\end{ejercicio}


\extractionlabel{guia}
\begin{ejercicio}
Dada una lista de números enteros y un entero k, escribir una función que:
\begin{partes}
\item Devuelva tres listas, una con los menores, otra con los mayores y
otra con los iguales a k.
\item Devuelva una lista con aquellos que son múltiplos de k.
\end{partes}
\end{ejercicio}


\extractionlabel{guia}
\begin{ejercicio}
Escribir una función que reciba una lista de tuplas (Apellido,
Nombre, Inicial\_segundo\_nombre) y devuelva una lista de cadenas
donde cada una contenga primero el nombre, luego la inicial con un punto, y luego el
apellido.
\end{ejercicio}


\extractionlabel{guia}
\begin{ejercicio}{\bf Inversión de listas}
\begin{partes}
\item Realizar una función que, dada una lista, devuelva una nueva lista cuyo
contenido sea igual a la original pero invertida. Así, dada la lista
\lstinline!['Di', 'buen', 'día', 'a', 'papa']!, deberá devolver
\lstinline!['papa', 'a', 'día', 'buen', 'Di']!.

\item Realizar otra función que invierta la lista, pero en lugar de devolver
una nueva, modifique la lista dada para invertirla, {\bf sin}  usar listas
auxiliares.
\end{partes}
\end{ejercicio}


\extractionlabel{guia}
\begin{ejercicio}
Escribir una función \texttt{empaquetar} para una lista, donde
epaquetar significa indicar la repetición de valores consecutivos
mediante una tupla (valor, cantidad de repeticiones). Por ejemplo,
\lstinline!empaquetar([1, 1, 1, 3, 5, 1, 1, 3, 3])! debe devolver
\lstinline![(1, 3), (3, 1), (5, 1), (1, 2), (3, 2)]!.
\end{ejercicio}


\extractionlabel{guia}
\begin{ejercicio}
{\bf Matrices.}
\begin{partes}
\item Escribir una función que reciba dos matrices y devuelva la suma.
\item Escribir una función que reciba dos matrices y devuelva el producto.
\item $\dificil$ Escribir una función que opere sobre una matriz y mediante
\emph{eliminación gaussiana} devuelva una matriz triangular superior.
\item $\dificil$ Escribir una función que indique si un grupo de vectores, recibidos
mediante una lista, son linealmente independientes o no.
\end{partes}
\end{ejercicio}


\extractionlabel{guia}
\begin{ejercicio}
$\dificil$ {\bf Plegado de un texto.} Escribir una función que reciba un
párrafo de texto (palabras separadas por espacios) y una longitud $n$, y
devuelva una lista conteniendo líneas de texto de longitud máxima $n$. Las
líneas deben ser cortadas correctamente en los espacios (sin cortar las
palabras). Asumir que ninguna palabra tiene longitud mayor a $n$. Ejemplo:

\begin{lstlisting}[numbers=none]
>>> plegar('El viejo Señor Gómez pedía queso, kiwi y habas, pero le ha tocado un saxofón', 20)
['El viejo Señor Gómez', 'pedía queso, kiwi y', 'habas, pero le ha', 'tocado un saxofón']
\end{lstlisting}
\end{ejercicio}


\extractionlabel{guia}
\begin{ejercicio}
{\bf Funciones que reciben funciones.}
\begin{partes}
\item Escribir una funcion llamada {\bf map}, que reciba una función y una
lista y devuelva la lista que resulta de aplicar la función recibida a
cada uno de los elementos de la lista recibida.
\item Escribir una función llamada {\bf filter}, que reciba una función y
una lista y devuelva una lista con los elementos de la lista recibida para
los cuales la función recibida devuelve un valor verdadero.
\item ¿En qué ejercicios de esta guía podría haber utilizado estas
funciones?
\end{partes}
\end{ejercicio}

\extractionlabel{guia}
\begin{ejercicio}
$\dificil$ {\bf Juegos sencillos (¡o no tanto!).}
Estos ejercicios permiten aplicar todo lo aprendido hasta ahora.
Recomendación: pensar bien el {\bf diseño}; y en particular elegir una {\bf
estructura de datos} apropiada para representar el
\emph{estado del juego}. Algunas preguntas que pueden ayudar:
\begin{itemize}
\item ¿Cómo actualizamos el estado del juego según cada acción realizada?
\item ¿Cómo detectamos si un movimiento es válido o no?
\item ¿Cómo detectamos si el juego ha terminado?
\item ¿Cómo se muestra la información en la consola a los jugadores?
\item ¿Cómo hacen los jugadores para ingresar las acciones a realizar?
\end{itemize}

\noindent Escribir un programa que permita jugar\footnote{No es el objetivo
implementar una \emph{inteligencia artificial}: en los juegos de dos o más
jugadores, solo aceptaremos jugadores humanos.}:
\begin{partes}
\item Torres de Hanói (\url{https://es.wikipedia.org/wiki/Torres_de_Hanoi})
\item Ta-Te-Ti (\url{https://es.wikipedia.org/wiki/Tres_en_linea})
\item Nim (\url{https://es.wikipedia.org/wiki/Nim_(juego)})
\item Generala (\url{https://es.wikipedia.org/wiki/Generala})
\end{partes}
\end{ejercicio}

\newpage
\begin{subappendices}
\section{Funciones de orden superior}
\label{map-filter}

En Python, se dice que \emph{las funciones son ciudadanos de primera clase}, ya
que una función es un valor (cuyo tipo de dato es |function|) que puede ser
manipulado como cualquier otro valor (|int|, |str|, etc.):

\begin{codigo-python-sn}
>>> def exclamar(s):
...     return '¡' + s + '!'
>>> exclamar
<function exclamar at 0x10532ef28>
>>> type(exclamar)
<class 'function'>
>>> exclamar('Hola')
'¡Hola!'
>>> f = exclamar
>>> f('Hola')
'¡Hola!'
\end{codigo-python-sn}

En el ejemplo, |exclamar| (sin los paréntesis) es una referencia a un valor de
tipo |function| (la función llamada ``exclamar''). Al hacer |f = exclamar|,
el resultado es que ambas variables (|exclamar| y |f|) hacen referencia a
\emph{la misma} función. Por lo tanto, podemos llamar a la función con
|f(...)| o con |exclamar(...)|.

Aprovechando esto, una función puede recibir otra función como parámetro:

\begin{codigo-python-sn}
>>> def aplicar((@f@), x, n):
...     """Devuelve el resultado de f(f(...f(f(x)))), aplicando en total n
...        veces la función f"""
...     for i in range(n):
...         x = (@f(x)@)
...     return x
>>> aplicar((@exclamar@), 'Hola', 5)
'¡¡¡¡¡Hola!!!!!'
>>> def alargar(s):
...     return s + s[-1]
>>> alargar('Hola')
'Holaa'
>>> aplicar((@alargar@), 'Hola', 5)
'Holaaaaaa'
\end{codigo-python-sn}

Una función también puede devolver otra función:

\begin{codigo-python-sn}
>>> def puntuador(inicio, fin):
...    """Recibe dos cadenas de inicio y fin, y devuelve una función
...       que permite agregar signos de puntuación a una cadena."""
...    (@def puntuar(cadena):@)
...        return inicio + cadena + fin
...    (@return puntuar@)
>>> exclamar = puntuador('¡', '!')
>>> type(exclamar)
<class 'function'>
>>> exclamar('Hola')
'¡Hola!'
>>> preguntar = puntuador('¿', '?')
>>> preguntar('Qué')
'¿Qué?'
\end{codigo-python-sn}

\begin{observacion}
Se dice que una función es \emph{de orden superior} cuando recibe una función
por parámetro, o bien cuando devuelve una función.
\end{observacion}

En nuestros ejemplos, |aplicar| y
|puntuador| son de orden superior. Las funciones de orden superior son
características de un estilo de programación llamado \emph{programación
funcional}.

\subsection*{Funciones anónimas}

Cuando trabajamos con funciones de orden superior, suele ser conveniente
definir \emph{funciones anónimas}. En Python, cualquier función que evalúa
una única expresión y la devuelve puede ser definida en forma anónima usando
la sintaxis |lambda|:

\begin{itemize}
\item |lambda x: x * 2| es ``una función que dado un número $x$ devuelve $2x$''.
\item |lambda x, y: x + y| es ``una función que dados $x$ e $y$ devuelve la suma de ambos''.
\end{itemize}

La sintaxis general es |lambda <parámetros>: <expresión>|. El valor de retorno de una
función definida con |lambda| es el resultado de la |<expresión>|; el |return| es
implícito.

\begin{sabias_que}
La sintaxis |lambda| hace referencia al \emph{cálculo lambda} (un sistema
formal introducido por Alonzo Church y Stephen Kleene en la década de
1930), en el que las funciones se escriben anteponiendo la letra griega
$\lambda$. Por ejemplo: una función que devuelve el valor absoluto de un
número (invocando a la función |abs|), en notación lambda se escribe
$\lambda x. (abs x)$, mientras que en Python se escribe |lambda x: abs(x)|.
\end{sabias_que}

Las funciones |exclamar| y |alargar| podrían haber sido definidas usando la
sintaxis |lambda|:

\begin{codigo-python-sn}
>>> (@exclamar = lambda s: '¡' + s + '!'@)
>>> type(exclamar)
<class 'function'>
>>> exclamar('Hola')
'¡Hola!'
>>> (@alargar = lambda s: s + s[-1]@)
>>> alargar('Hola')
'Holaa'
\end{codigo-python-sn}

Pero las funciones anónimas son especialmente útiles cuando trabajamos con
funciones de orden superior. Por ejemplo, podemos pasar un |lambda| a una
función que recibe una función:

\begin{codigo-python-sn}
>>> aplicar((@lambda x: 2 * x@), 1, 4)
16(~\footnote{Ya que $2 \cdot (2 \cdot (2 \cdot (2 \cdot 1))) = 16$.}~)
\end{codigo-python-sn}

Una función que devuelve una función también puede devolver un |lambda|:

\begin{codigo-python-sn}
>>> def puntuador(inicio, fin):
...    """Recibe dos cadenas de inicio y fin, y devuelve una función
...       que permite agregar signos de puntuación a una cadena."""
...    (@return lambda cadena: inicio + cadena + fin@)
>>> citar = puntuador('“', '”')
>>> citar('Del dicho al hecho hay mucho trecho')
'“Del dicho al hecho hay mucho trecho(~”~)'
>>> def componer(f, g):
...     """Devuelve una función que dado x devuelve g(f(x))"""
...     (@return lambda x: g(f(x))@)
>>> gritar = componer(alargar, exclamar)
>>> gritar('Hola')
'¡Holaa!'
\end{codigo-python-sn}

\subsection*{Funciones de orden superior y secuencias}

Las funciones |sorted|, |map| y |filter| son ejemplos de funciones de orden superior
disponibles en Python, que además permiten operar sobre secuencias.

\begin{itemize}

\item
La función |sorted| ya fue mencionada en la Sección~\ref{sorted}. Si recibe una
lista de cadenas, devuelve una nueva lista ordenada:

\begin{codigo-python-sn}
>>> L = ['zorro', '', '935', 'alpaca', '1312', '-----', 'gato']
>>> sorted(L)
['', '-----', '1312', '935', 'alpaca', 'gato', 'zorro']
\end{codigo-python-sn}

El comportamiento por omisión es ordenar las cadenas en orden \emph{lexicográfico}
(comparando el primer caracter, luego el segundo, etc.). Este comportamiento es
útil en la gran mayoría de los casos, pero ¿qué pasa si necesitamos ordenar las
cadenas por otro criterio?

Podemos pasarle a |sorted| un parámetro adicional, que es una función que determina para
cada elemento el valor utilizado para ordenar. Por ejemplo, si queremos ordenar
las cadenas según su longitud (de menor a mayor), tenemos que pasarle a
|sorted| una función que recibe una cadena y devuelve su longitud. Esa función
no es otra que la función |len|:

\begin{codigo-python-sn}
>>> sorted(L, (@key=len@))(~\footnote{\texttt{key=len} indica que el parámetro adicional es un \emph{parámetro con nombre} (en inglés \emph{keyword argument}): el nombre del parámetro es \texttt{key} y el valor es \texttt{len}.}~)
['', '935', '1312', 'gato', 'zorro', '-----', 'alpaca']
\end{codigo-python-sn}

\item
La función |map| recibe una función y una secuencia, y devuelve la secuencia
resultante de aplicar la función a cada uno de los elementos de la secuencia
original.

\begin{codigo-python-sn}
>>> map(lambda x: x * 2, [1, 2, 3, 4])
<map object at 0x10edcba20>
\end{codigo-python-sn}

El valor de retorno de |map| es una secuencia, pero no es una tupla ni una
lista. Por suerte, podemos convertir cualquier secuencia en una lista con la
función |list|:

\begin{codigo-python-sn}
>>> list(map((@lambda x: x * 2@), [1, 2, 3, 4, 5]))
[2, 4, 6, 8, 10]
>>> list(map((@exclamar@), ['Usando', 'la', 'función', 'map']))
['¡Usando!', '¡la!', '¡función!', '¡map!']
>>> list(map((@int@), [8.3, '42', True]))
[8, 42, 1]
\end{codigo-python-sn}

\item
La función |filter| recibe una función |f| y una secuencia, y devuelve una secuencia
que contiene los elementos de la secuencia original para los cuales el
resultado de |f(x)| es |True|~.

\begin{codigo-python-sn}
>>> list(filter((@lambda x: x % 2 == 0@), [1, 2, 3, 4, 5]))
[2, 4]
>>> list(filter((@lambda s: len(s) > 2@), ['Usando', 'la', 'función', 'filter']))
['Usando', 'función', 'filter']
\end{codigo-python-sn}

\end{itemize}
\end{subappendices}

\chapter[Más sobre listas]{Más sobre listas y secuencias}

\section{Búsqueda lineal}

\subsection*{El problema de la búsqueda}

Presentamos ahora uno de los problemas más clásicos de la computación, \emph{el
problema de la búsqueda}, que se puede enunciar de la siguiente manera:

{\bf Problema: } Dada una lista $xs$ y un valor $x$ devolver el índice de $x$
en $xs$ si $x$ está en $xs$, y $-1$ si $x$ no está en $xs$.

Alicia Hacker afirma que este problema tiene una solución muy sencilla en
Python: se puede usar directamente la poderosa función \lstinline+index()+ de
lista.

Probamos esa solución para ver qué pasa:

\begin{codigo-python-sn}
>>> [1, 3, 5, 7].index(5)
2
>>> [1, 3, 5, 7].index(20)
(^Traceback (most recent call last):
  File "<stdin>", line 1, in <module>
ValueError: list.index(x): x not in list^)
\end{codigo-python-sn}

Vemos que usar la función \lstinline+index()+ resuelve nuestro problema si el
valor buscado está en la lista, pero si el valor no está no sólo no devuelve
un $-1$, sino que se produce un error.

El problema es que para poder aplicar la función \lstinline+index()+ debemos
estar seguros de que el valor está en la lista, y para averiguar eso Python
nos provee del operador \lstinline+in+:

\begin{codigo-python-sn}
>>> 5 in [1, 3, 5, 7]
True
>>> 20 in [1, 3, 5, 7]
False
\end{codigo-python-sn}

O sea que si llamamos a la función \lstinline+index()+ sólo cuando el
resultado de \lstinline+in+ es verdadero, y devolvemos $-1$ cuando el
resultado de \lstinline+in+ es falso, estaremos resolviendo el problema
planteado usando sólo funciones provistas por Python. La solución se plantea a
continuación:

\begin{codigo-python-sn}
def busqueda_con_index(xs, x):
    """Busca un elemento x en una lista xs.

    Si x está en xs devuelve el índice,
    de lo contrario devuelve -1.
    """
    if x in xs:
        return xs.index(x)
    else:
        return -1
\end{codigo-python-sn}

Probamos la función \verb+busqueda_con_index()+:

\begin{codigo-python-sn}
>>> busqueda_con_index([1, 4, 54, 3, 0, -1], 1)
0
>>> busqueda_con_index([1, 4, 54, 3, 0, -1], -1)
5
>>> busqueda_con_index([1, 4, 54, 3, 0, -1], 3)
3
>>> busqueda_con_index([1, 4, 54, 3, 0, -1], 44)
-1
>>> busqueda_con_index([], 0)
-1
\end{codigo-python-sn}

\subsection*{¿Cuántas comparaciones hace este programa?}

Es decir, ¿cuánto esfuerzo computacional requiere
este programa? ¿Cuántas veces compara el valor que buscamos con los datos de
la lista? No lo sabemos porque no sabemos cómo están implementadas las
operaciones \lstinline+in+ e \lstinline+index()+. La pregunta queda planteada
por ahora pero daremos un método para averiguarlo más adelante en esta unidad.


\subsection*{Búsqueda lineal}

Nos interesa ver qué sucede si programamos la búsqueda usando operaciones más
elementales, y no las grandes primitivas \lstinline+in+ e \lstinline+index()+.
Esto nos permitirá estudiar una solución que puede portarse a otros lenguajes
que no tienen instrucciones tan poderosas.

Supongamos entonces que en nuestra versión de Python no existen ni \lstinline+in+
ni \lstinline+index()+. Podemos en cambio acceder a cada uno de los elementos
de la lista a través de una construcción \lstinline+for+, y también, por
supuesto, podemos acceder a un elemento de la lista mediante un índice.

Diseñamos una solución: Podemos comparar uno a uno los elementos de la
lista con el valor de \lstinline!x!, y retornar el valor de la posición
donde lo encontramos en caso de encontrarlo.

Si llegamos al final de la lista sin haber salido antes de la función es
porque el valor de \lstinline!x! no está en la lista, y en ese caso
retornamos $-1$.

En esta solución necesitamos una variable \lstinline!i! que cuente en cada
momento en qué posición de la lista estamos parados. Esta variable se
inicializa en $0$ antes de entrar en el ciclo y se incrementa en $1$ en
cada paso.

El programa nos queda entonces como se muestra a continuación:

\begin{codigo-python-sn}
def busqueda_lineal(lista, x):
    """Si x está en lista devuelve su posición en lista, de lo
    contrario devuelve -1.
    """

    i = 0
    for z in lista: (~\circled{1}~)
        if z == x: (~\circled{2}~)
            return i (~\circled{3}~)
        i += 1
    return -1
\end{codigo-python-sn}

Y ahora lo probamos:

\begin{codigo-python-sn}
>>> busqueda_lineal([1, 4, 54, 3, 0, -1], 44)
-1
>>> busqueda_lineal([1, 4, 54, 3, 0, -1], 3)
3
>>> busqueda_lineal([1, 4, 54, 3, 0, -1], 0)
4
>>> busqueda_lineal([], 42)
-1
\end{codigo-python-sn}

\subsection*{¿Cuántas comparaciones hace este programa?}
\label{busqueda-lineal}

Volvemos a preguntarnos lo mismo que en la sección anterior, pero con el nuevo
programa: ¿cuánto esfuerzo computacional requiere este programa?, ¿cuántas
veces compara el valor que buscamos con los datos de la lista? Ahora podemos
analizar el código de \lstinline!busqueda_lineal!:

\begin{itemize}
\item La línea \circled{1} es un ciclo que recorre uno a uno los
elementos de la lista, y en el cuerpo de ese ciclo, en \circled{2} se
compara cada elemento con el valor buscado. En el caso de encontrarlo
(\circled{3}) se devuelve la posición.

\item Si el valor no está en la lista se recorrerá la lista entera, haciendo
una comparación por cada elemento.
\end{itemize}

O sea que si el valor está en la posición $p$ de la lista se hacen $p$
comparaciones, y si el valor no está se hacen tantas comparaciones como
elementos tenga la lista.

Nuestra hipótesis es: {\bf Si la lista crece, la cantidad de comparaciones
para encontrar un valor arbitrario crecerá en forma proporcional al tamaño de
la lista}. Por lo tanto diremos que:

\begin{observacion}
El algoritmo de búsqueda lineal tiene un comportamiento \emph{proporcional a la
longitud de la lista involucrada}, o que es un algoritmo \emph{lineal}.
\end{observacion}

En la próxima sección veremos cómo probar esta hipótesis.

\section{Búsqueda sobre listas ordenadas}

Si podemos suponer que la lista está previamente ordenada,
¿podemos encontrar una manera más eficiente de buscar elementos sobre ella?

En principio hay una modificación muy simple que podemos hacer sobre el
algoritmo de búsqueda lineal: si estamos buscando el elemento $x$ en una
lista que está ordenada de menor a mayor, en cuanto encontremos algún elemento
mayor a $x$ podemos estar seguros de que $x$ no está en la lista, por lo que no
es necesario continuar recorriendo el resto.

\ejercicioc{Modificar la búsqueda lineal para el caso de listas ordenadas.
En el peor caso, ¿cuál es nuestra nueva hipótesis sobre comportamiento del
algoritmo? ¿Es realmente más eficiente?}

\subsection*{Búsqueda binaria}

¿Podemos hacer algo mejor? Trataremos de aprovechar el hecho de que la lista
está ordenada y vamos a hacer algo distinto: nuestro espacio de búsqueda se
irá achicando a segmentos cada vez menores de la lista original.
La idea es descartar segmentos de la lista donde el valor seguro que no puede
estar:

\begin{enumerate}
\item Consideramos como segmento inicial de búsqueda a la lista completa.

\item Analizamos el punto medio del segmento (el valor central); si es el valor
buscado, devolvemos el índice del punto medio.

\item Si el valor central es mayor al buscado, podemos descartar el segmento
que está desde el punto medio hacia la a derecha.

\item Si el valor central es menor al buscado, podemos descartar el segmento
que está desde el punto medio hacia la izquierda.

\item Una vez descartado el segmento que no nos interesa, volvemos a analizar
el segmento restante, de la misma forma.

\item Si en algún momento el segmento a analizar tiene longitud 0 o negativa
significa que el valor buscado no se encuentra en la lista.
\end{enumerate}

Para señalar la porción del segmento que se está analizando a cada paso,
utilizaremos dos variables (\lstinline!izq! y \lstinline!der!) que
contienen la posición de inicio y la posición de fin del segmento que se
está considerando. De la misma manera usaremos la varible \lstinline!medio!
para contener la posición del punto medio del segmento.

En la Figura~\ref{fig:busqbin} vemos qué pasa cuando se busca
el valor 18 en la lista [1, 3, 5, 7, 9, 11, 13, 15, 17, 19, 21, 23].

\def\elements{1, 3, 5, 7, 9, 11, 13, 15, 17, 19, 21, 23}

\tikzset{pics/arrayrow/.style n args={3}{code={
    \node[anchor=east] at (-1, 0) {#1};
    \foreach \x [count=\i] in \elements {
        \ifthenelse{\x>#2 \AND \x<#3}
            {\node[draw,fill=med-gray,minimum width=0.75cm,node font=\ttfamily] at (0.75*\i,0) (\x) {\x};}
            {\node[draw,fill=none,minimum width=0.75cm,node font=\ttfamily] at (0.75*\i,0) (\x) {\x};}
    }
}}}
\tikzset{pics/arraymark/.style n args={3}{code={
    \node[below=#3 of #1.south,anchor=north,node font=\ttfamily] (label) {#2};
    \draw[flecha] (label) -- (#1.south);
}}}

\begin{figure}[h!t]
\begin{center}
\begin{tikzpicture}
\pic {arrayrow={El arreglo inicial:}{0}{24}};
\pic {arraymark={1}{izq\,=\,0}{0.25cm}};
\pic {arraymark={11}{medio\,=\,5}{0.25cm}};
\pic {arraymark={23}{der\,=\,11}{0.25cm}};

\pic at (0, -2) {arrayrow={Paso 2 ({\texttt lista[5] < 18}):}{12}{24}};
\pic {arraymark={13}{izq\,=\,6}{0.25cm}};
\pic {arraymark={17}{medio\,=\,8}{0.25cm}};
\pic {arraymark={23}{der\,=\,11}{0.25cm}};

\pic at (0, -4) {arrayrow={Paso 3 ({\texttt lista[8] < 18}):}{18}{24}};
\pic {arraymark={19}{izq\,=\,9}{0.25cm}};
\pic {arraymark={21}{medio\,=\,10}{0.75cm}};
\pic {arraymark={23}{der\,=\,11}{0.25cm}};

\pic at (0, -6) {arrayrow={Paso 4 ({\texttt lista[9] >= 18}):}{18}{20}};
\pic {arraymark={19}{izq\,=\,der\,=\,medio\,=\,9}{0.25cm}};
\end{tikzpicture}
\end{center}
\caption{Ejemplo de una búsqueda usando el algoritmo de búsqueda binaria.
Como no se encontró al valor buscado, devuelve $-1$.}
\label{fig:busqbin}
\end{figure}

En el Código~\ref{busquedabinaria} mostramos una posible implementación de
este algoritmo.

\begin{codigo}{busqueda\_binaria.py}{Función de búsqueda binaria}
\label{busquedabinaria}
\lstinputlisting{src/8_busqueda/busb.py}
\end{codigo}

A continuación varias ejecuciones de prueba:

\begin{codigo-python-sn}
>>> busqueda_binaria([1, 3, 5], 0)
[DEBUG] izq: 0 der: 2 medio: 1
[DEBUG] izq: 0 der: 0 medio: 0
-1
>>> busqueda_binaria([1, 3, 5], 1)
[DEBUG] izq: 0 der: 2 medio: 1
[DEBUG] izq: 0 der: 0 medio: 0
0
>>> busqueda_binaria([1, 3, 5], 2)
[DEBUG] izq: 0 der: 2 medio: 1
[DEBUG] izq: 0 der: 0 medio: 0
-1
>>> busqueda_binaria([1, 3, 5], 3)
[DEBUG] izq: 0 der: 2 medio: 1
1
>>> busqueda_binaria([1, 3, 5], 5)
[DEBUG] izq: 0 der: 2 medio: 1
[DEBUG] izq: 2 der: 2 medio: 2
2
>>> busqueda_binaria([1, 3, 5], 6)
[DEBUG] izq: 0 der: 2 medio: 1
[DEBUG] izq: 2 der: 2 medio: 2
-1
>>> busqueda_binaria([], 0)
-1
>>> busqueda_binaria([1], 1)
[DEBUG] izq: 0 der: 0 medio: 0
0
>>> busqueda_binaria([1], 3)
[DEBUG] izq: 0 der: 0 medio: 0
-1
\end{codigo-python-sn}

\ejercicioc{En la línea 13 de |busqueda_binaria.py| efectuamos la división usando el
operador |//| en lugar de |/|. ¿Qué sucedería si utilizáramos |/|?  }

\subsection*{¿Cuántas comparaciones hace este programa?}

Para responder esto pensemos en el peor caso, es decir, que se descartaron
varias veces partes del segmento para finalmente llegar a un segmento vacío y
el valor buscado no se encontraba en la lista.

En cada paso el segmento se divide por la mitad y se desecha una de esas
mitades, y en cada paso se hace una comparación con el valor buscado. Por lo
tanto, la cantidad de comparaciones que hacen con el valor buscado es
aproximadamente igual a la cantidad de pasos necesarios para llegar a un
segmento de tamaño 1.
Veamos el caso más sencillo para razonar, y supongamos que la longitud de la
lista es una potencia de 2, es decir \lstinline+len(lista)+~$= 2^k$:

\begin{itemize}
\item Luego del primer paso, el segmento a tratar es de tamaño $2^k$.
\item Luego del segundo paso, el segmento a tratar es de tamaño $2^{k-1}$.
\item Luego del tercer paso, el segmento a tratar es de tamaño $2^{k-2}$.

$\ldots$

\item Luego del paso $k$, el segmento a tratar es de tamaño $2^{k-k}=1$.
\end{itemize}

Por lo tanto este programa hace aproximadamente $k$ comparaciones con el valor
buscado cuando \lstinline+len(lista)+~$= 2^k$.
Pero si despejamos $k$ de la ecuación anterior, podemos ver que este programa
realiza aproximadamente $\log_2($\lstinline+len(lista)+$)$ comparaciones.

Cuando \lstinline+len(lista)+ no es una potencia de 2 el razonamiento es menos
prolijo, pero también vale que este programa realiza aproximadamente
$\log_2$(\lstinline+len(lista)+$)$ comparaciones. Concluimos entonces que:

\begin{observacion}
Si podemos suponer que la lista está previamente ordenada, podemos utilizar el
algoritmo de búsqueda binaria, que es {\emph muchísimo} más eficiente que la
búsqueda lineal.
\end{observacion}

Veamos un ejemplo para entender cuánto más eficiente es la búsqueda binaria.
Supongamos que tenemos una lista de un millón de elementos.

\begin{itemize}
\item El algoritmo de búsqueda lineal hará una cantidad de operaciones proporcional
a un millón; es decir que en el peor caso hará 1.000.000 de comparaciones, y en
un caso promedio, 500.000 comparaciones.
\item El algoritmo de búsqueda binaria hará como máximo $\log_2(1\,000\,000)$
comparaciones, o sea ¡no más que 20 comparaciones!.
\end{itemize}

\section{Resumen}

\begin{itemize}

\item La {\bf búsqueda} de un elemento en una secuencia es un
algoritmo básico pero importante. El problema que intenta resolver puede
plantearse de la siguiente manera: Dada una secuencia de valores y un
valor, devolver el índice del valor en la secuencia, si se encuentra, de no
encontrarse el valor en la secuencia señalizarlo apropiadamente.

\item Una de las formas de resolver el problema es mediante la {\bf
búsqueda lineal}, que consiste en ir revisando uno a uno los elementos de
la secuencia y comparándolos con el elemento a buscar.  Este algoritmo no
requiere que la secuencia se encuentre ordenada.

\item Cuando la secuencia sobre la que se quiere buscar está ordenada, se
puede utilizar el algoritmo de {\bf búsqueda binaria}.  Al estar ordenada
la secuencia, se puede desacartar en cada paso la mitad de los elementos,
quedando entonces con una eficiencia algorítmica relativa al
$log($\lstinline!len(secuencia)!$)$. Este algoritmo sólo tiene sentido
utilizarlo sobre una secuencia ordenada.

\item El análisis del comportamiento de un algoritmo puede ser muy engañoso
si se tiene en cuenta el mejor caso, por eso suele ser mucho más
ilustrativo tener en cuenta el {\bf peor caso}.  En algunos casos
particulares podrá ser útil tener en cuenta, además, el {\bf caso
promedio}.
\end{itemize}


\newpage
\section{Ejercicios}

\extractionlabel{guia}
\begin{ejercicio}
Escribir una función que reciba una lista desordenada y un elemento, que:
\begin{partes}
\item Busque todos los elementos coincidan con el pasado por parámetro y
devuelva la cantidad de coincidencias encontradas.
\item Busque la primera coincidencia del elemento en la lista y devuelva su
posición.
\item Utilizando la función anterior, busque todos los elementos que coincidan
con el pasado por parámetro y devuelva una lista con las posiciones.
\end{partes}
\end{ejercicio}


\extractionlabel{guia}
\begin{ejercicio}
Escribir una función que reciba una lista de números no ordenada, que:
\begin{partes}
\item Devuelva el valor máximo.
\item Devuelva una tupla que incluya el valor máximo y su posición.
\item ¿Qué sucede si los elementos son cadenas de caracteres?
\end{partes}
{\bf Nota:} no utilizar \verb!lista.sort()!
\end{ejercicio}


\extractionlabel{guia}
\begin{ejercicio}
{\bf Agenda simplificada} \\
Escribir una función que reciba una cadena a buscar y una lista de tuplas
(nombre\_completo, telefono), y busque dentro de la lista, todas las
entradas que contengan en el nombre completo la cadena recibida (puede
ser el nombre, el apellido o sólo una parte de cualquiera de ellos).
Debe devolver una lista con todas las tuplas encontradas.
\end{ejercicio}


\extractionlabel{guia}
\begin{ejercicio}
{\bf Sistema de facturación simplificado} \\
Se cuenta con una lista ordenada de productos, en la que uno consiste en
una tupla de (identificador, descripción, precio), y una lista de los
productos a facturar, en la que cada uno consiste en una tupla de
(identificador, cantidad). \\
Se desea generar una factura que incluya la cantidad, la descripción, el
precio unitario y el precio total de cada producto comprado, y al final
imprima el total general. \\
Escribir una función que reciba ambas listas e imprima por
pantalla la factura solicitada.
\end{ejercicio}


\extractionlabel{guia}
\begin{ejercicio}
Escribir una función que reciba una lista ordenada y un elemento. Si el
elemento se encuentra en la lista, debe encontrar su posición mediante
búsqueda binaria y devolverlo.  Si no se encuentra, debe agregarlo a la
lista en la posición correcta y devolver esa nueva posición. (No utilizar
\verb!lista.sort()!.)
\end{ejercicio}


\newpage
\begin{subappendices}
\section{Filtros, transformaciones y acumulaciones}

Supongamos que manejamos una librería, y disponemos de una base de datos con el
inventario. Cada entrada del inventario está compuesta por el título del libro,
el autor, la cantidad disponible y el precio. Por ejemplo:

\begin{center}
\small
\rowcolors[]{2}{}{light-gray}
\begin{tabular}{p{8cm} p{3cm} r r}
{\bf Título} & {\bf Autor} & {\bf Cantidad} & {\bf Precio} \\
\hline
The Art of Computer Programming, Volumes 1-4 & Donald Knuth & 12 & 179.62 \\
Concrete Mathematics: A Foundation for Computer Science & Donald Knuth & 5 & 54.77 \\
The Pragmatic Programmer: From Journeyman to Master & Andrew Hunt, David Thomas & 3 & 33.17 \\
Clean Code: A Handbook of Agile Software Craftsmanship & Robert C. Martin & 7 & 38.99 \\
Code Complete: A Practical Handbook of Software Construction & Steve McConnell & 0 & 29.97 \\
Learning Python & Mark Lutz & 4 & 40.95 \\
\ldots & \ldots & \ldots & \ldots \\ \hline
\end{tabular}
\end{center}

Podemos representar nuestro inventario en Python utilizando una lista de
tuplas:

\begin{codigo-python-sn}
inventario = [
    ('The Art of Computer Programming, Volumes 1-4',
     'Donald Knuth', 12, 179.62),
    ('Concrete Mathematics: A Foundation for Computer Science',
     'Donald Knuth', 5, 54.77),
    ('The Pragmatic Programmer: From Journeyman to Master',
     'Andrew Hunt and David Thomas', 3, 33.17),
    ...
]
\end{codigo-python-sn}

Una vez que disponemos de nuestro inventario en una estructura de datos,
podemos sacar todo tipo de
reportes: la cantidad total de libros, el valor total del
inventario, el precio promedio por libro, etc.

Veamos un ejemplo simple: supongamos que queremos obtener la cantidad total de
libros de un autor determinado. Podemos diseñar un algoritmo muy simple:

\begin{codigo-python-sn}
def total_libros_autor(inventario, autor_buscado):
    total = 0
    for titulo, autor, cantidad, precio in inventario:
        if autor == autor_buscado:
            total += cantidad
    return total
\end{codigo-python-sn}

Otro ejemplo: queremos obtener la cantidad de títulos de los cuales no hay
suficiente stock (menos de 5 unidades):

\begin{codigo-python-sn}
def cantidad_poco_stock(inventario):
    total = 0
    for titulo, autor, cantidad, precio in inventario:
        if cantidad < 5:
            total += 1
    return total
\end{codigo-python-sn}

¿Y si quisiéramos obtener la lista de títulos cuyo precio supera los \$100?

\begin{codigo-python-sn}
def titulos_caros(inventario):
    titulos = []
    for titulo, autor, cantidad, precio in inventario:
        if precio > 100:
            titulos.append(titulo)
    return titulos
\end{codigo-python-sn}

Acabamos de ``inventar'' tres algoritmos para resolver tres problemas
diferentes\ldots\ pero ¿son realmente diferentes? ¿No tienen nada en común?

Si observamos con detenimiento, los tres algoritmos comparten un mismo esquema:

\begin{codigo-nohl-sn}
def f(L):
    inicializar acumulador
    por cada elemento en el la lista L:
        si se cumple alguna condicion:
            hacer algún cálculo en base al elemento y acumular
    devolver acumulador
\end{codigo-nohl-sn}

Y este esquema en el fondo puede pensarse como una composición de tres
problemas más simples:

\begin{enumerate}
    \item A partir de una lista, \emph{filtrar} la lista según una condición
        determinada y obtener una lista con los elementos que pasan la
        condición.
    \item A partir de una lista, aplicar una \emph{transformación} a cada elemento
        y obtener una lista con los resultados.
    \item A partir de una lista, \emph{acumular} los elementos según un
        criterio determinado.
\end{enumerate}

% TODO: dibujo

Por ejemplo, podemos repensar nuestro primer algoritmo, que nos permitía calcular
la cantidad total de libros de un determinado autor, como:

\begin{enumerate}
    \item A partir del inventario, \emph{filtrar} según el autor:
        el resultado será una lista que contiene los libros del autor buscado.
    \item A partir de la lista obtenida, \emph{transformar} cada una de las tuplas
        |(titulo, autor, cantidad, precio)| para descartar todo menos la
        |cantidad|. Es decir, nos quedamos con una lista de números
        enteros, donde cada número representa una cantidad de libros.
    \item A partir de la lista obtenida, \emph{acumular} los elementos sumando uno
        a uno.
\end{enumerate}

Una ventaja de pensar el algoritmo de esta manera es que Python nos provee una
forma muy fácil de implementar filtros y transformaciones: las \emph{listas por
comprensión}.

\subsection{Listas por comprensión}

Concentrémonos en el filtro según el autor propuesto en el ejemplo anterior.
Una forma de implementarlo es:

\begin{codigo-python-sn}
def filtrar_autor(inventario, autor_buscado):
    filtrado = []
    for libro in inventario:
        if libro[2] == autor_buscado:
            filtrado.append(libro)
    return filtrado
\end{codigo-python-sn}

En Python podemos obtener el mismo resultado utilizando una \emph{lista por
comprensión}:

\begin{codigo-python-sn}
def filtrar_autor(inventario, autor_buscado):
    return (@[libro for libro in inventario if libro[2] == autor_buscado]@)
\end{codigo-python-sn}

Lo que vemos aquí es una sintaxis especial, que nos permite crear una lista
filtrando una secuencia según una condición:

\begin{codigo-python-sn}
(@[@)<variable> (@for@) <variable> (@in@) <secuencia> (@if@) <condicion>(@]@)
\end{codigo-python-sn}

Para aplicar la transformación propuesta (quedándonos únicamente con las
cantidades), podríamos implementarlo de esta manera:

\begin{codigo-python-sn}
def obtener_cantidades(inventario):
    cantidades = []
    for titulo, autor, cantidad, precio in inventario:
        cantidades.append(cantidad)
    return cantidades
\end{codigo-python-sn}

Pero en este caso también podemos obtener el mismo resultado con una lista por
comprensión:

\begin{codigo-python-sn}
def obtener_cantidades(inventario):
    return (@[cantidad for titulo, autor, cantidad, precio in inventario]@)
\end{codigo-python-sn}

En este caso la sintaxis utilizada es un poco diferente:

\begin{codigo-python-sn}
(@[@)<expresión> (@for@) <variable> (@in@) <secuencia>(@]@)
\end{codigo-python-sn}

Opcionalmente podemos combinar el filtro y la transformación en una única lista
por comprensión:

\begin{codigo-python-sn}
def cantidades_autor(inventario, autor_buscado):
    return [
        cantidad
        for titulo, autor, cantidad, precio in inventario
        if autor == autor_buscado
    ]
\end{codigo-python-sn}

\begin{observacion}
Es decir que la sintaxis general de las listas por comprensión es:
\begin{codigo-python-sn}
(@[@)<expresión> (@for@) <variable> (@in@) <secuencia> (@if@) <condicion>(@]@)
\end{codigo-python-sn}
\end{observacion}

Volviendo a nuestro problema inicial: obtener la cantidad total de libros del
autor; ya tenemos una forma de filtrar y transformar, y lo único que nos falta
es la acumulación. Pero recordemos que ya conocíamos una forma simple de
acumular sumando elementos: ¡la función |sum|!

\begin{codigo-python-sn}
def total_libros_autor(inventario, autor_buscado):
    return sum([
        cantidad
        for titulo, autor, cantidad, precio in inventario
        if autor == autor_buscado
    ])
\end{codigo-python-sn}

Planteemos ahora las soluciones para los otros dos problemas utilizando
filtros, transformaciones y acumulaciones. Nuestro segundo problema era obtener
la cantidad de títulos de los cuales no hay suficiente stock.

\begin{enumerate}
    \item Filtramos según la cantidad de stock, quedándonos con los libros
        para los cuales |cantidad < 5|.
    \item No es necesario aplicar una transformación.
    \item Solo necesitamos la cantidad de títulos, y eso es simplemente la
        cantidad de elementos de la lista producida en el paso anterior. Es
        decir que nuestra función de acumulación es |len|.
\end{enumerate}

\begin{codigo-python-sn}
def cantidad_poco_stock(inventario):
    return len([libro for libro in inventario if libro[3] < 5])
\end{codigo-python-sn}

Nuestro tercer problema era obtener la lista de títulos cuyo precio supera los
\$100.

\begin{enumerate}
    \item Filtramos según el precio, quedándonos con la lista de libros con
        |precio > 100|.
    \item Transformamos cada tupla quedándonos únicamente con el |titulo|.
    \item No es necesario aplicar una acumulación.
\end{enumerate}

\begin{codigo-python-sn}
def titulos_caros(inventario):
    return [
        titulo
        for titulo, autor, cantidad, precio in inventario
        if precio > 100
    ]
\end{codigo-python-sn}

Comparando estas soluciones con las primeras tres soluciones
propuestas, vemos que son dos estilos de programación diferentes:

\begin{itemize}
    \item Las primeras soluciones corresponden a un estilo más \emph{procedural} e
        \emph{imperativo}. Cuando pensamos en este estilo nos concentramos en
        dar órdenes para especificar \emph{cómo} queremos que la computadora
        resuelva el problema, paso por paso.
    \item Las soluciones planteadas utilizando filtros, transformaciones y
        acumulaciones corresponden a un estilo más \emph{funcional} y
        \emph{declarativo}, en el cual dividimos el problema en sub-problemas
        más simples, y nos concentramos en especificar cómo es el flujo de datos.
\end{itemize}

La discusión acerca de si uno de los dos estilos es ``mejor'' que el otro queda
fuera del alcance de este apunte, pero en general se considera que el uso de
listas por comprensión es \emph{idiomático} en Python. Es decir, los
programadores Python experimentados van a preferir leer y escribir código que
utilice listas por comprensión en lugar de implementar los filtros y
transformaciones a mano.
\end{subappendices}

\chapter{Diccionarios}

En esta unidad analizaremos otra estructura de datos importante: los diccionarios.
Su importancia radica no sólo en las grandes posibilidades que presentan
como estructuras para almacenar información, sino también en que, en
Python, son utilizados por el propio lenguaje para realizar diversas
operaciones y para almacenar información de otras estructuras.

\section{Qué es un diccionario}

Según Wikipedia, ``[u]n diccionario es una obra de consulta de
palabras y/o términos que se encuentran generalmente ordenados
alfabéticamente. De dicha compilación de palabras o términos se
proporciona su significado, etimología, ortografía y, en el caso
de ciertas lenguas fija su pronunciación y separación silábica.''


Al igual que los diccionarios a los que se refiere Wikipedia, y
que usamos habitualmente en la vida diaria, los diccionarios de
Python son una lista de consulta de términos de los cuales se
proporcionan valores asociados. A diferencia de los diccionarios a
los que se refiere Wikipedia, los diccionarios de Python no están
ordenados.


En Python, un diccionario es una colección no-ordenada de valores
que son accedidos a traves de una clave.  Es decir, en lugar de
acceder a la información mediante el índice numérico, como es el
caso de las listas y tuplas, es posible acceder a los valores a
través de sus claves, que pueden ser de diversos tipos.

Las claves son únicas dentro de un diccionario, es decir que no puede haber
un diccionario que tenga dos veces la misma clave. Si se asigna un valor a
una clave ya existente, se reemplaza el valor anterior.

No hay una forma directa de acceder a una clave a través de su valor, y
nada impide que un mismo valor se encuentre asignado a distintas claves.

Si bien las claves almacenadas en los diccionarios respetan el orden
de inserción, como generalmente no sabremos cuál fuel el mismo y como no
hay forma de insertar claves en un lugar particular en la práctica los
utilizaremos sin hacer asunciones al respecto del orden de dichas claves.
Cabe señalar que dos diccionarios con el mismo contenido e idénticos en
la comparación pueden tener las claves en diferente orden dado que la
historia de los mismos puede ser distinta.

Al igual que las listas, los diccionarios son mutables. Esto significa que
podemos agregar, quitar y modificar los elementos de un diccionario
posteriormente a su creación.

Cualquier valor de tipo inmutable puede ser clave de un diccionario:
cadenas, enteros, tuplas (con valores inmutables en sus miembros), etc.  No hay
restricciones para los valores que el diccionario puede contener, cualquier
tipo puede ser el valor: listas, cadenas, tuplas, otros diccionarios,
etc.

\begin{sabias_que}
En otros lenguajes de programación, a los diccionarios se los llama \emph{arreglos asociativos},
\emph{mapas} o \emph{tablas}.
\end{sabias_que}

\section{Utilizando diccionarios en Python}

De la misma forma que con listas, es posible definir un diccionario
directamente con los miembros que va a contener, o bien inicializar el
diccionario vacío y luego agregar los valores de a uno o de a muchos.

Para definirlo junto con los miembros que va a contener, se encierra el
listado de valores entre llaves, las parejas de clave y valor se separan
con comas, y la clave y el valor se separan con ':'.

\begin{codigo-python-sn}
punto = {'x': 2, 'y': 1, 'z': 4}
\end{codigo-python-sn}

\begin{observacion}
En Python el tipo de dato asociado a los diccionarios se llama |dict|:

\begin{codigo-python-sn}
>>> type(punto)
<class 'dict'>
\end{codigo-python-sn}
\end{observacion}

Para declararlo vacío y luego ingresar los valores, se lo declara como un
par de llaves sin nada en medio, y luego se asignan valores directamente a
los índices.

\begin{codigo-python-sn}
materias = {}
materias["lunes"] = [6103, 7540]
materias["martes"] = [6201]
materias["miércoles"] = [6103, 7540]
materias["jueves"] = []
materias["viernes"] = [6201]
\end{codigo-python-sn}

Para acceder al valor asociado a una determinada clave, se lo hace
de la misma forma que con las listas, pero utilizando la clave
elegida en lugar del índice.

\begin{codigo-python-sn}
>>> materias["lunes"]
[6103, 7540]
\end{codigo-python-sn}

\begin{atencion}
El acceso por clave falla si se provee una clave que no está en el diccionario:

\begin{codigo-python-sn}
>>> materias["domingo"]
(^Traceback (most recent call last):
  File "<stdin>", line 1, in <module>
KeyError: 'domingo'^)
\end{codigo-python-sn}
\end{atencion}

El operador |in| nos permite preguntar si una clave se encuentra o no en el
diccionario:

\begin{codigo-python-sn}
>>> "lunes" in materias
True
>>> "domingo" in materias
False
\end{codigo-python-sn}

Además podemos utilizar la función \lstinline{get}, que recibe una
clave $k$ y un valor por omisión $v$, y devuelve el valor asociado a la clave
$k$, en caso de existir, o el valor $v$ en caso contrario.

\begin{codigo-python-sn}
>>> materias.get("lunes", [])
[6103, 7540]
>>> materias.get("domingo", [])
[]
\end{codigo-python-sn}

Existen diversas formas de recorrer un diccionario.  Es posible recorrer
sus claves y usar esas claves para acceder a los valores.

\begin{codigo-python-sn}
for dia in materias:
    print("El {} tengo que cursar {}".format(dia, materias[dia])
\end{codigo-python-sn}

Es posible, también, obtener los valores como tuplas donde el primer
elemento es la clave y el segundo el valor.

\begin{codigo-python-sn}
for dia, codigos in materias.items():
    print("El {} tengo que cursar {}".format(dia, codigos)
\end{codigo-python-sn}

Hay muchas otras operaciones que se
pueden realizar sobre los diccionarios, que permiten manipular la información
según sean nuestras necesidades. Algunos de estos métodos pueden verse en
la referencia al final de la unidad.

No es posible obtener porciones de un diccionario usando \lstinline![:]!,
dado que el orden de las claves no es relevante para su uso.

\begin{sabias_que}
En la sección \ref{lookup-listas} mencionamos que Python garantiza que para
cualquier lista |L| con $N$ elementos se cumple que |L[i]| es una operación de
\emph{tiempo constante}, sin importar el valor de $N$ o de |i|.

Los diccionarios en Python tienen la misma propiedad: para cualquier
diccionario |D| con $N$ pares clave-valor, y para cualquier clave |k|, la
operación |D[k]| es de tiempo constante.

Dado que las claves pueden ser de cualquier tipo (a diferencia de las listas, en
las que los índices son números enteros entre 0 y $N-1$), para garantizar esta
propiedad, el algoritmo utilizado para almacenar los datos en el diccionario
debe ser más sofisticado que el utilizado para las listas.

Los diccionarios de Python están implementados usando una estructura de datos
llamada \emph{tabla de hash}. Para cada clave se calcula un valor numérico
mediante un algoritmo llamado \emph{código de hash}, que produce valores
muy dispares dependiendo de la clave.  Por ejemplo, el hash de la cadena
|"Python"| es -539294296 mientras que el de |"python"|, una cadena que
difiere en un caracter, es 1142331976. Los pares clave-valor del diccionario
se guardan internamente en una lista, y el código de hash de la clave se
utiliza para determinar el índice en la lista donde se ubicará cada par.
\end{sabias_que}

\section{Algunos usos de diccionarios}

Los diccionarios son una herramienta muy versátil.  Se puede utilizar un
diccionario, por ejemplo, para contar cuántas apariciones de cada palabra
hay en un texto, o cuántas apariciones de cada letra.

Es posible utilizar un diccionario, también, para tener una agenda donde la
clave es el nombre de la persona, y el valor es una lista con los datos
correspondientes a esa persona.

También podría utilizarse un diccionario para mantener los datos de los
alumnos inscriptos en una materia, siendo la clave el número de padrón, y
el valor una lista con todas las notas asociadas a ese alumno.

En general, los diccionarios sirven para crear bases de datos muy simples,
en las que la clave es el identificador del elemento, y el valor son todos
los datos del elemento a considerar.

Otro posible uso de un diccionario sería para realizar
traducciones, donde la clave sería la palabra en el idioma original y el
valor la palabra en el idioma al que se quiere traducir.  Sin embargo esta
aplicación es poco destacable, ya que esta forma de traducir suele dar
resultados poco satisfactorios.

\section{Resumen}

\begin{itemize}
\item Los diccionarios son una estructura de datos
muy poderosa, que permite almacenar un conjunto de pares $clave \rightarrow valor$.
\item Las claves deben ser inmutables y únicas.
\item Los valores pueden ser de cualquier tipo, y pueden no ser únicos.
\item El orden de las claves no es relevante. Si bien se los puede recorrer y las
claves vendrán en orden de inserción generalmente no es posible controlar ese orden
durante la vida útil de la estructura.
\end{itemize}

\begin{referencia_python}

\begin{sintaxis}{\lstinline!\{clave1:valor1, clave2:valor2\}!}
Se crea un nuevo diccionario con los valores asociados a las claves.  Si no
se ingresa ninguna pareja de clave y valor, se crea un diccionario vacío.
\end{sintaxis}

\begin{sintaxis}{\lstinline{diccionario[clave]}}
Accede al valor asociado con \lstinline!clave! en el diccionario. Falla si la
clave no está en el diccionario.
\end{sintaxis}

\begin{sintaxis}{\lstinline{clave in diccionario}}
Indica si un diccionario tiene o no una determinada clave.
\end{sintaxis}

\begin{sintaxis}{\lstinline{diccionario.get(clave, valor_predeterminado)}}
Devuelve el valor asociado a la clave.  A diferencia del acceso directo
utilizando \lstinline{[clave]}, en el caso en que el valor no se
encuentre devuelve el |valor_predeterminado|.
\end{sintaxis}

\begin{sintaxis}{\lstinline{for clave in diccionario:}}
Permite recorrer una a una todas las claves almacenadas en
el diccionario.
\end{sintaxis}

\begin{sintaxis}{\lstinline{diccionario.keys()}}
Devuelve una secuencia desordenada, con todas las claves que se hayan ingresado
al diccionario
\end{sintaxis}

\begin{sintaxis}{\lstinline{diccionario.values()}}
Devuelve una secuencia desordenada, con todos los valores que se hayan
ingresado al diccionario.
\end{sintaxis}

\begin{sintaxis}{\lstinline{diccionario.items()}}
Devuelve una secuencia desordenada con tuplas de dos elementos, en las que el
primer elemento es la clave y el segundo el valor.
\end{sintaxis}

\begin{sintaxis}{\lstinline{diccionario.pop(clave)}}
Quita del diccionario la clave y su valor asociado, y devuelve el valor.
\end{sintaxis}
\end{referencia_python}


\newpage
\section{Ejercicios}

\extractionlabel{guia}
\begin{ejercicio}
Escribir una función que reciba una lista de tuplas, y que devuelva
un diccionario en donde las claves sean los primeros elementos de las
tuplas, y los valores una lista con los segundos.

Por ejemplo:
\begin{lstlisting}[numbers=none]
>>> l = [ ('Hola', 'don Pepito'), ('Hola', 'don Jose'),
          ('Buenos', 'días') ]
>>> print(tuplas_a_diccionario(l))
{ 'Hola': ['don Pepito', 'don Jose'], 'Buenos': ['días'] }
\end{lstlisting}
\end{ejercicio}

\extractionlabel{guia}
\begin{ejercicio}
{\bf Diccionarios usados para contar.}
\begin{partes}
  \item Escribir una función que reciba una cadena y devuelva un diccionario con
la cantidad de apariciones de cada palabra en la cadena.  Por ejemplo, si
recibe "Qué lindo día que hace hoy" debe devolver:
\lstinline!{ 'que': 2, 'lindo': 1, 'día': 1, 'hace': 1, 'hoy': 1}!.

  \item Escribir una función que cuente la cantidad de apariciones de cada
caracter en una cadena de texto, y los devuelva en un diccionario.

  \item Escribir una función que reciba una cantidad de iteraciones de una tirada
de 2 dados a realizar y devuelva la cantidad de veces que se observa cada valor
de la suma de los dos dados. \\
{\bf Nota}: utilizar el módulo \verb!random! para obtener tiradas aleatorias.
\end{partes}
\end{ejercicio}

\extractionlabel{guia}
\begin{ejercicio}
{\bf Continuación de la agenda.} \\
Escribir un programa que vaya solicitando al usuario que ingrese nombres.
\begin{partes}
  \item Si el nombre se encuentra en la agenda (\emph{implementada con un
diccionario}), debe mostrar el teléfono y, opcionalmente, permitir
modificarlo si no es correcto.
  \item Si el nombre no se encuentra, debe permitir ingresar el teléfono
correspondiente.
\end{partes}
El usuario puede utilizar la cadena "*", para salir del programa.
\end{ejercicio}

\extractionlabel{guia}
\begin{ejercicio}
Escribir una función que reciba un texto y para cada caracter presente en el
texto devuelva la cadena más larga en la que se encuentra ese caracter.
\end{ejercicio}

\newpage
\begin{subappendices}
\section{Conjuntos}

Supongamos que queremos modelar un registro de donantes de órganos.
Este registro se comporta como un \emph{conjunto} de elementos, donde cada
elemento es una persona, y el conjunto:

\begin{enumerate}
    \item no puede contener elementos repetidos: una persona puede ser donante
        o no, pero no puede figurar dos o más veces en el registro.
    \item debe permitir averiguar si un elemento pertenece o no al conjunto en
        \emph{tiempo constante}: no importa si hay 1, 10 o 100000 donantes,
        queremos tener la capacidad de averiguar si una persona determinada
        es donante o no rápidamente.
\end{enumerate}

Si implementáramos el registro usando una lista de Python, nos encontraríamos
con que no cumplimos con los requisitos:

\begin{enumerate}
    \item La lista puede contener elementos repetidos. Si bien podemos salvar este
        detalle preguntando si el elemento se encuentra o no en la lista antes
        de agregarlo, esto sería muy poco eficiente, ya que:
    \item Como vimos en la Sección~\ref{busqueda-lineal}, buscar un elemento en la
        lista consume una cantidad de tiempo proporcional a la cantidad de
        elementos presentes en la lista, con lo que no es posible cumplir con
        el requerimiento de tiempo constante.
\end{enumerate}

Una solución posible es usar un diccionario, guardando como clave el número de
documento de la persona donante, y como valor\ldots\ ¡cualquier cosa! No
importa qué vayamos asignar como valor, ya que lo único que queremos aprovechar
del diccionario es la capacidad de tener claves únicas y poder preguntar en
tiempo constante si una clave está o no presente. Por ejemplo, si usamos |True|
como valor:

\begin{codigo-python-sn}
>>> donantes = {12345: True, 23456: True}
>>> 23456 in donantes
True
>>> 34567 in donantes
False
>>> donantes.pop(23456)
>>> 23456 in donantes
False
\end{codigo-python-sn}

Sin embargo, utilizar un diccionario para modelar un conjunto de elementos no
es la solución más elegante. Hay una mejor forma de hacerlo, que es
utilizando el tipo de datos |set|\footnote{La palabra ``set'' significa
``conjunto'' en inglés.}.

Para crear un |set| usamos la sintaxis |{<expresión>, <expresión>, ...}|:

\begin{codigo-python-sn}
>>> donantes = {12345, 23456}
>>> type(donantes)
<class 'set'>
>>> 23456 in donantes
True
>>> 34567 in donantes
False
>>> donantes.remove(23456)
>>> 23456 in donantes
False
\end{codigo-python-sn}

Un |set| es una estructura de datos mutable (como las listas y los
diccionarios), que permite agregar y quitar elementos cumpliendo los requisitos de
unicidad y búsqueda en tiempo constante. Además es posible hacer operaciones
entre |set|s como unión, intersección y diferencia muy fácilmente:

\begin{codigo-python-sn}
>>> s1 = {1, 2, 3, 4}
>>> s1
{1, 2, 3, 4}
>>> s1.add(1)
>>> s1
{1, 2, 3, 4}
>>> s2 = {3, 4, 5, 6}
>>> s1.union(s2)
{1, 2, 3, 4, 5, 6}
>>> s1.intersection(s2)
{3, 4}
>>> s1.difference(s2)
{1, 2}
\end{codigo-python-sn}

Notar que la sintaxis para crear un conjunto es muy similar a la de creación de
diccionarios. El caso especial es cuando queremos crear un conjunto vacío: la
sintaxis |{}| no funcionará, ya que eso crea un diccionario vacío. Podemos
crear un conjunto vacío con: |set()|.

\begin{codigo-python-sn}
>>> type({})
<class 'dict'>
>>> type(set())
<class 'set'>
\end{codigo-python-sn}

La referencia completa del tipo de dato |set| puede verse en
\url{https://docs.python.org/3/library/stdtypes.html#set}.
\end{subappendices}

\chapter{Documentación, contratos y mutabilidad}

En esta unidad se le dará cierta formalización a algunos temas que habían sido
presentados informalmente, como por ejemplo la documentación de las funciones.

Se formalizarán las condiciones que debe cumplir un algoritmo al comenzar, en
su transcurso, y al terminar, y algunas técnicas para tener en cuenta estas
condiciones.

También se verá una forma de modelar el espacio donde \textit{viven} las
variables.

\section{Documentación}

Comenzamos formalizando un poco más acerca de la documentación, cuál es su
objetivo y las distintas formas de documentar.

\subsection{Comentarios vs documentación}

En Python tenemos dos convenciones diferentes para documentar nuestro código:
la \emph{documentación} propiamente dicha (lo que ponemos entre \verb|"| o
\verb|"""| al principio de cada función o módulo), y los \emph{comentarios}
(\verb|#|).  En la mayoría de los lenguajes de programación hay convenciones
similares. ¿Por qué tenemos dos formas diferentes de documentar?

La \emph{documentación} tiene como objetivo explicar \emph{qué} hace el código.
La documentación está dirigida a cualquier persona que desee utilizar la
función o módulo, para que pueda entender cómo usarla sin necesidad de leer el
código fuente.  Esto es útil incluso cuando quien implementó la función es la
misma persona que la va a utilizar, ya que permite separar responsabilidades.

Los \emph{comentarios} tienen como objetivo explicar \emph{cómo} funciona el
código, y \emph{por qué} se decidió implementarlo de esa manera. Los comentarios
están dirigidos a quien esté leyendo el código fuente.

Podemos ver la diferencia entre la documentación y los comentarios en la
función |elegir_codigo| de nuestra implementacion del juego Mastermind
(Código~\ref{cod:mastermind}):

\begin{codigo-python-sn}
def elegir_codigo():
    """Devuelve un codigo de 4 digitos elegido al azar"""
    digitos = ('0','1','2','3','4','5','6','7','8','9')
    codigo = ''
    for i in range(4):
        candidato = random.choice(digitos)
        # Debemos asegurarnos de no repetir digitos
        while candidato in codigo:
            candidato = random.choice(digitos)
        codigo = codigo + candidato
    return codigo
\end{codigo-python-sn}

\subsection{¿Por qué documentamos?}

Seamos sinceros: nadie quiere escribir documentación. ¿Para qué repetir con
palabras lo que ya está estipulado en el código? La documentación es algo que
muy a menudo se deja \emph{para después}, y cuando llega el tan angustioso
momento de escribirla, lo que se termina haciendo es escribir lo más
escueto posible que pueda pasar como ``documentación''.

Incluso es muy frecuente que durante el desarrollo de un proyecto el código
evolucione con el tiempo, pero que nos olvidemos de actualizar la documentación
para reflejar los cambios. En este caso no solamente tenemos documentación de
mala calidad, ¡sino que además es incorrecta!

Pese a todo esto, la realidad sigue siendo que una buena documentación es
componente esencial de cualquier proyecto exitoso. Esto en parte se debe a que
el código fuente transmite en detalle las operaciones individuales que componen
un algoritmo o programa, pero no suele transmitir en forma transparente cosas
como la \emph{intención} del programa, el \emph{diseño} de alto nivel, las
\emph{razones} por las que se decidió utilizar un algoritmo u otro, etc.

\subsection{Código autodocumentado}

En teoría, si nestro código pudiera transmitir en forma eficiente todos esos
conceptos, la documentación no sería necesaria. De hecho, existe una técnica de
programación llamada \emph{código autodocumentado}, en la que la idea principal
es elegir los nombres de funciones y variables de forma tal que la
documentación sea innecesaria.

Tomemos como ejemplo el siguiente código:

\begin{codigo-python-sn}
a = 9.81
b = 5
c = 0.5 * a * b**2
\end{codigo-python-sn}

Leyendo esas tres líneas de código podemos razonar cuál será el valor final de
las variables |a|, |b| y |c|, pero no hay nada que nos indique qué representan
esas variables, o cuál es la intención del código. Una opción para mejorarlo sería
utilizar comentarios para aclarar la intención:

\begin{codigo-python-sn}
a = 9.81   # Valor de la constante G (aceleración gravitacional), en m/s²
b = 5      # Tiempo en segundos
c = 0.5 * a * b**2  # Desplazamiento (en metros)
\end{codigo-python-sn}

Otra opción, según la técnica de código autodocumentado, es simplemente asignar
nombres descriptivos a las variables:

\begin{codigo-python-sn}
aceleracionGravitacional = 9.81
tiempoEnSegundos = 5
desplazamientoMetros = 0.5 * aceleracionGravitacional * tiempoEnSegundos ** 2
\end{codigo-python-sn}

De esta manera logramos que no sea necesario ningún comentario ni documentación
adicional, ya que la intención del código es mucho más descriptiva.

La técnica de código autodocumentado presenta varias limitaciones. Entre ellas:

\begin{itemize}
    \item Elegir buenos nombres es una tarea difícil, que
        requiere que tener en cuenta cosas como: qué tan descriptivo es el nombre
        (cuanto más, mejor), la longitud del identificador
        (cuanto más corto mejor), el alcance del identificador (cuánto más
        grande, más descriptivo debe ser el nombre), y convenciones (|i| para
        índices, |c| para caracteres, etc).
    \item La documentación de todas formas termina siendo necesaria, ya que por
        muy bien que elijamos los nombres, muchas veces la única forma de
        explicar la intención del código y todos sus detalles es en lenguaje
        coloquial.
    \item Sigue siendo deseable que quien quiera utilizar nuestra función o
        módulo pueda entender su funcionamiento sin necesidad de leer el código
        fuente.
\end{itemize}

\subsection{Un error común: la sobredocumentación}

Si bien la ausencia de documentación suele ser perjudicial, el otro extremo
también lo es: la \emph{sobredocumentación}. Después de todo, en la vida
diaria no necesitamos carteles que nos recuerden cosas como ``esta es la
puerta'', ``este es el picaporte'' y ``empujar hacia abajo para abrir''. De
la misma manera, podríamos decir que el siguiente código peca de ser sobredocumentado:

\begin{codigo-python-sn}
def buscarElemento(listaDeNumeros, numero):
    """Esta función devuelve el índice (contando desde 0) en el que se
       encuentra el número `numero` en la lista `listaDeNumeros`. Si el elemento
       no está en la lista devuelve -1.
    """
    # Recorremos todos los índices de la lista, desde 0 (inclusive) hasta N
    # (no inclusive)
    for indice in range(len(listaDeNumeros)):
        # Si el elemento en la posicion `indice` es el buscado
        if listaDeNumeros[indice] == numero:
            # Devolvemos sl índice en el que está el elememto
            return indice
    # No lo encontramos, devolvemos -1
    return -1
\end{codigo-python-sn}

Algunas cosas que podemos mejorar:

\begin{itemize}
\item En la firma de la función los nombres |buscarElemento|,
    |listaDeNumeros| y |numero| se pueden simplificar a |buscar|, |secuencia| y
    |elemento|. Cambiamos |listaDeNumeros| por |secuencia|, ya que la función
    puede recibir secuencias de cualquier tipo, con elementos de cualquier
    tipo, y no hay ninguna razón para limitar a que sea una lista de números.
\item Las variable interna |indice| también se pueden simplificar:
    por convención podemos usar |i|.
\item ``Esta función'' es redundante: cuando alguien lea la documentación ya va
    a saber que se trata de una función.
\item ``contando desde 0'' es redundante: en Python siempre contamos desde 0.
\item Los comentarios son excesivos: la función es suficientemente simple y
    cualquier persona que sepa programación básica podrá entender el algoritmo.
\end{itemize}

Corrigiendo todos estos detalles resulta:

\begin{codigo-python-sn}
def buscar(lista, elemento):
    """Devuelve el índice en el que se encuentra el `elemento` en la `lista`,
       o -1 si no está.
    """
    for i in range(len(lista)):
        if lista[i] == elemento:
            return i
    return -1
\end{codigo-python-sn}

\section{Contratos}

Cuando hablamos de \textit{contratos} o \textit{programación por
contratos}, nos referimos a la necesidad de estipular tanto lo que necesita
como lo que devuelve nuestro código. El contrato de una función suele ser
incluido en su documentación.

Algunos ejemplos de cosas que deben ser estipuladas como parte del contrato
son: cómo deben ser los parámetros recibidos, cómo va a ser lo que se devuelve,
y si la función provoca algún efecto secundario (como por ejemplo modificar
alguno de los parámetros recibidos o imprimir algo en la consola).

Algunas de estas condiciones deben estar dadas antes de ejecutar el código o
función; a estas condiciones las llamamos \emph{precondiciones}. Si se cumplen
las precondiciones, habrá un conjunto de condiciones sobre el estado en que
quedan las variables y el o los valores de retorno una vez finalizada la
ejecución, que llamamos \emph{postcondiciones}.

\subsection{Precondiciones}

Las precondiciones de una función son las condiciones que deben cumplirse antes
de ejecutarla, para que se comporte correctamente: cómo deben ser los
parámetros que recibe, cómo debe ser el estado global, etc.

Por ejemplo, en una función que divide dos números, las precondiciones son que los parámetros
son números, y que el divisor es distinto de 0.

Si estipulamos las precondiciones como parte de la documentación, en el cuerpo
de la función podremos asumir que son ciertas, y no será necesario escribir
código para lidiar con los casos en los que no se cumplen.

\subsection{Postcondiciones}

Las postcodiciones son las condiciones que se cumplirán una vez finalizada la
ejecución de la función (asumiendo que se cumplen las precondiciones): cómo
será el valor de retorno, si los parámetros recibidos o variables globales son
alteradas, si se imprimen cosas, si se modifican archivos, etc.

En el ejemplo anterior, la función división, dadas las precondiciones
puede asegurar que devolverá un número correspondiente al cociente solicitado.

\subsection{Aseveraciones}

Tanto las precondiciones como las postcondiciones son \textit{aseveraciones}
(en inglés \textit{assertions}). Es decir, afirmaciones realizadas en un momento
particular de la ejecución sobre el estado computacional. Si llegaran a ser
falsas significaría que hay algún error en el diseño o utilización del algoritmo.

En algunos casos puede ser útil comprobar estas afirmaciones desde el código, y
para ello podemos utilizar la instrucción \lstinline!assert!. Esta instrucción
recibe una condición a verificar (o sea, una expresión booleana).
Si la condición es |True|, la instrucción no hace nada; en caso contrario se
produce un error.

\begin{codigo-python-sn}
>>> assert True
>>> assert False
(^Traceback (most recent call last):
  File "<stdin>", line 1, in <module>
AssertionError^)
\end{codigo-python-sn}

Opcionalmente, la instrucción |assert| puede recibir
un mensaje de error que mostrará en caso que la condición no se cumpla.

\begin{codigo-python-sn}
>>> n = 0
>>> assert n != 0, "El divisor no puede ser 0"
(^Traceback (most recent call last):
  File "<stdin>", line 1, in <module>
AssertionError: El divisor no puede ser 0^)
\end{codigo-python-sn}

\begin{atencion}
Es importante tener en cuenta que \lstinline!assert! está pensado para ser
usado en la etapa de desarrollo. Un programa terminado nunca debería dejar
de funcionar por este tipo de errores.
\end{atencion}

\subsection{Ejemplos}

Usando los ejemplos anteriores, la función \lstinline!division! nos
quedaría de la siguiente forma:

\begin{codigo-python-sn}
def division(dividendo, divisor):
    """Cálculo de la división

    Pre: Recibe dos números, divisor debe ser distinto de 0.
    Post: Devuelve un número real, con el cociente de ambos.
    """
    assert divisor != 0, "El divisor no puede ser 0"
    return dividendo / divisor
\end{codigo-python-sn}

Otro ejemplo, tal vez más interesante, puede ser una función que implemente
una sumatoria ($\sum_{i=inicial}^{final} f(i)$).  En este caso hay que
analizar cuáles van a ser los parámetros que recibirá la función, y las
precondiciones que estos parámetros deberán cumplir.

La función |sumatoria| a escribir necesita de un valor inicial, un valor
final, y una función a la cual llamar en cada paso. Es decir que recibe
tres parámetros.

\begin{codigo-python-sn}
def sumatoria(inicial, final, f):
\end{codigo-python-sn}

Tanto \lstinline!inicial! como \lstinline!final! deben ser números enteros,
y dependiendo de la implementación a realizar o de la especificación
previa, puede ser necesario que \lstinline!final! deba ser mayor o igual a
\lstinline!inicial!.

Con respecto a \lstinline!f!, se trata de una función que será llamada con
un parámetro en cada paso y se requiere poder sumar el resultado, por lo
que debe ser una función que reciba un número y devuelva un número.

La declaración de la función queda, entonces, de la siguiente manera.

\begin{codigo-python-sn}
def sumatoria(inicial, final, f):
    """Calcula la sumatoria desde i=inicial hasta final de f(i)

    Pre: inicial y final son números enteros, f es una función que
         recibe un entero y devuelve un número.
    Post: Se devuelve el valor de la sumatoria de aplicar f a cada
          número comprendido entre inicial y final.
    """
\end{codigo-python-sn}

\begin{ejercicio}
Realizar la implementación correspondiente a la función \lstinline!sumatoria!.
\end{ejercicio}

En definitiva, la estipulación de pre y postcondiciones dentro de la
documentación de las funciones es una forma de especificar claramente el
comportamiento del código.  Las pre y postcondiciones son, en efecto, un
\textit{contrato} entre el código invocante y el invocado.

\section{Invariantes de ciclo}

% TODO: conseguir frase, la vida es siempre igual, siempre está cambiando.
%\begin{quote}
%``Dadme un punto de apoyo y moveré el mundo'' Arquímedes
%\end{quote}

\label{invariantes}
Los invariantes se refieren a estados o situaciones que no cambian dentro
de un contexto o porción de código.  Hay invariantes de ciclo, que son los
que veremos a continuación, e invariantes de estado, que se verán más
adelante.

El invariante de ciclo permite conocer cómo llegar desde las precondiciones
hasta las postcondiciones, cuando la implementación se compone de un ciclo.
El invariante de ciclo es, entonces, una
aseveración que debe ser verdadera al comienzo de cada iteración.

Por ejemplo, si el problema es ir desde el punto $A$ al punto $B$, las
precondiciones dicen que estamos parados en $A$ y las postcondiciones que
estamos parados en $B$, un invariante podría ser ``estamos en algún punto entre
$A$ y $B$, en el punto más cercano a $B$ que estuvimos hasta ahora.''.

Más específicamente, si analizamos el ciclo para buscar el máximo en una lista
desordenada, la precondición es que la lista contiene elementos que son
comparables y la postcondición es que se devuelve el elemento máximo de la
lista.

\begin{codigo-python-sn}
def maximo(lista):
    "Devuelve el elemento máximo de la lista o None si está vacía."
    if not lista:
        return None
    max_elem = lista[0]
    for elemento in lista:
        if elemento > max_elem:
            max_elem = elemento
    return max_elem
\end{codigo-python-sn}

En este caso, el invariante del ciclo es que \lstinline!max_elem! contiene el
valor máximo de la porción de lista analizada.

Los invariantes son de gran importancia al momento de demostrar que un
algoritmo funciona, pero aún cuando no hagamos una demostración formal es muy
útil tener los invariantes a la vista, ya que de esta forma es más fácil
entender cómo funciona un algoritmo y encontrar posibles errores.

Los invariantes, además, son útiles a la hora de determinar las condiciones
iniciales de un algoritmo, ya que también deben cumplirse para ese caso.  Por
ejemplo, consideremos el algoritmo para obtener la potencia \lstinline!n! de
un número.

\begin{codigo-python-sn}
def potencia(b, n):
    "Devuelve la potencia n del número b, con n entero mayor que 0."
    p = 1
    for i in range(n):
        p *= b
    return p
\end{codigo-python-sn}

En este caso, el invariante del ciclo es que la variable \lstinline!p!
contiene el valor de la potencia correspondiente a esa iteración. Teniendo en
cuenta esta condición, es fácil ver que \lstinline!p! debe comenzar el ciclo
con un valor de 1, ya que ese es el valor correspondiente a $p^0$.

De la misma manera, si la operación que se quiere realizar es sumar todos los
elementos de una lista, el invariante será que una variable \lstinline!suma!
contenga la suma de todos los elementos ya recorridos, por lo que es claro que
este invariante debe ser 0 cuando aún no se haya recorrido ningún elemento.

\begin{codigo-python-sn}
def suma(lista):
    "Devuelve la suma de todos los elementos de la lista."
    suma = 0
    for elemento in lista:
        suma += elemento
    return suma
\end{codigo-python-sn}

% TODO
% \subsection{Invariantes como medida de cuánto falta}

%Dependiendo del problema y las herramientas con las que contemos algunos
%invariantes se pueden medir retomando el ejemplo de ir de A a B, uno podría
%medir la distancia hasta B para esta, pero si para medir la distancia hay que ir hasta B
%y volver deja de tener sentido. O al estar buscando el mínimo en una
%secuencia, cómo hago para comprobar que paso a paso tengo el mínimo de la
%secuencia que ya recorrí sin usar

\subsection{Comprobación de invariantes desde el código}

Cuando la comprobación necesaria para saber si seguimos ``en camino'' es simple,
se la puede tener directamente dentro del código.  Evitando seguir avanzando
con el algoritmo si se produjo un error crítico.

Por ejemplo, en una búsqueda binaria, el elemento a buscar debe ser mayor que
el elemento inicial y menor que el elemento final, de no ser así, no tiene sentido
continuar con la búsqueda.  Es posible, entonces, agregar una instrucción
que compruebe esta condición y de no ser cierta realice alguna acción para
indicar el error, por ejemplo, utilizando la instrucción \lstinline!assert!,
vista anteriormente.

\section{Mutabilidad e Inmutabilidad}
\label{mutabilidad}

Hasta ahora cada vez que estudiamos un tipo de datos indicamos si son
mutables o inmutables.

Cuando un valor es de un tipo inmutable, como por ejemplo una cadena, es
posible asignar un nuevo valor a esa variable, pero no es posible modificar su
contenido.

\begin{codigo-python-sn}
>>> s = "ejemplo"
>>> s = "otro"
>>> s[2] = "c"
(^Traceback (most recent call last):
  File "<stdin>", line 1, in <module>
TypeError: 'str' object does not support item assignment^)
\end{codigo-python-sn}

Esto se debe a que cuando se realiza una nueva asignación, no se modifica la
cadena en sí, sino que la variable \lstinline!s! pasa a \emph{referenciar} a otra cadena.
En cambio, no es posible asignar un nuevo caracter en una posición, ya que
esto implicaría modificar la cadena inmutable.

En el caso de los parámetros mutables, la asignación tiene el mismo
comportamiento, es decir que las variables pasan a apuntar a un nuevo valor.

\begin{codigo-python-sn}
>>> lista1 = [10, 20, 30]
>>> lista2 = lista1
>>> lista1 = [3, 5, 7]
>>> lista1
[3, 5, 7]
>>> lista2
[10, 20, 30]
\end{codigo-python-sn}

Algo importante a tener en cuenta en el caso de las variables de tipo
mutable es que si hay dos o más variables que \textit{referencian} a un mismo
valor, y este valor se modifica, el cambio se verá reflejado en ambas variables.

\begin{codigo-python-sn}
>>> lista1 = [1, 2, 3]
>>> lista2 = lista1
>>> lista2[1] = 5
>>> lista1
[1, 5, 3]
\end{codigo-python-sn}

\begin{sabias_que}
En otros lenguajes, como C o C++, existe un tipo de variable especial
llamado \emph{puntero}, que se comporta como una referencia a un valor,
como es el caso de las variables mutables del ejemplo anterior.

En Python no hay punteros como los de C o C++, pero todas las variables son
referencias a una porción de memoria, de modo que cuando se asigna una
variable a otra, lo que se está asignando es la porción de memoria a la que
refieren.  Si esa porción de memoria cambia, el cambio se puede ver en
todas las variables que apuntan a esa porción.
\end{sabias_que}

% TODO: describir en más detalle el modelo de referencia de python ?

\subsection{Parámetros mutables e inmutables}

Las funciones reciben parámetros que pueden ser mutables o inmutables.

Si dentro del cuerpo de la función se modifica uno de estos parámetros para
que \textit{referencie} a otro valor, este cambio no se verá reflejado fuera de la
función.  Si, en cambio, se modifica el \textit{contenido} de alguno de los
parámetros mutables, este cambio \textit{sí} se verá reflejado fuera de la
función.

A continuación un ejemplo en el cual se asigna la variable recibida, a un
nuevo valor.  Esa asignación sólo tiene efecto dentro de la función.

\begin{codigo-python-sn}
>>> def no_cambia_lista(lista):
...     lista = [0, 1, 2, 3]
...     print("Dentro de la funcion lista =", lista)
...
>>> lista = [10, 20, 30, 40]
>>> no_cambia_lista(lista)
Dentro de la funcion lista = [0, 1, 2, 3]
>>> lista
[10, 20, 30, 40]
\end{codigo-python-sn}

A continuación un ejemplo en el cual se modifica la variable recibida. En este
caso, los cambios realizados tienen efecto tanto dentro como fuera de la
función.

\begin{codigo-python-sn}
>>> def cambia_lista(lista):
...     for i in range(len(lista)):
...         lista[i] = lista[i] ** 3
...
>>> lista = [1, 2, 3, 4]
>>> cambia_lista(lista)
>>> lista
[1, 8, 27, 64]
\end{codigo-python-sn}

\begin{atencion}
En general, se espera que una función que recibe parámetros mutables no los
modifique, ya que si se los modifica se podría perder información valiosa.

En el caso en que por una decisión de diseño o especificación se modifiquen
los parámetros recibidos, esto debe estar claramente documentado, dentro de
las postcondiciones.
\end{atencion}

\section{Resumen}

\begin{itemize}
\item La \textbf{documentación} tiene como objetivo explicar \emph{qué} hace el código,
    y está dirigida a quien desee utilizar la función o módulo.
\item Los \textbf{comentarios} tienen como objetivo explicar \emph{cómo} funciona el
    código y \emph{por qué} se decidió implementarlo de esa manera, y están dirigidos a
    quien esté leyendo el código fuente.
\item Las \textbf{precondiciones} son las condiciones que deben cumplir los
parámetros recibidos por una función.
\item Las \textbf{postcondiciones} son las condiciones cumplidads por los
resultados que la función devuelve y por los parámetros recibidos, siempre
que las precondiciones hayan sido válidas.
\item Los \textbf{invariantes de ciclo} son las condiciones que deben
cumplirse al comienzo de cada iteración de un ciclo.
\item En el caso en que estas \textbf{aseveraciones} no sean verdaderas, se
deberá a un error en el diseño o utilización del código.
\item En general una función no debe modificar el contenido de sus parámetros,
aún cuando esto sea posible, a menos que sea la funcionalidad explícita de esa
función.
\end{itemize}

\begin{referencia_python}

\begin{sintaxis}{Documentación: \lstinline|"..."| ó \lstinline|"""..."""|}
    Por convención, si la primera línea de una función o módulo es una cadena,
    esa será su documentación
\end{sintaxis}

\begin{sintaxis}{Comentarios: \lstinline|# ...|}
    El intérprete ignora cualquier texto que se encuentra desde el caracter
    |#| hasta el fin de la línea.
\end{sintaxis}

\begin{sintaxis}{\lstinline!assert condicion[, mensaje]!}
Verifica si la condición es verdadera.  En caso contrario, provoca un error
con el mensaje recibido por parámetro.
\end{sintaxis}

\end{referencia_python}

\newpage
\begin{subappendices}
\section{Acertijo MU}

El acertijo MU\footnote{%
\url{http://en.wikipedia.org/wiki/Invariant\_(computer\_science)}} es un buen
ejemplo de un problema lógico donde es útil determinar el invariante.  El
acertijo consiste en buscar si es posible convertir MI a MU, utilizando las
siguientes operaciones.

\begin{enumerate}
\item Si una cadena termina con una I, se le puede agregar una U (xI -> xIU)
\item Cualquier cadena luego de una M puede ser totalmente duplicada (Mx ->
Mxx)
\item Donde haya tres Is consecutivas (III) se las puede reemplazar por una U
(xIIIy -> xUy)
\item Dos Us consecutivas, pueden ser eliminadas (xUUy -> xy)
\end{enumerate}

Para resolver este problema, es posible pasar horas aplicando estas reglas
a distintas cadenas.  Sin embargo, puede ser más fácil encontrar una
afirmación que sea invariante para todas las reglas y que muestre si es o
no posible llegar a obtener MU.

Al analizar las reglas, la forma de deshacerse de las Is es conseguir tener
tres Is consecutivas en la cadena.  La única forma de deshacerse de todas las
Is es que haya un cantidad de Is consecutivas múltiplo de tres.

Es por esto que es interesante considerar la siguiente afirmación como
invariante: el número de Is en la cadena no es múltiplo de tres.

Para que esta afirmación sea invariante al acertijo, para
cada una de las reglas se debe cumplir que: si el invariante era verdadero
antes de aplicar la regla, seguirá siendo verdadero luego de aplicarla.

Para ver si esto es cierto o no, es necesario considerar la aplicación del
invariante para cada una de las reglas.

\begin{enumerate}
\item Se agrega una U, la cantidad de Is no varía, por lo cual se mantiene el
invariante.
\item Se duplica toda la cadena luego de la M, siendo $n$ la cantidad de
Is antes de la duplicación, si $n$ no es múltiplo de 3, $2n$ tampoco lo será.
\item Se reemplazan tres Is por una U.  Al igual que antes, siendo $n$ la
cantidad de Is antes del reemplazo, si $n$ no es múltiplo de 3, $n-3$ tampoco
lo será.
\item Se eliminan Us, la cantidad de Is no varía, por lo cual se mantiene el
invariante.
\end{enumerate}

Todo esto indica claramente que el invariante se mantiene para cada una de las
posibles transformaciones.  Esto significa que sea cual fuere la regla que se
elija, si la cantidad de Is no es un múltiplo de tres antes de aplicarla, no
lo será luego de hacerlo.

Teniendo en cuenta que hay una única I en la cadena inicial MI, y que uno no
es múltiplo de tres, es imposible llegar a MU con estas reglas, ya que MU
tiene cero Is, que sí es múltiplo de tres.
\end{subappendices}

\chapter{Manejo de archivos}
\label{uni:archivos}

Veremos en esta unidad cómo manipular archivos desde nuestros programas. Los
archivos permiten almacenar información que persistirá luego de que el programa
finalice su ejecución. Los archivos también se pueden compartir o ser
transmitidos entre diferentes computadoras, mediante dispositivos de
almacenamiento o redes como Internet.

\section{¿Qué es un archivo?}

Un archivo no es otra cosa más que una secuencia de bytes. Por ejemplo, un
archivo puede contener la secuencia de 4 bytes: |48 6f 6c 61| (notación
hexadecimal).

\begin{sabias_que}
\edef\myindent{\the\parindent}%
\noindent%
\begin{minipage}{.55\textwidth}
Un byte está formado de 8 bits, y según el valor de cada uno de los bits, un byte
puede representar $2^8 = 256$ combinaciones diferentes.
Si asignamos un valor numérico a cada una de esas combinaciones, comenzando
de 0, con un byte podemos representar cualquier número entre 0 y 255.

\setlength{\parindent}{\myindent}
Cuando se representan datos binarios, en lugar de utilizar la notación
binaria (base 2) o la decimal (base 10), se suele utilizar la notación
\emph{hexadecimal} (base 16).

En Python se puede escribir un número en notación binaria con el prefijo |0b|,
y en notación hexadecimal con el prefijo |0x|. Además, las funciones |bin|
y |hex| permiten obtener la representación de un número en binario y
hexadecimal, respectivamente.

\begin{codigo-python-sn}
>>> 0b11111101
253
>>> 0xfd
253
>>> bin(0xfd)
'0b11111101'
>>> hex(253)
'0xfd'
\end{codigo-python-sn}
\end{minipage}\hfill%
\begin{minipage}{.44\textwidth}
\begin{center}
\begin{tabular}{r r r}
{\bf Binario} & {\bf Decimal} & {\bf Hexadecimal} \\
\hline
\lstinline!00000000! & \lstinline!0! & \lstinline!00! \\
\lstinline!00000001! & \lstinline!1! & \lstinline!01! \\
\lstinline!00000010! & \lstinline!2! & \lstinline!02! \\
\lstinline!00000011! & \lstinline!3! & \lstinline!03! \\
\lstinline!00000100! & \lstinline!4! & \lstinline!04! \\
\lstinline!00000101! & \lstinline!5! & \lstinline!05! \\
\lstinline!00000110! & \lstinline!6! & \lstinline!06! \\
\lstinline!00000111! & \lstinline!7! & \lstinline!07! \\
\lstinline!00001000! & \lstinline!8! & \lstinline!08! \\
\lstinline!00001001! & \lstinline!9! & \lstinline!09! \\
\lstinline!00001010! & \lstinline!10! & \lstinline!0a! \\
\lstinline!00001011! & \lstinline!11! & \lstinline!0b! \\
\lstinline!00001100! & \lstinline!12! & \lstinline!0c! \\
\lstinline!00001101! & \lstinline!13! & \lstinline!0d! \\
\lstinline!00001110! & \lstinline!14! & \lstinline!0e! \\
\lstinline!00001111! & \lstinline!15! & \lstinline!0f! \\
\lstinline!00010000! & \lstinline!16! & \lstinline!10! \\
\lstinline!...!      & \lstinline!...! & \lstinline!...! \\
\lstinline!11111101! & \lstinline!253! & \lstinline!fd! \\
\lstinline!11111110! & \lstinline!254! & \lstinline!fe! \\
\lstinline!11111111! & \lstinline!255! & \lstinline!ff! \\
\end{tabular}
\end{center}
\end{minipage}
\end{sabias_que}

Un archivo se identifica con un \emph{nombre}, por ejemplo |hola.txt|. Para
facilitar la gestión y búsqueda eficiente, los archivos se organizan en
\emph{carpetas} y \emph{subcarpetas}, formando una estructura jerárquica: cada
carpeta puede contener archivos y otras carpetas, permitiendo una organización
lógica y estructurada de la información.

La ubicación de un archivo se identifica mediante una \emph{ruta}, que es una
cadena formada por la secuencia de carpetas y subcarpetas que lleva a dicho
archivo desde la \emph{carpeta raíz} (\lstinline!/!). Por ejemplo, si nuestro
archivo se encuentra dentro de la carpeta |home| y subcarpeta |alan|, su ruta
sería \verb!"/home/alan/hola.txt"!\footnote{En sistemas Windows se utiliza el
caracter {\tt \textbackslash} como separador en lugar del caracter
\lstinline!/!. Sin embargo, al especificar rutas en nuestros programas Python
se acepta que utilicemos el separador \lstinline!/!; de esta manera nuestros
programas podrán funcionar en cualquier sistema operativo.}. Una ruta se puede
escribir en forma \emph{absoluta} (comenzando con \lstinline!/! y conteniendo
la secuencia completa de carpetas y subcarpetas desde la carpeta raíz), o
\emph{relativa} a alguna carpeta (sin comenzar con \lstinline!/!). En nuestro
ejemplo, la ruta del archivo relativa a la carpeta \verb!"/home"! sería
\verb!"alan/hola.txt"!.

\section{Formatos de archivos}

Para cualquier información que se desee almacenar en un archivo, se debe elegir
una codificación que permita representar esa información mediante una secuencia
de bytes.

Por ejemplo, si deseamos almacenar el texto \texttt{"Hola mundo!"} en un archivo, lo
más simple sería elegir la codificación ASCII, en la que cada caracter se
almacena como 1 byte. De esta manera el archivo contendría en total 11 bytes:
|48 6f 6c 61 20 6d 75 6e 64 6f 21|.
Según la codificación ASCII, el caracter |H| se codifica con el valor
hexadecimal |48|, el caracter |o| con el valor |6f|, y así sucesivamente.

ASCII es un ejemplo de un \emph{formato de codificación de texto}, pero no es
el único. La limitación principal del formato ASCII es que solo permite
representar 128 caracteres. El formato UTF-8 es más complejo que ASCII,
ya que representa cada caracter con una cantidad variable de bytes, pero a
cambio permite representar más de un millón de caracteres. UTF-8 es la codificación
que se usa por defecto en la mayoría de los sistemas.

Si en lugar de texto deseamos almacenar una imagen en un archivo, tendríamos
que elegir otra codificación. En la Figura~\ref{fig-bmp} se muestra una imagen
codificada en formato BMP. Notar que en este caso el algoritmo de codificación
no es tan directo o intuitivo.

\begin{figure}[htb]
\hfill%
\begin{minipage}{.1\textwidth}
  \includegraphics[width=\textwidth]{graficos/bmp}
\end{minipage}%
\hfill%
\begin{minipage}{.85\textwidth}\raggedright%
\noindent{\footnotesize\tt%
4d 42 00 ba 00 00 00 00 00 00 00 8a 00 00 00 7c 00 00 00 04 00 00 00 04 00 00 \\
00 01 00 18 00 00 00 00 00 30 00 00 2e 23 00 00 2e 23 00 00 00 00 00 00 00 00 \\
00 00 00 00 00 ff ff 00 00 00 00 ff 00 00 00 00 00 00 47 42 73 52 00 00 00 00 \\
00 00 00 00 00 00 00 00 00 00 00 00 00 00 00 00 00 00 00 00 00 00 00 00 00 00 \\
00 00 00 00 00 00 00 00 00 00 00 00 00 00 00 00 00 00 00 02 00 00 00 00 00 00 \\
00 00 00 00 00 00 00 00 11 11 ff f9 00 00 11 11 ff f9 00 00 00 ff 11 00 f9 11 \\
00 ff 11 00 f9 11 11 11 ff f9 00 00 11 11 ff f9 00 00 00 ff 11 00 f9 11 00 ff \\
11 00 f9 11 \\}%
\end{minipage}
\hfill%
\caption{Una imagen de 4x4 pixels codificada en formato BMP.}
\label{fig-bmp}
\end{figure}

BMP, JPG y PNG son algunos ejemplos de formatos de codificación de imágenes.
También hay formatos de codificación para sonido, video y cualquier otro tipo
de información. Se dice que todos estos son \emph{formatos binarios}, en
contraste con los formatos de texto.\footnote{La distinción entre formatos
``de texto`` y ``binarios`` es histórica, pero cabe destacar que técnicamente
todos los formatos son binarios, incluso los de texto.}

Notar que el archivo está definido únicamente por su contenido (la secuencia de
bytes), independientemente de lo que representen esos bytes. Si solo tenemos el
contenido de un archivo dado, no hay manera de saber qué codificación se
utilizó para crear esos datos; es decir, de qué formato es el archivo. Es por
eso que se utiliza la convención de agregar una \emph{extensión} al nombre del
archivo para indicar su formato. Por ejemplo, a los archivos de texto se les
agrega la extensión |.txt|, a los archivos en formatp BMP se les agrega la
extensión |.bmp|, etc. Si bien es una práctica recomendable, no es obligatorio
que el nombre del archivo incluya una extensión, o que concuerde con el formato
real del archivo.

A continuación veremos cómo manipular archivos en nuestros programas Python.

\section{Abrir un archivo}

En Python (así como en la mayoría de los lenguajes de programación) hay dos
\emph{modos} para acceder a un archivo:

\begin{itemize}
    \item {\bf Modo texto} en el que podemos leer o escribir cadenas de texto (|str|), que
        serán codificadas automáticamente según la codificación elegida (ASCII, UTF-8, etc.).
        En la mayoría de los sistemas, el modo por defecto es texto con codificación UTF-8.
    \item {\bf Modo binario} en el que no se aplica ninguna codificación, y leemos o escribimos
        datos de tipo |bytes|.
\end{itemize}

Para poder leer o escribir un archivo, primero debemos pedirle permiso al
sistema operativo. Esta operación se llama \emph{abrir} el archivo. En Python,
para abrir un archivo usaremos la función \lstinline!open!, que recibe la
ruta del archivo a abrir.

\begin{codigo-python-sn}
archivo = open("archivo.txt")
\end{codigo-python-sn}

Esta función intentará abrir el archivo con el nombre indicado, por defecto en
modo texto. Si tiene éxito devolverá un valor de un tipo especial, que nos
permitirá manipular el archivo de diversas maneras.

\section{Leer un archivo de texto}

La operación más sencilla a realizar sobre un archivo es leer su contenido.
Podemos utilizar la función |read| para leer uno o más caracteres. Por ejemplo,
para almacenar en |s| una cadena con los 3 primeros caracteres leidos del archivo:

\begin{codigo-python-sn}
s = archivo.read(3)
\end{codigo-python-sn}

Si volvemos a invocar a |read(3)| se leerán los próximos 3 caracteres del archivo.
Esto es así ya que cada archivo que se encuentre abierto tiene una
posición asociada, que indica el último punto que fue leído.  Cada vez que
se lee un byte, avanza esa posición.

Lo más usual al trabajar con archivos de texto es procesarlos línea por línea;
es decir, leer hasta que encontramos el caracter \emph{nueva línea} o \verb!\n!. La
función |readline| hace esto de forma automática:

\begin{sabias_que}
Los archivos de texto son sencillos de manejar, pero existen por lo menos 3
formas distintas de marcar un fin de línea. En Unix tradicionalmente se usa
el caracter '\verb!\n!' (código ASCII~10, definido como ``nueva línea'') para
el fin de línea, mientras que en los sitemas de Apple el fin de línea se solía
representar como un '\verb!\r!' (valor ASCII~13, definido como retorno de
carro) y en Windows se usan ambos caracteres '\verb!\r\n!'.

Al leer archivos en modo texto, Python acepta cualquier tipo de fin
de línea como si fuese un |\n|. Al escribir archivos, Python elegirá
automáticamente el modo más apropiado. Si queremos modificar este
comportamiento podemos especificar el modo utilizando la opción |newline| de la
función |open|.
\end{sabias_que}

\begin{codigo-python-sn}
linea = archivo.readline()
while linea != '':
    # procesar linea
    linea = archivo.readline()
\end{codigo-python-sn}

El lenguaje Python nos permite hacer lo mismo de una manera más abreviada:

\begin{codigo-python-sn}
for linea in archivo:
    # procesar linea
\end{codigo-python-sn}

De esta manera, la variable \lstinline!linea! irá almacenando distintas cadenas
correspondientes a cada una de las líneas del archivo.

Es posible, además, obtener todas las líneas del archivo utilizando una
sola llamada a función:

\begin{codigo-python-sn}
lineas = archivo.readlines()
\end{codigo-python-sn}

En este caso, la variable \lstinline!lineas! tendrá una lista de cadenas con
todas las líneas del archivo.

\begin{atencion}
Es importante tener en cuenta que cuando se utilizan funciones como
\lstinline!archivo.readlines()!, se está cargando en la memoria de la computadora
el contenido completo del archivo.  Siempre que una instrucción cargue un archivo
completo en memoria debe tenerse cuidado de utilizarla sólo con archivos
pequeños, ya que de otro modo podría agotarse la memoria de la computadora.
\end{atencion}

\section{Cerrar un archivo}

Al terminar de trabajar con un archivo, es importante cerrarlo,
por diversos motivos: en algunos sistemas los archivos sólo pueden
ser abiertos de a un programa por la vez; en otros, lo que se haya
escrito no se guardará realmente hasta no cerrar el archivo; o el
límite de cantidad de archivos que puede manejar un programa puede
ser bajo, etc.

Para cerrar un archivo simplemente se debe llamar a:

\begin{codigo-python-sn}
archivo.close()
\end{codigo-python-sn}

Además, Python nos provee con una estructura que permite cerrar el archivo
automáticamente, sin necesidad de llamar a |close|:

\begin{codigo-python-sn}
with open("archivo.txt") as archivo:
    #
    # procesar el archivo
    #

# Cuando la ejecución sale del bloque 'with',
# el archivo se cierra automáticamente.
\end{codigo-python-sn}

\section{Ejemplo: procesamiento de archivos de texto}

Por ejemplo, para mostrar todas las líneas de un archivo,
precedidas por el número de línea, podemos hacerlo de la siguiente manera:

\begin{codigo-python-sn}
archivo = open("archivo.txt")
i = 1
for linea in archivo:
    linea = linea.rstrip("\n")
    print(f"{i}: {linea}")
    i += 1
archivo.close()
\end{codigo-python-sn}

La llamada a \lstinline!rstrip! es necesaria ya que cada línea que se lee del
archivo contiene un caracter especial llamado \emph{fin de línea} y con la llamada a
\lstinline!rstrip("\n")! se remueve.

Notar que sería equivalente usar el bloque |with| para ahorrarnos la llamada a
|close|:

\begin{codigo-python-sn}
(@with open("archivo.txt") as archivo:@)
    i = 1
    for linea in archivo:
        linea = linea.rstrip("\n")
        print(f"{i}: {linea}")
        i += 1
\end{codigo-python-sn}

También podemos utilizar la función |enumerate| (explicada en la
sección~\ref{enumerate}) para no tener que mantener el
contador |i| a mano:

\begin{codigo-python-sn}
with open("archivo.txt") as archivo:
    (@for i, linea in enumerate(archivo):@)
        linea = linea.rstrip("\n")
        print(f"{i}: {linea}")
\end{codigo-python-sn}

\section{Modo de apertura de los archivos}

La función \lstinline!open! recibe un parámetro opcional para indicar el
modo en que se abrirá el archivo. Como ya mencionamos, un archivo se puede abrir en \textbf{modo texto} (|'t'|) o \textbf{modo binario} (|'b'|).

También se pueden especificar tres modos en cuanto a los permisos de lectura y escritura:

\begin{itemize}
\item Modo de \textbf{sólo lectura} (\lstinline!'r'!).   En este caso no es
posible realizar modificaciones sobre el archivo, solamente leer su
contenido.

\item Modo de \textbf{sólo escritura} (\lstinline!'w'!). En este caso el
archivo es truncado (vaciado) si existe, y se lo crea si no existe.

\item Modo \textbf{sólo escritura posicionándose al final del archivo}
(\lstinline!a!). En este caso se crea el archivo, si no existe, pero en
caso de que exista se posiciona al final, manteniendo el contenido
original.
\end{itemize}

Por ejemplo, para abrir un archivo en modo binario y escritura:

\begin{codigo-python-sn}
archivo = open("imagen.jpg", "wb")
\end{codigo-python-sn}

En cualquiera de los modos |r|, |w| o |a| se puede agregar un
\lstinline!+! para pasar a un modo lectura-escritura. El comportamiento de
\lstinline!r+! y de \lstinline!w+! no es el mismo, ya que en el primer caso
se tiene el archivo completo, y en el segundo caso se trunca el archivo,
perdiendo así los datos.

\begin{observacion}
Si un archivo no existe y se lo intenta abrir en modo lectura, se generará
un error; en cambio si se lo abre para escritura, Python se encargará de
crear el archivo al momento de abrirlo, ya sea con \lstinline!'w'!,
\lstinline!'a'!, \lstinline!'w+'! o con \lstinline!'a+')!.
\end{observacion}

En caso de que no se especifique el modo, los archivos serán abiertos en
modo sólo lectura (\lstinline!r!).

\begin{atencion}
Si un archivo existente se abre en modo escritura (\lstinline!'w'! o
\lstinline!'w+'!), todos los datos anteriores son borrados y reemplazados
por lo que se escriba en él.
\end{atencion}

\section{Escribir un archivo de texto}

De la misma forma que para la lectura, existen dos formas distintas de
escribir a un archivo.  Mediante cadenas:

\begin{codigo-python-sn}
archivo.write(cadena)
\end{codigo-python-sn}

O mediante listas de cadenas:

\begin{codigo-python-sn}
archivo.writelines(lista_de_cadenas)
\end{codigo-python-sn}

Así como la función \lstinline!readline! devuelve las líneas con los caracteres
de fin de línea (\lstinline!\n!), será necesario agregar los caracteres de
fin de línea a las cadenas que se vayan a escribir en el archivo.

Por ejemplo, el siguiente programa genera a su vez el código de otro programa
Python y lo guarda en el archivo |saludo.py|:

\begin{codigo-python-sn}
with open("saludo.py", "w") as saludo:
    saludo.write("# Este programa fue generado por otro programa!\n")
    saludo.write("print('Hola Mundo')\n")
\end{codigo-python-sn}

\begin{atencion}
Si un archivo existente se abre en modo lectura-escritura, al escribir en
él se sobreescribirán los datos anteriores, a menos que se haya llegado al
final del archivo.

Este proceso de sobreescritura se realiza caracter por caracter, sin
consideraciones adicionales para los caracteres de fin de línea ni otros
caracteres especiales.
\end{atencion}

\section{Agregar información a un archivo}

Abrir un archivo en modo \emph{agregar al final} puede parece raro,
pero es bastante útil.

Uno de sus usos es para escribir un archivo de bitácora (o archivo de
{\textit log}), que nos permita ver los distintos eventos que se fueron
sucediendo, y así encontrar la secuencia de pasos (no siempre evidente) que
hace nuestro programa.

Esta es una forma muy habitual de buscar problemas o hacer un seguimiento
de los sucesos. Para los administradores de sistemas es una herramienta
esencial de trabajo.

En el Código~\ref{modulo_log} se muestra un módulo para manejo de logs, que
se encarga de la apertura del archivo, del guardado de las líneas una por
una y del cerrado final del archivo.

\begin{codigo}{log.py}{Módulo para manipulación de archivos de log}
\label{modulo_log}
\lstinputlisting{src/11_archivos/log.py}
\end{codigo}

En este módulo se utiliza el módulo de Python \lstinline!datetime! para
obtener la fecha y hora actual que se guardará en los archivos.  Es
importante notar que en el módulo mostrado no se abre o cierra un archivo
en particular, sino que las funciones están programadas de modo tal que
puedan ser utilizadas desde otro programa.

Se trata de un módulo genérico que podrá ser utilizado por diversos programas,
que requieran la funcionalidad de registrar los posibles errores o eventos que
se produzcan durante la ejecución.

Para utilizar este módulo, será necesario primero llamar a
\lstinline!log.abrir()! para abrir el archivo de log, luego llamar a
\lstinline!log.guardar()! por cada mensaje que se quiera registrar, y
finalmente llamar a \lstinline!log.cerrar()! cuando se quiera concluir la
registración de mensajes:

\begin{codigo-python-sn}
import log

archivo_log = log.abrir("log.txt")
# ...
# Hacer cosas que pueden dar error
if error:
    log.guardar(archivo_log, "ERROR: " + error)
# ...
# Finalmente cerrar el archivo
log.cerrar(archivo_log)
\end{codigo-python-sn}

Este código, que incluye el módulo \lstinline!log! mostrado anteriormente,
muestra una forma básica de utilizar un archivo de log.  Al iniciarse el
programa se abre el archivo de log, de forma que queda registrada la fecha
y hora de inicio.  Posteriormente se realizan tareas varias que podrían
provocar errores, y de haber algún error se lo guarda en el archivo de log.
Finalmente, al terminar el programa, se cierra el archivo de log, quedando
registrada la fecha y hora de finalización.

El archivo de log generado tendrá la forma:

\begin{codigo-nohl-sn}
2016-04-10 15:20:32.229556 Iniciando registro de errores
2016-04-10 15:20:50.721415 ERROR: no se pudo acceder al recurso
2016-04-10 15:21:58.625432 ERROR: formato de entrada inválido
2016-04-10 15:22:10.109376 Fin del registro de errores
\end{codigo-nohl-sn}

\section{Archivos binarios}

Para abrir un archivo y manejarlo de forma binaria es necesario agregarle
una \verb!'b'! al parametro de modo:

\begin{codigo-python-sn}
archivo_binario = open('imagen.jpg', 'rb')
\end{codigo-python-sn}

Para procesar el archivo de a bytes en lugar de líneas, se utiliza la
función \lstinline!contenido = archivo.read(n)! para leer \lstinline!n!
bytes y \lstinline!archivo.write(contenido)!, para
escribir \lstinline!contenido! en la posición actual del archivo.

Al manejar un archivo binario, frecuentemente es necesario conocer la
posición actual en el archivo y poder modificarla. Para obtener la
posición actual se utiliza \lstinline!archivo.tell()!, que
indica la cantidad de bytes desde el comienzo del archivo.

Para modificar la posición actual se utiliza
\lstinline!archivo.seek(posicion)!, que permite desplazarse hacia el byte
ubicado en la posición indicada.

\begin{codigo-python-sn}
>>> f = open('imagen.jpg', 'rb')
>>> f.tell()
0
>>> datos = f.read(3)
>>> datos
b'\xff\xd8\xff'
>>> type(datos)
<class 'bytes'>
>>> f.tell()
3
>>> f.seek(0)
0
>>> datos = f.read() # leer el contenido completo del archivo
>>> len(datos)
3150
\end{codigo-python-sn}

\begin{atencion}
Al trabajar con archivos binarios, la función |read| no devuelve cadenas de
caracteres (|str|), sino que devuelve una \emph{secuencia de bytes} (|bytes|).
Análogamente, la función |write| recibe una secuencia de bytes.
\end{atencion}

\section{Persistencia de datos}

Se llama {\bf persistencia} a la capacidad de guardar la
información de un programa para poder volver a utilizarla en otro
momento. Es lo que los usuarios conocen como \emph{Guardar el archivo}
y después \emph{Abrir el archivo}. Pero para un programador puede
significar más cosas y suele involucrar un proceso de
\emph{serialización} de los datos a un archivo o a una base de datos o a
algún otro medio similar, y el proceso inverso de recuperar los
datos a partir de la información \emph{serializada}.

% Ejemplo Highscores

Por ejemplo, supongamos que en el desarrollo de un juego se quiere guardar
en un archivo la información referente a los ganadores, el puntaje máximo
obtenido y el tiempo de juego en el que obtuvieron ese puntaje.

En el juego, esa información podría estar almacenada en una lista de
tuplas:
\begin{codigo-python-sn}
[(nombre1, puntaje1, tiempo1), (nombre2, puntaje2, tiempo2), ...]
\end{codigo-python-sn}

Esta información se puede guardar en un archivo de muchas formas distintas.
En este caso, para facilitar la lectura del archivo de puntajes para los
humanos, se decide guardarlos en un archivo de texto, donde cada tupla
ocupará una línea y los valores de las tuplas estarán separados por
comas.

En el Código~\ref{puntajes} se muestra un módulo capaz de guardar y
recuperar los puntajes en el formato especificado.

\begin{codigo}{puntajes.py}{Módulo para guardar y recuperar puntajes en un archivo}
\label{puntajes}
\lstinputlisting{src/11_archivos/puntajes.py}
\end{codigo}

Dadas las especificaciones del problema al guardar los valores en el
archivo, es necesario convertir el puntaje (que es un valor numérico) en
una cadena, y al abrir el archivo es necesario convertirlo nuevamente en un
valor numérico.

\begin{observacion}
Es importante notar que tanto la función que almacena los datos como la que
los recupera requieren que la información se encuentre de una forma
determinada y de no ser así, fallarán.  Es por eso que estas condiciones se
indican en la documentación de las funciones como sus precondiciones. En
próximas unidades veremos cómo evitar que falle una función si alguna de
sus condiciones no se cumple.
\end{observacion}

Es bastate sencillo probar el módulo programado y ver que lo que se guarda
es igual que lo que se recupera:

\begin{codigo-python-sn}
>>> import puntajes
>>> valores = [("Pepe", 108, "4:16"), ("Juana", 2315, "8:42")]
>>> puntajes.guardar_puntajes("puntajes.txt", valores)
>>> recuperado = puntajes.recuperar_puntajes("puntajes.txt")
>>> print(recuperado)
[('Pepe', 108, '4:16'), ('Juana', 2315, '8:42')]
\end{codigo-python-sn}

% Fin ejemplo.

Guardar el estado de un programa se puede hacer tanto en un
archivo de texto, como en un archivo binario. En muchas
situaciones es preferible guardar la información en un archivo de
texto, ya que de esta manera es posible modificarlo fácilmente
desde cualquier editor de textos.

En general, los archivos de texto consumen más
espacio, pero son más faciles de entender y fáciles de usar desde
cualquier programa.

Por otro lado, en un archivo binario bien definido se puede evitar el
desperdicio de espacio, o también hacer que sea más eficiente acceder a los
datos.

En definitiva, la decisión de qué formato usar queda a discreción del
programador. Es importante recordar que el sentido común es el valor más
preciado en un programador.

\subsection{Persistencia en archivos CSV}

Un formato que suele usarse para transferir datos entre programas es
\textbf{CSV} (del inglés \emph{comma separated values}: valores separados
por comas). Es un formato bastante sencillo, tanto para leerlo como para
procesarlo desde el código, parecido al formato visto en el ejemplo
anteriormente.

\begin{codigo-nohl-sn}
Nombre,Apellido,Telefono,Cumpleaños
"John","Smith","555-0101","1973-11-24"
"Jane","Smith","555-0101","1975-06-12"
\end{codigo-nohl-sn}

En el ejemplo se puede ver una pequeña base de datos. En la primera línea
del archivo tenemos los nombres de los campos, un dato opcional desde el
punto de vista del procesamiento de la información, pero que facilita
entender el archivo.

En las siguientes lineas se ingresan los datos de la base de datos, cada
campo separado por comas. Los campos que son cadenas se suelen escribir
entre comillas dobles. Si alguna cadena contiene alguna comilla doble se la
reemplaza por \verb!\"! y una contrabarra se escribe como \verb!\\!.

En Python es bastante sencillo procesar de este tipo de archivos, tanto
para la lectura como para la escritura, mediante el módulo |csv|.

La funciones del ejemplo anterior podrían programarse mediante el módulo
|csv|.  En el Código~\ref{puntajes_csv} se muestra una posible implementación
que utiliza este módulo.  Si se prueba este código, se obtiene un resultado
idéntico al obtenido anteriormente.

\begin{codigo}{puntajes\_csv.py}{Módulo para guardar y recuperar puntajes en un archivo CSV}
\label{puntajes_csv}
\lstinputlisting{src/11_archivos/puntajes_csv.py}
\end{codigo}

El código en este caso es muy similar, ya que en el ejemplo original se
hacían muy pocas consideraciones al respecto de los valores: se asumía que
el primero y el tercero eran cadenas mientras que el segundo necesitaba ser
convertido a cadena.

\begin{observacion}
Es importante notar, entonces, que al utilizar el módulo \lstinline!csv!
en lugar de hacer el procesamiento en forma manual, se obtiene un
comportamiento más robusto, ya que el módulo \lstinline!csv! tiene en
cuenta muchos más casos que nuestro código original no. Por ejemplo, el
código anterior no tenía en cuenta que el nombre pudiera contener una coma.
\end{observacion}

En el apéndice de esta unidad puede verse una aplicación completa de una
agenda, que almacena los datos del programa en archivos CSV.

\subsection{Persistencia en archivos binarios}

En el caso de que decidiéramos grabar los datos en un archivo binario,
Python incluye varios módulos que pueden ser de ayuda. Entre ellos se
destacan los módulos |pickle| y |struct|.

El módulo |pickle| permite codificar automáticamente un valor de cualquier tipo
a una secuencia de bytes, y luego decodificarlo. Hay que tener en cuenta, sin
embargo, que no es nada simple acceder a un archivo en el formato |pickle|
desde un programa que no esté escrito en Python.

El módulo |struct| permite codificar y decodificar a mano cada fragmento de
información. Es bastante más complicado de usar que |pickle|, pero es esencial
si deseamos codificar datos binarios de forma tal que podamos decodificarlos en
otros sistemas o lenguajes.

En el Código~\ref{puntajes_pickle} se muestra el mismo ejemplo de
almacenamiento de puntajes, utilizando el módulo \lstinline!pickle!.

\begin{codigo}{puntajes\_pickle.py}{Módulo para guardar y recuperar puntajes en
    un archivo que usa \emph{pickle}}
\label{puntajes_pickle}
\lstinputlisting{src/11_archivos/puntajes_pickle.py}
\end{codigo}

En el apéndice de esta unidad puede verse una aplicación completa de una
agenda, que utiliza el módulo|struct| para almacenar datos en archivos.

\section{Resumen}

\begin{itemize}
\item Para utilizar un archivo desde un programa, es necesario abrirlo, y
cuando ya no se lo necesite, se lo debe cerrar.
\item Las instrucciones más básicas para manejar un archivo son leer y escribir.
\item Cada archivo abierto tiene relacionada una posición que se puede
consultar o cambiar.
\item Los archivos de texto se procesan generalmente línea por línea y
sirven para intercambiar información entre diversos programas o entre
programas y humanos.
\item Los archivos binarios se procesan generalmente byte por byte. Suelen ser
más eficientes al ser interpretados por una computadora, pero son ilegibles
para humanos.
\item Es posible acceder de forma secuencial a los datos, o se puede ir
accediendo a posiciones en distintas partes del archivo, dependiendo de
cómo esté almacenada la información y qué se quiera hacer con ella.
\item Es posible leer todo el contenido de un archivo y almacentarlo en una
única variable, pero si el archivo es muy grande puede consumir memoria
innecesariamente.
\end{itemize}

\begin{referencia_python}

\begin{sintaxis}{\lstinline{archivo = open(nombre[, modo])}}

Abre un archivo. |nombre| es el nombre completo del archivo,
|modo| especifica si se va usar para lectura ('\verb!r!'), escritura
truncando el archivo ('\verb!w!'), o escritura agregando al final del archivo
('\verb!a!'), agregándole un '\verb!+!' al modo el archivo se abre en
lectura-escritura, agregándole una '\verb!b!' el archivo se maneja como archivo
binario, agregándole '\verb!U!' los fin de línea se manejan a mano.
\end{sintaxis}

\begin{sintaxis}{\lstinline!archivo.close()!}
Cierra el archivo.
\end{sintaxis}

\begin{sintaxis}{\lstinline!with open(nombre) as archivo:!}
Abre el archivo para procesar dentro del bloque |with|. El archivo se
cerrará automáticamente al salir del bloque.
\end{sintaxis}

\begin{sintaxis}{\lstinline!linea = archivo.readline()!}
Lee una línea de texto del archivo

Si la cadena devuelta es vacía, es que se ha llegado al
final del archivo.
\end{sintaxis}

\begin{sintaxis}{\lstinline!for linea in archivo:!}
Itera sobre las lineas del archivo.
\end{sintaxis}

\begin{sintaxis}{\lstinline!lineas = archivo.readlines()!}
Devuelve una lista con todas las líneas del archivo.
\end{sintaxis}

\begin{sintaxis}{\lstinline!datos = archivo.read([n])!}
Si se trata de un archivo de texto, devuelve la cadena de |n|
caracteres situada en la posición actual del archivo.

Si se trata de un archivo binario, devuelve una secuencia de |n| bytes.

Si la secuencia devuelta es vacía, es que se ha llegado al
final del archivo.

De omitirse el parámetro \lstinline!n!, devuelve todo el contenido del archivo.
\end{sintaxis}

\begin{sintaxis}{\lstinline!archivo.write(contenido)!}
Escribe \lstinline!contenido! en la posición actual de \lstinline!archivo!.
\end{sintaxis}

\begin{sintaxis}{\lstinline!posicion = archivo.tell()!}
Devuelve un número que indica la posición actual en \lstinline!archivo!,
equivalente a la cantidad de bytes desde el comienzo del archivo.
\end{sintaxis}

\begin{sintaxis}{\lstinline!archivo.seek(posicion)!}
Modifica la posición actual en \lstinline!archivo!, trasladándose
hasta el byte |posicion|.
\end{sintaxis}

\begin{sintaxis}{\lstinline!os.path.exists(ruta)!}
Indica si la ruta existe o no.
No nos dice si es una carpeta, un archivo u otro tipo de archivo especial
del sistema.
\end{sintaxis}

\begin{sintaxis}{\lstinline!os.path.isfile(ruta)!}
Indica si la ruta existe y es un archivo.
\end{sintaxis}

\begin{sintaxis}{\lstinline!os.path.isdir(ruta)!}
Indica si la ruta existe y es una carpeta (directorio).
\end{sintaxis}

\begin{sintaxis}{\lstinline!os.path.join(ruta, ruta1[, ... rutaN]])!}
Une las rutas con el caracter de separación de carpetas que le corresponda
al sistema en uso.
\end{sintaxis}

\end{referencia_python}

\newpage
\section{Ejercicios}

\extractionlabel{guia}
\begin{ejercicio}
Escribir una función, llamada \lstinline|head| que reciba un archivo y un número
\lstinline!N! e imprima las primeras \lstinline!N! líneas del archivo.
\end{ejercicio}

\extractionlabel{guia}
\begin{ejercicio}
Escribir una función, llamada \lstinline|cp|, que copie todo el contenido de un
archivo (sea de texto o binario) a otro, de modo que quede exactamente igual.\\
{\bf Nota}: utilizar \lstinline!archivo.read(bytes)! para leer como máximo
una cantidad de bytes.
\end{ejercicio}

\extractionlabel{guia}
\begin{ejercicio}
Escribir una función, llamada \lstinline|wc|, que dado un archivo de texto, lo procese e
imprima por pantalla cuántas líneas, cuantas palabras y cuántos caracteres
contiene el archivo.
\end{ejercicio}

\extractionlabel{guia}
\begin{ejercicio}
Escribir una función, llamada \lstinline|grep|, que reciba una cadena y un
archivo e imprima las líneas del archivo que contienen la cadena recibida.
\end{ejercicio}

\extractionlabel{guia}
\begin{ejercicio}
Escribir una función, llamada \lstinline|rot13|, que reciba un archivo de texto de
origen y uno de destino, de modo que para cada línea del archivo origen, se
guarde una línea \emph{cifrada} en el archivo destino.  El algoritmo de cifrado
a utilizar será muy sencillo: a cada caracter comprendido entre la a y la z, se
le suma 13 y luego se aplica el módulo 26, para obtener un nuevo caracter.
\end{ejercicio}

\extractionlabel{guia}
\begin{ejercicio} {\bf Persistencia de un diccionario}
\begin{partes}
  \item Escribir una función \lstinline!cargar_datos! que reciba un nombre de
archivo, cuyo contenido tiene el formato \lstinline!clave, valor! y devuelva un
diccionario con el primer campo como clave y el segundo como diccionario.
  \item Escribir una función \lstinline!guardar_datos! que reciba un diccionario
y un nombre de archivo, y guarde el contenido del diccionario en el archivo,
con el formato \lstinline!clave, valor!.
\end{partes}
\end{ejercicio}

\newpage
\begin{subappendices}
\section{Agenda con archivos CSV}

A continuación se muestra un programa de agenda que utiliza archivos
CSV\@. Luego se muestran los cambios necesarios para que la agenda que utilice archivos
en formato binario utilizando el módulo |struct|, en lugar de CSV.

\tituloCodigo{\lstinline|agenda-csv.py| Agenda con los datos en formato CSV}
\lstinputlisting{src/11_archivos/agenda-csv.py}

\section{Agenda con archivos binarios}

\tituloCodigo{\lstinline|agenda-struct.py| Modificaciones a la agenda para
guardar los datos en formato binario, utilizando el módulo \texttt{struct}}
\lstinputlisting[firstline=1,lastline=49]{src/11_archivos/agenda-struct.py}

\end{subappendices}

\include{13_excepciones}
\include{12_procesamiento}
\chapter{Objetos}

Los \emph{objetos} son una manera de organizar datos y de relacionar esos datos
con el código apropiado para manejarlo.  Son los protagonistas de un
paradigma de programación llamado \emph{Programación Orientada a Objetos}.

Nosotros ya usamos objetos en Python sin mencionarlo explícitamente. Es más,
todos los tipos de datos que Python nos provee son, en realidad, objetos.

De forma que, cuando utilizamos \lstinline!cadena.upper()!, le estamos
diciendo a Python que llame a la función \lstinline!upper! del tipo
\lstinline!str! para \lstinline!cadena! que es lo mismo que decir que
llame al \emph{método} \lstinline!upper! del objeto \lstinline!cadena!.

A su vez, a las variables que un objeto contiene se las llama
\emph{atributos}.

\begin{sabias_que}
La Programación Orientada a Objetos introduce bastante terminología, y una
gran parte es simplemente darle un nuevo nombre a cosas que ya estuvimos
usando.  Esto si bien parece raro es algo bastante común en el aprendizaje
humano.

Para poder pensar abstractamente, los humanos necesitamos asignarle
distintos nombres a cada cosa o proceso. De la misma manera, para poder
hacer un cambio en una forma de ver algo ya establecido (realizar un
\emph{cambio de paradigma}), suele ser necesario cambiar la forma de nombrar a
los elementos que se comparten con el paradigma anterior, ya que sino es
muy difícil realizar el salto al nuevo paradigma.
\end{sabias_que}

\section{Tipos}

En los temas que vimos hasta ahora nos hemos encontrado con numerosos tipos
provistos por Python, los \emph{números}, las \emph{cadenas de caracteres},
las \emph{listas}, las \emph{tuplas}, los \emph{diccionarios}, los
\emph{archivos}, etc.  Cada uno de estos tipos tiene sus características, tiene
operaciones propias de cada uno y nos provee de una gran cantidad de
funcionalidades que podemos utilizar para nuestros programas.

Como ya se dijo en unidades anteriores, para saber de qué tipo es un
valor, utilizamos la función \lstinline!type!. Para saber qué métodos
y atributos tiene ese valor utilizamos la función \lstinline!dir!:

\begin{codigo-python-sn}
>>> f = open("archivo.txt")
>>> type(f)
<class '_io.TextIOWrapper'>
>>> dir(f)
['_CHUNK_SIZE', '__class__', '__del__', '__delattr__', '__dict__',
'__dir__', '__doc__', '__enter__', '__eq__', '__exit__', '__format__',
'__ge__', '__getattribute__', '__getstate__', '__gt__', '__hash__',
'__init__', '__iter__', '__le__', '__lt__', '__ne__', '__new__',
'__next__', '__reduce__', '__reduce_ex__', '__repr__', '__setattr__',
'__sizeof__', '__str__', '__subclasshook__', '_checkClosed',
'_checkReadable', '_checkSeekable', '_checkWritable', '_finalizing',
'buffer', 'close', 'closed', 'detach', 'encoding', 'errors', 'fileno',
'flush', 'isatty', 'line_buffering', 'mode', 'name', 'newlines',
'read', 'readable', 'readline', 'readlines', 'seek', 'seekable',
'tell', 'truncate', 'writable', 'write', 'writelines']
\end{codigo-python-sn}

En este caso, la función \lstinline!dir! nos muestra los métodos que tiene
un objeto del tipo \lstinline!_io.TextIOWrapper! (el tipo que Python le asigna
internamente a un archivo).  Podemos ver en el listado los métodos
que ya hemos visto al operar con archivos, junto con otros métodos con
nombres \emph{raros} como \lstinline!__str__!, o \lstinline!__doc__!. Estos
métodos son especiales en Python; más adelante veremos para qué sirven y
cómo se usan.

En el listado que nos da \lstinline!dir! están los atributos y métodos
mezclados.  Si necesitamos saber cuáles son atributos y cuáles son métodos,
podemos hacerlo nuevamente mediante el uso de \lstinline!type!.

\begin{codigo-python-sn}
>>> type(f.name)
<type 'str'>
>>> f.name
'archivo.txt'
>>> type (f.tell)
<type 'builtin_function_or_method'>
>>> f.tell()
0L
\end{codigo-python-sn}

Es decir que \lstinline!name! es un atributo del objeto (el nombre del
archivo), mientras que \lstinline!tell! es un método, que para utilizarlo
debemos llamarlo con paréntesis.

\begin{observacion}
En Python los métodos se invocan con la \emph{notación punto}:
\lstinline+archivo.tell()+, \lstinline+cadena.split(":")+.

Analicemos la segunda expresión.  El significado de ésta es: la variable
\lstinline!cadena! llama al método \lstinline+split+ (del cual es dueña
por tratarse de una variable de tipo \lstinline!str!) con el argumento
\lstinline+":"+.

Sería equivalente a llamar a la función \lstinline!split! pasándole como
primer parámetro la variable |cadena|, y como segundo parámetro el delimitador
|":"|.
Pero la diferencia de notación resalta que el método \lstinline!split! es
un método {\bf de} cadenas, y que no se lo puede utilizar con variables de
otros tipos.

Esta notación provocó un cambio de paradigma en la programación, y es uno de
los ejes de la \emph{Programación Orientada a Objetos}.
\end{observacion}

\section{Qué es un objeto}

En Python, todos los tipos son objetos.  Pero no en todos los lenguajes de
programación es así.  En general, podemos decir que un objeto es una forma
ordenada de agrupar datos (los \emph{atributos}) y operaciones a utilizar
sobre esos datos (los \emph{métodos}).

Es importante notar que cuando decimos \emph{objetos} podemos estar haciendo
referencia a dos cosas parecidas, pero distintas.

Por un lado, la definición del tipo, donde se indican cuáles son los
atributos y métodos que van a tener todas las variables que sean de ese
tipo.  Esta definición se llama específicamente, la {\bf clase} del objeto.

A partir de una clase es posible crear distintos valores que son de ese
tipo. A cada uno de los valores generados a partir de una clase se los llama
{\bf instancia} de esa clase.

\begin{observacion}
Se dice que los objetos tienen {\bf estado} y {\bf comportamiento}, ya que
los valores que tengan los atributos de una instancia determinan el estado
actual de esa instancia, y los métodos definidos en una clase determinan
cómo se va a comportar ese objeto.
\end{observacion}

\section{Definiendo nuevos tipos}

Si bien Python nos provee con un gran número de tipos ya definidos, en
muchas situaciones utilizar solamente los tipos provistos por el lenguaje
resultará insuficiente.  En estas situaciones queremos poder crear nuestros
propios tipos, que almacenen la información relevante para el problema a
resolver y contengan las funciones para operar con esa información.

Por ejemplo, si se quiere representar un punto en el plano, es posible
hacerlo mediante una tupla de dos elementos, pero esta implementación es
limitada, ya que si se quiere poder operar con distintos puntos (sumarlos,
restarlos o calcular la distancia entre ellos) se deberán tener funciones
\emph{sueltas} para realizar las diversas operaciones.

Podemos hacer algo mejor definiendo un nuevo tipo \lstinline!Punto!, que almacene
la información relacionada con el punto, y contenga las operaciones que nos
interese realizar sobre él.

\subsection{Nuestra primera clase: Punto}

Queremos definir nuestra clase que represente un punto en el plano.
Lo primero que debemos notar es que existen varias formas de representar un
punto en el plano, por ejemplo, coordenadas polares o coordenadas
cartesianas.
Además, existen varias operaciones que se pueden realizar sobre un punto
del plano, e implementarlas todas podría llevar mucho tiempo.

Si representamos nuestro punto utilizando coordenadas cartesianas,
vamos a necesitar dos atributos numéricos, |x| e |y|, para almacenar las
coordenadas. En la Figura~\ref{fig-punto}, se muestra un diagrama simple con el
diseño de la clase |Punto|.

\tikzset{pics/Punto/.style n args={2}{code={
    \node[umlattr,anchor=west] (-x) {x: #1};
    \node[umlattr,below=of -x.south west,anchor=north west] (y) {y: #2};
    \node[umlattrs,fit=(-x) (y)] (attrs)  {};
    \node[umltitle,above=of attrs] (titulo) {Punto};
    \node[umlclass,fit=(titulo) (attrs)] (-box)   {};
    \draw[-] (titulo.south-|-box.west)--(titulo.south-|-box.east);
}}}

\begin{figure}[htb]
\begin{tikzpicture}
    \pic (Punto) {Punto={<número>}{<número>}};
    \pic[right= 3cm of Punto-box] (p1) {Punto={5}{7}};
    \pic[right=of p1-box] {Punto={-3.6}{0}};
\end{tikzpicture}
\caption{Diseño de la \emph{clase} \texttt{Punto}, y dos posibles \emph{instancias}}
\label{fig-punto}
\end{figure}

Veamos cómo crear nuestra clase |Punto| en Python:

\begin{codigo-python-sn}
class Punto: (~\circled{1}~)
    """Representación de un punto en el plano en
       coordenadas cartesianas (x, y)"""

    def __init__(self, x, y): (~\circled{2}~)
        "Constructor de Punto. x e y deben ser numéricos"
        self.x = x (~\circled{3}~)
        self.y = y
\end{codigo-python-sn}

\circled{1} En la primera línea de código indicamos que vamos a crear una nueva
clase, llamada \lstinline!Punto!.  Por convención, en los nombres de las clases
definidas por el programador se escribe cada palabra del nombre con la primera
letra en mayúsculas (|Punto|, |Rectangulo|, |ListaEnlazada|,
|Hotel|, etc.).

\circled{2} Además definimos uno de los métodos especiales, \lstinline!__init__!, el
{\bf constructor} de la clase.  Este método se llama automáticamente cada
vez que se crea una nueva instancia de la clase.

Todos los métodos de cualquier clase reciben como
primer parámetro a la instancia sobre la que está trabajando.  Por
convención, a ese primer parámetro se lo suele llamar \lstinline!self! (que
podríamos traducir como \emph{yo mismo}), pero puede llamarse de cualquier
forma.

En nuestro caso el constructor |__init__| recibe la instancia |self| y dos
parámetros más, |x| e |y|. De esta forma indicamos que para crear una instancia
de |Punto| será necesario proveer el valor de ambas coordenadas.

\circled{3} En el cuerpo del constructor se definen los atributos, utilizando la notación
punto con la instancia |self|. En este caso los atributos se llamarán igual que
los parámetros del constructor, |x| e |y|, y se inicializan con el valor de
los parámetros.

Para utilizar esta clase que acabamos de definir, lo haremos de la
siguiente forma:

\begin{codigo-python-sn}
>>> p = Punto(5, 7)
>>> type(p)
<class '__main__.Punto'>
>>> p.x
5
>>> p.y
7
\end{codigo-python-sn}

Al realizar la llamada \lstinline!Punto(5, 7)!:

\begin{enumerate}
\item Se crea una nueva instancia de la clase |Punto|.
\item Se ejecuta el constructor |__init__|, con |self| $\rightarrow$ la
    instancia nueva, |x| $\rightarrow$ 5, |y| $\rightarrow$ 7.
\item El constructor asigna los atributos |self.x| $\rightarrow$ 5, |self.y|
    $\rightarrow$ 7.
\item Cuando finaliza la ejecución del constructor, se asigna |p| $\rightarrow$
    la instancia de |Punto| recién creada.
\end{enumerate}

\begin{figure}[htb]
\begin{tikzpicture}
    \pic (Punto) {Punto={5}{7}};
    \node[left=of Punto-box.west] (p) {p};
    \draw[flecha] (p)--(Punto-box);
\end{tikzpicture}
\caption{Una instancia de \texttt{Punto}, referenciada con la variable \texttt{p}}
\end{figure}
\subsection{Agregando validaciones al constructor}

Hemos creado una clase \lstinline!Punto! que permite guardar valores
\lstinline!x! e \lstinline!y!.  Sin embargo, por más que en la
documentación se indique que los valores deben ser numéricos, el código
mostrado hasta ahora no impide que a \lstinline!x! e \lstinline!y! se les
asigne un valor cualquiera, no numérico.

\begin{codigo-python-sn}
>>> q = Punto("A", True)
>>> q.x
'A'
>>> q.y
True
\end{codigo-python-sn}

Si queremos impedir que esto suceda, debemos agregar validaciones al
constructor, como las vistas en unidades anteriores.

Verificaremos que los valores pasados para \lstinline!x! e \lstinline!y!
sean numéricos, utilizando la función \lstinline!validar_numero!:

\begin{codigo-python-sn}
def validar_numero(valor):
    "Si el valor es numérico, lo devuelve. En caso contrario lanza TypeError."
    if not isinstance(valor, (int, float, complex)):
        raise TypeError("{:r} no es un valor numérico".format(valor))
    return valor
\end{codigo-python-sn}

El nuevo constructor quedará así:

\begin{codigo-python-sn}
    def __init__(self, x, y):
        """Constructor de Punto. x e y deben ser numéricos,
           de no ser así, se levanta una excepción TypeError"""
        self.x = validar_numero(x)
        self.y = validar_numero(y)
\end{codigo-python-sn}

Este constructor impide que se creen instancias con valores inválidos para
\lstinline!x! e \lstinline!y!.

\begin{codigo-python-sn}
>>> Punto("A", True)
(^Traceback (most recent call last):
  File "<stdin>", line 1, in <module>
  File "punto.py", line 12, in __init__
    self.x = validar_numero(x)
  File "punto.py", line 5, in validar_numero
    raise TypeError("{!r} no es un valor numérico".format(valor))
TypeError: 'A' no es un valor numérico^)
\end{codigo-python-sn}

\subsection{Agregando operaciones}

Hasta ahora hemos creado una clase \lstinline!Punto! que permite
construirla con un par de valores, que deben ser sí o sí numéricos, pero no
podemos operar con esos valores.  Para aprovechar la ventaja de los objetos,
tenemos que definir operaciones adicionales que vayamos a querer realizar
sobre esos puntos.

Queremos, por ejemplo, poder calcular la distancia entre dos puntos.  Para
ello definimos un nuevo método \lstinline!distancia! que recibe el punto de
la instancia actual y el punto para el cual se quiere calcular la
distancia.

\begin{codigo-python-sn}
    def distancia(self, otro):
        """Devuelve la distancia entre ambos puntos."""
        dx = self.x - otro.x
        dy = self.y - otro.y
        return (dx * dx + dy * dy) ** 0.5
\end{codigo-python-sn}

Una vez agregado este método a la clase, será posible obtener la distancia
entre dos puntos, de la siguiente manera:

\begin{codigo-python-sn}
>>> p = Punto(5, 7)
>>> q = Punto(2, 3)
>>> p.distancia(q)
5.0
\end{codigo-python-sn}

Podemos ver, sin embargo, que la operación para calcular la distancia
incluye la operación de restar dos puntos y la de obtener la norma de un
vector. Sería deseable incluir también estas dos operaciones dentro de la
clase \lstinline!Punto!.

Agregaremos, entonces, el método para restar dos puntos:

\begin{codigo-python-sn}
    def restar(self, otro):
        """Devuelve el Punto que resulta de la resta
           entre dos puntos."""
        return Punto(self.x - otro.x, self.y - otro.y)
\end{codigo-python-sn}

La resta entre dos puntos es un nuevo punto.  Es por ello que este método
devuelve una nueva instancia de |Punto|, en lugar de modificar las instancias
|self| u |otro|.

A continuación definimos el método para calcular la norma del vector que
se forma uniendo un punto con el origen.

\begin{codigo-python-sn}
    def norma(self):
        """Devuelve la norma del vector que va desde el origen
           hasta el punto. """
        return (self.x * self.x + self.y * self.y) ** 0.5
\end{codigo-python-sn}

En base a estos dos métodos podemos ahora volver a escribir el método
\lstinline!distancia! para que aproveche el código de ambos:

\begin{codigo-python-sn}
    def distancia(self, otro):
        """Devuelve la distancia entre ambos puntos."""
        return self.restar(otro).norma()
\end{codigo-python-sn}

En definitiva, hemos definido tres operaciones en la clase
\lstinline!Punto!, que nos sirve para calcular restas, normas de vectores
al origen, y distancias entre puntos.

\begin{codigo-python-sn}
>>> p = Punto(5, 7)
>>> q = Punto(2, 3)
>>> r = p.restar(q)
>>> (r.x, r.y)
(3, 4)
>>> r.norma()
5.0
>>> q.distancia(r)
1.41421356237
\end{codigo-python-sn}

\begin{atencion}
Cuando definimos los métodos que va a tener una determinada clase es
importante tener en cuenta que el listado de métodos debe ser lo más
conciso posible.

Es decir, si una clase tiene algunos métodos básicos que pueden combinarse
para obtener distintos resultados, no queremos implementar toda posible
combinación de llamadas a los métodos básicos, sino sólo los básicos y
aquellas combinaciones que sean muy frecuentes, o en las que
tenerlas como un método aparte implique una ventaja significativa en cuanto
al tiempo de ejecución de la operación.

Este concepto se llama {\bf ortogonalidad} de los métodos, basado en la
idea de que cada método debe realizar una operación independiente de los
otros.  Entre las motivaciones que puede haber para agregar métodos que no
sean ortogonales, se encuentran la \emph{simplicidad de uso} y la
\emph{eficiencia}.
\end{atencion}

\section{Métodos especiales}

Así como el constructor, \lstinline!__init__!, existen diversos métodos
especiales que, si están definidos en nuestra clase, Python los llamará por
nosotros cuando se utilice una instancia en situaciones particulares.

\subsection{Conversión a cadena de texto}

Veamos qué pasa cuando intentamos convertir la instancia de |Punto| a una
cadena de texto:

\begin{codigo-python-sn}
>>> p = Punto(5, 7)
>>> str(p)
'<__main__.Punto object at 0x7fa6da7694a8>'
\end{codigo-python-sn}

Esto no resulta muy satisfactorio. Lo mismo sucede cuando intentamos imprimir
el objeto con |print| (internamente la función |print| aplica la función |str|
a todos los objetos que recibe antes de imprimirlos):

\begin{codigo-python-sn}
>>> print(p)
<__main__.Punto object at 0x7fa6da7694a8>
\end{codigo-python-sn}

Para poder modificar el comportamiento del objeto al convertir a cadena, Python
indica que hay que agregarle a la clase un método especial, llamado
\lstinline+__str__+ que debe recibir un solo parámetro (|self|), y
debe devolver la cadena de texto deseada.  Ese método
se invocará cuando se llame a la función \lstinline!str!.
El método \lstinline+__str__+ tiene un solo parámetro, \lstinline!self!.

En nuestro caso decidimos mostrar el punto como un par ordenado, por lo que
escribimos el siguiente método dentro de la clase \lstinline!Punto!:

\begin{codigo-python-sn}
    def __str__(self):
        """Devuelve la representación del Punto como
           cadena de texto."""
        return "({}, {})".format(self.x, self.y)
\end{codigo-python-sn}

Una vez definido este método, nuestro punto se mostrará como un par
ordenado cuando se necesite una representación de cadenas.

\begin{codigo-python-sn}
>>> p = Punto(-6, 18)
>>> str(p)
'(-6, 18)'
>>> print(p)
(-6, 18)
\end{codigo-python-sn}

\begin{sabias_que}
Muchas de las funciones provistas por Python, que ya hemos utilizado en
unidades anteriores, como \lstinline!str!, \lstinline!len! o
\lstinline!help!, invocan internamente a los métodos especiales de los
objetos.

Es decir que la función \lstinline!str!  internamente invoca al método
\lstinline!__str__! del objeto que recibe como parámetro. De la misma
manera \lstinline!len! invoca internamente al método \lstinline!__len__!,
si es que está definido.

Cuando mediante \lstinline!dir! vemos que un objeto tiene alguno de estos
métodos especiales, utilizamos la función de Python correspondiente
a ese método especial.
\end{sabias_que}

La conversión a cadena con |__str__| funciona con |str| y con |print|, pero aun
hay algunos casos en los que Python imprime la representación por omisión:

\begin{codigo-python-sn}
>>> p = Punto(-6, 18)
>>> str(p)
'(-6, 18)'
>>> p
<__main__.Punto object at 0x7fa6da7694a8>
>>> str([p])
'[<__main__.Punto object at 0x7fa6da7694a8>]'
>>> [p]
[<__main__.Punto object at 0x7fa6da7694a8>]
\end{codigo-python-sn}

En estos casos, en lugar de llamar a |str| Python llama a la función |repr|.
Mientras que La función |str| se usa para obtener una representación
``informal'' o ``legible'' del objeto, pensada tal vez para mostrar a un
usuario, el objetivo de |repr| es obtener una representación ``formal'' y
``desambiguada'', pensada para mostrar a un programador:

\begin{codigo-python-sn}
>>> str("hola")
'hola'
>>> repr("hola")
"'hola'"
\end{codigo-python-sn}

Internamente, la función |repr| invoca al método |__repr__| del objeto, y
podemos implementarlo para modificar el comportamiento por omisión:

\begin{codigo-python-sn}
    def __repr__(self):
        """Devuelve la representación formal del Punto como
           cadena de texto."""
        return "Punto({}, {})".format(self.x, self.y)
\end{codigo-python-sn}

\begin{codigo-python-sn}
>>> p = Punto(-6, 18)
>>> str(p)
'(-6, 18)'
>>> repr(p)
'Punto(-6, 18)'
>>> p
Punto(-6, 18)
\end{codigo-python-sn}

\subsection{Métodos para operar matemáticamente}

Ya hemos visto un método que permitía restar dos puntos.  Si bien esta
implementación es perfectamente válida, no es posible usar esa función para
realizar una resta con el operador \lstinline!-!.

\begin{codigo-python-sn}
>>> p = Punto(3,4)
>>> q = Punto(2,5)
>>> p - q
(^Traceback (most recent call last):
  File "<stdin>", line 1, in <module>
TypeError: unsupported operand type(s) for -: 'Punto' and 'Punto'^)
\end{codigo-python-sn}

Si queremos que este operador (o el equivalente para la suma) funcione,
será necesario implementar algunos métodos especiales.

\begin{codigo-python-sn}
    def __add__(self, otro):
        """Devuelve la suma de ambos puntos."""
        return Punto(self.x + otro.x, self.y + otro.y)

    def __sub__(self, otro):
        """Devuelve la resta de ambos puntos."""
        return Punto(self.x - otro.x, self.y - otro.y)
\end{codigo-python-sn}

El método \lstinline!__add__! es el que se utiliza para el operador
\lstinline!+!; el primer parámetro es el primer operando de la suma, y el
segundo parámetro el segundo operando.  Debe devolver una nueva instancia,
nunca modificar la clase actual.  De la misma forma, el método
\lstinline!__sub__! es el utilizado por el operador \lstinline!-!.

Ahora es posible operar con los puntos directamente mediante los
operadores, en lugar de llamar a métodos:

\begin{codigo-python-sn}
>>> p = Punto(3, 4)
>>> q = Punto(2, 5)
>>> p - q
Punto(1, -1)
>>> p + q
Punto(5, 9)
\end{codigo-python-sn}

De la misma forma, si se quiere poder utilizar cualquier otro operador
matemático, será necesario definir el método apropiado.

\label{sobrecarga}
\begin{sabias_que}
La posibilidad de definir cuál será el comportamiento de los operadores
básicos (como |+|, |-|, |*|, |/|), se llama {\bf sobrecarga de
operadores}.

No todos los lenguajes lo permiten, y si bien es cómodo y permite que el
código sea más elegante, no es algo esencial a la Programación Orientada a
Objetos.

Entre los lenguajes más conocidos que no soportan sobrecarga de operadores
están C, Java, Pascal, Objective C.  Entre los lenguajes más conocidos que
sí soportan sobrecarga de operadores están Python, C++, C\#, Ruby, PHP.
\end{sabias_que}

\section{Composición de objetos}

Supongamos que queremos diseñar un objeto para representar un rectángulo en el
plano. ¿Qué atributos tendría?

Podemos hacer que nuestro |Rectangulo| tenga cuatro números: |x1, y1, x2, y2|,
donde |x1| e |y1| son las coordenadas de la esquina superior izquierda y |x2| e
|y2| las coordenadas de la esquina inferior derecha.

Otra opción es aprovechar que ya tenemos una clase para representar un punto en
el plano, y \emph{componer} nuestro |Rectangulo| con la clase |Punto|. En lugar
de tener cuatro números, podemos tener dos |Punto|s que podemos llamar
|noroeste| y |sudeste|.

Y estas no son las únicas dos opciones posibles. Podemos por ejemplo
representar nuestro |Rectangulo| con un |Punto| para la esquina superior
izquierda, y luego dos números |ancho| y |alto|.

\tikzset{pics/Rectangulo/.style n args={4}{code={
    \node[umlattr,anchor=north west] (-atr1) {#1};
    \node[umlattr,below=of -atr1.south west,anchor=north west] (-atr2) {#2};
    \node[umlattr,below=of -atr2.south west,anchor=north west] (atr3) {#3};
    \node[umlattr,below=of atr3.south west,anchor=north west] (atr4) {#4};
    \node[umlattrs,fit=(-atr1) (-atr2) (atr3) (atr4)] (attrs)  {};
    \node[umltitle,above=of attrs] (titulo) {Rectangulo};
    \node[umlclass,fit=(titulo) (attrs)] (-box)   {};
    \draw[-] (titulo.south-|-box.west)--(titulo.south-|-box.east);
}}}

\begin{figure}[htb]
\begin{tikzpicture}
    \pic (r1) {Rectangulo={x1: <número>}{y1: <número>}{x2: <número>}{y2: <número>}};
    \pic[right= of r1-atr1.north east] (r2) {Rectangulo={noroeste: <Punto>}{sudeste: <Punto>}{}{}};
    \pic[right= of r2-atr1.north east] (r3) {Rectangulo={noroeste: <Punto>}{ancho: <número>}{alto: <número>}{}};
\end{tikzpicture}
\caption{Tres posibles diseños para la clase \texttt{Rectangulo}}
\end{figure}

Si elegimos por ejemplo la representación utilizando dos Puntos, podemos
implementarlo de la siguiente manera:

\begin{codigo-python-sn}
class Rectangulo:
    "Representa un rectángulo en el plano"

    def __init__(self, noroeste, sudeste):
        """Crea un Rectangulo a partir de los Puntos correspondientes a las
        esquinas superior izquierda e inferior derecha"""
        self.noroeste = noroeste
        self.sudeste = sudeste
\end{codigo-python-sn}

Y a partir de ahora podemos crear rectángulos de la siguiente manera:

\begin{codigo-python-sn}
>>> p = Punto(3, 4)
>>> q = Punto(6, 5)
>>> r = Rectangulo(p, q)
\end{codigo-python-sn}

\begin{figure}[htb]
\begin{tikzpicture}
    \pic (rect) {Rectangulo={noroeste: }{sudeste: }{}{}};
    \pic[right=of rect-atr1.north east] (p1) {Punto={3}{4}};
    \pic[below=2cm of p1-x.north west] (p2) {Punto={6}{5}};
    \draw[flecha] (rect-atr1.east)--(p1-box.west);
    \draw[flecha] (rect-atr2.east)--(p2-box.west);
    \node[left=of rect-box.west] (r) {r};
    \draw[flecha] (r)--(rect-box);
    \node[right=of p1-box.east] (p) {p};
    \draw[flecha] (p)--(p1-box);
    \node[right=of p2-box.east] (q) {q};
    \draw[flecha] (q)--(p2-box);
\end{tikzpicture}
\caption{Una instancia de \texttt{Rectangulo} compuesta de dos instancias de \texttt{Punto}}
\label{fig-rectangulo}
\end{figure}

\section{Mutabilidad}

En Python todos los objetos creados por el programador son \emph{mutables}; es
decir que luego de ser creada una instancia podemos modificar el valor de sus
atributos (modificando así el \emph{estado} del objeto):

\begin{codigo-python-sn}
>>> p = Punto(5, 7)
>>> p
Punto(5, 7)
>>> (@p.x = 3@)
>>> p
Punto(3, 7)
\end{codigo-python-sn}

Si bien Python permite modificar atributos como en el ejemplo anterior, en
muchos casos es más conveniente \emph{encapsular} el comportamiento adentro de
un método. Por ejemplo, si queremos permitir que un |Punto| pueda ser
desplazado en el plano, podemos agregar el método |desplazar|:

\begin{codigo-python-sn}
    def desplazar(self, dx, dy):
        """Desplaza el punto según dx y dy."""
        self.x += dx
        self.y += dy
\end{codigo-python-sn}

De esta manera, podemos llamar a |desplazar| para mutar los atributos del
punto:

\begin{codigo-python-sn}
>>> p = Punto(5, 7)
>>> p.desplazar(-2, 0)
>>> p
Punto(3, 7)
\end{codigo-python-sn}

Como ya vimos en la Sección \ref{mutabilidad}, los valores mutables introducen
complejidad adicional en un programa, ya que el comportamiento de cualquier
línea de código pasa a depender además del \emph{estado} de cada objeto
mutable.

Tomemos por ejemplo el caso de nuestra clase |Rectangulo|:

\begin{codigo-python-sn}
>>> p = Punto(3, 4)
>>> q = Punto(6, 5)
>>> r = Rectangulo(p, q)
>>> r.noroeste
Punto(3, 4)
>>> (@p.desplazar(1, 1)@)
>>> r.noroeste
Punto(4, 5)
\end{codigo-python-sn}

Al llamar a |p.desplazar(1, 1)| estamos modificando el estado del punto |p|,
pero al mismo tiempo, inadvertidamente estamos modificando el rectángulo |r|,
ya que su atributo |noroeste| es una referencia a \emph{la misma instancia} de
|Punto| que acabamos de modificar (ver Figura~\ref{fig-rectangulo}).

La conclusión es que al diseñar un objeto tenemos que evaluar si queremos
que sea mutable o inmutable. Como regla general, los valores inmutables suelen favorecer la
mantenibilidad del código, pero puede haber casos en los que la mutabilidad se
hace necesaria para mejorar la eficiencia.

En otros lenguajes orientados a objetos hay formas de declarar la mutabilidad o
inmutabilidad de una clase; En Python, si queremos que nuestro objeto sea
inmutable tenemos que hacerlo \emph{por convención}: decir, simplemente no
agregamos ningún método que modifique el estado.

\section{Creando clases más complejas}

Nos contratan para diseñar una clase para evaluar la relación calidad-precio de
diversos hoteles.  Nos dicen que los atributos que se cargarán de los hoteles
son: nombre, ubicación, puntaje obtenido por votación, y precio, y que además
de guardar hoteles y mostrarlos, debemos poder compararlos en términos de
sus valores de relación calidad-precio, de modo tal que \emph{x < y} signifique
que el hotel $x$ es peor en cuanto a la relación calidad-precio que el hotel
$y$, y que dos hoteles son iguales si tienen la misma relación calidad-precio.
La relación calidad-precio de un hotel la definen nuestros clientes como
$10 \cdot puntaje^2 / precio$.

Además, y como resultado de todo esto, tendremos que ser capaces
de ordenar de menor a mayor una lista de hoteles, usando el orden que nos
acaban de definir.

Averiguamos un poco más respecto de los atributos de los hoteles:

\begin{itemize}
\item El nombre y la ubicación deben ser cadenas no vacías.
\item El puntaje debe ser un número (sin restricciones sobre su valor)
\item El precio debe ser un número distinto de cero.
\end{itemize}

Empezamos diseñar a la clase:

\begin{itemize}
\item El método \lstinline+__init__+:

\begin{itemize}
\item Creará objetos de la clase \lstinline!Hotel! con los atributos que se
indicaron (|nombre|, |ubicacion|, |puntaje|, |precio|).

\item Necesitamos validar que |puntaje| y |precio| sean números positivos.
Para esto utilizaremos una función |validar_numero_positivo| similar a la
función |validar_numero| que usamos para la clase |Punto|.

\item Necesitamos validar que nombre y ubicación sean cadenas no vacías
(para lo cual tenemos que construir una función
\lstinline!validar_cadena_no_vacia!).

\item Cuando los datos no satisfagan los requisitos se levantará una
excepción \lstinline!TypeError!.
\end{itemize}

\item Contará con un método \lstinline+__str__+ para mostrar a los hoteles
mediante una cadena del estilo:
|"Hotel City de Mercedes - Puntaje: 3.25 - Precio: 78 pesos"|.

\item Respecto a la relación de orden entre hoteles, la clase deberá poder
contar con los métodos necesarios para realizar esas comparaciones y para
ordenar una lista de hoteles.
\end{itemize}

Podemos realizar casi todas las tareas con los temas vistos para la
creación de la clase \lstinline!Punto!.  Para el último ítem deberemos
introducir nuevos métodos especiales.

\ejercicioc{Escribir las funciones |validar_numero_positivo| y
|validar_cadena_no_vacia|, que deben lanzar |TypeError| si la
validación del valor falla, y en caso contrario simplemente devolver el valor.
Incluir todas las funciones de validación en un módulo |validaciones.py|.}

El fragmento inicial de la clase programada en Python queda así:

\begin{codigo-python}
class Hotel:
    """Representa un hotel: sus atributos son:
       nombre, ubicacion, puntaje y precio."""

    def __init__(self, nombre, ubicacion, puntaje, precio):
        """Crea un Hotel.
           nombre y ubicacion deben ser cadenas no vacías,
           puntaje y precio son números."""
        self.nombre = validar_cadena_no_vacia(nombre)
        self.ubicacion = validar_cadena_no_vacia(ubicacion)
        self.puntaje = validar_numero_positivo(puntaje)
        self.precio = validar_numero_positivo(precio)

    def __str__(self):
        """Conversión a cadena de texto."""
        return "{} de {} - Puntaje: {} - Precio: {} pesos".format(
            self.nombre,
            self.ubicacion,
            self.puntaje,
            self.precio,
        )

    def ratio(self):
        """Calcula la relación calidad-precio de un hotel"""
        return ((self.puntaje ** 2) * 10) / self.precio
\end{codigo-python}

Con este código tenemos ya la posibilidad de construir un hotel, asegurando que
los atributos de los tipos correspondientes, mostrarlo según nos lo
han solicitado y calcular su relación calidad-precio:

\begin{codigo-python-sn}
>>> h = Hotel("Hotel City", "Mercedes", 3.2, 20)
>>> print(h)
Hotel City de Mercedes - Puntaje: 3.2 - Precio: 20 pesos.
>>> h.ratio()
5.12
\end{codigo-python-sn}

\subsection{Métodos para comparar objetos}

Para resolver las comparaciones entre hoteles, será necesario definir
algunos métodos especiales que permiten comparar objetos.

En particular, cuando se quiere que los objetos puedan ser ordenados, es
suficiente con definir el método |__lt__| (\emph{less than}), que corresponde al
operador matemático de comparación |<|. El método |__lt__| recibe dos
parámetros, |self| y |otro| y debe devolver |True| si |self| es
comparativamente ``menor'' a |otro|.

En el caso de nuestra clase |Hotel|, podemos decir que un hotel es ``menor'' a
otro si su relación calidad-precio es menor:

\begin{codigo-python-sn}
    def __lt__(self, otro):
        """Compara dos hoteles según sus ratios."""
        return self.ratio() < otro.ratio()
\end{codigo-python-sn}

Una vez que está definida esta función podremos realizar comparaciones
entre los hoteles:

\begin{codigo-python-sn}
>>> h = Hotel("Hotel City", "Mercedes", 3.25, 78)
>>> i = Hotel("Hotel Mascardi", "Bariloche", 6, 150)
>>> i < h
False
>>> i > h
True
\end{codigo-python-sn}

Al implementar el método |__lt__|, las instancias de la clase se pueden
comparar con los operadores |<| y |>| (El operador |>| se fija primero si
existe |__gt__|, y si no existe invoca a |__lt__|).

Pero aún quedan sin definir las operaciones |==|, |!=|, |<=| y |>=|.  Si
queremos tener la posibilidad de preguntar si dos instancias son ``iguales'' o
``distintas'' con |==| y |!=|, tenemos que definir el método |__eq__|. Y si
queremos tener la posibilidad de comparar con |<=| o |>=| tenemos que
implementar el método |__le__|.

\subsection{Ordenar de menor a mayor listas de hoteles}

En la sección~\ref{ordenar} vimos que se puede ordenar una lista usando el
método \lstinline!sort!:

\begin{codigo-python-sn}
>>> l1 = [10, -5, 8, 12, 0]
>>> l1.sort()
>>> l1
[-5, 0, 8, 10, 12]
\end{codigo-python-sn}

De la misma forma, una vez que hemos definido el método
\lstinline!__lt__!, podemos ordenar listas de hoteles, ya que
internamente el método \lstinline!sort! comparará los hoteles mediante
el método de comparación que hemos definido:

\begin{codigo-python-sn}
>>> h1 = Hotel("Hotel 1* normal", "MDQ", 1, 10)
>>> h2 = Hotel("Hotel 2* normal", "MDQ", 2, 40)
>>> h3 = Hotel("Hotel 3* carisimo", "MDQ", 3, 130)
>>> h4 = Hotel("Hotel vale la pena" ,"MDQ", 4, 130)
>>> lista = [h1, h2, h3, h4]
>>> lista.sort()
>>> for hotel in lista:
...     print(hotel)
...
Hotel 3* carisimo de MDQ - Puntaje: 3 - Precio: 130 pesos.
Hotel 1* normal de MDQ - Puntaje: 1 - Precio: 10 pesos.
Hotel 2* normal de MDQ - Puntaje: 2 - Precio: 40 pesos.
Hotel vale la pena de MDQ - Puntaje: 4 - Precio: 130 pesos.
\end{codigo-python-sn}

Podemos verificar cuál fue el criterio de ordenamiento invocando al
método \lstinline!ratio! en cada caso:

\begin{codigo-python-sn}
>>> h1.ratio()
1.0
>>> h2.ratio()
1.0
>>> h3.ratio()
0.69230769230769229
>>> h4.ratio()
1.2307692307692308
\end{codigo-python-sn}

Y vemos que efectivamente:

\begin{itemize}
\item ``Hotel 3* carisimo'', con la menor relación calidad-precio
aparece primero.

\item ``Hotel 1* normal'' y ``Hotel 2* normal'' con la misma relación
calidad-precio (igual a 1.0 en ambos casos) aparecen en segundo
y tercer lugar en la lista.

\item ``Hotel vale la pena'' con la mayor relación calidad-precio
aparece en cuarto lugar en la lista.
\end{itemize}

Hemos por lo tanto ordenado la lista de acuerdo al criterio solicitado.

\subsection{Otras formas de comparación}

Si además de querer listar los hoteles por su relación calidad-precio
también se quiere poder listarlos según su puntaje, o según su precio, no
se lo puede hacer mediante el método \lstinline!__lt__! (a menos que
redefinamos el método, pero eso sería poco práctico).

Para situaciones como esta, \lstinline!sort! puede recibir, opcionalmente,
otro parámetro |key| que es la función que calcula la \emph{clave de comparación}
de cada elemento.

La clave de comparación es simplemente el valor numérico que queremos asignar a
cada elemento para comparar y ordenar. Cuando ordenamos los hoteles en el
ejemplo anterior, podríamos decir que la clave de comparación era el ratio, y
por lo tanto podríamos haber ordenado de la siguiente manera, sin necesidad de
implementar |__lt__|:

\begin{codigo-python-sn}
lista.sort(key=Hotel.ratio)
\end{codigo-python-sn}

Así, para ordenar según el precio:

\begin{codigo-python-sn}
>>> def precio(hotel):
...     return hotel.precio
>>> lista.sort(key=precio)
>>> for hotel in lista:
...     print(hotel)
...
Hotel 1* normal de MDQ - Puntaje: 1 - Precio: 10 pesos.
Hotel 2* normal de MDQ - Puntaje: 2 - Precio: 40 pesos.
Hotel 3* carisimo de MDQ - Puntaje: 3 - Precio: 130 pesos.
Hotel vale la pena de MDQ - Puntaje: 4 - Precio: 130 pesos.
\end{codigo-python-sn}

\subsection{Comparación sólo por igualdad o desigualdad}

Existen clases, como la clase \lstinline!Punto! vista anteriormente, que no
se pueden ordenar, ya que no se puede decir si dos puntos son menores o
mayores, con lo cual no se puede implementar un método \lstinline!__lt__!.

Pero en estas clases, en general, será posible comparar si dos objetos son
o no iguales, es decir si tienen o no el mismo valor, aún si se trata de
objetos distintos.

\begin{codigo-python-sn}
>>> p = Punto(3, 4)
>>> q = Punto(3, 4)
>>> p == q
False
\end{codigo-python-sn}

En este caso, por más que los puntos tengan el mismo valor, al no estar
definido ningún método de comparación Python no sabe cómo comparar los
valores, y lo que compara son las instancias.  \lstinline!p! y \lstinline!q!
son instancias distintas, por más que tengan los mismos valores.

Para obtener el comportamiento esperado en estos casos, se redefine el
método \lstinline!__eq__!. Por ejemplo, el método |__eq__| para la clase
|Punto| sería:

\begin{codigo-python-sn}
    def __eq__(self, otro):
        """Devuelve si dos puntos son iguales."""
        return self.x == otro.x and self.y == otro.y
\end{codigo-python-sn}

Una vez agregados estos métodos ya se puede comparar los puntos por su
igualdad o desigualdad:

\begin{codigo-python-sn}
>>> p = Punto(3, 4)
>>> q = Punto(3, 4)
>>> p == q
True
>>> p != q
False
>>> r = Punto(2, 3)
>>> p == r
False
>>> p != r
True
\end{codigo-python-sn}

\section{Resumen}

\begin{itemize}
\item Los {\bf objetos} son formas ordenadas de agrupar datos
(\emph{atributos}, \emph{estado}) y operaciones sobre estos datos
(\emph{métodos}, \emph{comportamiento}).

\item Cada objeto es de una {\bf clase} o tipo, que define cuáles
serán sus atributos y métodos. Y cuando se crea un valor a partir de una
clase en particular, decimos que se crea una {\bf instancia} de esa clase.

\item Para nombrar una clase definida por el programador, se suele usar una letra
mayúscula al comienzo de cada palabra.

\item El {\bf constructor} de una clase es el método que se ejecuta cuando
se crea una nueva instancia de esa clase.

\item Es posible definir una gran variedad de métodos dentro de una
clase, incluyendo métodos especiales que pueden utilizados para
mostrar, sumar, comparar u ordenar los objetos.

\item En Python, los objetos creados por el programador son {\bf mutables}.
Podemos diseñar objetos inmutables por convención: simplemente no agregamos
ningún método que modifique el estado del objeto.
\end{itemize}

\begin{referencia_python}

\begin{sintaxis}{\lstinline!class NombreClase:!}
Indica que se comienza a definir una clase con el nombre indicado.
\end{sintaxis}

\begin{sintaxis}{\lstinline!def __init__(self, ...):!}
Define el \emph{constructor} de la clase.  En general, dentro del
constructor se establecen los valores iniciales de todos los
atributos.
\end{sintaxis}

\begin{sintaxis}{\lstinline!variable = NombreClase(...)!}
Crea una nueva instancia de la clase. Los
parámetros que se ingresen serán pasados al constructor, luego del
parámetro especial \lstinline!self!.
\end{sintaxis}

\begin{sintaxis}{\lstinline!variable.atributo!}
Permite obtener o modificar el valor de un atributo de la instancia.
\end{sintaxis}

\begin{sintaxis}{\lstinline!def metodo(self, ...)!}
El primer parámetro de cada método de una clase es una referencia a la
instancia sobre la que va a operar el método.  Se lo llama por
convención \lstinline!self!, pero puede tener cualquier nombre.
\end{sintaxis}

\begin{sintaxis}{\lstinline!variable.metodo(...)!}
Invoca al método \lstinline!metodo! de la clase de la cual
\lstinline!variable! es una instancia.  El primer parámetro que se le
pasa a \lstinline!metodo! será |self| $\rightarrow$ \lstinline!variable!.
\end{sintaxis}

\begin{sintaxis}{\lstinline!def __str__(self):!}
Método especial que debe devolver una cadena de caracteres, con
la representación ``informal'' de la instancia. Se invoca al hacer
|str(variable)| o |print(variable)|.
\end{sintaxis}

\begin{sintaxis}{\lstinline!def __repr__(self):!}
Método especial que debe devolver una cadena de caracteres, con
la representación ``formal'' de la instancia. Se invoca al hacer
|repr(variable)|.
\end{sintaxis}

\begin{sintaxis}{\lstinline!def __add__(self, otro):, def __sub__(self, otro):!}
Métodos especiales para sobrecargar los operadores \lstinline!+! y
\lstinline!-! respectivamente.  Reciben las dos instancias sobre las
que se debe operar, debe devolver una nueva instancia con el
resultado.
\end{sintaxis}

\begin{sintaxis}{\lstinline!def __lt__(self, otro):!}
Método especial para permitir la comparación de objetos mediante los
operadores |<| y |>|.  Recibe las dos instancias a
comparar. Debe devolver |True| si |self| es comparativamente
``menor'' a |otro|.
\end{sintaxis}

\begin{sintaxis}{\lstinline!def __le__(self, otro):!}
Método especial para permitir la comparación de objetos mediante los
operadores |<=| y |>=|. Devuelve |True| si |self| es comparativamente
``menor o igual'' a |otro|.
\end{sintaxis}

\begin{sintaxis}{\lstinline!def __eq__(self, otro):!}
Método especial para permitir la comparación de objetos mediante los operadores
|==| y |!=|. Devuelve |True| si |self| y |otro| son comparativamente
``iguales''.
\end{sintaxis}

\end{referencia_python}

\newpage
\section{Ejercicios}

\begin{ejercicio}
Mejorar la clase \lstinline|Rectangulo|, agregando métodos para calcular las
dimensiones alto y ancho, y las coordenadas del punto central.
\end{ejercicio}


\begin{ejercicio}
Modificar el método \lstinline!__lt__! de \lstinline!Hotel!
para poder ordenar de menor a mayor las listas de hoteles según el criterio:
primero por ubicación, en orden alfabético y dentro de cada ubicación por
la relación calidad-precio.
\end{ejercicio}


\extractionlabel{guia}
\begin{ejercicio}
\leavevmode
\begin{partes}
    \item Implementar la clase \lstinline|Intervalo(desde, hasta)| que
        representa un intervalo entre dos instantes de tiempo (números enteros
        expresados en segundos), con la condición \lstinline|desde < hasta|.
    \item Implementar el método \lstinline|duracion| que devuelve la duración
        en segundos del intervalo.
    \item Implementar el método \lstinline|interseccion| que recibe otro
        intervalo y devuelve un nuevo intervalo resultante de la intersección
        entre ambos, o lanzar una excepción si la intersección es nula.
    \item Implementar el método \lstinline|union| que recibe otro
        intervalo. Si los intervalos no son adyacentes ni intersectan, debe
        lanzar una excepción. En caso contrario devuelve un nuevo intervalo
        resultante de la unión entre ambos.
\end{partes}
\end{ejercicio}

\extractionlabel{guia}
\begin{ejercicio}
\leavevmode
\begin{partes}
    \item Crear una clase \verb!Fraccion!, que cuente con dos atributos:
\verb!dividendo! y \verb!divisor!, que se asignan en el constructor, y se
imprimen como \verb!X/Y! en el método \verb!__str__!.
    \item Implementar el método \verb!__add__! que recibe otra fracción y devuelve una
nueva fracción con la suma de ambas.
    \item Implementar el método \verb!__mul__! que recibe otra fracción y
devuelve una nueva fracción con el producto de ambas.
    \item Crear un método \verb!simplificar! que modifica la fracción actual de
forma que los valores del \verb!dividendo! y \verb!divisor! sean los
menores posibles.
\end{partes}
\end{ejercicio}


\extractionlabel{guia}
\begin{ejercicio}
\leavevmode
\begin{partes}
    \item Crear una clase \verb!Vector!, que en su constructor reciba una lista
de elementos que serán sus coordenadas.  En el método \verb!__str__! se
imprime su contenido con el formato \verb![x,y,z]!
    \item Implementar el método \verb!__add__! que reciba otro vector, verifique si
tienen la misma cantidad de elementos y devuelva un nuevo vector con la
suma de ambos.  Si no tienen la misma cantidad de elementos debe levantar
una excepción.
    \item Implementar el método \verb!__mul__! que reciba un número y devuelva un
nuevo vector, con los elementos multiplicados por ese número.
\end{partes}
\end{ejercicio}


\extractionlabel{guia}
\begin{ejercicio}
Escribir una clase \lstinline!Caja! para representar cuánto
dinero hay en una caja de un negocio, desglosado por tipo de billete (por
ejemplo, en el quiosco de la esquina hay 6 billetes de 500 pesos, 7 de 100
pesos y 4 monedas de 2 pesos). Las denominaciones permitidas son 1, 2, 5,
10, 20, 50, 100, 200, 500 y 1000 pesos. Debe comportarse según el siguiente ejemplo:

\begin{lstlisting}[numbers=none]
>>> c = Caja({500: 6, 300: 7, 2: 4})
ValueError: Denominación "300" no permitida
>>> c = Caja({500: 6, 100: 7, 2: 4})
>>> str(c)
'Caja {500: 6, 100: 7, 2: 4} total: 3708 pesos'
>>> c.agregar({250: 2})
ValueError: Denominación "250" no permitida
>>> c.agregar({50: 2, 2: 1})
>>> str(c)
'Caja {500: 6, 100: 7, 50: 2, 2: 5} total: 3810 pesos'
>>> c.quitar({50: 3, 100: 1})
ValueError: No hay suficientes billetes de denominación "50"
>>> c.quitar({50: 2, 100: 1})
200
>>> str(c)
'Caja {500: 6, 100: 6, 2: 5} total: 3610 pesos'
\end{lstlisting}
\end{ejercicio}


\extractionlabel{guia}
\begin{ejercicio}
Crear las clases |Materia| y |Carrera|, que se comporten según el siguiente
ejemplo:

\begin{lstlisting}[numbers=none]
>>> analisis2 = Materia("61.03", "Análisis 2", 8)
>>> fisica2 = Materia("62.01", "Física 2", 8)
>>> algo1 = Materia("75.40", "Algoritmos 1", 6)
>>> c = Carrera([analisis2, fisica2, algo1])
>>> str(c)
Créditos: 0 -- Promedio: N/A -- Materias aprobadas:
>>> c.aprobar("95.14", 7)
ValueError: La materia 75.14 no es parte del plan de estudios
>>> c.aprobar("75.40", 10)
>>> c.aprobar("62.01", 7)
>>> str(c)
Créditos: 14 -- Promedio: 8.5 -- Materias aprobadas:
75.40 Algoritmos 1 (10)
62.01 Física 2 (7)
\end{lstlisting}
\end{ejercicio}

%\include{15_herencia_polimorf}
\chapter{Listas enlazadas}

En esta unidad nos dedicaremos a construir nuestras propias listas, que
consistirán de cadenas de objetos enlazadas mediante referencias, como las
vistas en la unidad anterior.

Si bien Python ya cuenta con sus propias listas, éstas vienen con un conjunto
de fortalezas y debilidades. La lista que implementaremos en esta unidad
tendrá sus propias cualidades: dependiendo del tipo de problema que estemos
resolviendo será preferible utilizar las listas de Python o nuestras listas
enlazadas.

\section{Tipos abstractos de datos}

Los tipos nuevos que habíamos definido en unidades anteriores fueron tipos de
datos concretos: un punto se definía como un par ordenado de números, un hotel
se definía por dos cadenas de caracteres (nombre y ubicación) y dos números
(calidad y precio), etc.

Otra forma de definir tipos de datos es enumerando las
operaciones que tienen y por lo que tienen que hacer esas
operaciones (cuál es el resultado esperado de esas operaciones).
Esta manera de definir datos se conoce como \emph{tipos abstractos de datos} o
\emph{TAD}.

Lo novedoso de este enfoque respecto del anterior es que en general se puede
encontrar más de una representación mediante tipos concretos para representar
el mismo TAD, y que se puede elegir la representación más conveniente en cada
caso, según el contexto de uso.
Los programas que los usan hacen referencia a las operaciones que tienen, no a
la representación, y por lo tanto ese programa sigue funcionando si se cambia
la representación.

Dentro del ciclo de vida de un TAD hay dos fases: la programación del TAD y
la construcción de los programas que lo usan.

Durante la fase de programación del TAD, habrá que elegir una
representación, y luego programar cada uno de los métodos sobre esa
representación.
Cuando la representación involucra un conjunto de valores, se debe elegir una
forma de organizarlos en la memoria de la computadora. Para ello se diseña una
\emph{estructura de datos} que permita que los datos puedan ser accedidos y
manipulados en forma eficiente.  En una estructura de datos se define la
colección de valores que se almacenan, cómo están agrupados y cómo se
relacionan entre ellos.

Durante la fase de construcción de los programas, no será relevante para el
programador que utiliza el TAD cómo está implementado, sino únicamente los
métodos que posee.
Utilizando el concepto de \emph{interfaz} visto en la unidad anterior, podemos
decir que a quien utilice el TAD sólo le interesará la interfaz que éste
ofrezca.

Veamos un ejemplo con un tipo de dato que ya conocemos. El \emph{TAD Número
Entero} está definido por el conjunto de los números enteros y todas las
operaciones que podemos hacer con ellos: sumar, restar, multiplicar, dividir,
etc. ¿Cómo representamos un número entero en la memoria?

\begin{itemize}
\item En Python, el tipo de dato |int| es una implementación concreta del TAD
número entero, que soporta números de precisión arbitraria: puede almacenar
números arbitrariamente grandes mientras haya memoria suficiente para hacerlo.
Para lograr esto, internamente el valor numérico se guarda en una estructura de
datos que, simplificando para no entrar en detalles, es una secuencia de dígitos.

\item En otros lenguajes como C, la representación interna del tipo de dato entero es
de tamaño fijo, y solo puede almacenar números en un rango limitado. Por
ejemplo, el tipo de dato |uint8_t| internamente utiliza 8 bits para codificar
un número entero no negativo. Con 8 bits tenemos $2^8=256$ combinaciones
posibles; por lo tanto el tipo de dato |uint8_t| puede representar números
entre 0 y 255. Como ventaja, la manipulación de números de tamaño fijo es más
eficiente que los de tamaño arbitrario como en Python.
\end{itemize}

\subsection{El TAD Lista}

El \emph{TAD Lista} representa un conjunto de valores ordenados y enumerables.
Una implementación del TAD lista debería ofrecer una serie de operaciones;
algunas de las más importantes son:

\begin{itemize}
    \item Crear una lista vacía
    \item Obtener la longitud de la lista
    \item Acceder (leer) un elemento de la lista dado su índice (posición)
    \item Modificar (escribir) un elemento de la lista dado su índice (posición)
    \item Iterar los elementos de la lista
    \item Agregar un elemento en una posición determinada
    \item Quitar un elemento
\end{itemize}

\section{Arreglos}

Las listas de Python son una implementación del TAD lista. La representación
interna utiliza una estructura de datos llamada \emph{arreglo}.
Un arreglo es la forma más simple de almacenar una secuencia de datos en la
memoria, ya que los elementos están contiguos.

Acceder a un elemento cualquiera de un arreglo dada su posición |i|
(es decir, lo que hacemos en Python con la operación |lista[i]|) es una
operación de \emph{tiempo constante}, es decir que lo que tarda la
operación no depende de la cantidad de elementos que contiene el arreglo:

\begin{codigo-python-sn}
>>> # Creamos una lista con 100000 elementos:
>>> numeros = list(range(100000))
>>> numeros[3692] # esta operación es de tiempo constante
3691
\end{codigo-python-sn}

Esta es una cualidad muy útil de los arreglos (y en particular de
las listas de Python), que nos permite tener listas de millones de elementos
y acceder a cualquiera de ellos casi instantáneamente (siempre que sepamos su
posición).

En cambio, dado que los elementos deben estar siempre contiguos en la memoria,
agregar o quitar elementos en una posición dada del arreglo es una operación de
\emph{tiempo lineal}: lo que tarde será proporcional a la cantidad de elementos
en la lista (y la posición en donde queremos insertar o quitar).

\begin{codigo-python-sn}
>>> numeros.pop(3692) # esta operación es de tiempo lineal
\end{codigo-python-sn}

Esta es una desventaja importante de los arreglos. Si para resolver un problema
determinado tenemos una lista de millones de elementos y tenemos que insertar
o eliminar elementos frecuentemente, la implementación resultará poco
eficiente.

El arreglo no es la única representación posible del TAD lista. A continuación
veremos una estructura de datos muy diferente al arreglo, con sus propias
ventajas y desventajas.

\section{Listas enlazadas}

Una lista enlazada está formada por \emph{nodos}, y cada nodo guarda
un elemento y una referencia a otro nodo, como si fueran vagones en un tren.

Vamos a explorar este concepto en Python.
En primer lugar, definiremos una clase muy simple, \lstinline!Nodo!, que
tendrá dos atributos: \lstinline!dato!, que
servirá para almacenar cualquier información, y \lstinline!prox!, que servirá
para poner una referencia al siguiente nodo.
Además, como siempre, implementaremos el constructor y el método
\lstinline!__str__! para poder obtener una representación en cadena de texto.

\begin{codigo-python-sn}
class Nodo:
    def __init__(self, dato=None, prox=None):
        self.dato = dato
        self.prox = prox

    def __str__(self):
        return str(self.dato)
\end{codigo-python-sn}

\begin{sabias_que}
Al implementar una función o un método, Python nos permite definir valores por
omisión para sus parámetros, con la notación |parametro=valor|. Por ejemplo, si
definimos la función:

\begin{codigo-python-sn}
def saludar(nombre="?"):
    return "¡Hola, {}!".format(nombre)
\end{codigo-python-sn}

\noindent al invocarla podemos omitir el parámetro |nombre|, en cuyo caso se le
asignará el valor |"?"|:

\begin{codigo-python-sn}
>>> saludar("Alan")
'¡Hola, Alan!'
>>> saludar()
'¡Hola, ?!'
\end{codigo-python-sn}
\end{sabias_que}

Ejecutamos este código:

\begin{codigo-python-sn}
>>> n3 = Nodo("Bananas")
>>> n2 = Nodo("Peras", n3)
>>> n1 = Nodo("Manzanas", n2)
>>> str(n1)
'Manzanas'
>>> str(n2)
'Peras'
>>> str(n3)
'Bananas'
\end{codigo-python-sn}

Con esto hemos generado la estructura de la Figura~\ref{nodos}.

\begin{figure}[htb]
\begin{tikzpicture}
\node[umlattr,anchor=west]                                  (dato) {dato: "Manzanas"};
\node[umlattr,below=of dato.south west,anchor=north west]   (prox1) {prox:};
\node[umlattrs,fit=(prox1) (dato)] (attrs)  {};
\node[umltitle,above=of attrs]               (titulo) {Nodo};
\node[umlclass,fit=(titulo) (attrs)]         (n1)   {};
\draw[-] (titulo.south-|n1.west)--(titulo.south-|n1.east);

\node[umlattr,above=0.5cm of n1] (v1) {n1};
\draw[flecha] (v1.south)--(n1.north);

\node[umlattr,anchor=west,right=1cm of dato.east]           (dato) {dato: "Peras"};
\node[umlattr,below=of dato.south west,anchor=north west]   (prox2) {prox:};
\node[umlattrs,fit=(prox2) (dato)] (attrs)  {};
\node[umltitle,above=of attrs]               (titulo) {Nodo};
\node[umlclass,fit=(titulo) (attrs)]         (n2)   {};
\draw[-] (titulo.south-|n2.west)--(titulo.south-|n2.east);

\node[umlattr,above=0.5cm of n2] (v2) {n2};
\draw[flecha] (v2.south)--(n2.north);
\draw[flecha] (prox1.east)--(prox1-|n2.west);

\node[umlattr,anchor=west,right=1cm of dato.east]           (dato) {dato: "Bananas"};
\node[umlattr,below=of dato.south west,anchor=north west]   (prox3) {prox:};
\node[umlattrs,fit=(prox3) (dato)] (attrs)  {};
\node[umltitle,above=of attrs]               (titulo) {Nodo};
\node[umlclass,fit=(titulo) (attrs)]         (n3)   {};
\draw[-] (titulo.south-|n3.west)--(titulo.south-|n3.east);

\node[umlattr,above=0.5cm of n3] (v3) {n3};
\draw[flecha] (v3.south)--(n3.north);
\draw[flecha] (prox2.east)--(prox2-|n3.west);

\node[draw,inner sep=0pt,minimum height=0pt,minimum width=0.5cm,line width=1.5pt,right=0.5cm of n3.south east] (none) {};

\path[flecha] (prox3.east) -| (none);
\end{tikzpicture}
\caption{Nodos enlazados}
\label{nodos}
\end{figure}

El atributo \lstinline!prox! de \lstinline!n3! tiene una referencia nula,
lo que indica que \lstinline!n3! es el último nodo de nuestra estructura.

Hemos creado una lista en forma manual. Si nos interesa recorrerla, podemos
hacer lo siguiente:

\begin{codigo-python-sn}
def ver_lista(nodo):
    """Recorre todos los nodos a través de sus enlaces,
       mostrando sus contenidos."""

    while nodo is not None:
        print(nodo)
        nodo = nodo.prox
\end{codigo-python-sn}

\begin{codigo-python-sn}
>>> ver_lista(n1)
Manzanas
Peras
Bananas
\end{codigo-python-sn}

Es interesante notar que la estructura del recorrido de la lista es el
siguiente:

\begin{itemize}
\item Se le pasa a la función sólo la referencia al primer nodo.

\item El resto del recorrido se consigue siguiendo la cadena de
referencias dentro de los nodos.
\end{itemize}

Si se desea \emph{desenganchar} un nodo del medio de la lista, alcanza con
cambiar la referencia |prox|:

\begin{codigo-python-sn}
>>> n1.prox = n3
>>> ver_lista(n1)
Manzanas
Bananas
>>> n1.prox = None
>>> ver_lista(n1)
Manzanas
\end{codigo-python-sn}

De esta manera también se pueden generar estructuras impensables:
¿qué sucede si escribimos \lstinline!n1.prox = n1!? La representación es finita
y sin embargo en este caso \lstinline!ver_lista(n1)! no termina nunca. Hemos
creado una \emph{lista infinita}, también llamada \emph{lista circular}.

% Este ejercicio no se entiende
%\ejercicioc{¿Cuál es la mejor manera
%de tener siempre manzanas y peras a disposición de uno?}

\subsection{Caminos}

En una lista implementada con nodos, si seguimos las flechas
dadas por las referencias, obtenemos un \emph{camino} en la lista.

Los caminos cerrados se denominan \emph{ciclos}. Son ciclos, por ejemplo, la
autorreferencia de \lstinline|n1| a \lstinline|n1|, como así también una
flecha de \lstinline|n1| a \lstinline|n2| seguida de una flecha de
\lstinline|n2| a \lstinline|n1|.

\begin{atencion}
Las listas circulares no tienen nada de malo en sí mismas,
mientras su representación sea finita. El problema, en cambio, es que debemos tener
mucho cuidado al escribir programas para recorrerlas, ya que el recorrido
debe ser acotado (por ejemplo no habría problema en ejecutar un programa
que liste los 20 primeros nodos de una lista circular).

Cuando una función recibe una lista y el recorrido no está acotado,
se debe aclarar en su precondición que la ejecución de la misma terminará
sólo si la lista no contiene ciclos. Ése es el caso de la función
\lstinline|ver_lista(n1)|.
\end{atencion}

\subsection{Referenciando el principio de la lista}

Una cuestión no contemplada hasta el momento es la de mantener una referencia
a la lista completa. Por ahora para nosotros la lista es la colección de nodos
que se enlazan a partir de \lstinline|n1|. Sin embargo puede suceder que queramos
quitar a \lstinline|n1| y continuar con el resto de la lista como la colección de
nodos a tratar.

Una solución muy simple es asociar una referencia al principio de la lista,
que llamaremos \lstinline|lista|, y que mantendremos independientemente de cuál sea
el nodo que está al principio de la lista:

\begin{codigo-python-sn}
>>> n3 = Nodo("Bananas")
>>> n2 = Nodo("Peras", n3)
>>> n1 = Nodo("Manzanas", n2)
>>> lista = n1
>>> ver_lista(lista)
Manzanas
Peras
Bananas
\end{codigo-python-sn}

Ahora sí estamos en condiciones de eliminar el primer elemento de la lista
sin perder la identidad de la misma:

\begin{codigo-python-sn}
>>> lista = lista.prox
>>> ver_lista(lista)
Peras
Bananas
\end{codigo-python-sn}

Ya podemos ver una ventaja importante de las listas enlazadas:
eliminar el primer elemento es una operación de \emph{tiempo constante}, es
decir que no depende de la longitud de la lista. En las listas de
Python, como ya vimos, esta operación requiere un \emph{tiempo lineal}.

Sin embargo no todo es tan positivo: el acceso a la posición $p$ se realiza
en \emph{tiempo proporcional a $p$}, mientras que en las listas de Python esta
operación se realiza en \emph{tiempo constante}.

\section{La clase {\tt ListaEnlazada}}

Basándonos en los nodos implementados anteriormente, pero buscando
desligar al programador que desea usar la lista de la responsabilidad de
manipular las referencias, definiremos ahora la clase
\lstinline!ListaEnlazada!, de modo tal que no haya que operar mediante las
referencias internas de los nodos, sino que se lo pueda hacer a través de
operaciones de lista.

Se podrá notar que implementaremos los mismos métodos de las listas de Python,
de modo que más allá del funcionamiento interno, ambas serán implementaciones
del TAD lista.

Definimos a continuación las operaciones que inicialmente deberá cumplir la
clase \lstinline!ListaEnlazada!.

\begin{itemize}
\item \lstinline|__str__|, para obtener una representación en cadena de texto.

\item \lstinline|__len__|, para calcular la longitud de la lista.

\item \lstinline|append(x)|, para agregar un elemento al final de la lista.

\item \lstinline|insert(i, x)|, para agregar el elemento \lstinline!x! en la
posición \lstinline!i! (levanta una excepción si la posición \lstinline!i! es
inválida).

\item \lstinline|remove(x)|, para eliminar la primera aparición de
\lstinline!x! en la lista (levanta una excepción si \lstinline!x! no está).

\item \lstinline|pop([i])|, para eliminar el elemento que está en la posición
\lstinline!i! y devolver su valor. Si no se especifica el valor de
\lstinline!i!, \lstinline|pop()| elimina y devuelve el elemento que está en
el último lugar de la lista (levanta una excepción si se hace referencia a
una posición no válida de la lista).

\item \lstinline|index(x)|, devuelve la posición de la primera aparición de
\lstinline!x! en la lista (levanta una excepción si \lstinline!x! no está).
\end{itemize}

Más adelante podrá agregarse a la lista otros métodos que también están
implementados por las listas de Python.

Valen ahora algunas consideraciones más antes de empezar a implementar la clase:

\begin{itemize}

\item Por lo dicho anteriormente, es claro que la lista deberá tener como
atributo la referencia al primer nodo que la compone.

\item Una implementación trivial del método |__len__| podría
recorrer todos los nodos de la lista y contar la cantidad de elementos,
pero si la lista tiene muchos elementos esto podría ser poco eficiente.

Para mejorar la eficiencia alcanza con agregar un atributo numérico que
contenga la cantidad de nodos. Así, este atributo que llamaremos |len|
se inicializará en $0$ cuando se cree la lista vacía, se
incrementará en $1$ cada vez que se agregue un elemento y se decrementará en $1$
cada vez que se elimine un elemento.

\begin{atencion}
Al agregar el atributo |len| estamos \emph{duplicando información}, ya que habrá
dos formas de obtener la longitud de la lista:

\begin{enumerate}
\item contar los nodos
\item obtener el valor de |len|
\end{enumerate}

Como consecuencia, vamos a tener que prestar especial atención para que el
atributo |len| siempre contenga un valor consistente; es decir que su valor sea
siempre igual a la cantidad de nodos que contiene la lista.

En la sección~\ref{invariante-objetos} se explica más formalmente este
concepto.
\end{atencion}

\item Por otro lado, como vamos a incluir todas las operaciones de listas
que sean necesarias para operar con ellas, no es necesario que la clase
\lstinline!Nodo! esté disponible para que otros programadores puedan
modificar (y romper) las listas a voluntad usando operaciones de nodos. Para eso
incluiremos la clase \lstinline!Nodo! de manera \emph{privada} (es
decir oculta), de modo que la podamos usar nosotros como dueños
(fabricantes) de la clase, pero no cualquier programador que utilice la
lista.
\end{itemize}

Python tiene una convención para hacer que atributos, métodos o clases
dentro de una clase dada no puedan ser usados por los usuarios, y sólo
tengan acceso a ellos quienes programan la clase: su nombre tiene que
empezar con un guión bajo y terminar sin guión bajo. Así que para hacer que
los nodos sean privados, cambiaremos el nombre de la clase a
|_Nodo|\footnote{Se trata solo de una convención; aun con el nombre
\lstinline!_Nodo! la clase es visible desde otros módulos.}.

\subsection{Construcción de la lista}

Empezamos escribiendo la clase con su constructor.

\begin{codigo-python-sn}
class ListaEnlazada:
    """Modela una lista enlazada."""

    def __init__(self):
        """Crea una lista enlazada vacía."""
        # referencia al primer nodo (None si la lista está vacía)
        self.prim = None
        # cantidad de elementos de la lista
        self.len = 0
\end{codigo-python-sn}

Si la lista contiene dos elementos (|"Peras"| y |"Bananas"|), su estructura
será como la representada por la Figura~\ref{lista_enlazada}.

\begin{figure}[htb]
\begin{tikzpicture}
\node[umlattr,anchor=west]                                  (len) {len: 2};
\node[umlattr,below=of len.south west,anchor=north west]   (prim) {prim:};
\node[umlattrs,fit=(prim) (len)] (attrs)  {};
\node[umltitle,above=of attrs]               (titulo) {ListaEnlazada};
\node[umlclass,fit=(titulo) (attrs)]         (n1)   {};
\draw[-] (titulo.south-|n1.west)--(titulo.south-|n1.east);

\node[umlattr,anchor=west,right=2cm of len.east]           (dato) {dato: "Peras"};
\node[umlattr,below=of dato.south west,anchor=north west]   (prox2) {prox:};
\node[umlattrs,fit=(prox2) (dato)] (attrs)  {};
\node[umltitle,above=of attrs]               (titulo) {Nodo};
\node[umlclass,fit=(titulo) (attrs)]         (n2)   {};
\draw[-] (titulo.south-|n2.west)--(titulo.south-|n2.east);

\draw[flecha] (prim.east)--(prim-|n2.west);

\node[umlattr,anchor=west,right=1cm of dato.east]           (dato) {dato: "Bananas"};
\node[umlattr,below=of dato.south west,anchor=north west]   (prox3) {prox:};
\node[umlattrs,fit=(prox3) (dato)] (attrs)  {};
\node[umltitle,above=of attrs]               (titulo) {Nodo};
\node[umlclass,fit=(titulo) (attrs)]         (n3)   {};
\draw[-] (titulo.south-|n3.west)--(titulo.south-|n3.east);

\draw[flecha] (prox2.east)--(prox2-|n3.west);
\node[draw,inner sep=0pt,minimum height=0pt,minimum width=0.5cm,line width=1.5pt,right=0.5cm of n3.south east] (none) {};

\path[flecha] (prox3.east) -| (none);
\end{tikzpicture}
\caption{Una lista enlazada con dos elementos.}
\label{lista_enlazada}
\end{figure}

\ejercicioc{Escribir los métodos \lstinline!__str__! y \lstinline!__len__!
para la lista}.

\subsection{Eliminar un elemento de una posición}

Analizaremos a continuación \lstinline|pop([i])|, que elimina el elemento que
está en la posición \lstinline!i! y devuelve su valor. Si no se especifica
el valor de \lstinline!i!, \lstinline|pop()| elimina y devuelve el elemento
que está en el último lugar de la lista.  Por otro lado, levanta una
excepción si se hace referencia a una posición no válida de la lista.

Dado que se trata de una función con cierta complejidad, separaremos el
código en las diversas consideraciones a tener en cuenta.

\begin{itemize}

\item Si no se indica la posición, \lstinline!i! toma la última posición de la
lista:

\begin{codigo-python-sn}
if i is None:
    i = self.len - 1
\end{codigo-python-sn}

\item Si la posición es inválida (\lstinline!i! menor que $0$ o mayor o
igual a la longitud de la lista), se considera error y se levanta la
excepción |IndexError|.

Esto se resuelve con este fragmento de código:

\begin{codigo-python-sn}
if i < 0 or i >= self.len:
    raise IndexError("Índice fuera de rango")
\end{codigo-python-sn}

\item Cuando la posición es $0$ se trata de un caso particular, ya que en ese
caso hay que cambiar la referencia de
\lstinline!self.prim! para que apunte al nodo siguiente.  Es decir, pasar de
\lstinline!self.prim! $\rightarrow$ |nodo0| $\rightarrow$ |nodo1| a
\lstinline!self.prim! $\rightarrow$ |nodo1|).

\begin{codigo-python-sn}
if i == 0:
    dato = self.prim.dato
    self.prim = self.prim.prox
\end{codigo-python-sn}

\noindent (Guardamos el |dato| del nodo descartado para poder devolverlo al
finalizar la función).

\item Vemos ahora el caso general:

Mediante un ciclo, se deben ubicar los nodos $n_{i - 1}$ y $n_i$ que
están en las posiciones $i-1$ e $i$ de la lista, respectivamente, de modo de
poder ubicar no sólo el nodo que se descartará, sino también estar en condiciones
de saltear el nodo descartado en los enlaces de la lista.  La lista debe pasar de
contener el camino $n_{i-1} \rightarrow n_i \rightarrow n_{i+1}$
a contener el camino $n_{i-1} \rightarrow n_{i+1}$.

Nos basaremos un esquema muy simple y útil que se denomina \emph{máquina de parejas}:

Si nuestra secuencia tiene la forma $ABCDE$, se itera sobre ella de modo de
tener las parejas $AB$, $BC$, $CD$, $DE$. En la pareja $XY$, llamaremos a $X$ el
\emph{elemento anterior}
y a $Y$ el \emph{elemento actual}. En general estos ciclos terminan o bien cuando
no hay más parejas que formar, o bien cuando el elemento actual cumple con una determinada
condición.

En nuestro problema, tenemos la siguiente situación:

\begin{itemize}
\item Las parejas son parejas de nodos.

\item Para avanzar en la secuencia se usa la referencia al próximo nodo de la lista.

\item La condición de terminación es siempre que la posición del nodo en la
lista sea igual al valor buscado.  En este caso particular no debemos
preocuparnos por la terminación de la lista porque la validez del índice
buscado ya fue verificada más arriba.
\end{itemize}

Esta es la porción de código correspondiente a la búsqueda. Llamamos |n_ant| y
|n_act| a los elementos anterior y actual de la pareja de nodos:

\begin{codigo-python-sn}
n_ant = self.prim
n_act = n_ant.prox
for pos in range(1, i):
    n_ant = n_act
    n_act = n_ant.prox
\end{codigo-python-sn}

Al finalizar el ciclo, \lstinline!n_ant! será una referencia al nodo $i-1$ y
\lstinline!n_act! una referencia al nodo $i$.

Una vez obtenidas las referencias, se obtiene el dato y se cambia el camino
para descartar el nodo |n_act|:

\begin{codigo-python-sn}
dato = n_act.dato
n_ant.prox = n_act.prox
\end{codigo-python-sn}

\item Finalmente, en todos los casos de éxito se debe devolver el dato que contenía
el nodo descartado y decrementar la longitud en 1:

\begin{codigo-python-sn}
self.len -= 1
return dato
\end{codigo-python-sn}

\end{itemize}

Finalmente, en el Código~\ref{lista_enlazada_pop} se incluye el código completo
del método \lstinline!pop!.

\begin{codigo}{pop}{Método pop de la lista enlazada}
\label{lista_enlazada_pop}
\begin{codigo-python}
def pop(self, i=None):
    """Elimina el nodo de la posición i, y devuelve el dato contenido.
       Si i está fuera de rango, se levanta la excepción IndexError.
       Si no se recibe la posición, devuelve el último elemento."""

    if i is None:
        i = self.len - 1

    if i < 0 or i >= self.len:
        raise IndexError("Índice fuera de rango")

    if i == 0:
        # Caso particular: saltear la cabecera de la lista
        dato = self.prim.dato
        self.prim = self.prim.prox
    else:
        # Buscar los nodos en las posiciones (i-1) e (i)
        n_ant = self.prim
        n_act = n_ant.prox
        for pos in range(1, i):
            n_ant = n_act
            n_act = n_ant.prox

        # Guardar el dato y descartar el nodo
        dato = n_act.dato
        n_ant.prox = n_act.prox

    self.len -= 1
    return dato
\end{codigo-python}
\end{codigo}

\subsection{Eliminar un elemento por su valor}

Análogamente se resuelve \lstinline|remove(self, x)|, que debe eliminar la
primera aparición de \lstinline!x! en la lista, o bien levantar una excepción
si \lstinline!x! no se encuentra en la lista.

Nuevamente, dado que se trata de un método de cierta complejidad, lo
resolveremos por partes, teniendo en cuenta los casos particulares y el caso
general.

\begin{itemize}

\item Si la lista está vacía levantamos una excepción. Tenemos que tratarlo
como un caso particular, ya que en todos los siguientes casos necesitamos
que haya al menos un nodo.

\begin{codigo-python-sn}
if self.prim is None
    raise ValueError("La lista está vacía")
\end{codigo-python-sn}

\item El caso en el que x está en el primer nodo también es particular, ya
que modificar la referencia |self.prim|:

\begin{codigo-python-sn}
if self.prim.dato == x:
    self.prim = self.prim.prox
\end{codigo-python-sn}

\item El caso general también implica un recorrido con máquina de parejas, sólo
que esta vez la condición de terminación es: o bien la lista se terminó o bien
encontramos un nodo con el valor \lstinline!x! buscado.

\begin{codigo-python-sn}
n_ant = self.prim
n_act = n_ant.prox
while n_act is not None and n_act.dato != x:
    n_ant = n_act
    n_act = n_ant.prox
\end{codigo-python-sn}

En este caso, al terminarse el ciclo será necesario corroborar si se terminó
porque llegó al final de la lista, y de ser así levantar una excepción; o si se
terminó porque encontró el dato, y de ser así eliminarlo.

\begin{codigo-python-sn}
if n_act is None:
    raise ValueError("El valor no está en la lista.")
n_ant.prox = n_act.prox
\end{codigo-python-sn}

\item Finalmente, en todos los casos de éxito debemos decrementar en 1 el valor
de \lstinline|self.len|.

\end{itemize}

En el Código~\ref{lista_enlazada_remove} se incluye el código completo
del método \lstinline!remove!.

\begin{codigo}{remove}{Método remove de la lista enlazada}
\label{lista_enlazada_remove}
\begin{codigo-python}
def remove(self, x):
    """Borra la primera aparición del valor x en la lista.
       Si x no está en la lista, levanta ValueError"""

    if self.len == 0:
        raise ValueError("Lista vacía")

    if self.prim.dato == x:
        # Caso particular: saltear la cabecera de la lista
        self.prim = self.prim.prox
    else:
        # Buscar el nodo anterior al que contiene a x (n_ant)
        n_ant = self.prim
        n_act = n_ant.prox
        while n_act is not None and n_act.dato != x:
            n_ant = n_act
            n_act = n_ant.prox

        if n_act == None:
            raise ValueError("El valor no está en la lista.")

        # Descartar el nodo
        n_ant.prox = n_act.prox

    self.len -= 1
\end{codigo-python}
\end{codigo}

\subsection{Insertar nodos}

Debemos programar ahora \lstinline|insert(i, x)|, que debe agregar el elemento
\lstinline!x! en la posición \lstinline!i!  (y levantar una excepción si la
posición \lstinline!i! es inválida).

Veamos qué debemos tener en cuenta para programar esta función.

\begin{itemize}

\item Si se intenta insertar en una posición menor que cero o mayor que la
longitud de la lista debe levantarse una excepción.

\begin{codigo-python-sn}
if i < 0 or i > self.len:
    raise IndexError("Posición inválida")
\end{codigo-python-sn}

\item Para los demás casos hay que crear un nodo, que será el que se insertará
en la posición que corresponda. Construimos un nodo \lstinline|nuevo| cuyo
\lstinline|dato| será \lstinline|x|.

\begin{codigo-python-sn}
nuevo = _Nodo(x)
\end{codigo-python-sn}

\item Si se quiere insertar en la posición 0, hay que cambiar la referencia de
\lstinline|self.prim|.

\begin{codigo-python-sn}
if i == 0:
    nuevo.prox = self.prim
    self.prim = nuevo
\end{codigo-python-sn}

\item Para los demás casos, nuevamente será necesaria la máquina de parejas.
Obtenemos el nodo anterior a la posición en la que queremos insertar.

\begin{codigo-python-sn}
n_ant = self.prim
for pos in range(1, i):
    n_ant = n_ant.prox

nuevo.prox = n_ant.prox
n_ant.prox = nuevo
\end{codigo-python-sn}

\item En todos los casos de éxito se debe incrementar en 1 la longitud de la lista.

\end{itemize}

En el Código~\ref{lista_enlazada_insert} se incluye el código resultante
del método \lstinline!insert!.

\begin{codigo}{insert}{Método insert de la lista enlazada}
\label{lista_enlazada_insert}
\begin{codigo-python}
def insert(self, i, x):
    """Inserta el elemento x en la posición i.
       Si la posición es inválida, levanta IndexError"""

    if i < 0 or i > self.len:
        raise IndexError("Posición inválida")

    nuevo = _Nodo(x)

    if i == 0:
        # Caso particular: insertar al principio
        nuevo.prox = self.prim
        self.prim = nuevo
    else:
        # Buscar el nodo anterior a la posición deseada
        n_ant = self.prim
        for pos in range(1, i):
            n_ant = n_ant.prox

        # Intercalar el nuevo nodo
        nuevo.prox = n_ant.prox
        n_ant.prox = nuevo

    self.len += 1
\end{codigo-python}
\end{codigo}

\ejercicioc{Completar la clase \lstinline|ListaEnlazada| con los métodos que
faltan: \lstinline|append| e \lstinline|index|}.

\ejercicioc{En los bucles de \emph{máquina de parejas} mostrados
anteriormente, no siempre es necesario tener la referencia al nodo actual,
puede alcanzar con la referencia al nodo anterior.  Donde sea posible,
eliminar la referencia al nodo actual.  Una vez hecho esto, analizar el
código resultante, ¿Es más elegante?}

\ejercicioc{\label{append_constante} {\bf Mantenimiento:} Con esta representación conseguimos que la
inserción en la posición 0 se realice en tiempo constante, sin embargo ahora
\lstinline|append| es lineal en la longitud de la lista. Si queremos mejorar
esto, debemos agregar un atributo más a los objetos de la
clase: la referencia al último nodo, y modificar \lstinline|append| para que se
pueda ejecutar en tiempo constante. Por supuesto que además hay que modificar
todos los métodos de la clase para que se mantenga la propiedad de que ese
atributo siempre es una referencia al útimo nodo.}

\section{Invariantes de objetos}

\label{invariante-objetos}
Los invariantes son condiciones que deben ser siempre ciertas.  En la
sección~\ref{invariantes} mencionamos los invariantes de ciclos, que son condiciones que deben
permanecer ciertas durante la ejecución de un ciclo.  Existen también los
invariantes de objetos, que son condiciones que deben ser ciertas a lo
largo de toda la existencia de un objeto.

La clase \lstinline!ListaEnlazada! presentada en la sección anterior,
cuenta con dos invariantes que siempre debemos mantener.  Por un lado, el
atributo \lstinline!len! debe contener siempre la cantidad de nodos de la
lista.  Es decir, siempre que se modifique la lista, agregando o quitando
un nodo, se debe actualizar \lstinline!len! como corresponda.

Por otro lado, el atributo \lstinline!prim! referencia siempre al primer
nodo de la lista. Si se agrega o elimina este primer nodo, es necesario
actualizar esta referencia.

Cuando se desarrolla una estructura de datos como la lista enlazada, es
importante destacar cuáles serán sus invariantes, ya que en cada método
habrá que tener especial cuidado de que los invariantes permanezcan siempre
ciertos.

Así, si se modifica la lista para que la inserción al final pueda hacerse en
tiempo constante (como se pide en el ejercicio~\ref{append_constante}),
se está agregando a la lista un nuevo invariante (un atributo de la lista
que apunte siempre al último elemento) y no es sólo el método
\lstinline!append! el que hay que modificar, sino todos los métodos que
puedan de una u otra forma cambiar la referencia al último elemento de la
lista.

\section{Otras listas enlazadas}

Las listas presentadas hasta aquí son las \emph{listas simplemente
enlazadas}, que son sencillas y útiles cuando se quiere poder insertar o
eliminar nodos de una lista en tiempo constante.

Existen otros tipos de listas enlazadas, cada uno con sus ventajas y
desventajas.

\subsection*{Listas doblemente enlazadas}

Las listas doblemente enlazadas son aquellas en que los nodos cuentan no
sólo con una referencia al siguiente, sino también con una referencia al
anterior.  Esto permite que la lista pueda ser recorrida en ambas
direcciones.

En una lista doblemente enlazada es posible, por ejemplo, eliminar un
nodo sin necesidad de saber cuál es el anterior.

Entre las desventajas podemos mencionar que al tener que mantener dos
referencias el código se vuelve más complejo, y también que ocupa más
espacio en memoria.

\subsection*{Listas circulares}

Las listas circulares, que ya fueron mencionadas al comienzo de esta
unidad, son aquellas en las que el último nodo contiene una referencia al
primero.  Pueden ser tanto simplemente como doblemente enlazadas.

Se las utiliza para modelar situaciones en las cuales los elementos no
tienen un primero o un último, sino que forman una cadena infinita, que se
recorre una y otra vez.

\begin{sabias_que}
El código del kernel Linux, que está programado en C, incluye una
implementación de lista enlazada circular utilizada en la mayoría de los
subsistemas.

Por ejemplo, la lista de tareas que se están ejecutando es una lista
circular.  El \emph{scheduler} (planificador) del kernel permite que cada tarea
utilice el procesador durante una porción de tiempo y luego pasa a la
siguiente, aplicando así una ``ronda de turnos'' sin que haya una primera o una
última tarea.
\end{sabias_que}

\section{Iteradores}

A la interfaz de nuestra lista enlazada le falta una operación muy importante:
una que permita \emph{iterar} la lista; es decir ``recorrer'' cada uno de los
elementos para procesarlos de alguna manera (por ejemplo imprimirlos, insertar
uno nuevo entre otros dos, eliminar los que cumplan alguna condición, etc.).

Para iterar la lista vamos a tener que guardar en algún lugar la posición
actual en la lista, que empezará en el primer nodo, irá avanzando a medida que
iteramos hasta que llega al último nodo. Este estado de iteración (la posición
actual) no sería apropiado guardarlo en la clase |ListaEnlazada|, ya que la
iteración es una operación externa a la lista, y no es su responsabilidad saber
en qué estado se encuentra.

Es por eso que vamos a implementar una clase nueva para guardar el estado de
iteración. Nuestra clase se llamará |IteradorListaEnlazada|, y la forma de
utilizarla será:

\begin{codigo-python-sn}
>>> l = ListaEnlazada()
>>> l.append(7)
>>> l.append(3)
>>> l.append(5)
>>> it = IteradorListaEnlazada(l)
>>> it.esta_al_final()
False
>>> it.dato_actual()
7
>>> it.avanzar()
>>> it.esta_al_final()
False
>>> it.dato_actual()
3
>>> it.avanzar()
>>> it.esta_al_final()
False
>>> it.dato_actual()
5
>>> it.avanzar()
>>> it.esta_al_final()
True
\end{codigo-python-sn}

Otra forma de utilizar el iterador sería mediante un ciclo |while|:

\begin{codigo-python-sn}
>>> it = IteradorListaEnlazada(l)
>>> while not it.esta_al_final():
...     print(it.dato_actual())
...     it.avanzar()
7
3
5
\end{codigo-python-sn}

Para cumplir con la interfaz propuesta, se muestra la implementación de la
clase |IteradorListaEnlazada| en el Código~\ref{cod:iterador-le}:

\begin{codigo}{IteradorListaEnlazada}{Iterador de la lista enlazada}
\label{cod:iterador-le}
\begin{codigo-python}
class IteradorListaEnlazada:
    """Almacena el estado de una iteración sobre la ListaEnlazada."""

    def __init__(self, lista):
        """Crea un iterador para la lista dada"""
        self.lista = lista
        self.anterior = None
        self.actual = lista.prim

    def avanzar(self):
        """Avanza la iteración un paso hacia adelante.
        Pre: la iteración no debe haber llegado al final.
        """
        self.anterior = self.actual
        self.actual = self.actual.prox

    def dato_actual(self):
        """Devuelve el elemento en la posición actual de iteración.
        Pre: la iteración no debe haber llegado al final.
        """
        return self.actual.dato

    def esta_al_final(self):
        """Devuelve verdadero si la iteración llegó al final de la lista."""
        return self.actual is None
\end{codigo-python}
\end{codigo}

Notar que en el constructor guardamos una referencia a la instancia de
|ListaEnlazada|, que no se utiliza en ninguno de los otros métodos. Además
guardamos una referencia al nodo anterior además del actual, cuando sería
suficiente con solo guardar el nodo actual.  Estas referencias serán de
utilidad cuando implementemos métodos para insertar y eliminar elementos,
como se muestra a continuación.

\subsection{Insertar y eliminar}

Ahora que tenemos una manera de iterar la lista enlazada, resulta muy fácil
agregar a nuestro iterador métodos para insertar o eliminar elementos en el
medio de la iteración. A diferencia de los métodos |insert| y |remove| de la
clase |ListaEnlazada|, estos nuevos métodos nos permitirán modificar la lista
en \emph{tiempo constante} una vez que tengamos una referencia a un nodo
cualquiera mediante un iterador.

Supongamos que tenemos una lista |l| de palabras y queremos insertar la palabra
|'mundo'| después de todas las apariciones de la palabra |'hola'|. La idea
sería utilizar nuestro iterador, por cada elemento verificar si es o no la
palabra buscada y en caso de que lo sea insertar el nodo nuevo.

\begin{codigo-python-sn}
>>> it = IteradorListaEnlazada(l)
>>> while not it.esta_al_final():
...     if it.dato_actual() == 'hola'
...         it.avanzar() # queremos agregar 'mundo' después de 'hola'
...         it.insertar('mundo')
...     it.avanzar()
\end{codigo-python-sn}

La implementación del método |insertar| del iterador se muestra en el
Código~\ref{cod:iterador-le-insertar}:

\begin{codigo}{insertar}{Método insertar del iterador}
\label{cod:iterador-le-insertar}
\begin{codigo-python}
    def insertar(self, x):
        """Insertar un elemento en el lugar de la iteración actual.
        Una vez insertado, el nuevo elemento será el actual de la iteración,
        y el elemento que antes era el actual será el siguiente.
        """
        nuevo = _Nodo(x)
        if self.anterior:
            nuevo.prox = self.anterior.prox
            self.anterior.prox = nuevo
        else:
            nuevo.prox = self.lista.prim
            self.lista.prim = nuevo
        self.actual = nuevo
\end{codigo-python}
\end{codigo}

Nuestro iterador también puede permitirnos eliminar elementos. Por ejemplo, si
dada nuestra lista de palabras queremos eliminar todas las que contengan |'ñ'|,
haríamos algo como esto:

\begin{codigo-python-sn}
>>> it = IteradorListaEnlazada(l)
>>> while not it.esta_al_final():
...     if 'ñ' in it.dato_actual()
...         it.eliminar()
...         # luego de eliminar ya estamos en el nodo siguiente
...     else:
...         it.avanzar()
\end{codigo-python-sn}

La implementación del método |eliminar| del iterador se muestra en el
Código~\ref{cod:iterador-le-eliminar}:

\begin{codigo}{insertar}{Método eliminar del iterador}
\label{cod:iterador-le-eliminar}
\begin{codigo-python}
    def eliminar(self):
        dato = self.dato_actual()
        if self.anterior:
            self.anterior.prox = self.actual.prox
            self.actual = self.anterior.prox
        else:
            self.lista.prim = self.actual.prox
            self.actual = self.lista.prim
        return dato
\end{codigo-python}
\end{codigo}

\subsection{Iteradores de Python}

En la unidad anterior se hizo referencia a que todas las secuencias
pueden ser recorridas mediante una misma estructura de control
(\lstinline!for variable in secuencia!), ya que todas implementan el método
especial \lstinline!__iter__!.  Este método debe devolver un iterador
capaz de recorrer la secuencia como corresponda. Los iteradores de Python son
muy parecidos al iterador que hicimos para la lista enlazada, con la diferencia
de que deben implementar una interfaz particular definida por el lenguaje.

Tomemos a la función |range| como ejemplo, y veamos en detalle qué ocurre
cuando la llamamos:

\begin{codigo-python-sn}
>>> r = range(3)
>>> type(r)
<class 'range'>
\end{codigo-python-sn}

Lo primero que observamos es que la función |range| no devuelve una lista de
números, sino que devuelve una instancia de la clase |range|. Para obtener un
iterador a partir de este objeto podemos aplicar la función |iter| (que a su
vez llamará al método |__iter__|):

\begin{codigo-python-sn}
>>> it = iter(r)
>>> type(it)
<class 'range_iterator'>
\end{codigo-python-sn}

En Python los iteradores implementan un método
\lstinline!__next__! que debe devolver los elementos, de a uno por vez,
comenzando por el primero.  Y al llegar al final de la estructura, debe
levantar una excepción de tipo \lstinline!StopIteration!.

La función |next| permite invocar manualmente al método |__next__|:

\begin{codigo-python-sn}
>>> next(it)
0
>>> next(it)
1
>>> next(it)
2
>>> next(it)
(^Traceback (most recent call last):
  File "<stdin>", line 1, in <module>
StopIteration^)
\end{codigo-python-sn}

Cuando hacemos |for x in range(3)|, el bucle |for| automáticamente llama a las
funciones |iter| y |next|, asignando a |x| los valores que devuelve |next| en
cada iteración.

Es decir que las siguientes estructuras son equivalentes:

\noindent\begin{minipage}{.45\textwidth}
\begin{codigo-python-sn}
for elemento in secuencia:
	# hacer algo con elemento
\end{codigo-python-sn}
\end{minipage}\hfill%
\begin{minipage}{.45\textwidth}
\begin{codigo-python-sn}
iterador = iter(secuencia)
while True:
    try:
        elemento = next(iterador)
    except StopIteration:
        break
    # hacer algo con elemento
\end{codigo-python-sn}
\end{minipage}

\subsection{Iterador de Python para la lista enlazada}

Veamos qué pasa si intentamos iterar nuestra lista enlazada utilizando un ciclo
|for|:

\begin{codigo-python-sn}
>>> l = ListaEnlazada()
>>> l.append(7)
>>> l.append(3)
>>> l.append(5)
>>> for valor in l:
...    print(valor)
...
(^Traceback (most recent call last):
  File "<stdin>", line 1, in <module>
TypeError: 'ListaEnlazada' object is not iterable^)
\end{codigo-python-sn}

Nuestra lista enlazada aun no es \emph{iterable} para Python, ya que si bien
hemos implementado un iterador, el mismo no cumple con la interfaz definida por
Python. La buena noticia es que es muy fácil adaptarlo para que lo haga.

Lo primero que debemos hacer es que la clase |ListaEnlazada| responda al método
|__iter__|, devolviendo un iterador:

\begin{codigo-python-sn}
class ListaEnlazada:
    def __iter__(self):
        return IteradorListaEnlazada(self)
\end{codigo-python-sn}

El iterador, por su parte, debe responder al método |__next__|. Este método,
como ya vimos, debe hacer dos cosas: devolver el elemento actual y además
avanzar la iteración. Y en caso de haber llegado al final, lanzar
|StopIteration|:

\begin{codigo-python-sn}
class IteradorListaEnlazada:
    def __next__(self):
        if self.esta_al_final():
            raise StopIteration("No hay más elementos en la lista")
        dato = self.dato_actual()
        self.avanzar()
        return dato
\end{codigo-python-sn}

Habiendo hecho estos cambios, será posible recorrer nuestra lista con el ciclo
|for| al que estamos acostumbrados:

\begin{codigo-python-sn}
>>> l = ListaEnlazada()
>>> l.append(7)
>>> l.append(3)
>>> l.append(5)
>>> for valor in l:
...     print(valor)
...
7
3
5
\end{codigo-python-sn}

No solo eso, sino que además podremos utilizar la lista enlazada en cualquier
operación que requiera una secuencia iterable:

\begin{codigo-python-sn}
>>> list(l) # convertir a una lista de Python
[7, 3, 5]
>>> sum(l) # sumar los elementos
15
>>> max(l) # obtener el máximo elemento
7
\end{codigo-python-sn}

\section{Resumen}

\begin{itemize}

\item Un {\bf tipo abstracto de datos} (TAD) es un tipo de datos que está
definido por las operaciones que contiene y cómo se comportan (su
\emph{interfaz}), no por la forma en la que esas operaciones están implementadas.

\item Una {\bf estructura de datos} es un formato para organizar un conjunto de
datos en la memoria de la computadora, de forma tal de que la información pueda
ser accedida y manipulada en forma eficiente.

\item Las listas de Python son una implementación del TAD lista, utilizando una
estructura de datos llamada {\bf arreglo}.  En un arreglo, es \emph{barato}
(tiempo constante) acceder a cualquier elemento dada su posición, y es
\emph{caro} (tiempo lineal) insertar o eliminar elementos.

\item Una {\bf lista enlazada} es otra implementación del TAD lista.
Se trata de una secuencia compuesta por nodos, en la que
cada nodo contiene un dato y una referencia al nodo que le sigue.

\item En las listas enlazadas, es \emph{barato} (tiempo constante) insertar o eliminar
elementos, ya que simplemente se deben alterar un par de referencias; pero
es \emph{caro} (tiempo lineal) acceder a un elemento en particular, ya que es necesario
pasar por todos los anteriores para llegar a él.

\item Un {\bf iterador} es un objeto que permite recorrer uno a uno los
elementos de una secuencia, y procesarlos a medida que avanza la iteración.

\end{itemize}

\begin{referencia_python}

\begin{sintaxis}{\lstinline{__iter__(self)}}
Método especial que debe devolver un iterador para el objeto. El iterador debe
ser un objeto que implementa el método |__next__|.
\end{sintaxis}

\begin{sintaxis}{\lstinline{__next__(self)}}
Devuelve el elemento actual de la iteración y avanza al siguiente.
Si se llegó al final de la iteración lanza |StopIteration|.
\end{sintaxis}

\begin{sintaxis}{\lstinline{iter(objeto)}}
Equivalente a invocar |objeto.__iter__()|.
\end{sintaxis}

\begin{sintaxis}{\lstinline{next(objeto, [valor])}}
Equivalente a invocar |objeto.__next__()|. Si se invoca con el parámetro
adicional |valor|, al llegar al final de la iteración se devuelve el |valor| en
lugar de lanzar |StopIteration|.
\end{sintaxis}
\end{referencia_python}

\newpage
\section{Ejercicios}

\extractionlabel{guia}
\begin{ejercicio}
Implementar el método \verb!__str__! de \verb|ListaEnlazada|, para que se
genere una salida legible de lo que contiene la lista, similar a las listas de
python.
\end{ejercicio}

\extractionlabel{guia}
\begin{ejercicio}
Agregar a \verb|ListaEnlazada| un método \verb!extend! que reciba una
\verb|ListaEnlazada| y agregue a la lista actual los elementos que se encuentran
en la lista recibida.
\end{ejercicio}

\extractionlabel{guia}
\begin{ejercicio}
Implementar el método \verb|remover_todos(elemento)| de \verb|ListaEnlazada|, que recibe
un elemento y remueve de la lista todas las apariciones del mismo, devolviendo la cantidad
de elementos removidos. La lista debe ser recorrida una sola vez.
\end{ejercicio}

\extractionlabel{guia}
\begin{ejercicio}
Implementar el método \verb|duplicar(elemento)| de \verb|ListaEnlazada|, que recibe
un elemento y duplica todas las apariciones del mismo. Ejemplo:

\begin{verbatim}
L = 1 -> 5 -> 8 -> 8 -> 2 -> 8
L.duplicar(8) => L = 1 -> 5 -> 8 -> 8 -> 8 -> 8 -> 2 -> 8 -> 8
\end{verbatim}
\end{ejercicio}

\extractionlabel{guia}
\begin{ejercicio}
Implementar el método \verb|filter(f)| de \verb|ListaEnlazada|, que recibe
una función y devuelve {\bf una nueva lista enlazada} con los elementos para los
cuales la aplicación de \verb|f| devuelve \verb|True|. La nueva lista debe ser construida
recorriendo los nodos una sola vez (es decir, no se puede llamar a \verb|append|). Ejemplo:

\begin{verbatim}
L = 1 -> 5 -> 8 -> 8 -> 2 -> 8
L.filter(es_primo) -> L2 = 5 -> 2
\end{verbatim}
\end{ejercicio}

\extractionlabel{guia}
\begin{ejercicio}
Escribir un método de la clase \verb|ListaEnlazada| que invierta el orden
de la lista (es decir, el primer elemento queda como último y
viceversa).
\end{ejercicio}

\extractionlabel{guia}
\begin{ejercicio}
Una {\bf lista circular} es una lista cuyo último nodo está ligado al primero,
de modo que es posible recorrerla infinitamente.
Escribir la clase \verb|ListaCircular|, incluyendo los métodos \verb!insert!,
\verb!append!, \verb!remove! y \verb!pop!.
\end{ejercicio}

\extractionlabel{guia}
\begin{ejercicio}
Una {\bf lista doblemente enlazada} es una lista en la cual cada nodo tiene
una referencia al anterior además de al próximo de modo que es posible
recorrerla en ambas direcciones.
Escribir la clase \verb|ListaDobleEnlazada|, incluyendo los métodos
\verb!insert!, \verb!append!, \verb!remove! y \verb!pop!.
\end{ejercicio}
 % 15
\include{17_PilasColas} % 16
\chapter{Modelo de ejecución de funciones y recursión}

\section{La pila de ejecución de las funciones}

% TODO:
% Esta sección debería estar en un capítulo muy anterior.  Estos conceptos
% los venimos usando desde bastante antes de ver objetos.

Si miramos el siguiente segmento de código y su ejecución podemos comprobar
que, pese a tener el mismo nombre, la variable de \lstinline!x! de la función
\lstinline!f! y la variable de \lstinline!x! de la función \lstinline!g! no
tienen nada que ver: una y otra se refieren a valores distintos, y modificar
una no modifica a la otra.

\begin{codigo-python-sn}
def f():
    x = 50
    a = 20
    print("En f, x vale", x)

def g():
    x = 10
    b = 45
    print("En g, antes de llamar a f, x vale", x)
    f()
    print("En g, después de llamar a f, x vale", x)
\end{codigo-python-sn}

Esta es la ejecución de \lstinline!g()!:

\begin{codigo-python-sn}
>>> g()
En g, antes de llamar a f, x vale 10
En f, x vale 50
En g, después de llamar a f, x vale 10
\end{codigo-python-sn}

Este comportamiento lo dimos por sentado desde el principio, pero nunca nos
detuvimos a pensar por qué sucede.  Vamos a ver en esta sección cómo se
ejecutan las llamadas a funciones, para comprender cuál es la razón de este
comportamiento.

Cada función tiene asociado por un lado un código (el texto del programa)
que se ejecutará, y por el otro un conjunto de variables que le son propias
(en este caso \lstinline!x! y \lstinline!a! se asocian con \lstinline!f! y
\lstinline!x! y \lstinline!b! se asocian con \lstinline!g!) y que no se
confunden entre sí pese a tener el mismo nombre. No debería llamarnos la
atención, ya que después de todo conocemos a muchas personas que tienen el
mismo nombre.

Estos nombres asociados a una función los va \emph{descubriendo} el intérprete de
Python a medida que va ejecutando el programa (hay otros lenguajes en los
que los nombres se descubren todos juntos antes de iniciar la ejecución).

La ejecución del programa se puede modelar por el siguiente diagrama, en el
cual los nombres asociados a cada función se encerrarán en una caja o
\emph{marco}:

\newcommand{\currentframe}{\raisebox{1pt}{\tiny$\blacktriangleright$\normalsize} }

\begin{enumerate}

\item  \verb|g()   | \hspace{1.5cm}
	\begin{tabular}{r|r|}
	\hline
	\currentframe \verb|g|&\phantom{\verb|x| $\rightarrow$ 10} \\
	\hline
	\end{tabular}

\item  \verb|x = 10| \hspace{1.5cm}
	\begin{tabular}{r|r|}
	\hline
	\currentframe \verb|g|& \verb|x| $\rightarrow$ 10 \\
	\hline
	\end{tabular}

\item  \verb|b = 45| \hspace{1.5cm}
	\begin{tabular}{r|r|}
	\hline
	\currentframe \verb|g|& \verb|x| $\rightarrow$ 10 \\
	        & \verb|b| $\rightarrow$ 45 \\
	\hline
	\end{tabular}

\item  \verb|print | \hspace{1.5cm}
	\begin{tabular}{r|r|}
	\hline
	\currentframe \verb|g|& \verb|x| $\rightarrow$ 10 \\
	             & \verb|b| $\rightarrow$ 45 \\
	\hline
	\end{tabular}
	\hspace{1cm}
	\begin{tabular}{l}
	Imprime: \\
	{\small\tt En g, antes de llamar a f, x vale 10}
	\end{tabular}

\item  \verb|f()   | \hspace{1.5cm}
	\begin{tabular}{r|r|}
	\hline
	\currentframe \verb|f|&\\
	\hline
	\hline
	\verb|g|& \verb|x| $\rightarrow$ 10 \\
	        & \verb|b| $\rightarrow$ 45 \\
	\hline
	\end{tabular}

\item  \verb|x = 50| \hspace{1.5cm}
	\begin{tabular}{r|r|}
	\hline
	\currentframe \verb|f|& \verb|x| $\rightarrow$ 50 \\
	\hline
	\hline
	\verb|g|& \verb|x| $\rightarrow$ 10 \\
	        & \verb|b| $\rightarrow$ 45 \\
	\hline
	\end{tabular}

\item  \verb|a = 20| \hspace{1.5cm}
	\begin{tabular}{r|r|}
	\hline
	\currentframe \verb|f|& \verb|x| $\rightarrow$ 50 \\
	             & \verb|a| $\rightarrow$ 20 \\
	\hline
	\hline
	\verb|g|& \verb|x| $\rightarrow$ 10 \\
	        & \verb|b| $\rightarrow$ 45 \\
	\hline
	\end{tabular}

\item  \verb|print | \hspace{1.5cm}
	\begin{tabular}{r|r|}
	\hline
	\currentframe \verb|f|& \verb|x| $\rightarrow$ 50 \\
	             & \verb|a| $\rightarrow$ 20 \\
	\hline
	\hline
	\verb|g|& \verb|x| $\rightarrow$ 10 \\
	        & \verb|b| $\rightarrow$ 45 \\
	\hline
	\end{tabular}
	\hspace{1cm}
	\begin{tabular}{l}
	Imprime: \\
	{\small\tt En f, x vale 50}
	\end{tabular}

\item  \verb|print | \hspace{1.5cm}
	\begin{tabular}{r|r|}
	\hline
    \currentframe \verb|g|& \verb|x| $\rightarrow$ 10 \\
	        & \verb|b| $\rightarrow$ 45 \\
	\hline
	\end{tabular}
	\hspace{1cm}
	\begin{tabular}{l}
	Imprime: \\
	{\small\tt En g, despues de llamar a f, x vale 10}
	\end{tabular}

\item  \verb|      | \hspace{1.5cm}
	\begin{tabular}{r|r|}
	\hline
	\currentframe pila vacía\\
	\hline
	\end{tabular}

\end{enumerate}

Se puede observar que:
\begin{itemize}

\item Cuando se invoca a \lstinline|g|, se arma un \emph{marco} vacío para
contener las referencias a las variables asociadas con \lstinline|g|. Ese
marco se apila sobre una \emph{pila vacía}. El marco que está en el tope de la
pila es el \emph{marco actual} y se marca con \currentframe en el diagrama.

\item Cuando se ejecuta dentro de \lstinline|g| la invocación
\lstinline|f()| (en el paso 5) se \emph{apila} un marco vacío que va a alojar
las variables asociadas con \lstinline|f|, que pasa a ser el marco actual,
y se transfiere el control del
programa a la primera instrucción de \lstinline|f|.  El marco de
\lstinline|g| queda debajo del tope de la pila, y por lo tanto el
intérprete no lo ve.

\item Mientras se ejecuta \lstinline|f|, el intérprete busca los
valores que necesita usando el marco que está en el tope de la pila. Si alguna
línea de código de |f| intentara acceder a una variable llamada |b|, el
intérprete no la encontraría y lanzaría una excepción.

\item Después de ejecutar 8, se encuentra el final de la ejecución de
\lstinline|f|.  Se desapila el marco de \lstinline|f| y reaparece el marco
de \lstinline|g| en el tope de la pila. Sigue ejecutando \lstinline|g| a
partir de donde se suspendió por la invocación a \lstinline|f|.
\lstinline|g| sólo ve su marco en el tope de la pila.

\item Después de ejecutar 9, se encuentra el final de la ejecución de
\lstinline|g|.  Se desapila el marco de \lstinline|g| y queda la pila vacía.

\end{itemize}

El {\bf ámbito de definición} de una variable está constituido por todas las
partes del programa desde donde esa variable \emph{es visible}.

\section{Pasaje de parámetros}

Un parámetro es una variable más dentro del marco de una función.
Sólo hay que tener en cuenta que si en la invocación se le pasa
un valor a ese parámetro, en el marco inicial esa variable ya aparecerá
ligada a un valor. Analicemos el siguiente código de ejemplo:

\begin{codigo-python-sn}
def fun1(a):
    print(a + 1)

def fun2(b):
    fun1(b + 5)
    print("Volvio a fun2")
\end{codigo-python-sn}

Con la siguiente ejecución:

\begin{codigo-python-sn}
>>> fun2(43)
49
Volvio a fun2
\end{codigo-python-sn}

En este caso, la ejecución se puede representar de la siguiente manera:

% Tablas de ejecución
\begin{enumerate}

\item  \verb|fun2(43)  | \hspace{1.5cm}
	\begin{tabular}{r|r|}
	\hline
	\currentframe \verb|fun2|&\verb|b| $\rightarrow$ 43\\
	\hline
	\end{tabular}

\item  \verb|fun1(b+5) | \hspace{1.5cm}
	\begin{tabular}{r|r|}
	\hline
	\currentframe \verb|fun1|&\verb|a| $\rightarrow$ 48\\
	\hline
	\hline
	\verb|fun2|&\verb|b| $\rightarrow$ 43\\
	\hline
	\end{tabular}

\item  \verb|print(a+1)| \hspace{1.5cm}
	\begin{tabular}{r|r|}
	\hline
	\currentframe \verb|fun1|&\verb|a| $\rightarrow$ 48\\
	\hline
	\hline
	\verb|fun2|&\verb|b| $\rightarrow$ 43\\
	\hline
	\end{tabular}
	\hspace{1cm}
	\begin{tabular}{l}
	Imprime: \\
	{\small\tt 49}
	\end{tabular}

\item  \verb|print     | \hspace{1.5cm}
	\begin{tabular}{r|r|}
	\hline
	\currentframe \verb|fun2|&\verb|b| $\rightarrow$ 43\\
	\hline
	\end{tabular}
	\hspace{1cm}
	\begin{tabular}{l}
	Imprime: \\
	{\small\tt Volvio a fun2}
	\end{tabular}

\item  \verb|          | \hspace{1.5cm}
	\begin{tabular}{r|r|}
	\hline
	\currentframe pila vacía\\
	\hline
	\end{tabular}

\end{enumerate}

Cuando se pasan objetos como parámetros, las dos variables hacen referencia al \emph{mismo}
objeto. Eso significa que si el objeto pasado es mutable, cualquier modificación que
la función invocada realice sobre su parámetro se reflejará en el argumento de la función llamadora,
como se puede ver en el siguiente ejemplo:

\begin{codigo-python-sn}
def modif(lista):
    lista[0] = 5

def main():
    ls = [1, 2, 3, 4]
    print(ls)
    modif(ls)
    print(ls)
\end{codigo-python-sn}

Y esta es la ejecución:
\begin{codigo-python-sn}
>>> main()
[1, 2, 3, 4]
[5, 2, 3, 4]
\end{codigo-python-sn}

\begin{itemize}

\item Cuando se invoca a \lstinline|modif(ls)| desde \lstinline|main|, el
esquema de la pila es
el siguiente:

\verb|modif(ls)   | \hspace{1.5cm}
	\begin{tabular}{r|l||l|}
	\hline
	\currentframe \verb|modif|&\verb|lista|\tikzmark{ls1}\verb|  | & \\
	\cline{1-2}
	\cline{1-2}
	              \verb|main|&\verb|ls|\tikzmark{ls2}\verb|     | & \multirow{-2}{*}{\verb|  |\tikzmark{t}\verb|[1, 2, 3, 4]|}\\
	\hline
	\end{tabular}
\tikz[overlay,remember picture] \draw[-latex,thick] ($(ls1.north east)$) -- ($(t.north)$);
\tikz[overlay,remember picture] \draw[-latex,thick] ($(ls2.north east)+(0,-0.04)$) -- ($(t)$);

\item Cuando se modifica la lista desde \lstinline|modif|, el esquema de la
pila es el siguiente:

\verb|lista[0] = 5| \hspace{1.5cm}
	\begin{tabular}{r|l||l|}
	\hline
	\currentframe \verb|modif|&\verb|lista|\tikzmark{ls1}\verb|  | & \\
	\cline{1-2}
	\cline{1-2}
	              \verb|main|&\verb|ls|\tikzmark{ls2}\verb|     | & \multirow{-2}{*}{\verb|  |\tikzmark{t}\verb|[5, 2, 3, 4]|}\\
	\hline
	\end{tabular}
\tikz[overlay,remember picture] \draw[-latex,thick] ($(ls1.north east)$) -- ($(t.north)$);
\tikz[overlay,remember picture] \draw[-latex,thick] ($(ls2.north east)+(0,-0.04)$) -- ($(t)$);

\item Cuando la ejecución vuelve a \lstinline|main|, \lstinline!ls!
seguirá apuntando a la lista \lstinline|[5, 2, 3, 4]|.

\end{itemize}

En cambio, cuando el parámetro cambia la referencia que se le pasó por una
referencia a otro objeto, la función |main| no se entera:

\begin{codigo-python-sn}
def cambia_ref(lista):
    lista = [5, 2, 3, 4]

def main():
    ls = [1, 2, 3, 4]
    print(ls)
    cambia_ref(ls)
    print(ls)
\end{codigo-python-sn}

\begin{codigo-python-sn}
>>> main()
[1, 2, 3, 4]
[1, 2, 3, 4]
\end{codigo-python-sn}

\begin{itemize}

\item Cuando se invoca a \lstinline|cambia_ref(ls)| desde
\lstinline|main|, el esquema de la pila es el siguiente:

\verb|cambia_ref() | \hspace{1.5cm}
	\begin{tabular}{r|l||l|}
	\hline
	\currentframe \verb|cambia_ref|&\verb|lista|\tikzmark{ls1}\verb|  | & \\
	\cline{1-2}
	\cline{1-2}
	              \verb|main|&\verb|ls|\tikzmark{ls2}\verb|     | &
                    \multirow{-2}{*}{\verb|  |\tikzmark{t}\verb|[1, 2, 3, 4]|}\\
	\hline
	\end{tabular}
\tikz[overlay,remember picture] \draw[-latex,thick] ($(ls1.north east)$) -- ($(t.north)$);
\tikz[overlay,remember picture] \draw[-latex,thick] ($(ls2.north east)+(0,-0.04)$) -- ($(t)$);

\item Cuando se cambia referencia a la lista desde \verb|cambia_ref|, {\bf se crea
una nueva instancia de lista}, y el esquema de la pila es el siguiente:

\verb|lista = [...]| \hspace{1.5cm}
	\begin{tabular}{r|l||l|}
	\hline
	\currentframe \verb|cambia_ref|&\verb|lista|\tikzmark{ls1}\verb|  | &
        \verb|  |\tikzmark{t1}\verb|[5, 2, 3, 4]|\\
	\cline{1-2}
	\cline{1-2}
	              \verb|main|&\verb|ls|\tikzmark{ls2}\verb|     | &
        \verb|  |\tikzmark{t2}\verb|[1, 2, 3, 4]|\\
	\hline
	\end{tabular}
\tikz[overlay,remember picture] \draw[-latex,thick] ($(ls1.north east)+(0,-0.04)$) -- ($(t1.north west)+(0,-0.04)$);
\tikz[overlay,remember picture] \draw[-latex,thick] ($(ls2.north east)+(0,-0.04)$) -- ($(t2.north west)+(0,-0.04)$);

\item Cuando la ejecución vuelve a \lstinline|main|, \lstinline!ls!
seguirá apuntando a la lista \lstinline|[1, 2, 3, 4]|.

\end{itemize}

\section{Devolución de resultados}

Finalmente, para completar los distintos seguimientos, debemos tener en
cuenta que los resultados que devuelve la función invocada, se \emph{reciben}
en la expresión correspondiente de la función invocante.

\begin{codigo-python-sn}
def cuad(valor):
    c = valor * valor
    return c

def main():
    c = cuad(6)
    print(c)
\end{codigo-python-sn}

En este caso, si hacemos el seguimiento de la funcion invocada:
\begin{codigo-python-sn}
>>> main()
36
\end{codigo-python-sn}

Veremos algo como lo siguiente:

% Tablas de ejecución
\begin{enumerate}

\item \makebox[5cm][l]{\verb|main()|}
	\begin{tabular}{r|l|}
	\hline
	\currentframe \verb|main| & \phantom{\verb|valor| $\rightarrow$ 5} \\
	\hline
	\end{tabular}

\item \makebox[5cm][l]{\verb|c = cuad(6)|}
	\begin{tabular}{r|l|}
	\hline
	\currentframe \verb|main| & \phantom{\verb|valor| $\rightarrow$ 5} \\
	\hline
	\end{tabular}
	\begin{tabular}{l}
	Se suspende la ejecución.\\
	Se invoca a \verb|cuad(6)|.
	\end{tabular}

\item \makebox[5cm][l]{\verb|cuad(6)|}
	\begin{tabular}{r|l|}
	\hline
	\currentframe \verb|cuad| & \verb|valor| $\rightarrow$ 6\\
	\hline
	\hline
	              \verb|main| & \\
	\hline
	\end{tabular}
	\hspace{1cm}

\item \makebox[5cm][l]{\verb|c = valor * valor|}
	\begin{tabular}{r|l|}
	\hline
	\currentframe \verb|cuad|&\verb|valor| $\rightarrow$ 6\\
	                         &\verb|c| $\rightarrow$ 36\\
	\hline
	\hline
	              \verb|main| & \\
	\hline
	\end{tabular}

\item \makebox[5cm][l]{\verb|return c|}
	\begin{tabular}{r|l|}
	\hline
	\currentframe \verb|cuad|&\verb|valor| $\rightarrow$ 6\\
	                         &\verb|c| $\rightarrow$ 36\\
	\hline
	\hline
	              \verb|main| & \\
	\hline
	\end{tabular}
	\begin{tabular}{l}
	\verb|cuad| devuelve 36.\\
	\end{tabular}

\item \makebox[5cm][l]{\verb|c = cuad(6)|}
	\begin{tabular}{r|l|}
	\hline
	\currentframe \verb|main|&\verb|c| $\rightarrow$ 36\\
	\hline
	\end{tabular}

\item \makebox[5cm][l]{\verb|print(c)|}
	\begin{tabular}{r|l|}
	\hline
	\currentframe \verb|main|&\verb|c| $\rightarrow$ 36\\
	\hline
	\end{tabular}
	\begin{tabular}{l}
	Imprime:\\
	\verb|36|.
	\end{tabular}

\item \makebox[5cm][l]{}
	\begin{tabular}{r|l|}
	\hline
	\currentframe pila vacía\\
	\hline
	\end{tabular}

\end{enumerate}

Según se ve en el paso 6, al momento de devolver un valor, el valor de
retorno correspondiente a la función \lstinline!cuad! es el que se
asigna a la variable \lstinline!cuad!, a la vez que la llamada a la función
se elimina de la pila.

\newpage
\section{La recursión y cómo puede ser que funcione}

Estamos acostumbrados a escribir funciones que llaman a otras funciones.
Pero lo cierto es que nada impide que en Python (y en muchos otros
lenguajes) una función se llame a sí misma. Y lo más interesante es que
esta propiedad, que se llama \emph{recursión}, permite en muchos casos
encontrar soluciones muy elegantes para determinados problemas.

En materias de matemática se estudian los razonamientos por inducción para
probar propiedades de números enteros, la recursión no es más que una
generalización de la inducción a más estructuras: las listas, las cadenas
de caracteres, las funciones, etc.

\begin{figure}[h!]
  \centerline{\includegraphics[height=0.5\textheight]{graficos/droste}}
  \caption{Una imagen recursiva: la publicidad de Cacao Droste.}
\end{figure}

A continuación estudiaremos diversas situaciones en las cuales aparece la
recursión, veremos cómo es que esto puede funcionar, algunas situaciones en
las que es conveniente utilizarla y otras situaciones en las que no.

\vspace{2.5cm}

\section{Una función recursiva matemática}

Es muy común tener definiciones inductivas de operaciones, como por ejemplo:

\begin{align*}
0! &= 1 \\
x! &= x \, (x-1)! \quad \text{si}\; x>0
\end{align*}

Este tipo de definición se traduce naturalmente en una función en Python:

\begin{codigo-python-sn}
def factorial(n):
    """Precondición: n entero >= 0
       Devuelve: n!"""
    if n == 0:
        return 1
    return n * factorial(n - 1)
\end{codigo-python-sn}

Esta es la ejecución del factorial para \lstinline!n = 0! y para
\lstinline!n = 3!.

\begin{codigo-python-sn}
>>> factorial(0)
1
>>> factorial(3)
6
\end{codigo-python-sn}

El sentido de la instrucción
\lstinline|n * factorial(n - 1)| es exactamente el mismo que el de la
definición inductiva: para calcular el factorial de $n$ se debe multiplicar
$n$ por el factorial de $n-1$.

Dos piezas fundamentales para garantizar el funcionamiento de este programa
son:

\begin{itemize}
\item Que se defina un \emph{caso base} (en este caso la indicación \emph{no
recursiva} de cómo calcular \lstinline|factorial(0)|).

\item Que el argumento de la función respete la precondición
de que \lstinline!n! debe ser un entero mayor o igual que 0.
\end{itemize}

Dado que ya vimos la pila de evaluación y cómo funciona, no debería
llamarnos la atención que esto pueda funcionar adecuadamente en un lenguaje
de programación que utilice pila para evaluar.

Para poder analizar qué sucede a cada paso de la ejecución de la función,
utilizaremos una versión más detallada del mismo código, en la que el resultado
de cada paso se asigna a una variable.

\begin{codigo-python-sn}
def factorial(n):
    if n == 0:
        r = 1
        return r

    f = factorial(n-1)
    r = n * f
    return r
\end{codigo-python-sn}

Esta porción de código funciona exactamente igual que la anterior, pero nos
permite ponerles nombres a los resultados intermedios de cada operación
para poder estudiar qué sucede a cada paso.
Analicemos entonces la ejecución de \lstinline|factorial(3)|  mediante la pila de
evaluación:

\begin{enumerate}

\item \makebox[5cm][l]{\verb|factorial(3)|}
	\begin{tabular}{r|l|}
	\hline
	\currentframe \verb|factorial| & \verb|n| $\rightarrow$ 3 \\
	\hline
	\end{tabular}

\item \makebox[5cm][l]{\verb|if n == 0:|}
	\begin{tabular}{r|l|}
	\hline
	\currentframe \verb|factorial| & \verb|n| $\rightarrow$ 3 \\
	\hline
	\end{tabular}

\item \makebox[5cm][l]{\verb|f = factorial(n - 1)|}
	\begin{tabular}{r|l|}
	\hline
	\currentframe \verb|factorial| & \verb|n| $\rightarrow$ 3 \\
	\hline
	\end{tabular}
	\begin{tabular}{l}
	Se suspende el cálculo. \\
	Se invoca a \verb|factorial(2)|.
	\end{tabular}

\item \makebox[5cm][l]{\verb|factorial(2)|}
	\begin{tabular}{r|l|}
	\hline
	\currentframe \verb|factorial| & \verb|n| $\rightarrow$ 2 \\
	\hline
	\hline
	              \verb|factorial| & \verb|n| $\rightarrow$ 3 \\
	\hline
	\end{tabular}

\item \makebox[5cm][l]{\verb|if n == 0:|}
	\begin{tabular}{r|l|}
	\hline
	\currentframe \verb|factorial| & \verb|n| $\rightarrow$ 2 \\
	\hline
	\hline
	              \verb|factorial| & \verb|n| $\rightarrow$ 3 \\
	\hline
	\end{tabular}

\item \makebox[5cm][l]{\verb|f = factorial(n - 1)|}
	\begin{tabular}{r|l|}
	\hline
	\currentframe \verb|factorial| & \verb|n| $\rightarrow$ 2 \\
	\hline
	\hline
	              \verb|factorial| & \verb|n| $\rightarrow$ 3 \\
	\hline
	\end{tabular}
	\begin{tabular}{l}
	Se suspende el cálculo. \\
	Se invoca a \verb|factorial(1)|.
	\end{tabular}

\item \makebox[5cm][l]{\verb|factorial(1)|}
	\begin{tabular}{r|l|}
	\hline
	\currentframe \verb|factorial| & \verb|n| $\rightarrow$ 1 \\
	\hline
	\hline
	              \verb|factorial| & \verb|n| $\rightarrow$ 2 \\
	\hline
	\hline
	              \verb|factorial| & \verb|n| $\rightarrow$ 3 \\
	\hline
	\end{tabular}

\item \makebox[5cm][l]{\verb|if n == 0:|}
	\begin{tabular}{r|l|}
	\hline
	\currentframe \verb|factorial| & \verb|n| $\rightarrow$ 1 \\
	\hline
	\hline
	              \verb|factorial| & \verb|n| $\rightarrow$ 2 \\
	\hline
	\hline
	              \verb|factorial| & \verb|n| $\rightarrow$ 3 \\
	\hline
	\end{tabular}

\item \makebox[5cm][l]{\verb|f = factorial(n - 1)|}
	\begin{tabular}{r|l|}
	\hline
	\currentframe \verb|factorial| & \verb|n| $\rightarrow$ 1 \\
	\hline
	\hline
	              \verb|factorial| & \verb|n| $\rightarrow$ 2 \\
	\hline
	\hline
	              \verb|factorial| & \verb|n| $\rightarrow$ 3 \\
	\hline
	\end{tabular}
	\begin{tabular}{l}
	Se suspende el cálculo. \\
	Se llama a \verb|factorial(0)|.
	\end{tabular}

\item \makebox[5cm][l]{\verb|factorial(0)|}
	\begin{tabular}{r|l|}
	\hline
	\currentframe \verb|factorial| & \verb|n| $\rightarrow$ 0 \\
	\hline
	\hline
	              \verb|factorial| & \verb|n| $\rightarrow$ 1 \\
	\hline
	\hline
	              \verb|factorial| & \verb|n| $\rightarrow$ 2 \\
	\hline
	\hline
	              \verb|factorial| & \verb|n| $\rightarrow$ 3 \\
	\hline
	\end{tabular}

\item \makebox[5cm][l]{\verb|if n == 0:|}
	\begin{tabular}{r|l|}
	\hline
	\currentframe \verb|factorial| & \verb|n| $\rightarrow$ 0 \\
	\hline
	\hline
	              \verb|factorial| & \verb|n| $\rightarrow$ 1 \\
	\hline
	\hline
	              \verb|factorial| & \verb|n| $\rightarrow$ 2 \\
	\hline
	\hline
	              \verb|factorial| & \verb|n| $\rightarrow$ 3 \\
	\hline
	\end{tabular}

\item \makebox[5cm][l]{
    \begin{tabular}{l}
    \verb|r = 1|\\
    \verb|return r|
    \end{tabular}
    }
	\begin{tabular}{r|l|}
	\hline
	\currentframe \verb|factorial| & \verb|n| $\rightarrow$ 0 \\
	                               & \verb|r| $\rightarrow$ 1 \\
	\hline
	\hline
	              \verb|factorial| & \verb|n| $\rightarrow$ 1 \\
	\hline
	\hline
	              \verb|factorial| & \verb|n| $\rightarrow$ 2 \\
	\hline
	\hline
	              \verb|factorial| & \verb|n| $\rightarrow$ 3 \\
	\hline
	\end{tabular}
	\begin{tabular}{l}
    |factorial(0)| devuelve 1
	\end{tabular}

\item \makebox[5cm][l]{\verb|f = factorial(n - 1)|}
	\begin{tabular}{r|l|}
	\hline
	\currentframe \verb|factorial| & \verb|n| $\rightarrow$ 1 \\
	                               & \verb|f| $\rightarrow$ 1 \\
	\hline
	\hline
	              \verb|factorial| & \verb|n| $\rightarrow$ 2 \\
	\hline
	\hline
	              \verb|factorial| & \verb|n| $\rightarrow$ 3 \\
	\hline
	\end{tabular}

\item \makebox[5cm][l]{
    \begin{tabular}{l}
    \verb|r = n * f|\\
    \verb|return r|
    \end{tabular}
    }
	\begin{tabular}{r|l|}
	\hline
	\currentframe \verb|factorial| & \verb|n| $\rightarrow$ 1 \\
	                               & \verb|f| $\rightarrow$ 1 \\
	                               & \verb|r| $\rightarrow$ 1 \\
	\hline
	\hline
	              \verb|factorial| & \verb|n| $\rightarrow$ 2 \\
	\hline
	\hline
	              \verb|factorial| & \verb|n| $\rightarrow$ 3 \\
	\hline
	\end{tabular}
	\begin{tabular}{l}
    |factorial(1)| devuelve 1
	\end{tabular}

\item \makebox[5cm][l]{\verb|f = factorial(n - 1)|}
	\begin{tabular}{r|l|}
	\hline
	\currentframe \verb|factorial| & \verb|n| $\rightarrow$ 2 \\
	                               & \verb|f| $\rightarrow$ 1 \\
	\hline
	\hline
	              \verb|factorial| & \verb|n| $\rightarrow$ 3 \\
	\hline
	\end{tabular}

\item \makebox[5cm][l]{
    \begin{tabular}{l}
    \verb|r = n * f|\\
    \verb|return r|
    \end{tabular}
    }
	\begin{tabular}{r|l|}
	\hline
	\currentframe \verb|factorial| & \verb|n| $\rightarrow$ 2 \\
	                               & \verb|f| $\rightarrow$ 1 \\
	                               & \verb|r| $\rightarrow$ 2 \\
	\hline
	\hline
	              \verb|factorial| & \verb|n| $\rightarrow$ 3 \\
	\hline
	\end{tabular}
	\begin{tabular}{l}
    |factorial(2)| devuelve 2
	\end{tabular}

\item \makebox[5cm][l]{\verb|f = factorial(n - 1)|}
	\begin{tabular}{r|l|}
	\hline
	\currentframe \verb|factorial| & \verb|n| $\rightarrow$ 3 \\
	                               & \verb|f| $\rightarrow$ 2 \\
	\hline
	\end{tabular}

\item \makebox[5cm][l]{
    \begin{tabular}{l}
    \verb|r = n * f|\\
    \verb|return r|
    \end{tabular}
    }
	\begin{tabular}{r|l|}
	\hline
	\currentframe \verb|factorial| & \verb|n| $\rightarrow$ 3 \\
	                               & \verb|f| $\rightarrow$ 2 \\
	                               & \verb|r| $\rightarrow$ 6 \\
	\hline
	\end{tabular}
	\begin{tabular}{l}
    |factorial(3)| devuelve 6
	\end{tabular}

\end{enumerate}

\begin{figure}[ht]
  \centerline{\includegraphics[width=0.7\textwidth]{graficos/recursive}}
  \caption{Otra imagen recursiva: captura de pantalla de RedHat.}
\end{figure}

\section{Algoritmos recursivos y algoritmos iterativos}

Llamaremos \emph{algoritmos recursivos} a aquellos que realizan llamadas
recursivas para llegar al resultado, y \emph{algoritmos iterativos} a
aquellos que llegan a un resultado a través de una iteración mediante un
ciclo definido o indefinido.

Todo algoritmo recursivo puede expresarse como iterativo y viceversa.  Sin
embargo, según las condiciones del problema a resolver podrá ser preferible
utilizar la solución recursiva o la iterativa.

Una posible implementación iterativa de la función \lstinline!factorial!
vista anteriormente sería:

\begin{codigo-python-sn}
def factorial(n):
    """Precondición: n entero >= 0
       Devuelve: n!"""
    fact = 1
    for num in range(n, 1, -1):
        fact *= num
    return fact
\end{codigo-python-sn}

Se puede ver que en este caso no es necesario incluir un caso base, ya que
el mismo ciclo incluye una condición de corte, pero que sí es necesario
incluir un acumulador, que en el caso recursivo no era necesario.

Por otro lado, si hiciéramos el seguimiento de esta función, como se hizo
para la versión recursiva, veríamos que la pila de ejecución siempre tiene un
único marco, en el cual se van modificando los valores de \lstinline!num! y
\lstinline!fact!.

Es por esto que, en general, las versiones recursivas de los algoritmos
utilizan más memoria (ya que la pila de ejecución se guarda en
memoria) pero suelen ser más elegantes.

\section{Un ejemplo de recursión elegante}
\label{recursion_potencia}

Consideremos ahora otro problema que puede ser resuelto de forma elegante
mediante un algoritmo recursivo.

La función \lstinline!potencia(b, n)!, vista en unidades anteriores,
realizaba \lstinline!n! iteraciones para poder obtener el valor de $b^n$.
Sin embargo, es posible optimizarla teniendo en cuenta que:

\begin{align*}
b^n &= b^{n/2} \cdot b^{n/2} &&\text{si}\;n\;\text{es par.} \\
b^n &= b^{(n-1)/2} \cdot b^{(n-1)/2} \cdot b &&\text{si}\;n\;\text{es impar.} \\
\end{align*}

Antes de programar cualquier función recursiva es necesario decidir cuál
será el \emph{caso base} y cuál el \emph{caso recursivo}.  Para esta función,
tomaremos $n=0$ como el caso base, en el que devolveremos $1$; y el caso
recursivo tendrá dos partes, correspondientes a los dos posibles grupos de
valores de $n$.

\begin{codigo-python-sn}
def potencia(b,n):
    """Precondición: n >= 0
       Devuelve: b^n."""

    if n <= 0:
        # caso base
        return 1

    if n % 2 == 0:
        # caso n par
        p = potencia(b, n // 2)
        return p * p
    else:
        # caso n impar
        p = potencia(b, (n - 1) // 2)
        return p * p * b
\end{codigo-python-sn}

El uso de la variable \lstinline!p! en este caso no es optativo, ya que
es una de las ventajas principales de esta implementación: se aprovecha el
resultado calculado en lugar de tener que calcularlo dos veces. Vemos que
este código funciona correctamente:

\begin{codigo-python-sn}
>>> potencia(2, 10)
1024
>>> potencia(3, 3)
27
>>> potencia(5, 0)
1
\end{codigo-python-sn}

El orden de las llamadas, haciendo un seguimiento simplificado de la
función será:

\begin{enumerate}
\item \verb!potencia(2, 10)!
\item \hspace{1cm} \verb!p = potencia(2, 5) !
\hspace{4cm} \begin{tabular}{|c|c|}\verb|b| $\rightarrow$ 2 & \verb|n| $\rightarrow$ 10\end{tabular}
\item \hspace{2cm} \verb!p = potencia(2, 2) !
\hspace{3cm} \begin{tabular}{|c|c|}\verb|b| $\rightarrow$ 2 & \verb|n| $\rightarrow$ 5$\;\,$\end{tabular}
\item \hspace{3cm} \verb!p = potencia(2, 1) !
\hspace{2cm} \begin{tabular}{|c|c|}\verb|b| $\rightarrow$ 2 & \verb|n| $\rightarrow$ 2$\;\,$\end{tabular}
\item \hspace{4cm} \verb!p = potencia(2, 0) !
\hspace{1cm} \begin{tabular}{|c|c|}\verb|b| $\rightarrow$ 2 & \verb|n| $\rightarrow$ 1$\;\,$\end{tabular}
\item \hspace{5cm} \verb!return 1           !
\hspace{0cm} \begin{tabular}{|c|c|}\verb|b| $\rightarrow$ 2 & \verb|n| $\rightarrow$ 0$\;\,$\end{tabular}
\item \hspace{4cm} \verb!return 1 * 1 * 2   !
\hspace{1cm} \begin{tabular}{|c|c|c|}\verb|b| $\rightarrow$ 2 & \verb|n| $\rightarrow$ 1$\;\,$
& \verb|p| $\rightarrow$ 1$\;\,$ \end{tabular}
\item \hspace{3cm} \verb!return 2 * 2       !
\hspace{2cm} \begin{tabular}{|c|c|c|}\verb|b| $\rightarrow$ 2 & \verb|n| $\rightarrow$ 2$\;\,$
& \verb|p| $\rightarrow$ 2$\;\,$ \end{tabular}
\item \hspace{2cm} \verb!return 4 * 4 * 2   !
\hspace{3cm} \begin{tabular}{|c|c|c|}\verb|b| $\rightarrow$ 2 & \verb|n| $\rightarrow$ 5$\;\,$
& \verb|p| $\rightarrow$ 4$\;\,$ \end{tabular}
\item \hspace{1cm} \verb!return 32 * 32     !
\hspace{4cm} \begin{tabular}{|c|c|c|}\verb|b| $\rightarrow$ 2 & \verb|n| $\rightarrow$ 10
& \verb|p| $\rightarrow$ 32 \end{tabular}
\end{enumerate}

Se puede ver, entonces, que para calcular $2^{10}$ se realizaron 5 llamadas a
\lstinline!potencia!, mientras que en la implementación más sencilla se
realizaban 10 iteraciones. Y esta optimización será cada vez más importante
a medida que aumenta \lstinline!n!: por ejemplo para $n = 100$ se
realizarán 8 llamadas recursivas, y para $n = 1000$ 11 llamadas.

% Esto no es para darlo, es sólo para que esté

Para transformar este algoritmo recursivo en un algoritmo iterativo, es
necesario \emph{simular} la pila de llamadas a funciones mediante una pila que
almacene los valores que sean necesarios.  En este caso, lo que apilaremos será
si el valor de \lstinline!n! es par o no.

\begin{codigo-python-sn}
def potencia(b, n):
    """Precondición: n >= 0
       Devuelve: b^n."""

    pila = []
    while n > 0:
        if n % 2 == 0:
            pila.append(True)
            n //= 2
        else:
            pila.append(False)
            n = (n - 1) // 2

    p = 1
    while pila:
        es_par = pila.pop()
        if es_par:
            p *= p
        else:
            p *= p * b

    return p
\end{codigo-python-sn}

Como se puede ver, este código es mucho más complejo que la versión recursiva.
Esto se debe a que utilizando recursión el uso de la pila de llamadas a
funciones oculta el proceso de apilado y desapilado y permite concentrarse
en la parte importante del algoritmo.

\section{Un ejemplo de recursión poco eficiente}

Del ejemplo anterior se podría deducir que siempre es mejor utilizar algoritmos
recursivos; sin embargo ---como ya se dijo--- cada situación debe ser analizada por
separado.

Un ejemplo clásico en el cual la recursión tiene un resultado muy poco
eficiente es el de los números de Fibonacci.  La sucesión de Fibonacci está
definida por la siguiente relación:

\begin{align*}
F_n &= 0 &&\text{si}\;n = 0\\
F_n &= 1 &&\text{si}\;n = 1\\
F_n &= F_{n - 1} + F_{n - 2} &&\text{si}\;n > 1
\end{align*}

Los primeros números de esta sucesión son: $0$, $1$, $1$, $2$, $3$, $5$, $8$,
$13$, $21$, $34$, $55$.

Dada la definición recursiva de la sucesión, puede resultar muy tentador
escribir una función que calcule en valor de \lstinline!fib(n)! de la siguiente
forma:

\begin{codigo-python-sn}
def fib(n):
    """Precondición: n >= 0.
       Devuelve: el número de Fibonacci número n."""
    if n == 0 or n == 1:
        return n
    return fib(n - 1) + fib(n - 2)
\end{codigo-python-sn}

Si bien esta implementación es muy sencilla y elegante, también es extremadamente
poco eficiente: para calcular \lstinline!fib(n - 1)! es necesario calcular
\lstinline!fib(n - 2)!, que luego volverá a ser calculado para obtener el valor
\lstinline!fib(n)!.

Por ejemplo, una simple llamada a \lstinline!fib(5)!, generaría
recursivamente todas las llamadas ilustradas en la Figura~\ref{fibonacci}.
Puede verse que muchas de estas llamadas están repetidas, generando un
total de 15 llamadas a la función \lstinline!fib!, sólo para devolver el
valor $F_5$.

\begin{figure}[htb]
\begin{tikzpicture}[
    every node/.style={rectangle,node font=\ttfamily},
    -latex,
    level distance=1cm,
    level 1/.style={sibling distance=6cm},
    level 2/.style={sibling distance=3cm},
    level 3/.style={sibling distance=1.5cm},
]

\node {fib(5)}
child {
    node {fib(4)}
    child { node {fib(3)}
        child { node {fib(2)}
            child { node {fib(1)} }
            child { node {fib(0)} }
        }
        child { node {fib(1)} }
    }
    child { node {fib(2)}
        child { node {fib(1)} }
        child { node {fib(0)} }
    }
}
child {
    node {fib(3)}
    child { node {fib(2)}
        child { node {fib(1)} }
        child { node {fib(0)} }
    }
    child { node {fib(1)} }
};

\end{tikzpicture}
\caption{Árbol de llamadas para \lstinline!fib(5)!}
\label{fibonacci}
\end{figure}

En este caso, será mucho más conveniente utilizar una versión iterativa,
que vaya almacenando los valores de las dos variables anteriores a medida
que los va calculando.

\begin{codigo-python-sn}
def fib(n):
    """Precondición: n >= 0.
       Devuelve: el número de Fibonacci número n."""
    if n == 0 or n == 1:
        return n
    ant2 = 0
    ant1 = 1
    for i in range(2, n + 1):
        fibn = ant1 + ant2
        ant2 = ant1
        ant1 = fibn
    return fibn
\end{codigo-python-sn}

Vemos que el caso base es el mismo para ambos algoritmos, pero que en el
caso iterativo se calcula el número de Fibonacci de forma incremental, de
modo que para obtener el valor de \lstinline!fib(n)! se harán $n-1$
iteraciones.\footnote{¿Y es la iterativa la mejor implementación?
Particularmente para el caso de la sucesión de Fibonacci, puede resolverse
matemáticamente la ecuación de recurrencia para llegar a una fórmula
cerrada. Así se obtiene $F_n = \left[ \frac{\varphi^n}{\sqrt5} \right]$, con
$\varphi = \frac{1 + \sqrt5}2$ el ``número de oro'' y el operador
$\left[ \dots \right]$ como el redondeo entero.}

\begin{atencion}
En definitiva, vemos que un algoritmo recursivo {\bf no} es mejor que uno
iterativo, ni viceversa.  En cada situación será conveniente analizar cuál
algoritmo provee la solución al problema de forma más clara y eficiente.
\end{atencion}

\section{Diseño de algoritmos recursivos}

Hasta el momento vimos que hay muchas funciones matemáticas que se definen
o que pueden desarrollarse de forma recursiva, pero puede aplicarse recursividad
a muchos problemas que no sean explicitamente recursivos. Diseñar un
algoritmo recursivo es un proceso sistematizable.

En general en el proceso para plantear un algoritmo recursivo necesitamos
resolver estos tres problemas:

\begin{description}
\item[Caso base:] Necesitamos definir uno o más casos bases de acuerdo a
nuestro problema. Como regla general tratamos de pensar como caso base a
las condiciones sobre las cuales es más fácil resolver nuestro problema.
Por ejemplo, si estruviéramos trabajando sobre listas o cadenas probablemente
sepamos la respuesta a nuestro problema si tuviéramos una secuencia vacía,
o si estuviéramos trabajando sobre conjuntos de elementos probablemente la
respuesta fuera evidente cuando tengamos un solo elemento.
\item[Caso recursivo] o caso general: Este es el caso que va a efectuar
la llamada recursiva. La idea de este caso es reducir el problema a un
problema más sencillo, del cual se hará cargo la llamada recursiva, y luego
poder ensamblar la solución al problema original. Ampliaremos esto más adelante.
\item[Convergencia:] Necesitamos que la reducción que se haga en el caso
recursivo converja hacia los casos bases, de modo que la recursión alguna
vez termine. Esto es, si dijimos que el caso base se resolvía cuando teníamos
una lista vacía, las operaciones del caso recursivo tienen que reducir
reiteradamente la lista hasta que la misma quede vacía.
\end{description}

Si podemos hacer estas tres cosas, tendremos un algoritmo recursivo para
nuestro problema.

\section{Un primer diseño recursivo}

Supongamos que queremos programar una función \lstinline!sumar(lista)! que
determine en forma recursiva la suma de una secuencia \lstinline!lista! de
números.

Como caso base debemos elegir un caso sencillo de verificar. El caso más
sencillo de verificar es uno en el que ni siquiera necesitamos
computar algo: Si la lista está vacía es evidente que la suma da cero.

Nuestro caso base será algo así como:

\begin{codigo-python-sn}
    if len(lista) == 0:
        return 0
\end{codigo-python-sn}

Queremos converger a que dada cualquier lista de la reducción de nuestro
problema terminemos en el caso base. Hay muchas maneras de reducir una lista
para terminar teniendo cero elementos pero para este caso vamos a proponer la
más fácil: si cada llamada recursiva saca un elemento, tarde o temprano
covergeremos a una lista vacía.

Nuestra llamada recursiva podría ser algo así como:

\begin{codigo-python-sn}
   sumar(lista[1:])
\end{codigo-python-sn}

Lo más complejo ahora es pensar el caso general.

Dijimos que íbamos a retirar un elemento de la lista por vez y hacer una
llamada recursiva. Olvidémonos por un momento de la recursividad e imaginemos
que \emph{ya} tenemos una función |sumar2| que sabe sumar los elementos de una
|lista| y que lo hace bien.  Intentemos resolver el problema inverso: si
agregamos un elemento |x| al principio de la lista (obteniendo |[x] + lista|),
¿podemos calcular la suma de la nueva lista?  ¿Podemos resolver el problema más
grande con la solución al problema más pequeño? La solución es sencilla: La
suma de la lista ampliada será |x| más la suma de la lista original (que podemos
calcular como |sumar2(lista)|).

Es decir, la solución al problema este que planteamos sería así:
\begin{codigo-python-sn}
def sumar(lista):
   """Precondición: len(lista) >= 1.
      Devuelve: La suma de los elementos en la lista."""
   return lista[0] + sumar2(lista[1:])
\end{codigo-python-sn}

Podemos ver que si tuviéramos implementada \lstinline!sumar2! entonces
\lstinline!sumar! funcionaría bien. Volvamos ahora a recursividad: Si sabemos
resolver el caso general en función a la solución del caso simplificado de la
llamada recursiva, si existen casos bases que cortan la recursión y si además
la recursión converge hacia los casos bases tenemos resuelto el problema
completo. La función que asumimos que funcionaba \emph{es la misma} que
acabamos de implementar.

Cuando diseñamos una función recursiva tenemos que dar este \emph{salto de fé}:
asumir que la función del paso recursivo ya funciona; nosotros lo que vamos a implementar
es una función que logra concatenar el resultado del subproblema y ensamblarlo con
nuestro problema mayor. Si hacemos esto bien entonces todo funciona.

Finalmente nuestra primera función recursiva quedaría:
\begin{codigo-python-sn}
def sumar(lista):
   """Devuelve la suma de los elementos en la lista."""
   if len(lista) == 0:
       return 0
   return lista[0] + sumar(lista[1:])
\end{codigo-python-sn}

\section{Pasaje de la información}

Dentro de los problemas recursivos no siempre es inmediato establecer cómo
se va a propagar la información entre las llamadas recursivas, es decir, la
reducción de la solución de los subproblemas en la solución del problema
general.

En todos los ejemplos presentados hasta el momento la información del resultado
se propagó desde las hojas del árbol de llamadas (los casos bases) hacia las
funciones invocantes (mediante la instrucción |return|). Por ejemplo, para
resolver el resultado de Fibonacci $F_5$ se utilizan únicamente los resultados
computados por $F_4$ y $F_3$, y no se recibe ningún dato adicional de la función
invocante (más allá del parámetro |n=5|).
Esto no siempre es así, en algunos problemas sí se hace necesario propagar
información ``hacia abajo``. Y en otros casos, si bien no es necesario,
puede tener ventajas adicionales.

Por ejemplo, podríamos reescribir la función |sumar| de esta forma:
\begin{codigo-python-sn}
def sumar(lista, suma=0):
    """Devuelve la suma de los elementos en la lista."""
    if len(lista) == 0:
        return suma
    return sumar(lista[1:], lista[0] + suma)
\end{codigo-python-sn}
Puede observarse que en esta implementación en vez de \emph{esperar} a que se
resuelva el cómputo de la parte recursiva para ensamblar la solución e ir
resolviendo los cálculos parciales desde el final de la lista hacia el
principio, le \emph{pasamos} la solución parcial a la llamada recursiva.
Finalmente el caso base devuelve la suma de los cálculos que se realizaron
de principio a final y cada llamada recursiva devuelve este resultado.

No profundizaremos más en el tema, pero la particularidad de que lo último
que se realice en el caso general sea la llamada recursiva (sin realizar
ninguna operación adicional sobre el resultado de esta llamada) se conoce como
\emph{recursividad de cola}. La recursividad de cola es de interés porque
implica muy poco esfuerzo reescribir una versión iterativa y no recursiva
del algoritmo. Esto es inmediato: como lo último que se hace es la llamada
recursiva entonces no hace falta seguir \emph{recordando} el contexto de la
llamada anterior cuando se hace la siguiente, entonces no es necesario utilizar
la pila de ejecución. El código anterior puede reescribirse como
\begin{codigo-python-sn}
def sumar(lista):
    """Devuelve la suma de los elementos en la lista."""
    suma = 0
    while True:
        if len(lista) == 0:
            return suma
        lista, suma = lista[1:], lista[0] + suma
\end{codigo-python-sn}
tan solo reemplazando la recursión por un bucle y actualizando las
variables según los parámetros de la llamada recursiva.

\section{Modificación de la firma}

La \emph{firma} de una función es su nombre, más los
parámetros que recibe, más los valores que devuelve. Para invocar una función
cualquiera, es suficiente con saber cómo es su firma, y no es necesario saber
cómo es la implementación interna. Ahora bien, si cambiamos la lista
de parámetros o el tipo de dato del valor de retorno de la función, estamos
cambiando su firma, y eso nos obliga a cambiar cualquier lugar del código
que contenga alguna llamada a la función.

En el ejemplo de \lstinline!sumar! implementada con recursividad de cola nos
vimos obligados a modificar la firma de la función agregando el parámetro
\lstinline!suma! que no formaba parte del problema inicial. Pudimos hacerlo
elegantemente utilizando un valor por omisión (|suma=0|), pero la firma de
todos modos quedó confusa.

Hay casos en los que no podemos salvar un cambio en la firma.  Por ejemplo,
supongamos que queremos diseñar una función recursiva que calcule el promedio
de una secuencia de números.

Como ya sabemos diseñar funciones recursivas intuimos que el caso base será
cuando la lista esté vacía y que la reduciremos sacando de a un elemento por
vez. El cuerpo de nuestra función será algo así:
\begin{codigo-python-sn}
def promediar(lista):
    if len(lista) == 0:
        return ???
    promediar(lista[1:]) ???
\end{codigo-python-sn}
Ahora bien, con esto no alcanza para resolver el problema.

Para calcular un promedio necesitamos computar tanto una suma como contar
la cantidad de elementos. Las funciones van a estar computando dos valores
cuando el resultado del problema es evidentemente uno solo. Si bien puede
elaborarse una solución similar a la que ya ensayamos con \lstinline!sumar!
complicaría innecesariamente el código. Es preferible modificar la firma
de la función.

Implementemos el problema resolviendo primero la llamada recursiva (en una
función diferente que llamaremos |_promediar|) y luego ensamblando:
\begin{codigo-python-sn}
def _promediar(lista):
    if len(lista) == 0:
        return 0, 0
    suma, cantidad = _promediar(lista[1:])
    return lista[0] + suma, cantidad + 1
\end{codigo-python-sn}
Puede verse que esta función cumple con las reglas de diseño de recursividad
que describimos antes. Con lo que no cumple esta función es con la firma
natural de la función |promediar| que queríamos diseñar, ya que |_promediar|
devuelve dos cosas y no una.

Esto no invalida nuestra solución, pero la misma está incompleta. Lo que
debemos hacer es implementar una función \emph{wrapper} (envoltorio) que lo
que haga es operar como \emph{cara visible} para el usuario de la función que
hace realmente el trabajo. A esta función sí la vamos a llamar |promediar|, ya
que va a cumplir con la firma deseada:

\begin{codigo-python-sn}
def promediar(lista):
   """Devuelve el promedio de los elementos de una lista de números."""
   suma, cantidad = _promediar(lista)
   return suma / cantidad
\end{codigo-python-sn}

Notar que si bien la función visible \lstinline!promediar! no es recursiva, sí lo
es la función \lstinline!_promediar! que es la que realiza el trabajo, por
lo que el conjunto se considera recursivo.

Además de para adaptar la firma de la función recursiva, las funciones wrapper
se suelen utilizar para simplificar el código de las funciones recursivas. Por
ejemplo, si quisiéramos hacer validaciones de los parámetros, no
querríamos que las mismas se reiteraran en cada iteración recursiva porque
consumirían recursos innecesarios. Entonces las podemos resolver en la función
wrapper, antes de empezar la recursión.

Por ejemplo, en la sección~\ref{recursion_potencia} implementamos la potencia
en forma recursiva con la restricción $n \geq 0$. Pero dado que
$b^n = \left(\frac1b\right)^{-n}$ podemos aprovechar el código implementado para
resolver para cualquier $n$ entero. Podríamos modificar el código de \lstinline!potencia! para
incluir este caso, pero se reiteraría la comprobación en cada recursión.
Para un caso así sería más sencillo construir una función wrapper
e incluir ahí todo lo que consideremos necesario\footnote{Y más allá de lo
que se mencionó, ¿hace falta resolver la recursión si $b=0$, $b=1$, $b=-1$?}.
Habiendo renombrado la función original como |_potencia|, nuestro wrapper sería:
\begin{codigo-python-sn}
def potencia(b, n):
    """Precondición: n es entero
       Devuelve: b^n."""
    if n < 0:
        b = 1 / b
        n = -n
    return _potencia(b, n)
\end{codigo-python-sn}

\section{Limitaciones}

Si creamos una función sin \emph{caso base}, obtendremos el equivalente
recursivo de un bucle infinito.  Sin embargo, como cada llamada recursiva
agrega un elemento a la pila de llamadas a funciones y la memoria de
nuestras computadoras no es infinita, el ciclo deberá terminarse cuando se
agote la memoria disponible.

En particular, en Python, para evitar que la memoria se termine, la pila de
ejecución de funciones tiene un límite. Es decir, que si se ejecuta un
código como el que sigue:

\begin{codigo-python-sn}
def inutil(n):
    return inutil(n - 1)
\end{codigo-python-sn}

Se obtendrá un resultado como el siguiente:

\begin{codigo-python-sn}
>>> inutil(1)
  File "<stdin>", line 2, in inutil
  File "<stdin>", line 2, in inutil
  (...)
  File "<stdin>", line 2, in inutil
RecursionError: maximum recursion depth exceeded
\end{codigo-python-sn}

El límite por omisión es de 1000 llamadas recursivas. Es posible modificar
el tamaño máximo de la pila de recursión mediante la instrucción
\lstinline!sys.setrecursionlimit(n)!.  Sin embargo, si se está alcanzando
este límite suele ser una buena idea pensar si realmente el algoritmo
recursivo es el que mejor resuelve el problema.

\begin{sabias_que}
Existen algunos lenguajes \emph{funcionales}, como Haskell, ML, o Scheme, en
los cuales la recursión es la única forma de realizar un ciclo.  Es
decir, no existen construcciones |while| ni |for|.

Estos lenguajes cuentan con optimización de recursión de cola,
una optimización para que cuando se identifique que la recursión es de cola,
no se apile el estado de la función
innecesariamente, evitando el consumo adicional de memoria mencionado
anteriormente.

La ejecución de todas las funciones con recursión de cola vistas en esta
unidad podrían ser optimizada por el compilador o intérprete del lenguaje.
\end{sabias_que}

\section{Resumen}

\begin{itemize}

\item A medida que se realizan llamadas a funciones, el estado cada
función se almacena en la \emph{pila de ejecución}.

\item Esto permite que sea posible que una función se llame a sí misma,
pero que las variables dentro de la función tomen distintos valores.

\item La {\bf recursión} es el proceso en el cual una función se invoca a
sí misma.  Este proceso permite crear un nuevo tipo de ciclos.

\item Siempre que se escribe una función recursiva es importante considerar
el {\bf caso base} (el que detendrá la recursión) y el {\bf caso
recursivo} (el que realizará la llamada recursiva).  Una función recursiva
sin caso base es equivalente a un bucle infinito.

\item Una función no es mejor ni peor por ser recursiva.  En cada situación
a resolver puede ser conveniente utilizar una solución recursiva o una
iterativa.  Para elegir una o la otra será necesario analizar las
características de elegancia y eficiencia.

\item Al diseñar funciones recursivas muchas veces puede ser útil
implementar una función {\bf wrapper}, por ejemplo para adaptar
la firma de la función, validar parámetros, inicializar datos o manejar excepciones.

\end{itemize}


\newpage
\section{Ejercicios}

\extractionlabel{guia}
\begin{ejercicio}
Escribir una función recursiva que reciba un número positivo $n$ y devuelva
la cantidad de dígitos que tiene.
\end{ejercicio}

\extractionlabel{guia}
\begin{ejercicio}
Escribir una función recursiva que simule el siguiente experimento:
Se tiene una rata en una jaula con 3 caminos, entre los cuales elige
al azar (cada uno tiene la misma probabilidad), si elige el \emph{1} luego
de 3 minutos vuelve a la jaula, si elige el \emph{2} luego de 5 minutos vuelve a
la jaula, en el caso de elegir el \emph{3} luego de 7 minutos sale de la jaula.
La rata no aprende, siempre elige entre los 3 caminos con la misma probabilidad,
pero quiere su libertad, por lo que recorrerá los caminos hasta salir de la jaula.

La función debe devolver el tiempo que tarda la rata en salir de la jaula.
\end{ejercicio}

\extractionlabel{guia}
\begin{ejercicio}
Escribir una función recursiva que reciba 2 enteros \emph{n} y \emph{b} y devuelva
\verb!True! si \emph{n} es potencia de \emph{b}.

Ejemplos:
\begin{verbatim}
es_potencia(8, 2) -> True
es_potencia(64, 4) -> True
es_potencia(70, 10) -> False
\end{verbatim}
\end{ejercicio}

\extractionlabel{guia}
\begin{ejercicio}
Escribir una funcion recursiva que reciba como parámetros dos cadenas \emph{a} y
\emph{b}, y devuelva una lista con las posiciones en donde se encuentra \emph{b}
dentro de \emph{a}.

Ejemplo:
\begin{verbatim}
posiciones_de("Un tete a tete con Tete", "te") -> [3, 5, 10, 12, 21]
\end{verbatim}
\end{ejercicio}

\extractionlabel{guia}
\begin{ejercicio}
Escribir dos funciones mutualmente recursivas \verb|par(n)| e \verb|impar(n)| que
determinen la paridad del numero natural dado, conociendo solo que:
\begin{itemize}
    \item 1 es impar.
    \item Si un número es impar, su antecesor es par; y viceversa.
\end{itemize}
\end{ejercicio}

\extractionlabel{guia}
\begin{ejercicio}
Escribir una función recursiva que calcule recursivamente el n-ésimo número
triangular (el número $1 + 2 + 3 + \cdots + n$).
\end{ejercicio}

\extractionlabel{guia}
\begin{ejercicio}
Escribir una función que calcule recursivamente cuántos elementos
hay en una pila, suponiendo que la pila sólo tiene los métodos \verb|apilar|
y \verb|desapilar|, y no altere el contenido de la pila.
\end{ejercicio}

\extractionlabel{guia}
\begin{ejercicio}
Escribir una funcion recursiva que encuentre el mayor elemento de una lista.
\end{ejercicio}

\extractionlabel{guia}
\begin{ejercicio}
Escribir una función recursiva para replicar los elementos de una lista
una cantidad n de veces. Por ejemplo:
\begin{verbatim}
replicar([1, 3, 3, 7], 2) -> ([1, 1, 3, 3, 3, 3, 7, 7])
\end{verbatim}
\end{ejercicio}

\extractionlabel{guia}
\begin{ejercicio}
Escribir una funcion recursiva que dada una cadena determine si en la misma
hay más letras A o letras E.
\end{ejercicio}

\extractionlabel{guia}
\begin{ejercicio}
El triángulo de Pascal es un arreglo triangular de números que se define de la
siguiente manera: Las filas se enumeran desde $n = 0$, de arriba hacia
abajo. Los valores de cada fila se enumeran desde $k = 0$ (de izquierda a
derecha). Los valores que se encuentran en los bordes del triángulo son
$1$. Cualquier otro valor se calcula sumando los dos valores contiguos de
la fila de arriba.

\begin{center}
\begin{tabular}{l<{\hspace{12pt}}*{13}{c}}
$n=0$ &&&&&&&1&&&&&&\\
$n=1$ &&&&&&1&&1&&&&&\\
$n=2$ &&&&&1&&2&&1&&&&\\
$n=3$ &&&&1&&3&&3&&1&&&\\
$n=4$ &&&1&&4&&6&&4&&1&&\\
$n=5$ &&1&&5&&10&&10&&5&&1&\\
$n=6$ &1&&6&&15&&20&&15&&6&&1
\end{tabular}
\end{center}

Escribir la función recursiva \verb|pascal(n, k)| que calcula el valor que se
encuentra en la fila \verb|n| y la columna \verb|k|. Ejemplo:
\verb|pascal(5, 2) -> 10|
\end{ejercicio}

\extractionlabel{guia}
\begin{ejercicio}
Ya sabemos que la implementación recursiva del cálculo del número de Fibonacci
($F_n = F_{n-1} + F_{n-2}$, $F_0 = 0$, $F_1 = 1$)
es ineficiente porque muchas de las ramas calculan reiteradamente los mismos
valores.

Escribir una función \lstinline!fibonacci(n)! que calcule el enésimo número
de Fibonacci de forma recursiva pero que utilice un diccionario para almacenar
los valores ya computados y no computarlos más de una vez.

{\bf Nota}: Será necesario implementar una función wrapper para cumplir con la
firma de la función pedida.
\end{ejercicio}
 % 17
\include{19_ordenamiento} % 18
\include{20_ordenamiento_recursivo} % 19
\begin{extract}
\lstset{
    language=C,
}

\chapter{Lenguaje C}

\begin{ejercicio}
Escribir una función que permita calcular el área de un rectángulo dada
su base y altura.
\end{ejercicio}

\begin{ejercicio}
Escribir una función que reciba un número entero |n| y calcule el factorial de
|n|.
\begin{partes}
    \item En forma iterativa.
    \item En forma recursiva.
\end{partes}
\end{ejercicio}

\begin{ejercicio}
Escribir una función que reciba un arreglo de números y la cantidad de
elementos, y devuelva el promedio.
\end{ejercicio}

\begin{ejercicio}
    Usando las funciones |printf| y |sizeof|, escribir un programa que imprima
    el tamaño en bytes de cada uno de los siguientes tipos: |bool|, |char|,
    |short|, |int|, |long|, |float|, |double|, |bool*|, |char*|,
    |short*|, |int*|, |long*|, |float*|, |double*|.
\end{ejercicio}

\begin{ejercicio}
Implementar la función |unsigned int strlen(const char *s)| que devuelve la
longitud de la cadena |s| (sin contar el último caracter |'\0'|).
\begin{partes}
    \item En forma iterativa.
    \item En forma recursiva.
\end{partes}
\end{ejercicio}

\begin{ejercicio}
Implementar la función |void invertir(char *s)| que invierte la cadena
|s| ({\it in-place}).
\end{ejercicio}

\begin{ejercicio}
Implementar la función |void strcpy(const char *origen, char *destino)| que
copia la cadena |origen| en la dirección de memoria apuntada por |destino|.
Asumir que |destino| apunta a un arreglo de caracteres con espacio suficiente
(|strlen(origen) + 1|).
\end{ejercicio}

\begin{ejercicio}
Implementar una función que recibe un arreglo de números y su longitud y
lo ordena mediante el algoritmo de ordenamiento por selección.
\end{ejercicio}

\begin{ejercicio}
Implementar una función que reciba un |mensaje| y dos números enteros |min| y
|max|. La función debe pedir al usuario que ingrese un número entero entre
|min| y |max| (inclusive) y devolverlo. Si el usuario ingresa un valor
inválido se le debe informar y pedir que ingrese un nuevo valor, repitiendo
hasta que ingrese un número válido.
\end{ejercicio}

\begin{ejercicio}
Implementar una función que reciba una cadena de texto y
luego imprima la cadena enmarcada entre asteriscos (|*|). Asumir que la cadena
no contiene ningún caracter |\n|. Por ejemplo, si se recibe la cadena
|"Lenguaje C"|, debe imprimir:

\begin{verbatim}
**************
* Lenguaje C *
**************
\end{verbatim}
\end{ejercicio}

\begin{ejercicio}
Implementar una función que reciba una cadena de texto |s| y un número entero |n|, y
devuelva una cadena de texto que contiene la cadena |s| repetida |n| veces.
La cadena producida debe ser alojada en el \emph{heap}, y debe terminar con un
|\0|.
\end{ejercicio}

\begin{ejercicio}
Analizar el funcionamiento de la siguiente función en lenguaje C. ¿Hay algún
problema en su implementación?

\begin{lstlisting}[numbers=none]
void f() {
    int *p = malloc(sizeof(int) * 100);
    for (int i = 0; i < 10; i++) {
        p[i] = i+1000;
    }
    for (int j = 0; j < 100; j++) {
        printf("%d\n", p[i]);
    }
}
\end{lstlisting}
\end{ejercicio}

\begin{ejercicio}
Considerar el siguiente ejemplo. ¿La invocación a |free| realmente libera la
memoria apuntada por el puntero |a|?

\begin{lstlisting}[numbers=none]
int *a = malloc(42);
int *b = a;
free(b);
\end{lstlisting}
\end{ejercicio}

\begin{ejercicio}
Considerar el siguiente ejemplo. ¿Hay algún problema con la implementación?

\begin{lstlisting}[numbers=none]
char *nombre;
fputs(“Ingresa tu nombre: “, stdout);
fgets(nombre, 100, stdin);
printf(“Hola %s!\n”, nombre);
\end{lstlisting}
\end{ejercicio}

\begin{ejercicio}
Escribir una función que se comporte como la función |input| de Python. Debe
recibir un mensaje a imprimir, y luego debe leer una línea de texto de la
consola y devolverla. La cadena producida debe ser alojada en el \emph{heap},
y debe terminar con un |\0|. Asumir que la línea de texto ingresada
no puede superar los 100 caracteres. Utilizar las funciones de la
biblioteca estándar |fputs| y |fgets|.
\end{ejercicio}
 % 20
\end{extract}

\include{copyright}

\end{document}
