\chapter{Objetos}
\label{objetos}

%TODO: este capítulo es muy largo, lleva 2 o 3 clases darlo.  Para seguir la
%línea de los otros capítulos habría que dividirlo en partes de modo que sea
%una por clase.

Los \emph{objetos} son una manera de organizar datos y de relacionar esos datos
con el código apropiado para manejarlo.  Son los protagonistas de un
paradigma de programación llamado \emph{Programación Orientada a Objetos}.

Nosotros ya usamos objetos en Python sin mencionarlo explícitamente. Es más,
todos los tipos de datos que Python nos provee son, en realidad, objetos.

De forma que, cuando utilizamos \lstinline!cadena.upper()!, le estamos
diciendo a Python que llame a la función \lstinline!upper! del tipo
\lstinline!str! para \lstinline!cadena! que es lo mismo que decir que
llame al \emph{método} \lstinline!upper! del objeto \lstinline!cadena!.

A su vez, a las variables que un objeto contiene se las llama
\emph{atributos}.

\begin{sabias_que}
La Programación Orientada a Objetos introduce bastante terminología, y una
gran parte es simplemente darle un nuevo nombre a cosas que ya estuvimos
usando.  Esto si bien parece raro es algo bastante común en el aprendizaje
humano.

Para poder pensar abstractamente, los humanos necesitamos asignarle
distintos nombres a cada cosa o proceso. De la misma manera, para poder
hacer un cambio en una forma de ver algo ya establecido (realizar un
\emph{cambio de paradigma}), suele ser necesario cambiar la forma de nombrar a
los elementos que se comparten con el paradigma anterior, ya que sino es
muy difícil realizar el salto al nuevo paradigma.
\end{sabias_que}

\section{Tipos}

En los temas que vimos hasta ahora nos hemos encontrado con numerosos tipos
provistos por Python, los \emph{números}, las \emph{cadenas de caracteres},
las \emph{listas}, las \emph{tuplas}, los \emph{diccionarios}, los
\emph{archivos}, etc.  Cada uno de estos tipos tiene sus características, tiene
operaciones propias de cada uno y nos provee de una gran cantidad de
funcionalidades que podemos utilizar para nuestros programas.

Como ya se dijo en unidades anteriores, para saber de qué tipo es un
valor, utilizamos la función \lstinline!type!. Para saber qué métodos
y atributos tiene ese valor utilizamos la función \lstinline!dir!:

\begin{codigo-python-sn}
>>> f = open("archivo.txt")
>>> type(f)
<class '_io.TextIOWrapper'>
>>> dir(f)
['_CHUNK_SIZE', '__class__', '__del__', '__delattr__', '__dict__',
'__dir__', '__doc__', '__enter__', '__eq__', '__exit__', '__format__',
'__ge__', '__getattribute__', '__getstate__', '__gt__', '__hash__',
'__init__', '__iter__', '__le__', '__lt__', '__ne__', '__new__',
'__next__', '__reduce__', '__reduce_ex__', '__repr__', '__setattr__',
'__sizeof__', '__str__', '__subclasshook__', '_checkClosed',
'_checkReadable', '_checkSeekable', '_checkWritable', '_finalizing',
'buffer', 'close', 'closed', 'detach', 'encoding', 'errors', 'fileno',
'flush', 'isatty', 'line_buffering', 'mode', 'name', 'newlines',
'read', 'readable', 'readline', 'readlines', 'seek', 'seekable',
'tell', 'truncate', 'writable', 'write', 'writelines']
\end{codigo-python-sn}

En este caso, la función \lstinline!dir! nos muestra los métodos que tiene
un objeto del tipo \lstinline!_io.TextIOWrapper! (el tipo que Python le asigna
internamente a un archivo).  Podemos ver en el listado los métodos
que ya hemos visto al operar con archivos, junto con otros métodos con
nombres \emph{raros} como \lstinline!__str__!, o \lstinline!__doc__!. Estos
métodos son especiales en Python; más adelante veremos para qué sirven y
cómo se usan.

En el listado que nos da \lstinline!dir! están los atributos y métodos
mezclados.  Si necesitamos saber cuáles son atributos y cuáles son métodos,
podemos hacerlo nuevamente mediante el uso de \lstinline!type!.

\begin{codigo-python-sn}
>>> type(f.name)
<type 'str'>
>>> f.name
'archivo.txt'
>>> type (f.tell)
<type 'builtin_function_or_method'>
>>> f.tell()
0L
\end{codigo-python-sn}

Es decir que \lstinline!name! es un atributo del objeto (el nombre del
archivo), mientras que \lstinline!tell! es un método, que para utilizarlo
debemos llamarlo con paréntesis.

\begin{observacion}
Como ya sabemos, en Python los métodos se invocan con la \emph{notación punto}:
\lstinline+archivo.tell()+, \lstinline+cadena.split(":")+.

Analicemos la segunda expresión.  El significado de ésta es: la variable
\lstinline!cadena! llama al método \lstinline+split+ (del cual es dueña
por tratarse de una variable de tipo \lstinline!str!) con el argumento
\lstinline+":"+.

Sería equivalente a llamar a la función \lstinline!split! pasándole como
primer parámetro la variable |cadena|, y como segundo parámetro el delimitador
|":"|.
Pero la diferencia de notación resalta que el método \lstinline!split! es
un método {\bf de} cadenas, y que no se lo puede utilizar con variables de
otros tipos.

Esta notación provocó un cambio de paradigma en la programación, y es uno de
los ejes de la \emph{Programación Orientada a Objetos}
\end{observacion}.

\section{Qué es un objeto}

En Python, todos los tipos son objetos.  Pero no en todos los lenguajes de
programación es así.  En general, podemos decir que un objeto es una forma
ordenada de agrupar datos (los \emph{atributos}) y operaciones a utilizar
sobre esos datos (los \emph{métodos}).

Es importante notar que cuando decimos \emph{objetos} podemos estar haciendo
referencia a dos cosas parecidas, pero distintas.

Por un lado, la definición del tipo, donde se indican cuáles son los
atributos y métodos que van a tener todas las variables que sean de ese
tipo.  Esta definición se llama específicamente, la {\bf clase} del objeto.

A partir de una clase es posible crear distintos valores que son de ese
tipo. A cada uno de los valores generados a partir de una clase se los llama
{\bf instancia} de esa clase.

\begin{observacion}
Se dice que los objetos tienen {\bf estado} y {\bf comportamiento}, ya que
los valores que tengan los atributos de una instancia determinan el estado
actual de esa instancia, y los métodos definidos en una clase determinan
cómo se va a comportar ese objeto.
\end{observacion}

\section{Definiendo nuevos tipos}

Sin bien Python nos provee con un gran número de tipos ya definidos, en
muchas situaciones utilizar solamente los tipos provistos por el lenguaje
resultará insuficiente.  En estas situaciones queremos poder crear nuestros
propios tipos, que almacenen la información relevante para el problema a
resolver y contengan las funciones para operar con esa información.

Por ejemplo, si se quiere representar un punto en el plano, es posible
hacerlo mediante una tupla de dos elementos, pero esta implementación es
limitada, ya que si se quiere poder operar con distintos puntos (sumarlos,
restarlos o calcular la distancia entre ellos) se deberán tener funciones
\emph{sueltas} para realizar las diversas operaciones.

Podemos hacer algo mejor definiendo un nuevo tipo \lstinline!Punto!, que almacene
la información relacionada con el punto, y contenga las operaciones nos
interese realizar sobre él.

\subsection{Nuestra primera clase: Punto}

Queremos definir nuestra clase que represente un punto en el plano.
Lo primero que debemos notar es que existen varias formas de representar un
punto en el plano, por ejemplo, coordenadas polares o coordenadas
cartesianas.
Además, existen varias operaciones que se pueden realizar sobre un punto
del plano, e implementarlas todas podría llevar mucho tiempo.

En esta primera implementación, optaremos por utilizar la representación de
coordenadas cartesianas, e iremos implementando las operaciones a medida
que las vayamos necesitando.

En primer lugar, creamos una clase \lstinline!Punto! que simplemente
almacena las coordenadas.

\begin{codigo-python-sn}
class Punto:
    """Representación de un punto en el plano en
       coordenadas cartesianas (x, y)"""

    def __init__(self, x, y):
        "Constructor de Punto. x e y deben ser numéricos"
        self.x = x
        self.y = y
\end{codigo-python-sn}

En la primera línea de código indicamos que vamos a crear una nueva clase,
llamada \lstinline!Punto!.

\begin{observacion}
Por convención, en los nombres de las clases definidas por el programador se
escribe cada palabra del nombre con la primera letra en mayúsculas.  Ejemplos:
\lstinline!Punto!, \lstinline!ListaEnlazada!, \lstinline!Hotel!.
\end{observacion}

Además definimos uno de los métodos especiales, \lstinline!__init__!, el
{\bf constructor} de la clase.  Este método se llama cada vez que se crea
una nueva instancia de la clase.

Este método, al igual que todos los métodos de cualquier clase, recibe como
primer parámetro a la instancia sobre la que está trabajando.  Por
convención a ese primer parámetro se lo suele llamar \lstinline!self! (que
podríamos traducir como \emph{yo mismo}), pero puede llamarse de cualquier
forma.

En nuestro caso el constructor recibe dos parámetros más, |x| e |y|. De esta
forma indicamos que para crear una instancia de |Punto| será necesario proveer
el valor de ambas coordenadas.

En el cuerpo del constructor se definen los atributos, utilizando la notación
punto con la instancia |self|. En este caso los atributos se llamarán igual que
los parámetros del constryuctor, |x| e |y|, y se inicializan con el valor de
los parámetros.

Para utilizar esta clase que acabamos de definir, lo haremos de la
siguiente forma:

\begin{codigo-python-sn}
>>> p = Punto(5, 7)
>>> type(p)
<class '__main__.Punto'>
>>> p.x
5
>>> p.y
7
\end{codigo-python-sn}

Al realizar la llamada \lstinline!Punto(5, 7)!:

\begin{itemize}
\item Se crea una nueva instancia de la clase |Punto|.
\item Se ejecuta el constructor |__init__|, con |self| $\rightarrow$ la
    instancia nueva, |x| $\rightarrow$ 5, |y| $\rightarrow$ 7.
\item El constructor asigna los atributos |self.x| $\rightarrow$ 5, |self.y|
    $\rightarrow$ 7.
\item Cuando finaliza la ejecución del constructor, se asigna |p| $\rightarrow$
    la instancia de |Punto| recién creada.
\end{itemize}

\subsection{Agregando validaciones al constructor}

Hemos creado una clase \lstinline!Punto! que permite guardar valores
\lstinline!x! e \lstinline!y!.  Sin embargo, por más que en la
documentación se indique que los valores deben ser numéricos, el código
mostrado hasta ahora no impide que a \lstinline!x! e \lstinline!y! se les
asigne un valor cualquiera, no numérico.

\begin{codigo-python-sn}
>>> q = Punto("A", True)
>>> q.x
'A'
>>> q.y
True
\end{codigo-python-sn}

Si queremos impedir que esto suceda, debemos agregar validaciones al
constructor, como las vistas en unidades anteriores.

Verificaremos que los valores pasados para \lstinline!x! e \lstinline!y!
sean numéricos, utilizando la función \lstinline!validar_numero!:

\begin{codigo-python-sn}
def validar_numero(valor):
    """Si el valor es numérico, lo devuelve.
       En caso contrario lanza TypeError."""
    if not isinstance(valor, (int, float, complex)):
        raise TypeError("{:r} no es un valor numérico".format(valor))
    return valor
\end{codigo-python-sn}

El nuevo constructor quedará así:

\begin{codigo-python-sn}
    def __init__(self, x, y):
        """Constructor de Punto. x e y deben ser numéricos,
           de no ser así, se levanta una excepción TypeError"""
        self.x = validar_numero(x)
        self.y = validar_numero(y)
\end{codigo-python-sn}

Este constructor impide que se creen instancias con valores inválidos para
\lstinline!x! e \lstinline!y!.

\begin{codigo-python-sn}
>>> Punto("A", True)
Traceback (most recent call last):
  File "<stdin>", line 1, in <module>
  File "punto.py", line 12, in __init__
    self.x = validar_numero(x)
  File "punto.py", line 5, in validar_numero
    raise TypeError("{!r} no es un valor numérico".format(valor))
TypeError: 'A' no es un valor numérico
\end{codigo-python-sn}

\subsection{Agregando operaciones}

Hasta ahora hemos creado una clase \lstinline!Punto! que permite
construirla con un par de valores, que deben ser sí o sí numéricos, pero no
podemos operar con esos valores.  Para apreciar la potencia de los objetos,
tenemos que definir operaciones adicionales que vayamos a querer realizar
sobre esos puntos.

Queremos, por ejemplo, poder calcular la distancia entre dos puntos.  Para
ello definimos un nuevo método \lstinline!distancia! que recibe el punto de
la instancia actual y el punto para el cual se quiere calcular la
distancia.

\begin{codigo-python-sn}
    def distancia(self, otro):
        """Devuelve la distancia entre ambos puntos."""
        dx = self.x - otro.x
        dy = self.y - otro.y
        return (dx * dx + dy * dy) ** 0.5
\end{codigo-python-sn}

Una vez agregado este método a la clase, será posible obtener la distancia
entre dos puntos, de la siguiente manera:

\begin{codigo-python-sn}
>>> p = Punto(5, 7)
>>> q = Punto(2, 3)
>>> p.distancia(q)
5.0
\end{codigo-python-sn}

Podemos ver, sin embargo, que la operación para calcular la distancia
incluye la operación de restar dos puntos y la de obtener la norma de un
vector. Sería deseable incluir también estas dos operaciones dentro de la
clase \lstinline!Punto!.

Agregaremos, entonces, el método para restar dos puntos:

\begin{codigo-python-sn}
    def restar(self, otro):
        """Devuelve el Punto que resulta de la resta
           entre dos puntos."""
        return Punto(self.x - otro.x, self.y - otro.y)
\end{codigo-python-sn}

La resta entre dos puntos es un nuevo punto.  Es por ello que este método
devuelve una nueva instancia de |Punto|, en lugar de modificar las instancias
|self| u |otro|.

A continuación definimos el método para calcular la norma del vector que
se forma uniendo un punto con el origen.

\begin{codigo-python-sn}
    def norma(self):
        """Devuelve la norma del vector que va desde el origen
           hasta el punto. """
        return (self.x * self.x + self.y * self.y) ** 0.5
\end{codigo-python-sn}

En base a estos dos métodos podemos ahora volver a escribir el método
\lstinline!distancia! para que aproveche el código ambos:

\begin{codigo-python-sn}
    def distancia(self, otro):
        """Devuelve la distancia entre ambos puntos."""
        return self.restar(otro).norma()
\end{codigo-python-sn}

En definitiva, hemos definido tres operaciones en la clase
\lstinline!Punto!, que nos sirve para calcular restas, normas de vectores
al origen, y distancias entre puntos.

\begin{codigo-python-sn}
>>> p = Punto(5, 7)
>>> q = Punto(2, 3)
>>> r = p.restar(q)
>>> (r.x, r.y)
(3, 4)
>>> r.norma()
5.0
>>> q.distancia(r)
1.41421356237
\end{codigo-python-sn}

\begin{atencion}
Cuando definimos los métodos que va a tener una determinada clase es
importante tener en cuenta que el listado de métodos debe ser lo más
conciso posible.

Es decir, si una clase tiene algunos métodos básicos que pueden combinarse
para obtener distintos resultados, no queremos implementar toda posible
combinación de llamadas a los métodos básicos, sino sólo los básicos y
aquellas combinaciones que sean muy frecuentes, o en las que
tenerlas como un método aparte implique una ventaja significativa en cuanto
al tiempo de ejecución de la operación.

Este concepto se llama {\bf ortogonalidad} de los métodos, basado en la
idea de que cada método debe realizar una operación independiente de los
otros.  Entre las motivaciones que puede haber para agregar métodos que no
sean ortogonales, se encuentran la \emph{simplicidad de uso} y la
\emph{eficiencia}.
\end{atencion}

\section{Métodos especiales}

Así como el constructor, \lstinline!__init__!, existen diversos métodos
especiales que, si están definidos en nuestra clase, Python los llamará por
nosotros cuando se utilice una instancia en situaciones particulares.

\subsection{Conversión a cadena de texto}

Veamos qué pasa cuando intentamos convertir la instancia de |Punto| a una
cadena de texto:

\begin{codigo-python-sn}
>>> p = Punto(5, 7)
>>> str(p)
'<__main__.Punto object at 0x7fa6da7694a8>'
\end{codigo-python-sn}

Esto no resulta muy satisfactorio. Lo mismo sucede cuando intentamos imprimir
el objeto con |print| (internamente la función |print| aplica la función |str|
a todos los objetos que recibe antes de imprimirlos):

\begin{codigo-python-sn}
>>> print(p)
<__main__.Punto object at 0x7fa6da7694a8>
\end{codigo-python-sn}

Para poder modificar el comportamiento del objeto al convertir a cadena, Python
indica que hay que agregarle a la clase un método especial, llamado
\lstinline+__str__+ que debe recibir un solo parámetro (|self|), y
debe devolver la cadena de texto deseada.  Ese método
se invocará cuando se llame a la función \lstinline!str!.
El método \lstinline+__str__+ tiene un solo parámetro, \lstinline!self!.

En nuestro caso decidimos mostrar el punto como un par ordenado, por lo que
escribimos el siguiente método dentro de la clase \lstinline!Punto!:

\begin{codigo-python-sn}
    def __str__(self):
        """Devuelve la representación del Punto como
           cadena de texto."""
        return "({}, {})".format(self.x, self.y)
\end{codigo-python-sn}

Una vez definido este método, nuestro punto se mostrará como un par
ordenado cuando se necesite una representación de cadenas.

\begin{codigo-python-sn}
>>> p = Punto(-6, 18)
>>> str(p)
'(-6, 18)'
>>> print(p)
(-6, 18)
\end{codigo-python-sn}

\begin{sabias_que}
Muchas de las funciones provistas por Python, que ya hemos utilizado en
unidades anteriores, como \lstinline!str!, \lstinline!len! o
\lstinline!help!, invocan internamente a los métodos especiales de los
objetos.

Es decir que la función \lstinline!str!  internamente invoca al método
\lstinline!__str__! del objeto que recibe como parámetro. De la misma
manera \lstinline!len! invoca internamente al método \lstinline!__len__!,
si es que está definido.

Cuando mediante \lstinline!dir! vemos que un objeto tiene alguno de estos
métodos especiales, utilizamos la función de Python correspondiente
a ese método especial.
\end{sabias_que}

La conversión a cadena con |__str__| funciona con |str| y con |print|, pero aun
hay algunos casos en los que Python imprime la representación por omisión:

\begin{codigo-python-sn}
>>> p = Punto(-6, 18)
>>> str(p)
'(-6, 18)'
>>> p
<__main__.Punto object at 0x7fa6da7694a8>
>>> str([p])
'[<__main__.Punto object at 0x7fa6da7694a8>]'
>>> [p]
[<__main__.Punto object at 0x7fa6da7694a8>]
\end{codigo-python-sn}

En estos casos, en lugar de llamar a |str| Python llama a la función |repr|.
Mientras que La función |str| se usa para obtener una representación
``informal'' o ``legible'' del objeto, pensada tal vez para mostrar a un
usuario, el objetivo de |repr| es obtener una representación ``formal'' y
``desambiguada'', pensada para mostrar a un programador:

\begin{codigo-python-sn}
>>> str("hola")
'hola'
>>> repr("hola")
"'hola'"
\end{codigo-python-sn}

Internamente, la función |repr| invoca al método |__repr__| del objeto, y
podemos implementarlo para modificar el comportamiento por omisión:

\begin{codigo-python-sn}
    def __repr__(self):
        """Devuelve la representación formal del Punto como
           cadena de texto."""
        return "Punto({}, {})".format(self.x, self.y)
\end{codigo-python-sn}

\begin{codigo-python-sn}
>>> p = Punto(-6, 18)
>>> str(p)
'(-6, 18)'
>>> repr(p)
'Punto(-6, 18)'
>>> p
Punto(-6, 18)
\end{codigo-python-sn}

\subsection{Métodos para operar matemáticamente}

Ya hemos visto un método que permitía restar dos puntos.  Si bien esta
implementación es perfectamente válida, no es posible usar esa función para
realizar una resta con el operador \lstinline!-!.

\begin{codigo-python-sn}
>>> p = Punto(3,4)
>>> q = Punto(2,5)
>>> p - q
Traceback (most recent call last):
  File "<stdin>", line 1, in <module>
TypeError: unsupported operand type(s) for -: 'Punto' and 'Punto'
\end{codigo-python-sn}

Si queremos que este operador (o el equivalente para la suma) funcione,
será necesario implementar algunos métodos especiales.

\begin{codigo-python-sn}
    def __add__(self, otro):
        """Devuelve la suma de ambos puntos."""
        return Punto(self.x + otro.x, self.y + otro.y)

    def __sub__(self, otro):
        """Devuelve la resta de ambos puntos."""
        return Punto(self.x - otro.x, self.y - otro.y)
\end{codigo-python-sn}

El método \lstinline!__add__! es el que se utiliza para el operador
\lstinline!+!; el primer parámetro es el primer operando de la suma, y el
segundo parámetro el segundo operando.  Debe devolver una nueva instancia,
nunca modificar la clase actual.  De la misma forma, el método
\lstinline!__sub__! es el utilizado por el operador \lstinline!-!.

Ahora es posible operar con los puntos directamente mediante los
operadores, en lugar de llamar a métodos:

\begin{codigo-python-sn}
>>> p = Punto(3, 4)
>>> q = Punto(2, 5)
>>> p - q
(1, -1)
>>> p + q
(5, 9)
\end{codigo-python-sn}

De la misma forma, si se quiere poder utilizar cualquier otro operador
matemático, será necesario definir el método apropiado.

\label{sobrecarga}
\begin{sabias_que}
La posibilidad de definir cuál será el comportamiento de los operadores
básicos (como |+|, |-|, |*|, |/|), se llama {\bf sobrecarga de
operadores}.

No todos los lenguajes lo permiten, y si bien es cómodo y permite que el
código sea más elegante, no es algo esencial a la Programación Orientada a
Objetos.

Entre los lenguajes más conocidos que no soportan sobrecarga de operadores
están C, Java, Pascal, Objective C.  Entre los lenguajes más conocidos que
sí soportan sobrecarga de operadores están Python, C++, C\#, Ruby, PHP.
\end{sabias_que}

\section{Creando clases más complejas}

Nos contratan para diseñar una clase para evaluar la relación calidad-precio de
diversos hoteles.  Nos dicen que los atributos que se cargarán de los hoteles
son: nombre, ubicación, puntaje obtenido por votación, y precio, y que además
de guardar hoteles y mostrarlos, debemos poder compararlos en términos de
sus valores de relación calidad-precio, de modo tal que \emph{x < y} signifique
que el hotel $x$ es peor en cuanto a la relación calidad-precio que el hotel
$y$, y que dos hoteles son iguales si tienen la misma relación calidad-precio.
La relación calidad-precio de un hotel la definen nuestros clientes como
$10 \cdot puntaje^2 / precio$.

Además, y como resultado de todo esto, tendremos que ser capaces
de ordenar de menor a mayor una lista de hoteles, usando el orden que nos
acaban de definir.

Averiguamos un poco más respecto de los atributos de los hoteles:

\begin{itemize}
\item El nombre y la ubicación deben ser cadenas no vacías.
\item El puntaje debe ser un número (sin restricciones sobre su valor)
\item El precio debe ser un número distinto de cero.
\end{itemize}

Empezamos diseñar a la clase:

\begin{itemize}
\item El método \lstinline+__init__+:

\begin{itemize}
\item Creará objetos de la clase \lstinline!Hotel! con los atributos que se
indicaron (|nombre|, |ubicacion|, |puntaje|, |precio|).

\item Necesitamos validar que |puntaje| y |precio| sean números positivos.
Para esto utilizaremos una función |validar_numero_positivo| similar a la
función |validar_numero| que usamos para la clase |Punto|.

\item Necesitamos validar que nombre y ubicación sean cadenas no vacías
(para lo cual tenemos que construir una función
\lstinline!validar_cadena_no_vacia!).

\item Cuando los datos no satisfagan los requisitos se levantará una
excepción \lstinline!TypeError!.
\end{itemize}

\item Contará con un método \lstinline+__str__+ para mostrar a los hoteles
mediante una cadena del estilo:
|"Hotel City de Mercedes - Puntaje: 3.25 - Precio: 78 pesos"|.

\item Respecto a la relación de orden entre hoteles, la clase deberá poder
contar con los métodos necesarios para realizar esas comparaciones y para
ordenar una lista de hoteles.
\end{itemize}

Podemos realizar casi todas las tareas con los temas vistos para la
creación de la clase \lstinline!Punto!.  Para el último ítem deberemos
introducir nuevos métodos especiales.

\ejercicioc{Escribir las funciones |validar_numero_positivo| y
|validar_cadena_no_vacia|, que deben lanzar |TypeError| si la
validación del valor falla, y en caso contrario simplemente devolver el valor.
Incluir todas las funciones de validación en un módulo |validaciones.py|.}

El fragmento inicial de la clase programada en Python queda así:

\begin{codigo-python}
class Hotel:
    """Representa un hotel: sus atributos son:
       nombre, ubicacion, puntaje y precio."""

    def __init__(self, nombre, ubicacion, puntaje, precio):
        """Crea un Hotel.
           nombre y ubicacion deben ser cadenas no vacías,
           puntaje y precio son números."""
        self.nombre = validar_cadena_no_vacia(nombre)
        self.ubicacion = validar_cadena_no_vacia(ubicacion)
        self.puntaje = validar_numero_positivo(puntaje)
        self.precio = validar_numero_positivo(precio)

    def __str__(self):
        """Conversión a cadena de texto."""
        return "{} de {} - Puntaje: {} - Precio: {} pesos".format(
            self.nombre,
            self.ubicacion,
            self.puntaje,
            self.precio,
        )

    def ratio(self):
        """Calcula la relación calidad-precio de un hotel"""
        return ((self.puntaje ** 2) * 10) / self.precio
\end{codigo-python}

Con este código tenemos ya la posibilidad de construir un hotel, asegurando que
los atributos de los tipos correspondientes, mostrarlo según nos lo
han solicitado y calcular su relación calidad-precio:

\begin{codigo-python-sn}
>>> h = Hotel("Hotel City", "Mercedes", 3.2, 20)
>>> print(h)
Hotel City de Mercedes - Puntaje: 3.2 - Precio: 20 pesos.
>>> h.ratio()
5.12
\end{codigo-python-sn}

\subsection{Métodos para comparar objetos}

Para resolver las comparaciones entre hoteles, será necesario definir
algunos métodos especiales que permiten comparar objetos.

En particular, cuando se quiere que los objetos puedan ser ordenados, es
suficiente con definir el método |__lt__| (\emph{less than}), que corresponde al
operador matemático de comparación |<|. El método |__lt__| recibe dos
parámetros, |self| y |otro| y debe devolver |True| si |self| es
comparativamente ``menor'' a |otro|.

En el caso de nuestra clase |Hotel|, podemos decir que un hotel es ``menor'' a
otro si su relación calidad-precio es menor:

\begin{codigo-python-sn}
    def __lt__(self, otro):
        """Compara dos hoteles según sus ratios."""
        return self.ratio() < otro.ratio()
\end{codigo-python-sn}

Una vez que está definida esta función podremos realizar comparaciones
entre los hoteles:

\begin{codigo-python-sn}
>>> h = Hotel("Hotel City", "Mercedes", 3.25, 78)
>>> i = Hotel("Hotel Mascardi", "Bariloche", 6, 150)
>>> i < h
False
>>> i > h
True
\end{codigo-python-sn}

Al implementar el método |__lt__|, las instancias de la clase se pueden
comparar con los operadores |<| y |>| (El operador |>| se fija primero si
existe |__gt__|, y si no existe invoca a |__lt__|).

Pero aún quedan sin definir las operaciones |==|, |!=|, |<=| y |>=|.  Si
queremos tener la posibilidad de preguntar si dos instancias son ``iguales'' o
``distintas'' con |==| y |!=|, tenemos que definir el método |__eq__|. Y si
queremos tener la posibilidad de comparar con |<=| o |>=| tenemos que
implementar el método |__le__|.

\subsection{Ordenar de menor a mayor listas de hoteles}

En la sección~\ref{ordenar} vimos que se puede ordenar una lista usando el
método \lstinline!sort!:

\begin{codigo-python-sn}
>>> l1 = [10, -5, 8, 12, 0]
>>> l1.sort()
>>> l1
[-5, 0, 8, 10, 12]
\end{codigo-python-sn}

De la misma forma, una vez que hemos definido el método
\lstinline!__lt__!, podemos ordenar listas de hoteles, ya que
internamente el método \lstinline!sort! comparará los hoteles mediante
el método de comparación que hemos definido:

\begin{codigo-python-sn}
>>> h1 = Hotel("Hotel 1* normal", "MDQ", 1, 10)
>>> h2 = Hotel("Hotel 2* normal", "MDQ", 2, 40)
>>> h3 = Hotel("Hotel 3* carisimo", "MDQ", 3, 130)
>>> h4 = Hotel("Hotel vale la pena" ,"MDQ", 4, 130)
>>> lista = [h1, h2, h3, h4]
>>> lista.sort()
>>> for hotel in lista:
...     print(hotel)
...
Hotel 3* carisimo de MDQ - Puntaje: 3 - Precio: 130 pesos.
Hotel 1* normal de MDQ - Puntaje: 1 - Precio: 10 pesos.
Hotel 2* normal de MDQ - Puntaje: 2 - Precio: 40 pesos.
Hotel vale la pena de MDQ - Puntaje: 4 - Precio: 130 pesos.
\end{codigo-python-sn}

Podemos verificar cuál fue el criterio de ordenamiento invocando al
método \lstinline!ratio! en cada caso:

\begin{codigo-python-sn}
>>> h1.ratio()
1.0
>>> h2.ratio()
1.0
>>> h3.ratio()
0.69230769230769229
>>> h4.ratio()
1.2307692307692308
\end{codigo-python-sn}

Y vemos que efectivamente:

\begin{itemize}
\item ``Hotel 3* carisimo'', con la menor relación calidad-precio
aparece primero.

\item ``Hotel 1* normal'' y ``Hotel 2* normal'' con la misma relación
calidad-precio (igual a 1.0 en ambos casos) aparecen en segundo
y tercer lugar en la lista.

\item ``Hotel vale la pena'' con la mayor relación calidad-precio
aparece en cuarto lugar en la lista.
\end{itemize}

Hemos por lo tanto ordenado la lista de acuerdo al criterio solicitado.

\subsection{Otras formas de comparación}

Si además de querer listar los hoteles por su relación calidad-precio
también se quiere poder listarlos según su puntaje, o según su precio, no
se lo puede hacer mediante el método \lstinline!__lt__! (a menos que
redefinamos el método, pero eso sería poco práctico).

Para situaciones como esta, \lstinline!sort! puede recibir, opcionalmente,
otro parámetro |key| que es la función que calcula la \emph{clave de comparación}
de cada elemento.

La clave de comparación es simplemente el valor numérico que queremos asignar a
cada elemento para comparar y ordenar. Cuando ordenamos los hoteles en el
ejemplo anterior, podríamos decir que la clave de comparación era el ratio, y
por lo tanto podríamos haber ordenado de la siguiente manera, sin necesidad de
implementar |__lt__|:

\begin{codigo-python-sn}
lista.sort(key=Hotel.ratio)
\end{codigo-python-sn}

Así, para ordenar según el precio:

\begin{codigo-python-sn}
>>> def precio(hotel):
...     return hotel.precio
>>> lista.sort(key=precio)
>>> for hotel in lista:
...     print(hotel)
...
Hotel 1* normal de MDQ - Puntaje: 1 - Precio: 10 pesos.
Hotel 2* normal de MDQ - Puntaje: 2 - Precio: 40 pesos.
Hotel 3* carisimo de MDQ - Puntaje: 3 - Precio: 130 pesos.
Hotel vale la pena de MDQ - Puntaje: 4 - Precio: 130 pesos.
\end{codigo-python-sn}

\subsection{Comparación sólo por igualdad o desigualdad}

Existen clases, como la clase \lstinline!Punto! vista anteriormente, que no
se pueden ordenar, ya que no se puede decir si dos puntos son menores o
mayores, con lo cual no se puede implementar un método \lstinline!__lt__!.

Pero en estas clases, en general, será posible comparar si dos objetos son
o no iguales, es decir si tienen o no el mismo valor, aún si se trata de
objetos distintos.

\begin{codigo-python-sn}
>>> p = Punto(3, 4)
>>> q = Punto(3, 4)
>>> p == q
False
\end{codigo-python-sn}

En este caso, por más que los puntos tengan el mismo valor, al no estar
definido ningún método de comparación Python no sabe cómo comparar los
valores, y lo que compara son las instancias.  \lstinline!p! y \lstinline!q!
son instancias distintas, por más que tengan los mismos valores.

Para obtener el comportamiento esperado en estos casos, se redefine el
método \lstinline!__eq__!. Por ejemplo, el método |__eq__| para la clase
|Punto| sería:

\begin{codigo-python-sn}
    def __eq__(self, otro):
        """Devuelve si dos puntos son iguales."""
        return self.x == otro.x and self.y == otro.y
\end{codigo-python-sn}

Una vez agregados estos métodos ya se puede comparar los puntos por su
igualdad o desigualdad:

\begin{codigo-python-sn}
>>> p = Punto(3, 4)
>>> q = Punto(3, 4)
>>> p == q
True
>>> p != q
False
>>> r = Punto(2, 3)
>>> p == r
False
>>> p != r
True
\end{codigo-python-sn}

\section{Resumen}

\begin{itemize}
\item Los {\bf objetos} son formas ordenadas de agrupar datos
(\emph{atributos}) y operaciones sobre estos datos (\emph{métodos}).

\item Cada objeto es de una {\bf clase} o tipo, que define cuáles
serán sus atributos y métodos. Y cuando se crea un valor a partir de una
clase en particular, decimos que se crea una {\bf instancia} de esa clase.

\item Para nombrar una clase definida por el programador, se suele usar una letra
mayúscula al comienzo de cada palabra.

\item El {\bf constructor} de una clase es el método que se ejecuta cuando
se crea una nueva instancia de esa clase.

\item Es posible definir una gran variedad de métodos dentro de una
clase, incluyendo métodos especiales que pueden utilizados para
mostrar, sumar, comparar u ordenar los objetos.
\end{itemize}

\begin{referencia_python}

\begin{sintaxis}{\lstinline!class NombreClase:!}
Indica que se comienza a definir una clase con el nombre indicado.
\end{sintaxis}

\begin{sintaxis}{\lstinline!def __init__(self, ...):!}
Define el \emph{constructor} de la clase.  En general, dentro del
constructor se establecen los valores iniciales de todos los
atributos.
\end{sintaxis}

\begin{sintaxis}{\lstinline!variable = NombreClase(...)!}
Crea una nueva instancia de la clase. Los
parámetros que se ingresen serán pasados al constructor, luego del
parámetro especial \lstinline!self!.
\end{sintaxis}

\begin{sintaxis}{\lstinline!variable.atributo!}
Permite obtener o modificar el valor de un atributo de la instancia.
\end{sintaxis}

\begin{sintaxis}{\lstinline!def metodo(self, ...)!}
El primer parámetro de cada método de una clase es una referencia a la
instancia sobre la que va a operar el método.  Se lo llama por
convención \lstinline!self!, pero puede tener cualquier nombre.
\end{sintaxis}

\begin{sintaxis}{\lstinline!variable.metodo(...)!}
Invoca al método \lstinline!metodo! de la clase de la cual
\lstinline!variable! es una instancia.  El primer parámetro que se le
pasa a \lstinline!metodo! será |self| $\rightarrow$ \lstinline!variable!.
\end{sintaxis}

\begin{sintaxis}{\lstinline!def __str__(self):!}
Método especial que debe devolver una cadena de caracteres, con
la representación ``informal'' de la instancia. Se invoca al hacer
|str(variable)| o |print(variable)|.
\end{sintaxis}

\begin{sintaxis}{\lstinline!def __repr__(self):!}
Método especial que debe devolver una cadena de caracteres, con
la representación ``formal'' de la instancia. Se invoca al hacer
|repr(variable)|.
\end{sintaxis}

\begin{sintaxis}{\lstinline!def __add__(self, otro):, def __sub__(self, otro):!}
Métodos especiales para sobrecargar los operadores \lstinline!+! y
\lstinline!-! respectivamente.  Reciben las dos instancias sobre las
que se debe operar, debe devolver una nueva instancia con el
resultado.
\end{sintaxis}

\begin{sintaxis}{\lstinline!def __lt__(self, otro):!}
Método especial para permitir la comparación de objetos mediante los
operadores |<| y |>|.  Recibe las dos instancias a
comparar. Debe devolver |True| si |self| es comparativamente
``menor'' a |otro|.
\end{sintaxis}

\begin{sintaxis}{\lstinline!def __le__(self, otro):!}
Método especial para permitir la comparación de objetos mediante los
operadores |<=| y |>=|. Devuelve |True| si |self| es comparativamente
``menor o igual'' a |otro|.
\end{sintaxis}

\begin{sintaxis}{\lstinline!def __eq__(self, otro):!}
Método especial para permitir la comparación de objetos mediante los operadores
|==| y |!=|. Devuelve |True| si |self| y |otro| son comparativamente
``iguales''.
\end{sintaxis}

\end{referencia_python}

\newpage
\section{Ejercicios}

\begin{ejercicio}
Modificar el método \lstinline!__lt__! de \lstinline!Hotel!
para poder ordenar de menor a mayor las listas de hoteles según el criterio:
primero por ubicación, en orden alfabético y dentro de cada ubicación por
la relación calidad-precio.
\end{ejercicio}

\begin{ejercicio}
Escribir una clase \lstinline!Caja! para representar cuánto
dinero hay en una caja de un negocio, desglosado por tipo de billete (por
ejemplo, en el quiosco de la esquina hay 5 billetes de 10 pesos, 7 monedas
de 25 centavos y 4 monedas de 10 centavos).

Se tiene que poder comparar cajas por la cantidad de dinero que
hay en cada una, y además ordenar una lista de cajas
de menor a mayor según la cantidad de dinero disponible.
\end{ejercicio}

\extractionlabel{guia}
\begin{ejercicio}
Fracciones
\begin{partes}
    \item Crear una clase \verb!Fraccion!, que cuente con dos atributos:
\verb!dividendo! y \verb!divisor!, que se asignan en el constructor, y se
imprimen como \verb!X/Y! en el método \verb!__str__!.
    \item Crear un método \verb!sumar! que recibe otra fracción y devuelve una
nueva fracción con la suma de ambas.
    \item Crear un método \verb!multiplicar! que recibe otra fracción y
devuelve una nueva fracción con el producto de ambas.
    \item Crear un método \verb!simplificar! que modifica la fracción actual de
forma que los valores del \verb!dividendo! y \verb!divisor! sean los
menores posibles.
\end{partes}
\end{ejercicio}


\extractionlabel{guia}
\begin{ejercicio}
Vectores
\begin{partes}
    \item Crear una clase \verb!Vector!, que en su constructor reciba una lista
de elementos que serán sus coordenadas.  En el método \verb!__str__! se
imprime su contenido con el formato \verb![x,y,z]!
    \item Crear un método \verb!escalar! que reciba un número y devuelva un
nuevo vector, con los elementos multiplicados por ese número.
    \item Crear un método \verb!sumar! que recibe otro vector, verifica si
tienen la misma cantidad de elementos y devuelve un nuevo vector con la
suma de ambos.  Si no tienen la misma cantidad de elementos debe levantar
una excepción.
\end{partes}
\end{ejercicio}


\extractionlabel{guia}
\begin{ejercicio}
Botella y Sacacorchos
\begin{partes}
    \item Escribir una clase \emph{Corcho}, que contenga un atributo
\emph{bodega} (cadena con el nombre de la bodega).
    \item Escribir una clase \emph{Botella} que contenga un atributo
\emph{corcho} con una referencia al corcho que la tapa, o \verb!None! si está
destapada.
    \item Escribir una clase \emph{Sacacorchos} que tenga un método
\emph{destapar} que le reciba una botella, le saque el corcho y se guarde una
referencia al corcho sacado.  Debe lanzar una excepción en el caso en que
la botella ya esté destapada, o si el sacacorchos ya contiene un corcho.
    \item Agregar un método \emph{limpiar}, que saque el corcho del sacacorchos,
o lance una excepción en el caso en el que no haya un corcho.
\end{partes}
\end{ejercicio}


\extractionlabel{guia}
\begin{ejercicio}
Modelar una clase \emph{Mate} que describa el funcionamiento de la
conocida bebida tradicional local. La clase debe contener como miembros:
\begin{partes}
    \item Un atributo para la cantidad de cebadas restantes hasta que se lava
el mate (representada por un número).
    \item Un atributo para el estado (lleno o vacío).
    \item El constructor debe recibir como parámetro \verb!n!, la cantidad
máxima de cebadas en base a la cantidad de yerba vertida en el recipiente.
    \item Un método \verb!cebar!, que llena el mate con agua. Si se intenta
cebar con el mate lleno, se debe lanzar una excepción que imprima el
mensaje \lstinline|'Cuidado! Te quemaste!'|
    \item Un método \verb!beber!, que vacía el mate y le resta una cebada
disponible. Si se intenta beber un mate vacío, se debe lanzar una excepción
que imprima el mensaje \lstinline|'El mate está vacío!'|
    \item Es posible seguir cebando y bebiendo el mate aunque no haya cebadas
disponibes. En ese caso la cantidad de cebadas restantes se mantendrá
en 0, y cada vez que se intente beber se debe imprimir un mensaje de
aviso: \lstinline|'Advertencia: el mate está lavado.'|, pero no se debe lanzar una
excepción.
\end{partes}
\end{ejercicio}

