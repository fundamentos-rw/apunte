\chapter{Diccionarios}

En esta unidad analizaremos otra estructura de datos importante: los diccionarios.
Su importancia radica no sólo en las grandes posibilidades que presentan
como estructuras para almacenar información, sino también en que, en
Python, son utilizados por el propio lenguaje para realizar diversas
operaciones y para almacenar información de otras estructuras.

\section{Qué es un diccionario}

Según Wikipedia, ``[u]n diccionario es una obra de consulta de
palabras y/o términos que se encuentran generalmente ordenados
alfabéticamente. De dicha compilación de palabras o términos se
proporciona su significado, etimología, ortografía y, en el caso
de ciertas lenguas fija su pronunciación y separación silábica.''


Al igual que los diccionarios a los que se refiere Wikipedia, y
que usamos habitualmente en la vida diaria, los diccionarios de
Python son una lista de consulta de términos de los cuales se
proporcionan valores asociados. A diferencia de los diccionarios a
los que se refiere Wikipedia, los diccionarios de Python no están
ordenados.


En Python, un diccionario es una colección no-ordenada de valores
que son accedidos a traves de una clave.  Es decir, en lugar de
acceder a la información mediante el índice numérico, como es el
caso de las listas y tuplas, es posible acceder a los valores a
través de sus claves, que pueden ser de diversos tipos.

Las claves son únicas dentro de un diccionario, es decir que no puede haber
un diccionario que tenga dos veces la misma clave, si se asigna un valor a
una clave ya existente, se reemplaza el valor anterior.

No hay una forma directa de acceder a una clave a través de su valor, y
nada impide que un mismo valor se encuentre asignado a distintas claves.

La informacion almacenada en los diccionarios no tiene un orden
particular.  Ni por clave ni por valor, ni tampoco por el orden en
que han sido agregados al diccionario.

Al igual que las listas, los diccionarios son mutables. Esto significa que
podemos agregar, quitar y modificar los elementos de un diccionario
posteriormente a su creación.

Cualquier valor de tipo inmutable puede ser clave de un diccionario:
cadenas, enteros, tuplas (con valores inmutables en sus miembros), etc.  No hay
restricciones para los valores que el diccionario puede contener, cualquier
tipo puede ser el valor: listas, cadenas, tuplas, otros diccionarios,
etc.

\begin{sabias_que}
En otros lenguajes, a los diccionarios se los llama \emph{arreglos asociativos},
\emph{mapas}, \emph{tablas} o también \emph{tablas de hash}.
\end{sabias_que}

\section{Utilizando diccionarios en Python}

De la misma forma que con listas, es posible definir un diccionario
directamente con los miembros que va a contener, o bien inicializar el
diccionario vacío y luego agregar los valores de a uno o de a muchos.

Para definirlo junto con los miembros que va a contener, se encierra el
listado de valores entre llaves, las parejas de clave y valor se separan
con comas, y la clave y el valor se separan con ':'.

\begin{codigo-python-sn}
punto = {'x': 2, 'y': 1, 'z': 4}
\end{codigo-python-sn}

\begin{observacion}
En Python el tipo de dato asociado a los diccionarios se llama |dict|:

\begin{codigo-python-sn}
>>> type(punto)
<class 'dict'>
\end{codigo-python-sn}
\end{observacion}

Para declararlo vacío y luego ingresar los valores, se lo declara como un
par de llaves sin nada en medio, y luego se asignan valores directamente a
los índices.

\begin{codigo-python-sn}
materias = {}
materias["lunes"] = [6103, 7540]
materias["martes"] = [6201]
materias["miércoles"] = [6103, 7540]
materias["jueves"] = []
materias["viernes"] = [6201]
\end{codigo-python-sn}

Para acceder al valor asociado a una determinada clave, se lo hace
de la misma forma que con las listas, pero utilizando la clave
elegida en lugar del índice.

\begin{codigo-python-sn}
>>> materias["lunes"]
[6103, 7540]
\end{codigo-python-sn}

\begin{atencion}
El acceso por clave falla si se provee una clave que no está en el diccionario:

\begin{codigo-python-sn}
>>> materias["domingo"]
Traceback (most recent call last):
  File "<stdin>", line 1, in <module>
KeyError: 'domingo'
\end{codigo-python-sn}
\end{atencion}

El operador |in| nos permite preguntar si una clave se encuentra o no en el
diccionario:

\begin{codigo-python-sn}
>>> "lunes" in materias
True
>>> "domingo" in materias
False
\end{codigo-python-sn}

Además podemos utilizar la función \lstinline{get}, que recibe una
clave $k$ y un valor por omisión $v$, y devuelve el valor asociado a la clave
$k$, en caso de existir, o el valor $v$ en caso contrario.

\begin{codigo-python-sn}
>>> materias.get("lunes", [])
[6103, 7540]
>>> materias.get("domingo", [])
[]
\end{codigo-python-sn}

Existen diversas formas de recorrer un diccionario.  Es posible recorrer
sus claves y usar esas claves para acceder a los valores.

\begin{codigo-python-sn}
for dia in materias:
    print("El {} tengo que cursar {}".format(dia, materias[dia])
\end{codigo-python-sn}

Es posible, también, obtener los valores como tuplas donde el primer
elemento es la clave y el segundo el valor.

\begin{codigo-python-sn}
for dia, codigos in materias.items():
    print("El {} tengo que cursar {}".format(dia, codigos)
\end{codigo-python-sn}

Hay muchas otras operaciones que se
pueden realizar sobre los diccionarios, que permiten manipular la información
según sean nuestras necesidades. Algunos de estos métodos pueden verse en
la referencia al final de la unidad.

\begin{sabias_que}
El algoritmo que usa Python internamente para buscar un elemento en un
diccionario es muy distinto que el que utiliza para buscar en listas.

Para buscar en las listas, se utiliza un algoritmos de comparación que
tarda cada vez más a medida que la lista se hace más larga.  En cambio,
para buscar en diccionarios se utiliza un algoritmo llamado \emph{hash},
que se basa en realizar un cálculo numérico sobre la clave del elemento,
y tiene una propiedad muy interesante: sin importar cuántos elementos
tenga el diccionario, el tiempo de búsqueda es siempre aproximadamente
igual.

Este algoritmo de \emph{hash} es también la razón por la cual las claves de
los diccionarios deben ser inmutables, ya que la operación hecha sobre las
claves debe dar siempre el mismo resultado, y si se utilizara una variable
mutable esto no sería posible.
\end{sabias_que}

No es posible obtener porciones de un diccionario usando \lstinline![:]!,
ya que al no tener un orden determinado para los elementos, no sería
posible tomarlos en orden.

\section{Algunos usos de diccionarios}

Los diccionarios son una herramienta muy versátil.  Se puede utilizar un
diccionario, por ejemplo, para contar cuántas apariciones de cada palabra
hay en un texto, o cuántas apariciones de cada letra.

Es posible utilizar un diccionario, también, para tener una agenda donde la
clave es el nombre de la persona, y el valor es una lista con los datos
correspondientes a esa persona.

También podría utilizarse un diccionario para mantener los datos de los
alumnos inscriptos en una materia, siendo la clave el número de padrón, y
el valor una lista con todas las notas asociadas a ese alumno.

En general, los diccionarios sirven para crear bases de datos muy simples,
en las que la clave es el identificador del elemento, y el valor son todos
los datos del elemento a considerar.

Otro posible uso de un diccionario sería para realizar
traducciones, donde la clave sería la palabra en el idioma original y el
valor la palabra en el idioma al que se quiere traducir.  Sin embargo esta
aplicación es poco destacable, ya que esta forma de traducir suele dar
resultados poco satisfactorios.

\section{Resumen}

\begin{itemize}
\item Los diccionarios (también llamados \emph{arreglos asociativos},
\emph{mapas} o \emph{tablas de hash} en otros lenguajes), son una estructura de datos
muy poderosa, que permite asociar un valor a una clave.
\item Las claves deben ser de tipo inmutable, los valores
pueden ser de cualquier tipo.
\item Los diccionarios no están ordenados.  Si bien se los puede recorrer,
el orden en el que se tomarán los elementos no está determinado.
\end{itemize}

\begin{referencia_python}

\begin{sintaxis}{\lstinline!\{clave1:valor1, clave2:valor2\}!}
Se crea un nuevo diccionario con los valores asociados a las claves.  Si no
se ingresa ninguna pareja de clave y valor, se crea un diccionario vacío.
\end{sintaxis}

\begin{sintaxis}{\lstinline{diccionario[clave]}}
Accede al valor asociado con \lstinline!clave! en el diccionario. Falla si la
clave no está en el diccionario.
\end{sintaxis}

\begin{sintaxis}{\lstinline{clave in diccionario}}
Indica si un diccionario tiene o no una determinada clave.
\end{sintaxis}

\begin{sintaxis}{\lstinline{diccionario.get(clave, valor_predeterminado)}}
Devuelve el valor asociado a la clave.  A diferencia del acceso directo
utilizando \lstinline{[clave]}, en el caso en que el valor no se
encuentre devuelve el |valor_predeterminado|.
\end{sintaxis}

\begin{sintaxis}{\lstinline{for clave in diccionario:}}
Permite recorrer una a una todas las claves almacenadas en
el diccionario.
\end{sintaxis}

\begin{sintaxis}{\lstinline{diccionario.keys()}}
Devuelve una lista desordenada, con todas las claves que se hayan ingresado
al diccionario
\end{sintaxis}

\begin{sintaxis}{\lstinline{diccionario.values()}}
Devuelve una lista desordenada, con todos los valores que se hayan
ingresado al diccionario.
\end{sintaxis}

\begin{sintaxis}{\lstinline{diccionario.items()}}
Devuelve una lista desordenada con tuplas de dos elementos, en las que el
primer elemento es la clave y el segundo el valor.
\end{sintaxis}

\begin{sintaxis}{\lstinline{diccionario.pop(clave)}}
Quita del diccionario la clave y su valor asociado, y devuelve el valor.
\end{sintaxis}
\end{referencia_python}


\newpage
\section{Ejercicios}

\extractionlabel{guia}
\begin{ejercicio}
Escribir una función que reciba una lista de tuplas, y que devuelva
un diccionario en donde las claves sean los primeros elementos de las
tuplas, y los valores una lista con los segundos.

Por ejemplo:
\begin{lstlisting}[numbers=none]
>>> l = [ ('Hola', 'don Pepito'), ('Hola', 'don Jose'),
          ('Buenos', 'días') ]
>>> print(tuplas_a_diccionario(l))
{ 'Hola': ['don Pepito', 'don Jose'], 'Buenos': ['días'] }
\end{lstlisting}
\end{ejercicio}

\extractionlabel{guia}
\begin{ejercicio}
{\bf Diccionarios usados para contar.}
\begin{partes}
  \item Escribir una función que reciba una cadena y devuelva un diccionario con
la cantidad de apariciones de cada palabra en la cadena.  Por ejemplo, si
recibe "Qué lindo día que hace hoy" debe devolver:
\lstinline!{ 'que': 2, 'lindo': 1, 'día': 1, 'hace': 1, 'hoy': 1}!.

  \item Escribir una función que cuente la cantidad de apariciones de cada
caracter en una cadena de texto, y los devuelva en un diccionario.

  \item Escribir una función que reciba una cantidad de iteraciones de una tirada
de 2 dados a realizar y devuelva la cantidad de veces que se observa cada valor
de la suma de los dos dados. \\
{\bf Nota}: utilizar el módulo \verb!random! para obtener tiradas aleatorias.
\end{partes}
\end{ejercicio}

\extractionlabel{guia}
\begin{ejercicio}
{\bf Continuación de la agenda.} \\
Escribir un programa que vaya solicitando al usuario que ingrese nombres.
\begin{partes}
  \item Si el nombre se encuentra en la agenda (\emph{implementada con un
diccionario}), debe mostrar el teléfono y, opcionalmente, permitir
modificarlo si no es correcto.
  \item Si el nombre no se encuentra, debe permitir ingresar el teléfono
correspondiente.
\end{partes}
El usuario puede utilizar la cadena "*", para salir del programa.
\end{ejercicio}

\extractionlabel{guia}
\begin{ejercicio}
Escribir una función que reciba un texto y para cada caracter presente en el
texto devuelva la cadena más larga en la que se encuentra ese caracter.
\end{ejercicio}

