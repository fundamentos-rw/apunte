\chapter{Más sobre ciclos}

El último problema analizado en la unidad anterior decía:

\begin{quote}
Leer un número. Si el número es positivo escribir un mensaje
``Numero positivo'', si el número es igual a 0 un mensaje ``Igual a
0'', y si el número es negativo escribir un mensaje ``Numero
negativo''.
\end{quote}

Se nos plantea a continuación un nuevo problema, similar al anterior:

\problemac{El usuario debe poder ingresar muchos números y
cada vez que se ingresa uno debemos informar si es positivo, cero o
negativo.} \\

Utilizando los ciclos definidos vistos en las primeras unidades, es posible
preguntarle al usuario cada vez, al inicio del programa, cuántos números va a
ingresar para consultar. La solución propuesta resulta:

\begin{codigo-python-sn}
def muchos_pcn():
    (@i = int(input("Cuantos numeros quiere procesar?: "))@)
    (@for j in range(0, i):@)
        x = int(input("Ingrese un numero: "))
        if x > 0:
            print("Numero positivo")
        elif x == 0:
            print("Igual a 0")
        else:
            print("Numero negativo")
\end{codigo-python-sn}

Su ejecución es exitosa:

\begin{codigo-python-sn}
>>> muchos_pcn()
Cuantos numeros quiere procesar: 3
Ingrese un numero: 25
Numero positivo
Ingrese un numero: 0
Igual a 0
Ingrese un numero: -5
Numero negativo
>>>
\end{codigo-python-sn}

Sin embargo, el uso de este programa no resulta muy intuitivo, porque
obliga al usuario a contar de antemano cuántos números va a querer procesar, sin
equivocarse, en lugar de ingresar uno a uno los números hasta procesarlos a
todos.

\section{Ciclos indefinidos}

Para poder resolver este problema sin averiguar primero la cantidad de números
a procesar, debemos introducir una instrucción que nos permita construir ciclos
que no requieran que se informe de antemano la cantidad de veces que se
repetirá el cálculo del cuerpo.  Se trata de \emph{ciclos indefinidos} en los
cuales se repite el cálculo del cuerpo mientras una cierta condición es
verdadera.

Un ciclo indefinido es de la forma

\begin{codigo-python-sn}
while <condición>:
    <hacer algo>
\end{codigo-python-sn}

Donde \lstinline+while+ es una palabra reservada, y la condición es una expresión
booleana, igual que en las instrucciones \lstinline!if!. El cuerpo es, como
siempre, una o más instrucciones de Python.

El funcionamiento de esta instrucción es el siguiente:

\begin{enumerate}
\item Evaluar la condición.
\item Si la condición es falsa, salir del ciclo.
\item Si la condición es verdadera, ejecutar el cuerpo.
\item Volver a 1.
\end{enumerate}

\section{Ciclo interactivo}

¿Cuál es la condición y cuál es el cuerpo del ciclo en nuestro problema?
Claramente, el cuerpo del ciclo es el ingreso de datos y la verificación de si
es positivo, negativo o cero.  En cuanto a la condición, es que haya más datos
para seguir calculando.

Definimos una variable \lstinline!hay_mas_datos!, que valdrá ``Si'' mientras
haya datos.

Se le debe preguntar al usuario, después de cada cálculo, si hay o no más datos.
Cuando el usuario deje de responder ``Si'', dejaremos de ejecutar el cuerpo del
ciclo.

Una primera aproximación al código necesario para resolver este problema podría
ser:

\begin{codigo-python-sn}
def muchos_pcn():
    (@while hay_mas_datos == "Si":@)
        (@x = int(input("Ingrese un numero: "))@)
        if x > 0:
            print("Numero positivo")
        elif x == 0:
            print("Igual a 0")
        else:
            print("Numero negativo")

        (@hay_mas_datos = input("¿Quiere seguir? <Si-No>: ")@)
\end{codigo-python-sn}

Veamos qué pasa si ejecutamos la función tal como fue presentada:

\begin{codigo-python-sn}
>>> muchos_pcn()
(^Traceback (most recent call last):
  File "<pyshell#25>", line 1, in <module>
    muchos_pcn()
  File "<pyshell#24>", line 2, in muchos_pcn
    while hay_mas_datos == "Si":
UnboundLocalError: local variable 'hay_mas_datos' referenced before assignment^)
\end{codigo-python-sn}

El problema que se presentó en este caso, es que \lstinline!hay_mas_datos! no
tiene un valor asignado en el momento de evaluar la condición del ciclo por
primera vez.

\begin{observacion}
Es importante prestar atención a cuáles son las variables que hay que
inicializar antes de ejecutar un ciclo, para asegurar que la expresión
booleana que lo controla sea evaluable.
\end{observacion}

Una posibilidad es preguntarle al usario, antes de evaluar la condición, si
tiene datos; otra posibilidad es suponer que si llamó a este programa es porque
tenía algún dato para calcular, y darle el valor inicial ``Si'' a
\lstinline!hay_mas_datos!.

Encararemos la segunda opción:

\begin{codigo-python-sn}
def muchos_pcn():
    (@hay_mas_datos = "Si"@)
    while hay_mas_datos == "Si":
        x = int(input("Ingrese un numero: "))
        if x > 0:
            print("Numero positivo")
        elif x == 0:
            print("Igual a 0")
        else:
            print("Numero negativo")

        hay_mas_datos = input("Quiere seguir? <Si-No>: ")
\end{codigo-python-sn}

El esquema del ciclo interactivo es el siguiente:

\begin{codigo-nohl-sn}
hay_mas_datos hace referencia a "Si"
Mientras hay_mas_datos haga referencia a "Si":
    Pedir datos
    Realizar cálculos
    Preguntar al usuario si hay más datos ("Si" cuando los hay)
    hay_mas_datos hace referencia al valor ingresado
\end{codigo-nohl-sn}

Ésta es una ejecución:

\begin{codigo-python-sn}
>>> muchos_pcn()
Ingrese un numero: 25
Numero positivo
Quiere seguir? <Si-No>: "Si"
Ingrese un numero: 0
Igual a 0
Quiere seguir? <Si-No>: "Si"
Ingrese un numero: -5
Numero negativo
Quiere seguir? <Si-No>: "No"
\end{codigo-python-sn}

\section{Ciclo con centinela}
\label{centinela}

Un problema que tiene nuestra primera solución es que resulta poco amigable
preguntarle al usuario después de cada cálculo si desea continuar. Para evitar
esto, se puede usar el método del \emph{centinela}: un valor arbitrario que, si
se lee, le indica al programa que el usuario desea salir del ciclo. En este
caso, podemos suponer que si el usuario ingresa el caracter \lstinline!*!, es
una indicación de que desea terminar.

El esquema del ciclo con centinela es el siguiente:

\begin{codigo-nohl-sn}
Pedir datos
Mientras el dato pedido no coincida con el centinela:
    Realizar cálculos
    Pedir datos
\end{codigo-nohl-sn}

El programa resultante es el siguiente:

\begin{codigo-python-sn}
def muchos_pcn():
    (@centinela = input("Ingrese un numero (* para terminar): ")@)

    (@while centinela != "*":@)
        (@x = int(centinela)@)
        if x > 0:
            print("Numero positivo")
        elif x == 0:
            print("Igual a 0")
        else:
            print("Numero negativo")

        (@centinela = input("Ingrese un numero (* para terminar): ")@)
\end{codigo-python-sn}

Notar que no podemos hacer |centinela = int(input(...))| porque cuando el
usuario ingrese |'*'| la llamada a |int| fallaría (al no poder convertir |'*'|
a un valor entero). Por eso es que por un lado hacemos la llamada a |input|, y
una vez que sabemos que el valor |centinela| no es un |'*'|, lo convertimos a
entero llamando a |int|.

Y ahora lo ejecutamos:

\begin{codigo-python-sn}
>>> muchos_pcn()
Ingrese un numero (* para terminar): 25
Numero positivo
Ingrese un numero (* para terminar): 0
Igual a 0
Ingrese un numero (* para terminar): -5
Numero negativo
Ingrese un numero (* para terminar): *
\end{codigo-python-sn}

El ciclo con centinela es muy claro pero tiene un problema: hay una línea de
código repetida, que es la línea en donde el usuario ingresa el dato:

\begin{codigo-python-sn}
centinela = input("Ingrese un numero (* para terminar): ")
\end{codigo-python-sn}

Si en la etapa de mantenimiento tuviéramos que realizar un
cambio en el ingreso del dato (por ejemplo, cambiar el mensaje) deberíamos
estar atentos y corregir ambas líneas. En principio no parece ser un problema
muy grave, pero a medida que el programa y el código se hacen más complejos,
se hace mucho más difícil llevar la cuenta de todas las líneas de código
duplicadas, y por lo tanto se hace mucho más fácil cometer el error de cambiar
una de las líneas y olvidar hacer el cambio en la línea duplidada.

\begin{observacion}
El código duplicado suele incrementar el esfuerzo necesario para hacer
modificaciones en la etapa de mantenimiento. Es conveniente prestar atención en
a etapa de implementación, y modificar el código para eliminar la duplicación.
\end{observacion}

Veamos cómo eliminar el código duplicado en nuestro ejemplo.
Lo ideal sería leer el dato |centinela| en un único punto del
programa. Una opción es \emph{extraer} el código duplicado en una función:

\begin{codigo-python-sn}
(@def leer_centinela():@)
    (@return input("Ingrese un numero (* para terminar): ")@)

def muchos_pcn():
    (@centinela = leer_centinela()@)
    while centinela != "*":
        x = int(centinela)
        if x > 0:
            print("Numero positivo")
        elif x == 0:
            print("Igual a 0")
        else:
            print("Numero negativo")

        (@centinela = leer_centinela()@)
\end{codigo-python-sn}

Esta implementación es mejor que la anterior: si tuviéramos que cambiar el
mensaje sólo tendríamos que modificar una línea de código. Pero aun tenemos que
recordar inicializar la variable |centinela| antes de comenzar el ciclo. ¿Habrá
alguna manera de evitarlo?

\section{Cómo cortar un ciclo}

Un ciclo que no puede depender de valores leídos o calculados dentro
de él será de la forma:

\begin{codigo-nohl-sn}
Repetir indefinidamente:
    Hacer algo
\end{codigo-nohl-sn}

Y esto se traduce a Python como:

\begin{codigo-python-sn}
while True:
    <hacer algo>
\end{codigo-python-sn}

Un ciclo cuya condición es \lstinline!True! parece ser un ciclo infinito (o
sea que nunca va a terminar). ¡Pero eso es gravísimo! ¡Nuestros programas
tienen que terminar!

Afortunadamente hay una instrucción de Python, \lstinline!break!, que nos
permite salir de adentro de un ciclo (tanto sea \lstinline!for! como
\lstinline!while!) en medio de su ejecución.

En esta construcción

\begin{codigo-python-sn}
while <condicion_while>:
    <hacer algo_1>
    if <condicion_if>:
        break
    <hacer algo_2>
\end{codigo-python-sn}

El funcionamiento de la instrucción \lstinline!break! es el siguiente:

\begin{enumerate}
\item Se evalúa la |<condicion_while>| y si es falsa se sale del ciclo.
\item Se ejecuta |<hacer_algo_1>|.
\item Se evalúa la |<condicion_if>| y si es verdadera se sale del ciclo (al
    llegar a la sintrucción \lstinline!break!).
\item Se ejecuta |<hacer_algo_2>|.
\item Se vuelve al paso 1.
\end{enumerate}

Diseñamos entonces:

\begin{codigo-nohl-sn}
Repetir indefinidamente:
    Pedir dato
    Si el dato ingresado es el centinela, salir del ciclo
    Operar con el dato
\end{codigo-nohl-sn}

Codificamos en Python la solución al problema de los números usando ese
esquema:

\begin{codigo-python-sn}
def muchos_pcn():
    (@while True:@)
        (@centinela = input("Ingrese un numero ('*' para terminar): ")@)
        (@if centinela == '*':@)
            (@break@)

        x = int(centinela)
        if x > 0:
            print("Numero positivo")
        elif x == 0:
            print("Igual a 0")
        else:
            print("Numero negativo")
\end{codigo-python-sn}

\subsection{Otras formas de controlar el flujo de un ciclo}

Ya vimos que la instrucción |break| permite cortar la ejecución de un ciclo:

\begin{observacion}
Cuando la ejecución del cuerpo de un ciclo |for| o |while| llega a una
instrucción |break|, la ejecución ``sale'' del ciclo inmediatamente.

\begin{codigo-python-sn}
def f():
    ...
    while ...:
        ...
        if ...:
            (@break@)
        ...
    ... # (~$\leftarrow$~) luego del break, la ejecución continúa aquí
\end{codigo-python-sn}
\end{observacion}

Hay otras dos instrucciones que permiten modificar el flujo del ciclo. Una de
ellas es la instrucción |continue|:

\begin{observacion}
Cuando la ejecución del cuerpo de un ciclo |while| llega a una
instrucción |continue|, la ejecución ``saltea'' la iteración actual del ciclo y
pasa a la siguiente.

\begin{codigo-python-sn}
def f():
    ...
    while ...: # (~$\leftarrow$~) luego del continue, la ejecución continúa aquí
        ...
        if ...:
            (@continue@)
        ...
    ...
\end{codigo-python-sn}
\end{observacion}

Si la instrucción |continue| aparece dentro de un ciclo |while|, la condición
del ciclo vuelve a evaluarse, y se sale del ciclo si es falsa.

Si la instrucción |continue| aparece dentro de un ciclo |for|, la variable de
control del ciclo tomará el siguiente valor del rango de iteración.

\begin{codigo-python-sn}
for x in range(10): # (~$\leftarrow$~) luego del continue, la ejecución continúa aquí,
                    # y x tomará el siguiente valor del rango
    print("x vale", x)
    if x % 2 == 0:
        (@continue@)
    print("x es impar")
\end{codigo-python-sn}

Otra instrucción que puede cortar la ejecución del ciclo es |return|:

\begin{observacion}
La instrucción |return| puede aparecer únicamente dentro de una función. Si
además está dentro del cuerpo de un bucle |while| o |for|, cortará la ejecución
del ciclo y de la función.

\begin{codigo-python-sn}
def f():
    ...
    while ...:
        ...
        if ...:
            (@return@)
        ...
    ...

f()
# (~$\leftarrow$~) luego del return, la ejecución continúa aquí
\end{codigo-python-sn}
\end{observacion}

\begin{sabias_que}
Desde hace mucho tiempo los ciclos infinitos vienen trayéndoles dolores de
cabeza a los programadores.  Cuando un programa deja de responder y se
utiliza todos los recursos de la computadora, suele deberse a que entró 
en un ciclo del que no puede salir.

Estos bucles pueden aparecer por una gran variedad de causas.  A
continuación algunos ejemplos de ciclos de los que no se puede salir,
siempre o para ciertos parámetros.  Queda como ejercicio encontrar el error
en cada uno.

\begin{codigo-python-sn}
def menor_factor_primo(x):
    """Devuelve el menor factor primo del número x."""
    n = 2
    while n <= x:
        if x % n == 0:
            return n
\end{codigo-python-sn}

\begin{codigo-python-sn}
def buscar_impar(x):
    """Divide el número recibido por 2 hasta que sea impar."""
    while x % 2 == 0:
        x = x / 2
    return x
\end{codigo-python-sn}
\end{sabias_que}

\section{Ejercicios}

\ejercicioc{Nuevamente, se desea facturar el uso de un teléfono.  Para ello
se informa la tarifa por segundo y la duración de cada comunicación
expresada en horas, minutos y segundos.  Como resultado se informa la
duración en segundos de cada llamado y su costo. Resolver este problema 
usando

\begin{enumerate}
\item Un ciclo definido.
\item Un ciclo interactivo.
\item Un ciclo con centinela.
\item Un ciclo ``infinito'' que se corta.
\end{enumerate}
}

\ejercicioc{Mantenimiento del tarifador: al final del día se debe informar
cuántas llamadas hubo y el total facturado. Hacerlo con todos los esquemas
anteriores.}

\ejercicioc{Nos piden que escribamos una función que le pida al usuario que
ingrese un número positivo. Si el usuario ingresa cualquier cosa que no sea
lo pedido se le debe informar de su error mediante un mensaje y volverle a
pedir el número.

Resolver este problema usando

\begin{enumerate}
\item Un ciclo interactivo.
\item Un ciclo con centinela.
\item Un ciclo ``infinito'' que se corta.
\end{enumerate}

¿Tendría sentido hacerlo con ciclo definido? Justificar.
}

\newpage
\section{Resumen}
\label{ciclos:resumen}

\begin{itemize}

\item Además de los ciclos definidos, en los que se sabe cuáles son los
posibles valores que tomará una determinada variable, existen los ciclos
indefinidos, que se terminan cuando no se cumple una determinada condición.

\item La condición que termina el ciclo puede estar relacionada con una entrada
de usuario o depender del procesamiento de los datos.

\item Se puede utilizar el método del \emph{centinela} cuando se quiere que un ciclo se
repita hasta que el usuario indique que no quiere continuar.

\item Además de la condición que hace que el ciclo se termine, es posible
interrumpir su ejecución con código específico dentro del ciclo.

\end{itemize}

\begin{referencia_python}

\begin{sintaxis}{\lstinline!while <condicion>:!}
Introduce un ciclo indefinido, que se termina cuando la condición sea falsa.
\begin{codigo-python-sn}
while <condición>:
    # acciones a ejecutar mientras condición sea verdadera
\end{codigo-python-sn}
\end{sintaxis}

\begin{sintaxis}{\lstinline!break!}
Interrumpe la ejecución del ciclo actual. Puede utilizarse tanto para ciclos
definidos como indefinidos.
\end{sintaxis}

\begin{sintaxis}{\lstinline!continue!}
``Saltea'' la ejecución del ciclo a la siguiente iteración. Puede utilizarse
tanto para ciclos definidos como indefinidos.
\end{sintaxis}

\begin{sintaxis}{\lstinline!return [valor]!}
Finaliza la ejecución de una función, y además corta la ejecución del ciclo
actual, en caso de estar dentro del cuerpo del ciclo.
\end{sintaxis}
\end{referencia_python}

\newpage
\section{Ejercicios}

\extractionlabel{guia}
\begin{ejercicio}
Escribir un programa que permita al usuario ingresar un conjunto de notas,
preguntando a cada paso si desea ingresar más notas, e imprimiendo el
promedio correspondiente.
\end{ejercicio}

\extractionlabel{guia}
\begin{ejercicio}
Escribir una función que reciba un número entero $k$ e imprima su
descomposición en factores primos.
\end{ejercicio}

\extractionlabel{guia}
\begin{ejercicio}
{\bf Manejo de contraseñas}
\begin{partes}
    \item Escribir un programa que contenga una contraseña inventada, que le
pregunte al usuario la contraseña, y no le permita continuar hasta que la
haya ingresado correctamente.
    \item Modificar el programa anterior para que solamente permita una
cantidad fija de intentos.
    \item Modificar el programa anterior para que después de cada intento
agregue una pausa cada vez mayor, utilizando la función \verb!sleep! del
módulo \verb!time!.
    \item Modificar el programa anterior para que sea una función que devuelva
si el usuario ingresó o no la contraseña correctamente, mediante un valor
booleano (\verb!True! o \verb!False!).
\end{partes}
\end{ejercicio}


\extractionlabel{guia}
\begin{ejercicio}
Utilizando la función \verb!randrange! del módulo \verb!random!,
escribir un programa que obtenga un número aleatorio secreto, y luego
permita al usuario ingresar números y le indique si son menores o mayores
que el número a adivinar, hasta que el usuario ingrese el número correcto.
\end{ejercicio}


\extractionlabel{guia}
\begin{ejercicio}
{\bf Algoritmo de Euclides}
\begin{partes}
    \item Implementar el algoritmo de Euclides para calcular el máximo
común divisor de dos números $n$ y $m$, dado por los siguientes pasos.
    \begin{enumerate}
        \item Teniendo $n$ y $m$, se obtiene $r$, el resto de la
división entera de $m/n$.
        \item Si $r$ es cero, $n$ es el mcd de los valores iniciales.
        \item Se reemplaza $m \leftarrow n$, $n \leftarrow r$, y se vuelve al
primer paso.
    \end{enumerate}
    \item Hacer un seguimiento del algoritmo implementado para los siguientes
pares de números: $(15,9); (9,15); (10,8); (12,6)$.
\end{partes}
\end{ejercicio}

\extractionlabel{guia}
\begin{ejercicio}
{\bf Potencias de dos.}
\begin{partes}
    \item Escribir una función \verb!es_potencia_de_dos! que reciba como parámetro
un número natural, y devuelva \verb!True! si el número es una potencia de 2,
y \verb!False! en caso contrario.
    \item Escribir una función que, dados dos números naturales pasados como
parámetros, devuelva la suma de todas las potencias de 2 que hay en el
rango formado por esos números (0 si no hay ninguna potencia de 2 entre los
dos). Utilizar la función \verb!es_potencia_de_dos!, descripta en el
punto anterior.
\end{partes}
\end{ejercicio}


\extractionlabel{guia}
\begin{ejercicio}
{\bf Números perfectos y números amigos}
\begin{partes}
    \item Escribir una función que devuelva la suma de todos los divisores de
un número $n$, sin incluirlo.
    \item Usando la función anterior, escribir una función que imprima los
primeros $m$ números tales que la suma de sus divisores sea igual a sí
mismo (es decir los primeros $m$ números \emph{perfectos}).
    \item Usando la primera función, escribir una función que imprima las
primeras $m$ parejas de números ($a$, $b$), tales que la suma de los
divisores de $a$ es igual a $b$ y la suma de los divisores de $b$ es igual
a $a$ (es decir las primeras $m$ parejas de números \emph{amigos}).
    \item Proponer optimizaciones a las funciones anteriores para disminuir el
tiempo de ejecución.
\end{partes}
\end{ejercicio}

\extractionlabel{guia}
\begin{ejercicio}
Escribir un programa que le pida al usuario que ingrese una sucesión
de números naturales (primero uno, luego otro, y así hasta que el
usuario ingrese '-1' como condición de salida). Al final, el programa
debe imprimir cuántos números fueron ingresados, la suma total de los
valores y el promedio.
\end{ejercicio}

\extractionlabel{guia}
\begin{ejercicio}
Escribir una función que reciba dos números como parámetros, y
devuelva cuántos múltiplos del primero hay, que sean menores que el
segundo.
\begin{partes}
    \item Implementarla utilizando un ciclo \verb!for!, desde el primer número
hasta el segundo.
    \item Implementarla utilizando un ciclo \verb!while!, que multiplique el primer
número hasta que sea mayor que el segundo.
    \item Comparar ambas implementaciones: ¿Cuál es más clara? ¿Cuál realiza menos
operaciones?
\end{partes}
\end{ejercicio}

\extractionlabel{guia}
\begin{ejercicio}
Escribir una función que reciba un número natural e imprima todos
los números primos que hay hasta ese número.
\end{ejercicio}

\extractionlabel{guia}
\begin{ejercicio}
Escribir una función que reciba un dígito y un número natural, y
decida numéricamente si el dígito se encuentra en la notación decimal
del segundo.
\end{ejercicio}

\extractionlabel{guia}
\begin{ejercicio}
Escribir una función que dada la cantidad de ejercicios de un examen,
y el porcentaje necesario de ejercicios bien resueltos necesario para aprobar
dicho examen, revise un grupo de examenes. Para ello, en cada paso debe
preguntar la cantidad de ejercicios resueltos por el alumno, indicando con un
valor centinela que no hay más examenes a revisar. Debe mostrar por pantalla
el porcentaje correspondiente a la cantidad de ejercicios resueltos respecto a
la cantidad de ejercicios del examen y una leyenda que indique si aprobó o no.
\end{ejercicio}
