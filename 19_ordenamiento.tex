% Copyright (C) 2008-2010 Rosita Wachenchauzer <rositaw@gmail.com>
%               Margarita Manterola <margamanterola@gmail.com>

% Esta obra está licenciada de forma dual, bajo las licencias Creative
% Commons:
%  * Atribución-Compartir Obras Derivadas Igual 2.5 Argentina
%    http://creativecommons.org/licenses/by-sa/2.5/ar/
%  * Atribución-Compartir Obras Derivadas Igual 3.0 Unported
%    http://creativecommons.org/licenses/by-sa/3.0/deed.es_AR.
%
% A su criterio, puede utilizar una u otra licencia, o las dos.
% Para ver una copia de las licencias, puede visitar los sitios
% mencionados, o enviar una carta a Creative Commons,
% 171 Second Street, Suite 300, San Francisco, California, 94105, USA.

\chapter{Ordenar listas}

Al estudiar las listas de Python, vimos que poseen un método \lstinline!sort!
que las ordena de menor a mayor de acuerdo a una clave (e incluso de acuerdo a
una relación de orden que se desee, dada a través del parámetro
\lstinline!cmp!).

Sin embargo, no todas las estructuras cuentan con un método
\lstinline!sort! que las ordene.  Es por ello que en esta unidad nos
plantearemos cómo se hace para ordenar cuando no hay un método
\lstinline!sort!, y cuánto cuesta ordenar.

Ante todo una advertencia: hay varias maneras de ordenar, y no todas
cuestan lo mismo. Vamos a empezar viendo las más sencillas de escribir
(que en general suelen ser las más caras).

\section{Ordenamiento por selección}
Éste método de ordenamiento se basa en la siguiente idea:

\begin{itemize}

\item {\bf Paso 1.1}: Buscar el mayor de todos los elementos de la lista.
\\

\hspace{0.75cm}
\begin{tabular}[c]{|c|c|c|c|c|c|}
\hline
3& 2&-1&5&0&2\\
\hline
\end{tabular}
\hspace{0.75cm}
\begin{tabular}{p{9cm}}
Encuentra el valor $5$ en la posición $3$.
\end{tabular}\\

\item {\bf Paso 1.2}: Poner el mayor al final (intercambiar el que está en la última
posición de la lista con el mayor encontrado).\\

\hspace{0.75cm}
\begin{tabular}[c]{|c|c|c|c|c|c|}
\hline
3& 2&-1&2&0&5\\
\hline
\end{tabular}
\hspace{0.75cm}
\begin{tabular}{p{9cm}}
Intercambia el elemento de la posición $3$ con el de la posición $5$. \\
{\it En la última posición de la lista está el mayor de todos.}
\end{tabular} \\

\item {\bf Paso 2.1}: Buscar el mayor de todos los elementos del segmento de la lista
entre la primera y la anteúltima posición. \\

\hspace{0.75cm}
\begin{tabular}[c]{|c|c|c|c|c|c|}
\hline
3& 2&-1&2&0&5\\
\hline
\end{tabular}
\hspace{0.75cm}
\begin{tabular}{p{9cm}}
Encuentra el valor $3$ en la posición $0$.
\end{tabular} \\

\item {\bf Paso 2.2}: Poner el mayor al final del segmento (intercambiar el que está en la última
posición del segmento --o sea anteúltima posición de la lista-- con el mayor encontrado). \\

\hspace{0.75cm}
\begin{tabular}[c]{|c|c|c|c|c|c|}
\hline
0& 2&-1&2&3&5\\
\hline
\end{tabular}
\hspace{0.75cm}
\begin{tabular}{p{9cm}}
Intercambia el elemento de la posición $0$ con el valor de la posición $4$. \\
{\it En la anteúltima y última posición de la lista están los dos mayores en su posición definitiva.}
\end{tabular} \\

$\dots$\\

\item {\bf Paso n}: Se termina cuando queda un único elemento sin tratar: el que está
en la primera posición de la lista, y que es el menor de todos porque todos los
mayores fueron reubicados. \\

\hspace{0.75cm}
\begin{tabular}[c]{|c|c|c|c|c|c|}
\hline
-1& 0&2&2&3&5\\
\hline
\end{tabular}
\hspace{0.75cm}
\begin{tabular}{p{9cm}}
{\it La lista se encuentra ordenada}.
\end{tabular}\\
\end{itemize}

\begin{codigo}{seleccion.py}{Ordena una lista por selección}
\label{ord_seleccion}
\lstinputlisting{src/19_ordenamiento/seleccion.py}
\end{codigo}

La implementación en Python puede verse en el Código \ref{ord_seleccion}.


La función principal, \lstinline!ord_seleccion! es la encargada de recorrer
la lista, ubicando el mayor elemento al final del segmento y luego
reduciendo el segmento a analizar.

Mientras que \lstinline!buscar_max! es una función que ya se estudió
previamente, que busca el mayor elemento de la lista y devuelve su
posición.

A continuación, algunas una ejecuciones de prueba de ese código:

\begin{codigo-python-sn}
>>> l=[3, 2, -1, 5, 0, 2]
>>> ord_seleccion(l)
DEBUG:  3 5 [3, 2, -1, 2, 0, 5]
DEBUG:  0 4 [0, 2, -1, 2, 3, 5]
DEBUG:  1 3 [0, 2, -1, 2, 3, 5]
DEBUG:  1 2 [0, -1, 2, 2, 3, 5]
DEBUG:  0 1 [-1, 0, 2, 2, 3, 5]
>>> print l
[-1, 0, 2, 2, 3, 5]
>>> l=[]
>>> ord_seleccion(l)
>>> l=[1]
>>> ord_seleccion(l)
>>> print l
[1]
>>> l=[1,2,3,4,5]
>>> ord_seleccion(l)
DEBUG:  4 4 [1, 2, 3, 4, 5]
DEBUG:  3 3 [1, 2, 3, 4, 5]
DEBUG:  2 2 [1, 2, 3, 4, 5]
DEBUG:  1 1 [1, 2, 3, 4, 5]
\end{codigo-python-sn}

Puede verse que aún cuando la lista está ordenada, se la recorre buscando
los mayores elementos y ubicándolos en la misma posición en la que se
encuentran.

\subsection{Invariante en el ordenamiento por selección}

Todo ordenamiento tiene un invariante que permite asegurarse de que cada
paso que se toma va en la dirección de obtener una lista ordenada.

En el caso del ordenamiento por selección, el invariante es que los
elementos desde \lstinline!n+1! hasta el final de la lista están ordenados y
son mayores que los elementos de \lstinline!0! a \lstinline!n!, es decir
que ya están en su posición definitiva.

\subsection{¿Cuánto cuesta ordenar por selección?}

Como se puede ver en el código de la función \lstinline!buscar_max!, para
buscar el máximo elemento en un segmento de lista se debe recorrer todo ese
segmento, por lo que en nuestro caso debemos recorrer en el primer paso $N$
elementos, en el segundo paso $N-1$ elementos, en el tercer paso $N-2$
elementos, etc. Cada visita a un elemento implica una cantidad constante y
pequeña de comparaciones (que no depende de $N$). Por lo tanto tenemos que

\begin{equation}
T(N) \approx c * (2 + 3 + \ldots + N) \approx c * N * (N+1)/2 \sim N^2
\end{equation}

O sea que ordenar por selección una lista de tamaño $N$ insume tiempo del
orden de $N^2$.  Como ya se vio, este tiempo es independiente de si la
lista estaba previamente ordenda o no.

En cuanto al espacio utilizado, sólo se tiene en memoria la
lista que se desea ordenar y algunas variables de tamaño 1.

\section{Ordenamiento por inserción}

Éste otro método de ordenamiento se basa en la siguiente idea:

\begin{itemize}

\item {\bf Paso 0}: Partimos de la misma lista de ejemplo utilizada para el ordenamiento
por selección. \\

\hspace{0.75cm}
\begin{tabular}[c]{|c|c|c|c|c|c|}
\hline
3& 2&-1&5&0&2\\
\hline
\end{tabular}
\hspace{0.75cm}

\item {\bf Paso 1}: Considerar el segundo elemento de la lista,
y ordenarlo respecto del primero, deplazándolo hasta la
posición correcta, si corresponde. \\

\hspace{0.75cm}
\begin{tabular}[c]{|c|c|c|c|c|c|}
\hline
2& 3&-1&5&0&2\\
\hline
\end{tabular}
\hspace{0.75cm}
\begin{tabular}{p{9cm}}
Se desplaza el valor $2$ antes de $3$.
\end{tabular}

\item {\bf Paso 2}: Considerar el tercer elemento de la lista,
y ordenarlo respecto del primero y el segundo, deplazándolo hasta la
posición correcta, si corresponde. \\

\hspace{0.75cm}
\begin{tabular}[c]{|c|c|c|c|c|c|}
\hline
-1& 2&3&5&0&2\\
\hline
\end{tabular}
\hspace{0.75cm}
\begin{tabular}{p{9cm}}
Se desplaza el valor $-1$ antes de $2$ y de $3$.
\end{tabular}

\item {\bf Paso 3}: Considerar el cuarto elemento de la lista,
y ordenarlo respecto del primero, el segundo y el tercero, deplazándolo hasta la
posición correcta, si corresponde. \\

\hspace{0.75cm}
\begin{tabular}[c]{|c|c|c|c|c|c|}
\hline
-1& 2&3&5&0&2\\
\hline
\end{tabular}
\hspace{0.75cm}
\begin{tabular}{p{9cm}}
El $5$ está correctamente ubicado respecto de $-1$,$2$ y $3$ (como el segmento
hasta la tercera posición está ordenado, basta con comparar con el tercer elemento del
segmento para verificarlo).\\
\end{tabular}

$\dots$\\

\item {\bf Paso N-1}: \\

\hspace{0.75cm}
\begin{tabular}[c]{|c|c|c|c|c|c|}
\hline
-1& 0&2&3&5&2\\
\hline
\end{tabular}
\hspace{0.75cm}
\begin{tabular}{p{9cm}}
Todos los elementos excepto el ante-último ya se encuentran ordenados.
\end{tabular}

\item {\bf Paso N}:
Considerar el $N$--ésimo elemento de la lista, y ordenarlo respecto del
segmento formado por el primero hasta el $N-1$--ésimo, deplazándolo hasta
la posición correcta, si corresponde. \\

\hspace{0.75cm}
\begin{tabular}[c]{|c|c|c|c|c|c|}
\hline
-1& 0&2&2&3&5\\
\hline
\end{tabular}
\hspace{0.75cm}
\begin{tabular}{p{9cm}}
Se desplaza el valor $2$ antes de $3$ y de $5$.
\end{tabular}

\end{itemize}

\begin{codigo}{insercion.py}{Ordena una lista por Inserción}
\label{ord_insercion}
\lstinputlisting{src/19_ordenamiento/insercion.py}
\end{codigo}

Una posible implementación en Python de este algoritmo se incluye en el
Código \ref{ord_insercion}.

La función principal, \lstinline!ord_insercion!, recorre la lista desde el
segundo elemento hasta el último, y cuando uno de estos elementos no está
ordenado con respecto al anterior, llama a la función auxiliar
\lstinline!reubicar!, que se encarga de colocar el elemento en la posición
que le corresponde.

En la función \lstinline!reubicar! se busca la posición correcta donde debe
colocarse el elemento, a la vez que se van corriendo todos los elementos un
lugar a la derecha, de modo que cuando se encuentra la posición, el valor a
insertar reemplaza al valor que se encontraba allí anteriormente.

En las siguientes ejecuciones puede verse que funciona correctamente.

\begin{codigo-python-sn}
>>> l=[3, 2,-1,5, 0, 2]
>>> ord_insercion(l)
DEBUG:  [2, 3, -1, 5, 0, 2]
DEBUG:  [-1, 2, 3, 5, 0, 2]
DEBUG:  [-1, 2, 3, 5, 0, 2]
DEBUG:  [-1, 0, 2, 3, 5, 2]
DEBUG:  [-1, 0, 2, 2, 3, 5]
>>> print l
[-1, 0, 2, 2, 3, 5]
>>> l=[]
>>> ord_insercion(l)
>>> l=[1]
>>> ord_insercion(l)
>>> print l
[1]
>>> l=[1,2,3,4,5,6]
>>> ord_insercion(l)
DEBUG:  [1, 2, 3, 4, 5, 6]
DEBUG:  [1, 2, 3, 4, 5, 6]
DEBUG:  [1, 2, 3, 4, 5, 6]
DEBUG:  [1, 2, 3, 4, 5, 6]
DEBUG:  [1, 2, 3, 4, 5, 6]
>>> print l
[1, 2, 3, 4, 5, 6]
\end{codigo-python-sn}

\subsection{Invariante del ordenamiento por inserción}

En el ordenamiento por inserción, a cada paso se considera que los
elementos que se encuentran en el segmento de $0$ a $i$ están ordenados, de
manera que agregar un nuevo elemento implica colocarlo en la posición
correspondiente y el segmento seguirá ordenado.

\subsection{¿Cuánto cuesta ordenar por inserción?}

Del Código \ref{ord_insercion} se puede ver que la función principal avanza por la
lista de izquierda a derecha, mientras que la función \lstinline!reubicar!
cambia los elementos de lugar de derecha a izquierda.

Lo peor que le puede pasar a un elemento que está en la posición
$j$ es que deba ser ubicado al principio de la lista.  Y lo peor que le
puede pasar a una lista es que todos sus elementos deban ser reubicados.

Por ejemplo, en la lista \lstinline+[10, 8, 6, 2, -2, -5]+, todos los
elementos deben ser reubicados al principio de la lista.

En el primer paso, el segundo elemento se debe intercambiar con el primero;
en el segundo paso, el tercer elemento se compara con el segundo y el
primer elemento, y se ubica adelante de todo; en el tercer paso, el cuarto
elemento se compara con el tercero, el segundo y el primer elemento, y se
ubica adelante de todo; etc...

\begin{equation}
T(N) \approx c * (2 + 3 + \ldots + N) \approx c * N * (N+1)/2 \sim N^2
\end{equation}

Es decir que ordenar por inserción una lista de tamaño $N$ puede insumir
(en el peor caso) tiempo del orden de $N^2$. En cuanto al espacio
utilizado, nuevamente sólo se tiene en memoria la lista que se desea
ordenar y algunas variables de tamaño 1.

\subsection{Inserción en una lista ordenada}

Sin embargo, algo interesante a notar es que cuando la lista se encuentra
ordenada, este algoritmo no hace ningún movimiento de elementos,
simplemente compara cada elemento con el anterior, y si es mayor sigue
adelante.

Es decir que para el caso de una lista de $N$ elementos que se encuentra
ordenada, el tiempo que insume el algoritmo de inserción es:

\begin{equation}
T(N) \sim N
\end{equation}

% TODO: burbujeo

\section{Resumen}

\begin{itemize}

\item El ordenamiento por selección, es uno de los más sencillos, pero es
bastante ineficiente, se basa en la idea de buscar el máximo en una secuencia,
ubicarlo al final y seguir analizando la secuencia sin el último elemento.
Tiene como ventaja que hace una baja cantidad de ``intercambios'' ($N$), pero
como desventaja que necesita una alta cantidad de comparaciones ($N^2$).
Siempre tiene el mismo comportamiento.

\item El ordenamiento por inserción, es un algoritmo bastante intuitivo y se
suele usar para ordenar en la vida real. Se basa en la idea de ir insertando
ordenadamente, en cada paso se considera la inserción de un elemento más de
secuencia y la inserción se empieza a hacer desde el final de los datos ya
ordenados.

Tiene como ventaja que en el caso de tener los datos ya ordenados no hace
ningún intercambio (y hace sólo $N-1$ comparaciones). En el peor caso, cuando
la secuencia está invertida, se hace una gran cantidad de intercambios y
comparaciones ($N^2$). Si bien es un algoritmo ineficiente, para secuencias
cortas, el tiempo de ejecución es bastante bueno.

%\item Burbujeo
\end{itemize}

\newpage
\section{Ejercicios}

\begin{ejercicio}
Implementar una función que reciba una lista y devuelva otra lista con los
mismos elementos que la primera, ordenados de mayor a menor mediante el método
de inserción.
\end{ejercicio}
