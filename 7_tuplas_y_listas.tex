\chapter{Tuplas y listas}

Python cuenta con una gran variedad de tipos de datos que permiten
representar la información según cómo esté estructurada.  En esta unidad se
estudian las tuplas y las listas, que son tipos de datos utilizados cuando
se quiere agrupar elementos.

\section{Tuplas}

Al diseñar algoritmos, es muy común que queramos describir un agrupamiento de
datos de distintos tipos.

Esto es algo que ya hicimos anteriormente: en la conversión de un tiempo a
horas, minutos y segundos (sección~\ref{fun:multiple_return}) y también en el
juego Mastermind (sección \ref{mastermind}), usamos n-uplas, que en Python se
llaman \emph{tuplas}, y sirven para representar agrupaciones de datos ordenados.

Veamos más ejemplos:

\begin{itemize}

\item Una fecha la podemos querer representar como la terna día (un número
entero), mes (una cadena de caracteres), y año (un número entero), y
tendremos por ejemplo la tupla \lstinline!(25, "Mayo", 1810)!.

\item Como datos de los alumnos queremos guardar número de padrón, nombre y
apellido, como por ejemplo \lstinline!(89766, "Alicia", "Hacker")!.

\item {\bf Es posible anidar tuplas:} como datos de los alumnos
queremos guardar número de padrón, nombre, apellido y fecha de nacimiento,
como por ejemplo: \\
\lstinline!(89766, "Alicia", "Hacker", (9, "Julio", 1988))!.
\end{itemize}

\begin{observacion}
En Python el tipo de dato asociado a las tuplas se llama |tuple|:

\begin{codigo-python-sn}
>>> fecha = (25, "Mayo", 1810)
>>> type(fecha)
<class 'tuple'>
\end{codigo-python-sn}
\end{observacion}

\subsection{Elementos y segmentos de tuplas}

Las tuplas son \emph{secuencias}, igual que las cadenas, y se puede utilizar la
misma notación de índices que en las cadenas para obtener cada una de sus
componentes:
\begin{codigo-python-sn}
>>> fecha = (25, "Mayo", 1810)
>>> fecha[0]
25
>>> fecha[1]
'Mayo'
>>> fecha[2]
1810
\end{codigo-python-sn}

\begin{atencion}
Todas las secuencias en Python comienzan a numerarse desde 0.  Es por eso
que se produce un error si se quiere acceder al n-ésimo elemento de un
tupla:
\begin{codigo-python-sn}
>>> fecha[3]
(^Traceback (most recent call last):
  File "<stdin>", line 1, in <module>
IndexError: tuple index out of range^)
\end{codigo-python-sn}
\end{atencion}

También se puede utilizar la notación de rangos, que se vio aplicada a
cadenas para obtener una nueva tupla, con un subconjunto de componentes. Si
en el ejemplo de la fecha queremos quedarnos con un par que sólo contenga
día y mes podremos tomar el rango |[:2]| de la misma:

\begin{codigo-python-sn}
>>> fecha[:2]
(25, 'Mayo')
\end{codigo-python-sn}

\ejercicioc{¿Cuál es el resultado de obtener el cuarto elemento de la tupla
\lstinline!(89766, "Alicia", "Hacker", (9, "Julio", 1988))!?}

\subsection{Las tuplas son inmutables}

Al igual que con las cadenas, las componentes de las tuplas no pueden ser
modificadas:

\begin{codigo-python-sn}
>>> fecha[2] = 2018
(^Traceback (most recent call last):
  File "<stdin>", line 1, in <module>
TypeError: 'tuple' object does not support item assignment^)
\end{codigo-python-sn}

\subsection{Longitud de tuplas}

A las tuplas también se les puede aplicar la función \lstinline+len()+
para calcular su longitud. El valor de esta función aplicada a
una tupla nos indica cuántas componentes tiene esa tupla.

\begin{codigo-python-sn}
>>> len(fecha)
3
\end{codigo-python-sn}

\ejercicioc{¿Cuál es la longitud de la tupla
\lstinline!(89766, "Alicia", "Hacker", (9, "Julio", 1988))!?}

\begin{itemize}
\item Una \emph{tupla vacía} es una tupla con $0$ componentes, y se la
indica como \lstinline+()+.

\begin{codigo-python-sn}
>>> z = ()
>>> len(z)
0
>>> z[0]
(^Traceback (most recent call last):
  File "<stdin>", line 1, in <module>
IndexError: tuple index out of range^)
\end{codigo-python-sn}

\item Una \emph{tupla unitaria} es una tupla con una componente. Para
distinguir la tupla unitaria de la componente que contiene, la sintaxis de Python exige
que a la componente no sólo se la encierre entre paréntesis sino que se le
ponga una coma a continuación del valor de la componente (así,
\lstinline+(1810)+ es un número, pero \lstinline+(1810,)+ es la tupla
unitaria cuya única componente vale 1810).

\begin{codigo-python-sn}
>>> u = (1810)
>>> len(u)
(^Traceback (most recent call last):
  File "<stdin>", line 1, in <module>
TypeError: object of type 'int' has no len()^)
>>> u = (1810,)
>>> len(u)
1
>>> u[0]
1810
\end{codigo-python-sn}
\end{itemize}

\subsection{Empaquetado y desempaquetado de tuplas}
Si a una variable se le asigna una secuencia de valores separados por comas,
el valor de esa variable será la tupla formada por todos los valores asignados.
A esta operación se la denomina \emph{empaquetado de tuplas}.

\begin{codigo-python-sn}
>>> a = 125
>>> b = "#"
>>> c = "Ana"
>>> d = a, b, c
>>> len(d)
3
>>> d
(125, '#', 'Ana')
\end{codigo-python-sn}

Si se tiene una tupla de longitud \lstinline+k+, se puede asignar
la tupla a \lstinline+k+ variables distintas y en cada variable quedará
una de las componentes de la tupla. A esta operación se la denomina
\emph{desempaquetado de tuplas}.

\begin{codigo-python-sn}
>>> x, y, z = d
>>> x
125
>>> y
'#'
>>> z
'Ana'
\end{codigo-python-sn}

\begin{atencion}
Si las variables no son distintas, se pierden valores. Y si las variables
no son exactamente \lstinline+k+ se produce un error.

\begin{codigo-python-sn}
>>> p, p, p = d
>>> p
'Ana'
>>> m, n = d
(^Traceback (most recent call last):
  File "<stdin>", line 1, in <module>
ValueError: too many values to unpack^)
>>> m, n, o, p = d
(^Traceback (most recent call last):
  File "<stdin>", line 1, in <module>
ValueError: need more than 3 values to unpack^)
\end{codigo-python-sn}
\end{atencion}

\begin{sabias_que}
En la implementación de algunos programas suele ser necesario intercambiar el
valor de dos variables. Si queremos intercambiar el valor de |a| y |b|, una
posibilidad es utilizar una variable auxiliar:

\begin{codigo-python-sn}
aux = a
a = b
b = aux
\end{codigo-python-sn}

Pero un ``truco'' que permiten algunos lenguajes, entre ellos Python, es la
posibilidad de empaquetar y desempaquetar una tupla en una única operación:

\begin{codigo-python-sn}
a, b = b, a
\end{codigo-python-sn}
\end{sabias_que}

\subsection{Ejercicios con tuplas}

\ejercicioc{Cartas como tuplas.
\begin{partes}
\item Proponer una representación con tuplas para las cartas de la baraja
francesa.

\item Escribir una función \lstinline!poker! que reciba cinco cartas de la
baraja francesa e informe (devuelva el valor lógico correspondiente) si esas
cartas forman o no un \emph{poker} (es decir que hay 4 cartas con el mismo
número).
\end{partes}
}

\ejercicioc{El tiempo como tuplas.
\begin{partes}
\item Proponer una representación con tuplas para representar el tiempo.
\item Escribir una función \lstinline!sumar_tiempos! que reciba dos tiempos dados y
devuelva su suma.
\end{partes}
}

\ejercicioc{Escribir una función \lstinline!dia_siguiente! que dada una fecha
expresada como la terna |(Día, Mes, Año)| (donde |Día|, |Mes| y |Año| son números
enteros) calcule el día siguiente al dado, en el mismo formato.}

\ejercicioc{Escribir una función \lstinline!dia_siguiente_m! que dada una fecha
expresada como la terna |(Día, Mes, Año)| (donde |Día| y |Año| son números
enteros, y |Mes| es el texto \lstinline!"Ene"!, \lstinline!"Feb"!, $\ldots$,
\lstinline!"Dic"!, según corresponda) calcule el día siguiente al dado, en
el mismo formato.}

\section{Listas}

Presentaremos ahora una nueva estructura de datos: la \emph{lista}.
Usaremos listas para poder modelar datos compuestos, pero cuya cantidad y
valor varían a lo largo del tiempo. Son secuencias \emph{mutables} y vienen
dotadas de una variedad de operaciones muy útiles.

La notación para lista es una secuencia de valores encerrados entre
corchetes y separados por comas.  Por ejemplo, si representamos a los
alumnos mediante su número de padrón, se puede tener una lista de
inscriptos en la materia como la siguiente:
\lstinline![78455, 89211, 66540, 45750]!.
Al abrirse la inscripción, antes de que hubiera inscriptos, la lista de
inscriptos se representará por una lista vacía: \lstinline![]!.

\begin{observacion}
En Python el tipo de dato asociado a las listas se llama |list|:

\begin{codigo-python-sn}
>>> type([78455, 89211, 66540, 45750])
<class 'list'>
\end{codigo-python-sn}
\end{observacion}

\subsection{Longitud de la lista. Elementos y segmentos de listas}

\begin{itemize}

\item Como a las secuencias ya vistas, a las listas también se les puede
aplicar la función \lstinline+len()+ para conocer su longitud.

\item Para acceder a los distintos elementos de la lista se utilizará la
misma notación de índices de cadenas y tuplas, con valores que van de $0$ a la
longitud de la lista $- 1$.

\begin{codigo-python-sn}
>>> padrones = [78455, 89211, 66540, 45750]
>>> padrones[0]
78455
>>> len(padrones)
4
>>> padrones[4]
(^Traceback (most recent call last):
  File "<stdin>", line 1, in <module>
IndexError: list index out of range^)
>>> padrones[3]
45750
\end{codigo-python-sn}

\item Para obtener una sublista a partir de la lista original, se utiliza
la notación de rangos, como en las otras secuencias.

Para obtener la lista que contiene sólo a quién se inscribió en segundo
lugar podemos escribir:

\begin{codigo-python-sn}
>>> padrones[1:2]
[89211]
\end{codigo-python-sn}

Para obtener la lista que contiene al segundo y tercer inscriptos
podemos escribir:

\begin{codigo-python-sn}
>>> padrones[1:3]
[89211, 66540]
\end{codigo-python-sn}

Para obtener la lista que contiene al primero y segundo inscriptos
podemos escribir:

\begin{codigo-python-sn}
>>> padrones[:2]
[78455, 89211]
\end{codigo-python-sn}

\end{itemize}

\subsection{Cómo mutar listas}

Dijimos antes que las listas son secuencias mutables. Para lograr la
mutabilidad Python provee operaciones que nos permiten cambiarle valores,
agregarle valores y quitarle valores.

\begin{itemize}

\item Para cambiar una componente de una lista, se selecciona la componente
mediante su índice y se le asigna el nuevo valor:

\begin{codigo-python-sn}
>>> padrones[1] = 79211
>>> padrones
[78455, 79211, 66540, 45750]
\end{codigo-python-sn}

\item Para agregar un nuevo valor al final de la lista se utiliza la
operación \lstinline+append()+.  Escribimos \lstinline+padrones.append(47890)+ para
agregar el padrón 47890 al final de \lstinline+padrones+.

\begin{codigo-python-sn}
>>> padrones.append(47890)
>>> padrones
[78455, 79211, 66540, 45750, 47890]
\end{codigo-python-sn}

\item Para insertar un nuevo valor en la posición cuyo índice es
\lstinline+k+ (y desplazar un lugar el resto de la lista) se utiliza la
operación \lstinline+insert()+.

Escribimos \lstinline+padrones.insert(2, 54988)+ para insertar el padrón 54988 en
la tercera posición de \lstinline+padrones+.

\begin{codigo-python-sn}
>>> padrones.insert(2, 54988)
>>> padrones
[78455, 79211, 54988, 66540, 45750, 47890]
\end{codigo-python-sn}

\item Las listas no controlan si se insertan elementos repetidos. Si necesitamos
exigir unicidad, debemos hacerlo mediante el código de nuestros programas.
\begin{codigo-python-sn}
>>> padrones.insert(1,78455)
>>> padrones
[78455, 78455, 79211, 54988, 66540, 45750, 47890]
\end{codigo-python-sn}

\item Para eliminar un valor de una lista se utiliza la operación
\lstinline+remove()+.

Escribimos \lstinline+padrones.remove(45750)+ para borrar el padrón 45750 de la
lista de inscriptos:

\begin{codigo-python-sn}
>>> padrones.remove(45750)
>>> padrones
[78455, 78455, 79211, 54988, 66540, 47890]
\end{codigo-python-sn}

Si el valor a borrar está repetido, se borra sólo su primera aparición:

\begin{codigo-python-sn}
>>> padrones.remove(78455)
>>> padrones
[78455, 79211, 54988, 66540, 47890]
\end{codigo-python-sn}

\begin{atencion}
Si el valor a borrar no existe, se produce un error:

\begin{codigo-python-sn}
>>> padrones.remove(78)
(^Traceback (most recent call last):
  File "<stdin>", line 1, in <module>
ValueError: list.remove(x): x not in list^)
\end{codigo-python-sn}
\end{atencion}
\end{itemize}

\subsection{Cómo buscar dentro de las listas}

Queremos poder formular dos preguntas más respecto de la lista de
inscriptos:

\begin{itemize}
\item ¿Está la persona cuyo padrón es $v$ inscripta en esta materia?

\item ¿En qué orden se inscribió la persona cuyo padrón es $v$?.
\end{itemize}

Veamos qué operaciones sobre listas se pueden usar para lograr esos dos
objetivos:

\begin{itemize}
\item Para preguntar si un valor determinado es un elemento de una lista
usaremos la operación \lstinline+in+:

\begin{codigo-python-sn}
>>> padrones
[78455, 79211, 54988, 66540, 47890]
>>> 78 in padrones
False
>>> 66540 in padrones
True
\end{codigo-python-sn}

\begin{observacion}
El operador |in| se puede utilizar para todas las secuencias, incluyendo
tuplas y cadenas.
\end{observacion}

\item Para averiguar la posición de un valor dentro de una lista usaremos
la operación \lstinline+index()+.

\begin{codigo-python-sn}
>>> padrones.index(78455)
0
>>> padrones.index(47890)
4
\end{codigo-python-sn}

\begin{atencion}
Si el valor no se encuentra en la lista, se producirá un error:

\begin{codigo-python-sn}
>>> padrones.index(78)
(^Traceback (most recent call last):
  File "<stdin>", line 1, in <module>
ValueError: list.index(x): x not in list^)
\end{codigo-python-sn}
\end{atencion}

Si el valor está repetido, el índice que devuelve es el de la primera aparición:

\begin{codigo-python-sn}
>>> [10, 20, 10].index(10)
0
\end{codigo-python-sn}

\begin{observacion}
La función |index| también se puede utilizar con cadenas y tuplas.
\end{observacion}

\item Para iterar sobre todos los elementos de una lista usaremos una
construcción \lstinline+for+:

\begin{codigo-python-sn}
>>> for p in padrones:
...     print(p)
...
78455
79211
54988
66540
47890
\end{codigo-python-sn}

\begin{observacion}
El ciclo |for <variable> in <secuencia>:| se puede utilizar sobre cualquier
secuencia, incluyendo tuplas y cadenas.
\end{observacion}

\item Muchas veces, dentro del cuerpo del ciclo |for| es necesario contar con la
posición de cada elemento de la lista. Para esto es posible utilizar la función
|enumerate|: \label{enumerate}

\begin{codigo-python-sn}
>>> for i, p in enumerate(padrones):
...     print(i, p)
...
0 78455
1 79211
2 54988
3 66540
4 47890
\end{codigo-python-sn}
\end{itemize}

\begin{sabias_que}
En Python, las listas, las tuplas y las cadenas son parte del conjunto
de las \emph{secuencias}.  Todas las secuencias cuentan con las siguientes
operaciones:

\vspace{\medskipamount}
\begin{tabular}{l l}
{\bf Operación} & {\bf Resultado} \\
\hline
\lstinline!x in s! & Indica si el valor \lstinline!x! se encuentra en
    \lstinline!s! \\
\lstinline!s + t! & Concantena las secuencias \lstinline!s! y \lstinline!t! \\
\lstinline!s * n! & Concatena \lstinline!n! copias de \lstinline!s! \\
\lstinline!s[i]! & Elemento \lstinline!i! de \lstinline!s!, empezando por 0 \\
\lstinline!s[i:j]! & Porción de la secuencia \lstinline!s! desde
	\lstinline!i! hasta \lstinline!j! (no inclusive) \\
\lstinline!s[i:j:k]! & Porción de la secuencia \lstinline!s! desde
	\lstinline!i! hasta \lstinline!j! (no inclusive), con paso \lstinline!k!  \\
\lstinline!len(s)! & Cantidad de elementos de la secuencia \lstinline!s!  \\
\lstinline!min(s)! & Mínimo elemento de la secuencia \lstinline!s! \\
\lstinline!max(s)! & Máximo elemento de la secuencia \lstinline!s! \\
\lstinline!sum(s)! & Suma de los elementos de la secuencia \lstinline!s! \\
\lstinline!enumerate(s)! & Enumerar los elementos de \lstinline!s! junto con sus posiciones \\
\end{tabular}
\vspace{\medskipamount}

Además, es posible crear una lista o una tupla a partir de cualquier otra
secuencia, utilizando las funciones |list| y |tuple|, respectivamente:

\begin{codigo-python-sn}
>>> list("Hola")
['H', 'o', 'l', 'a']
>>> tuple("Hola")
('H', 'o', 'l', 'a')
>>> list((1, 2, 3, 4))
[1, 2, 3, 4]
\end{codigo-python-sn}
\end{sabias_que}

\problemac{Queremos escribir un programa que nos permita armar la lista de
los inscriptos de una materia.}

\begin{enumerate}

\item {\bf Análisis:} El usuario ingresa datos de padrones que se van
guardando en una lista.

\item {\bf Especificación:} El programa solicitará al usuario que ingrese
uno a uno los padrones de los inscriptos. Con esos números construirá una
lista, que al final se mostrará.

\item {\bf Diseño:}
\begin{itemize}
\item ¿Qué estructura tiene este programa? ¿Se parece a algo conocido?

Es claramente un ciclo en el cual se le pide al usuario que ingrese uno a
uno los padrones de los inscriptos, y estos números se agregan a una lista.
Y en algún momento, cuando se terminaron los inscriptos, el usuario deja de
cargar.

\item ¿El ciclo es definido o indefinido?

Para que fuera un ciclo definido deberíamos contar de antemano cuántos
inscriptos tenemos, y luego cargar exactamente esa cantidad, pero eso no
parece muy útil.

Estamos frente a una situación parecida al problema de la lectura de los
números, en el sentido de que no sabemos cuántos elementos queremos cargar
de antemano. Para ese problema, en la sección~\ref{centinela}, vimos una solución muy
sencilla y cómoda: se le piden datos al usuario y, cuando se cargaron todos
los datos se ingresa un valor arbitrario (que se usa sólo para indicar que
no hay más información). A ese diseño lo hemos llamado ciclo con centinela
y tiene el siguiente esquema:

\begin{codigo-nohl-sn}
Repetir indefinidamente:
    Pedir datos
    Si el dato pedido coincide con el centinela:
        Salir del ciclo
	Realizar cálculos
\end{codigo-nohl-sn}

Como sabemos que los números de padrón son siempre enteros positivos,
podemos considerar que el centinela puede ser cualquier número menor o
igual a cero.  También sabemos que en nuestro caso tenemos que ir armando
una lista que inicialmente no tiene ningún inscripto.

Modificamos el esquema anterior para ajustarnos a nuestra situación:

\begin{codigo-nohl-sn}
La lista de inscriptos es vacía
Repetir indefinidamente:
    Pedir padrón
    Si el padrón no es positivo:
        Salir del ciclo
	Agregar el padrón a la lista
Devolver la lista de inscriptos
\end{codigo-nohl-sn}

\end{itemize}

\item {\bf Implementación:}
De acuerdo a lo diseñado, el programa quedaría como
se muestra en el Código~\ref{ins0}.

\begin{observacion}
Para entender mejor la implementación propuesta, es una buena idea repasar los
conceptos de ciclos, en especial las instrucciones |break| y |continue|,
explicadas en la sección~\ref{ciclos:resumen}.
\end{observacion}

\begin{codigo}{inscripcion.py}{Permite ingresar padrones de alumnos
inscriptos}
\label{ins0}
\lstinputlisting{src/7_tuplas_y_listas/ins0.py}
\end{codigo}

\item {\bf Prueba:}
Para probarlo lo ejecutamos con algunos lotes de prueba (inscripción de
tres alumnos, inscripción de cero alumnos, inscripción de alumnos
repetidos):

\begin{codigo-nohl-sn}
(~\$~) python3 inscripcion.py
Inscripcion en el curso de Algoritmos y Programación I
Ingresa un padrón (<=0 para terminar): 30
Ingresa un padrón (<=0 para terminar): 40
Ingresa un padrón (<=0 para terminar): 50
Ingresa un padrón (<=0 para terminar): 0
La lista de inscriptos es: [30, 40, 50]

(~\$~) python3 inscripcion.py
Inscripcion en el curso de Algoritmos y Programación I
Ingresa un padrón (<=0 para terminar): 0
La lista de inscriptos es: []

(~\$~) python3 inscripcion.py
Inscripcion en el curso de Algoritmos y Programación I
Ingresa un padrón (<=0 para terminar): 30
Ingresa un padrón (<=0 para terminar): 40
Ingresa un padrón (<=0 para terminar): 40
Ingresa un padrón (<=0 para terminar): 30
Ingresa un padrón (<=0 para terminar): 50
Ingresa un padrón (<=0 para terminar): 0
La lista de inscriptos es: [30, 40, 40, 30, 50]
\end{codigo-nohl-sn}

Evidentemente el programa funciona de acuerdo a lo especificado, pero
hay algo que no tuvimos en cuenta: permite inscribir a una misma persona
más de una vez.

\item {\bf Mantenimiento:} No permitir que haya padrones repetidos.

\item {\bf Diseño revisado:} Para no permitir que haya padrones repetidos
debemos revisar que no exista el padrón antes de agregarlo en la lista:

\begin{codigo-nohl-sn}
La lista de inscriptos es vacía
Repetir indefinidamente:
    Pedir padrón
    Si el padrón no es positivo:
        Salir del ciclo
    (@Si el padrón está en la lista:@)
        (@Avisar que el padrón ya está en la lista@)
        (@Saltear a la siguiente iteración del ciclo@)
    Agregar el padrón a la lista
Devolver la lista de inscriptos
\end{codigo-nohl-sn}

\item {\bf Nueva implementación:}
De acuerdo a lo diseñado en el párrafo anterior, el programa ahora quedaría
como se muestra en el Código~\ref{ins1}.

\begin{codigo}{inscripcion.py}{Permite ingresar padrones, sin repetir}
\label{ins1}
\lstinputlisting{src/7_tuplas_y_listas/ins1.py}
\end{codigo}

\item {\bf Nueva prueba:}
Para probarlo lo ejecutamos con los mismos lotes de prueba anteriores
(inscripción de tres alumnos, inscripción de cero alumnos, inscripción de
alumnos repetidos):

\begin{codigo-nohl-sn}
(~\$~) python3 inscripcion.py
Inscripcion en el curso de Algoritmos y Programación I
Ingresa un padrón (<=0 para terminar): 30
Ingresa un padrón (<=0 para terminar): 40
Ingresa un padrón (<=0 para terminar): 50
Ingresa un padrón (<=0 para terminar): 0
La lista de inscriptos es: [30, 40, 50]

(~\$~) python3 inscripcion.py
Inscripcion en el curso de Algoritmos y Programación I
Ingresa un padrón (<=0 para terminar): 0
La lista de inscriptos es: []

(~\$~) python3 inscripcion.py
Inscripcion en el curso de Algoritmos y Programación I
Ingresa un padrón (<=0 para terminar): 30
Ingresa un padrón (<=0 para terminar): 40
(@Ingresa un padrón (<=0 para terminar): 40@)
(@El padrón ya está en la lista de inscriptos.@)
(@Ingresa un padrón (<=0 para terminar): 30@)
(@El padrón ya está en la lista de inscriptos.@)
Ingresa un padrón (<=0 para terminar): 50
Ingresa un padrón (<=0 para terminar): 0
La lista de inscriptos es: [30, 40, 50]
\end{codigo-nohl-sn}

Ahora el resultado es satisfactorio: no tenemos inscriptos repetidos.

\end{enumerate}

\ejercicioc{Permitir que los alumnos se puedan inscribir o borrar.}

\ejercicioc{Inscribir y borrar alumnos como antes, pero registrar también
el nombre y apellido de la persona inscripta, de modo de tener como lista
de inscriptos:
\lstinline![(20, "Ana", "García"), (10, "Juan", "Salas")]!.}

\section{Ordenar listas}
\label{ordenar}

Nos puede interesar que los elementos de una lista estén ordenados: una vez
que finalizó la inscripción en un curso, tener a los padrones de los
alumnos por orden de inscripción puede ser muy incómodo, siempre será
preferible tenerlos ordenados por número para realizar cualquier
comprobación.

Python provee dos operaciones para obtener una lista ordenada a partir de
una lista desordenada.

\begin{itemize}

\item Para dejar la lista original intacta pero obtener una nueva lista
ordenada a partir de ella, se usa la función \lstinline!sorted!.\label{sorted}

\begin{codigo-python-sn}
>>> bs = [5, 2, 4, 2]
>>> cs = sorted(bs)
>>> bs
[5, 2, 4, 2]
>>> cs
[2, 2, 4, 5]
\end{codigo-python-sn}

\item Para modificar directamente la lista original usaremos la operación
\lstinline+sort()+.

\begin{codigo-python-sn}
>>> ds = [5, 3, 4, 5]
>>> ds.sort()
>>> ds
[3, 4, 5, 5]
\end{codigo-python-sn}

\end{itemize}

\section{Listas y cadenas}

A partir de una cadena de caracteres, podemos obtener una lista con sus
componentes usando la función \lstinline!split!.

Si queremos obtener las palabras (separadas entre sí por espacios) que
componen la cadena \lstinline!padrones! escribiremos simplemente
\lstinline!padrones.split()!:

\begin{codigo-python-sn}
>>> c = "   Una    cadena   con    espacios    "
>>> c.split()
['Una', 'cadena', 'con', 'espacios']
\end{codigo-python-sn}

En este caso \lstinline!split! elimina todos los blancos de más, y devuelve
sólo las palabras que conforman la cadena.

Si en cambio el separador es otro carácter (por ejemplo la arroba,
\lstinline!"@"!), se lo debemos pasar como parámetro a la función
\lstinline!split!. En ese caso se considera una componente todo lo que se
encuentra entre dos arrobas consecutivas. En el caso particular de que el
texto contenga dos arrobas una a continuación de la otra, se devolverá una
componente vacía:

\begin{codigo-python-sn}
>>> d = "@@Una@@@cadena@@@con@@arrobas@"
>>> d.split("@")
['', '', 'Una', '', '', 'cadena', '', '', 'con', '', 'arrobas', '']
\end{codigo-python-sn}

La ``casi''--inversa de \lstinline!split! es una función \lstinline!join!
que tiene la siguiente sintaxis:
\begin{codigo-python-sn}
<separador>.join(<lista de componentes a unir>)
\end{codigo-python-sn}
y que devuelve la cadena que resulta de unir todas las componentes
separadas entre sí por medio del \emph{separador}:

\begin{codigo-python-sn}
>>> xs = ['aaa', 'bbb', 'cccc']
>>> " ".join(xs)
'aaa bbb cccc'
>>> ", ".join(xs)
'aaa, bbb, cccc'
>>> "@@".join(xs)
'aaa@@bbb@@cccc'
\end{codigo-python-sn}

\subsection{Ejercicios con listas y cadenas}

\ejercicioc{Escribir una función que reciba como parámetro una cadena de
palabras separadas por espacios y devuelva, como resultado, cuántas
palabras de más de cinco letras tiene la cadena dada.}

\ejercicioc{{\bf Procesamiento de telegramas.} Un oficial de correos decide
optimizar el trabajo de su oficina cortando todas las palabras de más de
cinco letras a sólo cinco letras (e indicando que una palabra fue cortada
con el agregado de una arroba). Además elimina todos los espacios en blanco
de más.

Por ejemplo, al texto
\lstinline+"   Llego   mañana   alrededor  del   mediodía   "+
se transcribe como \lstinline+"Llego mañan@ alred@ del medio@"+.

Por otro lado cobra un valor para las palabras cortas y otro valor para las
palabras largas (que deben ser cortadas).

\begin{partes}
\item Escribir una función que reciba un texto, la longitud máxima de las
palabras, el costo de cada palabra corta, el costo de cada palabra larga, y
devuelva como resultado el texto del telegrama y el costo del mismo.

\item Los puntos se reemplazan por la palabra especial "STOP", y el punto
final (que puede faltar en el texto original) se indica como "STOPSTOP".

Al texto: \\
\lstinline+"   Llego   mañana   alrededor  del   mediodía. Voy a almorzar "+ \\
Se lo transcribe como: \\
\lstinline+"Llego mañan@ alred@ del medio@ STOP Voy a almor@ STOPSTOP"+.

Extender la función anterior para agregar el tratamiento de los puntos.
\end{partes}
}

\section{Listas y tuplas anidadas}

Las listas y tuplas pueden contener valores de cualquier tipo, ¡incluso otras
listas y tuplas! Cuando una lista contiene otras listas decimos que tenemos
listas \emph{anidadas} (y lo mismo con tuplas).

\begin{codigo-python-sn}
>>> lista_de_listas = [[1, 2, 3], ["a", "b", "c", "d"]]
\end{codigo-python-sn}

Aquí |lista_de_listas| es una lista que contiene 2 elementos, que a su vez son
también listas. Así, |lista_de_listas[0]| sería la lista |[1, 2, 3]| y
|lista_de_listas[1]| sería |["a", "b", "c", "d"]|.

Si quisiéramos, por ejemplo, reemplazar la |"c"| por una |"C"|, podríamos
hacerlo en dos pasos: primero obteniendo una referencia a la segunda lista (que
es la que está en la posición 1), y luego mutándola como ya vimos:

\begin{codigo-python-sn}
>>> lista_de_letras = lista_de_listas[1]
>>> lista_de_letras[2] = "C"
\end{codigo-python-sn}

O bien podríamos hacerlo en una única operación:

\begin{codigo-python-sn}
>>> lista_de_listas[1][2] = "C"
\end{codigo-python-sn}

\subsection{Matrices}

Una aplicación muy frecuente de listas y tuplas anidadas es para representar
matrices (o, en general, estructuras de datos de más de una dimensión).
Por ejemplo, sea la siguiente matriz de números:

$$
\begin{pmatrix}
1 & 2 & 3 \\
4 & 5 & 6
\end{pmatrix}
$$

Para representarla en una variable en Python, debemos elegir una
\emph{estructura de datos} apropiada. Lo primero a tener en cuenta suele ser si
necesitamos que nuestra matriz sea mutable o inmutable; es decir, una vez
creada la matriz ¿será necesario modificarla? Si la respuesta es que sí,
seguramente sea conveniente usar listas; en caso contrario tuplas.

Supongamos que queremos que nuestra matriz sea mutable, así que decidimos usar
listas. Tenemos varias alternativas para representar la matriz:

\begin{enumerate}
\item Una lista de números:
\begin{codigo-python-sn}
>>> m = [1, 2, 3, 4, 5, 6]
\end{codigo-python-sn}

\item Una lista de listas de números, donde cada sublista representa una
    fila de la matriz:
\begin{codigo-python-sn}
>>> m = [[1, 2, 3], [4, 5, 6]]
\end{codigo-python-sn}

\item Una lista de listas de números, donde cada sublista representa una
    columna de la matriz:
\begin{codigo-python-sn}
>>> m = [[1, 4], [2, 5], [3, 6]]
\end{codigo-python-sn}
\end{enumerate}

Podríamos pensar más opciones, pero estas son las más intuitivas. Cada una de
ellas tiene ventajas y desventajas, y dependiendo del problema específico que
estemos resolviendo convendrá elegir una u otra. Analicemos cada una por
separado.

\begin{enumerate}
\item La primera opción (|m = [1, 2, 3, 4, 5, 6]|, una lista \emph{plana}) es,
tal vez, la menos intuitiva de las tres, pero en algunos casos puede ser la
más conveniente. Aquí asumimos que las dimensiones de la matriz son
conocidas (2 filas y 3 columnas) ya que no es posible deducirlo a
partir de la lista que tiene longitud 6 (podría ser de 2x3, 3x2, 1x6,
...).

\begin{codigo-python-sn}
>>> m = [1, 2, 3, 4, 5, 6]
>>> cantidad_de_filas = 2    # las filas posibles son: 0, 1
>>> cantidad_de_columnas = 3 # las columnas posibles son: 0, 1, 2
\end{codigo-python-sn}

Si quisiéramos acceder a un elemento particular sabiendo su fila y columna,
tendremos que calcular el índice de la lista correspondiente. Por ejemplo,
para imprimir la matriz fila por fila:

\begin{codigo-python-sn}
>>> def imprimir_matriz(m, cantidad_de_filas, cantidad_de_columnas):
>>>     for f in range(cantidad_de_filas):
>>>         for c in range(cantidad_de_columnas):
>>>             print((@m[fila * cantidad_de_columnas + columna]@), end=' ') (~\circled{1}~)
>>>         print() # para pasar a la siguiente línea
>>>
>>> m = [1, 2, 3, 4, 5, 6]
>>> imprimir_matriz(m, 2, 3)
1 2 3
4 5 6
\end{codigo-python-sn}

\circled{1} Sabiendo las dimensiones de la matriz, y la fila y columna de un elemento
en particular, podemos calcular su índice en la lista de la forma
|fila * cantidad_de_columnas + columna| (donde |fila| y |columna| cuentan desde 0).

\item Si la matriz está representada como una \emph{lista de filas}, todas las operaciones
se vuelven un poco más simples. Podemos deducir las dimensiones de la matriz a partir de las
longitudes de las listas, y ya no hace falta hacer cuentas para determinar la posición de un
elemento sabiendo su fila y columna:

\begin{codigo-python-sn}
>>> def imprimir_matriz(m):
>>>     (@cantidad_de_filas = len(m)@)
>>>     (@cantidad_de_columnas = len(m[0])@)
>>>     for f in range(cantidad_de_filas):
>>>         for c in range(cantidad_de_columnas):
>>>             print((@m[fila][columna]@), end=' ')
>>>         print() # para pasar a la siguiente línea
>>>
>>> m = [[1, 2, 3], [4, 5, 6]]
>>> imprimir_matriz(m)
1 2 3
4 5 6
\end{codigo-python-sn}


\item Lo mismo si la matriz está representada como una \emph{lista de columnas}, solo que tenemos que
tener cuidado en invertir el orden de los índices:

\begin{codigo-python-sn}
>>> def imprimir_matriz(m):
>>>     (@cantidad_de_columnas = len(m)@)
>>>     (@cantidad_de_filas = len(m[0])@)
>>>     for f in range(cantidad_de_filas):
>>>         for c in range(cantidad_de_columnas):
>>>             print((@m[columna][fila]@), end=' ')
>>>         print() # para pasar a la siguiente línea
>>>
>>> m = [[1, 4], [2, 5], [3, 6]]
>>> imprimir_matriz(m)
1 2 3
4 5 6
\end{codigo-python-sn}

\end{enumerate}

También es importante prestar atención a la hora de crear una nueva matriz en
forma \emph{programática}. Por ejemplo, sea la siguiente matriz de 4x4:

$$
\begin{pmatrix}
"a1" & "b1" & "c1" & "d1" \\
"a2" & "b2" & "c2" & "d2" \\
"a3" & "b3" & "c3" & "d3" \\
"a4" & "b4" & "c4" & "d4" \\
\end{pmatrix}
$$

Podríamos escribir esta matriz en Python en forma \emph{literal}:
|m = [["a1", "b1", "c1", "d1"], ...]|, pero ¿qué pasaría si necesitamos representar una
matriz similar de 100x100? ¿O si las dimensiones son ingresadas por el usuario?
Aquí es donde se vuelve necesario crear la matriz en forma programática:

\problemac{Escribir una función que reciba las dimensiones (cantidad de filas y columnas)
y devuelva una matriz donde cada elemento sea una cadena de la forma |"xy"|, donde |x| es
una letra representando la columna e |y| un número representando la fila.}

\begin{solucion}
Nuevamente, la forma de resolver el problema dependerá de cómo decidimos representar la matriz:

\begin{enumerate}

\item Si representamos la matriz como una lista plana:

\begin{codigo-python-sn}
LETRAS = 'abcdefghijklmnopqrstuvwxyz'(~\footnote{Por supuesto, esto limita nuestra solución a un máximo de 26 columnas. Decidimos ignorar ese detalle para no complicar el ejemplo.}~)

def crear_matriz_xy(filas, columnas):
    (@m = []@)
    for f in range(filas):
        y = str(f + 1)
        for c in range(columnas):
            x = LETRAS[c]
            (@m.append(x + y)@)
    (@return m@)
\end{codigo-python-sn}

\item Si representamos la matriz como una lista de filas:

\begin{codigo-python-sn}
def crear_matriz_xy(filas, columnas):
    (@m = []@)
    for f in range(filas): (~\circled{1}~)
        (@fila = []@)
        y = str(f + 1)
        for c in range(columnas):
            x = LETRAS[c]
            (@fila.append(x + y)@)
        (@m.append(fila)@)
    (@return m@)
\end{codigo-python-sn}


\item Si representamos la matriz como una lista de columnas:

\begin{codigo-python-sn}
def crear_matriz_xy(filas, columnas):
    (@m = []@)
    for c in range(columnas): (~\circled{2}~)
        (@columna = []@)
        x = LETRAS[c]
        for f in range(filas):
            y = str(f + 1)
            (@columna.append(x + y)@)
        (@m.append(columna)@)
    (@return m@)
\end{codigo-python-sn}

\begin{atencion}
\circled{1} \circled{2} Notar que el orden de los ciclos anidados es diferente; en cada caso
debemos analizar con cuidado si conviene iterar primero filas y luego columnas, o al revés.
\end{atencion}

\end{enumerate}
\end{solucion}



\section{Resumen}

\begin{itemize}

\item Python nos provee con varias estructuras que nos permiten agrupar los
datos que tenemos.  En particular, las {\bf tuplas} son estructuras
{\bf inmutables} que permiten agrupar valores al momento de crearlas, y las {\bf
listas} son estructuras {\bf mutables} que permiten agrupar valores, con la
posibilidad de agregar, quitar o reemplazar sus elementos.

\item Las tuplas se utilizan para modelar situaciones en las cuales al
momento de crearlas ya se sabe cuál va a ser la información a almacenar.
Por ejemplo, para representar una fecha, una carta de la baraja, una ficha
de dominó.

\item Las listas se utilizan en las situaciones en las que los elementos a
agrupar pueden ir variando a lo largo del tiempo.  Por ejemplo, para
representar un las notas de un alumno en diversas materias, los inscriptos
para un evento o la clasificación de los equipos en una competencia.

\item Las cadenas, tuplas y listas son tres tipos diferentes de {\bf
secuencias}. Las secuencias ofrecen un conjunto de operaciones básicas, como
obtener la longitud y recorrer sus elementos, que se aplican de la misma manera
sin importar qué tipo de secuencia es.

\end{itemize}

\begin{referencia_python}

\begin{sintaxis}{\lstinline!(valor1, valor2, valor3)!}
Las tuplas se definen como una sucesión de valores encerrados entre paréntesis
y separados por comas. Una vez definidas, no se pueden modificar los valores
asignados.

Casos particulares:
\begin{codigo-python-sn}
tupla_vacia = ()
tupla_unitaria = (3459,)
\end{codigo-python-sn}
\end{sintaxis}

\begin{sintaxis}{\lstinline![valor1, valor2, valor3]!}
Las listas se definen como una sucesión de valores encerrados entre corchetes y
separados por comas. Se les puede agregar, quitar o cambiar los valores que
contienen.

\begin{codigo-python-sn}
lista = [1, 2, 3]
lista[0] = 5
\end{codigo-python-sn}

Caso particular:
\begin{codigo-python-sn}
lista_vacia = []
\end{codigo-python-sn}
\end{sintaxis}

\begin{sintaxis}{\lstinline!x, y, z = secuencia!}
Es posible \emph{desempaquetar} una secuencia, asignando a la izquierda
tantas variables como elementos tenga la secuencia. Cada variable
tomará el valor del elemento que se encuentra en la misma posición.
\end{sintaxis}

\begin{sintaxis}{\lstinline!len(secuencia)!}
Devuelve la cantidad de elementos que contiene la secuencia, 0 si está vacía.
\end{sintaxis}

\begin{sintaxis}{\lstinline!for elemento in secuencia:!}
Itera uno a uno por los elementos de la secuencia.
\end{sintaxis}

\begin{sintaxis}{\lstinline!elemento in secuencia!}
Indica si el elemento se encuentra o no en la secuencia
\end{sintaxis}

\begin{sintaxis}{\lstinline!secuencia[i]!}
Corresponde al valor de la secuencia en la posición \lstinline!i!, comenzando
desde 0.

Si se utilizan números negativos, se puede acceder a los
elementos desde el último (\lstinline!-1!) hasta el primero
(\lstinline!-len(secuencia)!).

En el caso de las tuplas o cadenas (inmutables) sólo puede usarse para obtener el
valor, mientra que en las listas (mutables) puede usarse también para
modificar su valor.
\end{sintaxis}

\begin{sintaxis}{\lstinline!secuencia[i:j:k]!}
Permite obtener un segmento de la secuencia, desde la posición \lstinline!i!
inclusive, hasta la posición \lstinline!j! exclusive, con paso
\lstinline!k!.

En el caso de que se omita \lstinline!i!, se asume \lstinline!0!.  En el
caso de que se omita \lstinline!j!, se asume \lstinline!len(secuencia)!.
En el caso de que se omita \lstinline!k!, se asume 1. Si
se omiten todos, se obtiene una copia completa de la secuencia.
\end{sintaxis}

\begin{sintaxis}{\lstinline!lista.append(valor)!}
Agrega un elemento al final de la lista.
\end{sintaxis}

\begin{sintaxis}{\lstinline!lista.insert(posicion, valor)!}
Agrega un elemento a la lista, en la posición \lstinline!posicion!.
\end{sintaxis}

\begin{sintaxis}{\lstinline!lista.remove(valor)!}
Quita de la lista la primera aparción de elemento, si se encuentra.  De no
encontrarse en la lista, se produce un error.
\end{sintaxis}

\begin{sintaxis}{\lstinline!lista.pop()!}
Quita el elemento del final de la lista, y lo devuelve. Si la lista está
vacía, se produce un error.
\end{sintaxis}

\begin{sintaxis}{\lstinline!lista.pop(posicion)!}
Quita el elemento que está en la posición indicada, y lo devuelve. Si la
lista tiene menos de |posicion + 1| elementos, se produce un error.
\end{sintaxis}

\begin{sintaxis}{\lstinline!lista.index(valor)!}
Devuelve la posición de la primera aparición de valor.  Si no se encuentra
en la lista, se produce un error.
\end{sintaxis}

\begin{sintaxis}{\lstinline!sorted(secuencia)!}
Devuelve una lista nueva, con los elementos de la secuencia ordenados.
\end{sintaxis}

\begin{sintaxis}{\lstinline!lista.sort()!}
Ordena la misma lista.
\end{sintaxis}

\begin{sintaxis}{\lstinline!cadena.split(separador)!}
Devuelve una lista con los elementos de cadena, utilizando
\lstinline!separador! como separador de elementos.

Si se omite el separador, toma todos los espacios en blanco como
separadores.
\end{sintaxis}

\begin{sintaxis}{\lstinline!separador.join(lista)!}
Genera una cadena a partir de los elementos de \lstinline!lista!,
utilizando \lstinline!separador! como unión entre cada elemento y el
siguiente.
\end{sintaxis}
\end{referencia_python}


\newpage
\section{Ejercicios}

\extractionlabel{guia}
\begin{ejercicio}
Escribir una función que reciba una tupla de elementos e indique si se
encuentran ordenados de menor a mayor o no.
\end{ejercicio}


\extractionlabel{guia}
\begin{ejercicio}
{\bf Dominó.}
\begin{partes}
\item Escribir una función que indique si dos fichas de dominó
\emph{encajan} o no. Las fichas son recibidas en dos tuplas, por ejemplo:
\verb!(3,4)! y \verb!(5,4)!
\item Escribir una función que indique si dos fichas de dominó
\emph{encajan} o no. Las fichas son recibidas en una cadena, por ejemplo:
\verb!3-4 2-5!. {\bf Nota:} utilizar la función \verb!split! de las cadenas.
\end{partes}
\end{ejercicio}


\extractionlabel{guia}
\begin{ejercicio}{\bf Campaña electoral}
\begin{partes}
\item Escribir una función que reciba una tupla con nombres, y para cada
nombre imprima el mensaje \emph{Estimado <nombre>, vote por mí.}
\item Escribir una función que reciba una tupla con nombres, una posición
de origen \verb!p! y una cantidad \verb!n!, e imprima el mensaje anterior
para los \verb!n! nombres que se encuentran a partir de la posición
\verb!p!.
\item Modificar las funciones anteriores para que tengan en cuenta el
género del destinatario, para ello, deberán recibir una tupla de tuplas,
conteniendo el nombre y el género.
\end{partes}
\end{ejercicio}


\extractionlabel{guia}
\begin{ejercicio}
{\bf Vectores}
\begin{partes}
\item Escribir una función que reciba dos vectores y devuelva su producto
escalar.
\item Escribir una función que reciba dos vectores y devuelva si son o no
ortogonales.
\item Escribir una función que reciba dos vectores y devuelva si son
paralelos o no.
\item Escribir una función que reciba un vector y devuelva su norma.
\end{partes}
\end{ejercicio}


\extractionlabel{guia}
\begin{ejercicio}
Dada una lista de números enteros, escribir una función que:
\begin{partes}
\item Devuelva una lista con todos los que sean primos.
\item Devuelva la sumatoria y el promedio de los valores.
\item Devuelva una lista con el factorial de cada uno de esos números.
\end{partes}
\end{ejercicio}


\extractionlabel{guia}
\begin{ejercicio}
Dada una lista de números enteros y un entero k, escribir una función que:
\begin{partes}
\item Devuelva tres listas, una con los menores, otra con los mayores y
otra con los iguales a k.
\item Devuelva una lista con aquellos que son múltiplos de k.
\end{partes}
\end{ejercicio}


\extractionlabel{guia}
\begin{ejercicio}
Escribir una función que reciba una lista de tuplas (Apellido,
Nombre, Inicial\_segundo\_nombre) y devuelva una lista de cadenas
donde cada una contenga primero el nombre, luego la inicial con un punto, y luego el
apellido.
\end{ejercicio}


\extractionlabel{guia}
\begin{ejercicio}{\bf Inversión de listas}
\begin{partes}
\item Realizar una función que, dada una lista, devuelva una nueva lista cuyo
contenido sea igual a la original pero invertida. Así, dada la lista
\lstinline!['Di', 'buen', 'día', 'a', 'papa']!, deberá devolver
\lstinline!['papa', 'a', 'día', 'buen', 'Di']!.

\item Realizar otra función que invierta la lista, pero en lugar de devolver
una nueva, modifique la lista dada para invertirla, {\bf sin}  usar listas
auxiliares.
\end{partes}
\end{ejercicio}


\extractionlabel{guia}
\begin{ejercicio}
Escribir una función \texttt{empaquetar} para una lista, donde
epaquetar significa indicar la repetición de valores consecutivos
mediante una tupla (valor, cantidad de repeticiones). Por ejemplo,
\lstinline!empaquetar([1, 1, 1, 3, 5, 1, 1, 3, 3])! debe devolver
\lstinline![(1, 3), (3, 1), (5, 1), (1, 2), (3, 2)]!.
\end{ejercicio}


\extractionlabel{guia}
\begin{ejercicio}
{\bf Matrices.}
\begin{partes}
\item Escribir una función que reciba dos matrices y devuelva la suma.
\item Escribir una función que reciba dos matrices y devuelva el producto.
\item $\dificil$ Escribir una función que opere sobre una matriz y mediante
\emph{eliminación gaussiana} devuelva una matriz triangular superior.
\item $\dificil$ Escribir una función que indique si un grupo de vectores, recibidos
mediante una lista, son linealmente independientes o no.
\end{partes}
\end{ejercicio}


\extractionlabel{guia}
\begin{ejercicio}
$\dificil$ {\bf Plegado de un texto.} Escribir una función que reciba un
párrafo de texto (palabras separadas por espacios) y una longitud $n$, y
devuelva una lista conteniendo líneas de texto de longitud máxima $n$. Las
líneas deben ser cortadas correctamente en los espacios (sin cortar las
palabras). Asumir que ninguna palabra tiene longitud mayor a $n$. Ejemplo:

\begin{lstlisting}[numbers=none]
>>> plegar('El viejo Señor Gómez pedía queso, kiwi y habas, pero le ha tocado un saxofón', 20)
['El viejo Señor Gómez', 'pedía queso, kiwi y', 'habas, pero le ha', 'tocado un saxofón']
\end{lstlisting}
\end{ejercicio}


\extractionlabel{guia}
\begin{ejercicio}
{\bf Funciones que reciben funciones.}
\begin{partes}
\item Escribir una funcion llamada {\bf map}, que reciba una función y una
lista y devuelva la lista que resulta de aplicar la función recibida a
cada uno de los elementos de la lista recibida.
\item Escribir una función llamada {\bf filter}, que reciba una función y
una lista y devuelva una lista con los elementos de la lista recibida para
los cuales la función recibida devuelve un valor verdadero.
\item ¿En qué ejercicios de esta guía podría haber utilizado estas
funciones?
\end{partes}
\end{ejercicio}

\extractionlabel{guia}
\begin{ejercicio}
$\dificil$ {\bf Juegos sencillos (¡o no tanto!).}
Estos ejercicios permiten aplicar todo lo aprendido hasta ahora.
Recomendación: pensar bien el {\bf diseño}; y en particular elegir una {\bf
estructura de datos} apropiada para representar el
\emph{estado del juego}. Algunas preguntas que pueden ayudar:
\begin{itemize}
\item ¿Cómo actualizamos el estado del juego según cada acción realizada?
\item ¿Cómo detectamos si un movimiento es válido o no?
\item ¿Cómo detectamos si el juego ha terminado?
\item ¿Cómo se muestra la información en la consola a los jugadores?
\item ¿Cómo hacen los jugadores para ingresar las acciones a realizar?
\end{itemize}

\noindent Escribir un programa que permita jugar\footnote{No es el objetivo
implementar una \emph{inteligencia artificial}: en los juegos de dos o más
jugadores, solo aceptaremos jugadores humanos.}:
\begin{partes}
\item Torres de Hanói (\url{https://es.wikipedia.org/wiki/Torres_de_Hanoi})
\item Ta-Te-Ti (\url{https://es.wikipedia.org/wiki/Tres_en_linea})
\item Nim (\url{https://es.wikipedia.org/wiki/Nim_(juego)})
\item Generala (\url{https://es.wikipedia.org/wiki/Generala})
\end{partes}
\end{ejercicio}

\newpage
\begin{subappendices}
\section{Funciones de orden superior}
\label{map-filter}

En Python, se dice que \emph{las funciones son ciudadanos de primera clase}, ya
que una función es un valor (cuyo tipo de dato es |function|) que puede ser
manipulado como cualquier otro valor (|int|, |str|, etc.):

\begin{codigo-python-sn}
>>> def exclamar(s):
...     return '¡' + s + '!'
>>> exclamar
<function exclamar at 0x10532ef28>
>>> type(exclamar)
<class 'function'>
>>> exclamar('Hola')
'¡Hola!'
>>> f = exclamar
>>> f('Hola')
'¡Hola!'
\end{codigo-python-sn}

En el ejemplo, |exclamar| (sin los paréntesis) es una referencia a un valor de
tipo |function| (la función llamada ``exclamar''). Al hacer |f = exclamar|,
el resultado es que ambas variables (|exclamar| y |f|) hacen referencia a
\emph{la misma} función. Por lo tanto, podemos llamar a la función con
|f(...)| o con |exclamar(...)|.

Aprovechando esto, una función puede recibir otra función como parámetro:

\begin{codigo-python-sn}
>>> def aplicar((@f@), x, n):
...     """Devuelve el resultado de f(f(...f(f(x)))), aplicando en total n
...        veces la función f"""
...     for i in range(n):
...         x = (@f(x)@)
...     return x
>>> aplicar((@exclamar@), 'Hola', 5)
'¡¡¡¡¡Hola!!!!!'
>>> def alargar(s):
...     return s + s[-1]
>>> alargar('Hola')
'Holaa'
>>> aplicar((@alargar@), 'Hola', 5)
'Holaaaaaa'
\end{codigo-python-sn}

Una función también puede devolver otra función:

\begin{codigo-python-sn}
>>> def puntuador(inicio, fin):
...    """Recibe dos cadenas de inicio y fin, y devuelve una función
...       que permite agregar signos de puntuación a una cadena."""
...    (@def puntuar(cadena):@)
...        return inicio + cadena + fin
...    (@return puntuar@)
>>> exclamar = puntuador('¡', '!')
>>> type(exclamar)
<class 'function'>
>>> exclamar('Hola')
'¡Hola!'
>>> preguntar = puntuador('¿', '?')
>>> preguntar('Qué')
'¿Qué?'
\end{codigo-python-sn}

\begin{observacion}
Se dice que una función es \emph{de orden superior} cuando recibe una función
por parámetro, o bien cuando devuelve una función.
\end{observacion}

En nuestros ejemplos, |aplicar| y
|puntuador| son de orden superior. Las funciones de orden superior son
características de un estilo de programación llamado \emph{programación
funcional}.

\subsection*{Funciones anónimas}

Cuando trabajamos con funciones de orden superior, suele ser conveniente
definir \emph{funciones anónimas}. En Python, cualquier función que evalúa
una única expresión y la devuelve puede ser definida en forma anónima usando
la sintaxis |lambda|:

\begin{itemize}
\item |lambda x: x * 2| es ``una función que dado un número $x$ devuelve $2x$''.
\item |lambda x, y: x + y| es ``una función que dados $x$ e $y$ devuelve la suma de ambos''.
\end{itemize}

La sintaxis general es |lambda <parámetros>: <expresión>|. El valor de retorno de una
función definida con |lambda| es el resultado de la |<expresión>|; el |return| es
implícito.

\begin{sabias_que}
La sintaxis |lambda| hace referencia al \emph{cálculo lambda} (un sistema
formal introducido por Alonzo Church y Stephen Kleene en la década de
1930), en el que las funciones se escriben anteponiendo la letra griega
$\lambda$. Por ejemplo: una función que devuelve el valor absoluto de un
número (invocando a la función |abs|), en notación lambda se escribe
$\lambda x. (abs x)$, mientras que en Python se escribe |lambda x: abs(x)|.
\end{sabias_que}

Las funciones |exclamar| y |alargar| podrían haber sido definidas usando la
sintaxis |lambda|:

\begin{codigo-python-sn}
>>> (@exclamar = lambda s: '¡' + s + '!'@)
>>> type(exclamar)
<class 'function'>
>>> exclamar('Hola')
'¡Hola!'
>>> (@alargar = lambda s: s + s[-1]@)
>>> alargar('Hola')
'Holaa'
\end{codigo-python-sn}

Pero las funciones anónimas son especialmente útiles cuando trabajamos con
funciones de orden superior. Por ejemplo, podemos pasar un |lambda| a una
función que recibe una función:

\begin{codigo-python-sn}
>>> aplicar((@lambda x: 2 * x@), 1, 4)
16(~\footnote{Ya que $2 \cdot (2 \cdot (2 \cdot (2 \cdot 1))) = 16$.}~)
\end{codigo-python-sn}

Una función que devuelve una función también puede devolver un |lambda|:

\begin{codigo-python-sn}
>>> def puntuador(inicio, fin):
...    """Recibe dos cadenas de inicio y fin, y devuelve una función
...       que permite agregar signos de puntuación a una cadena."""
...    (@return lambda cadena: inicio + cadena + fin@)
>>> citar = puntuador('“', '”')
>>> citar('Del dicho al hecho hay mucho trecho')
'“Del dicho al hecho hay mucho trecho(~”~)'
>>> def componer(f, g):
...     """Devuelve una función que dado x devuelve g(f(x))"""
...     (@return lambda x: g(f(x))@)
>>> gritar = componer(alargar, exclamar)
>>> gritar('Hola')
'¡Holaa!'
\end{codigo-python-sn}

\subsection*{Funciones de orden superior y secuencias}

Las funciones |sorted|, |map| y |filter| son ejemplos de funciones de orden superior
disponibles en Python, que además permiten operar sobre secuencias.

\begin{itemize}

\item
La función |sorted| ya fue mencionada en la Sección~\ref{sorted}. Si recibe una
lista de cadenas, devuelve una nueva lista ordenada:

\begin{codigo-python-sn}
>>> L = ['zorro', '', '935', 'alpaca', '1312', '-----', 'gato']
>>> sorted(L)
['', '-----', '1312', '935', 'alpaca', 'gato', 'zorro']
\end{codigo-python-sn}

El comportamiento por omisión es ordenar las cadenas en orden \emph{lexicográfico}
(comparando el primer caracter, luego el segundo, etc.). Este comportamiento es
útil en la gran mayoría de los casos, pero ¿qué pasa si necesitamos ordenar las
cadenas por otro criterio?

Podemos pasarle a |sorted| un parámetro adicional, que es una función que determina para
cada elemento el valor utilizado para ordenar. Por ejemplo, si queremos ordenar
las cadenas según su longitud (de menor a mayor), tenemos que pasarle a
|sorted| una función que recibe una cadena y devuelve su longitud. Esa función
no es otra que la función |len|:

\begin{codigo-python-sn}
>>> sorted(L, (@key=len@))(~\footnote{\texttt{key=len} indica que el parámetro adicional es un \emph{parámetro con nombre} (en inglés \emph{keyword argument}): el nombre del parámetro es \texttt{key} y el valor es \texttt{len}.}~)
['', '935', '1312', 'gato', 'zorro', '-----', 'alpaca']
\end{codigo-python-sn}

\item
La función |map| recibe una función y una secuencia, y devuelve la secuencia
resultante de aplicar la función a cada uno de los elementos de la secuencia
original.

\begin{codigo-python-sn}
>>> map(lambda x: x * 2, [1, 2, 3, 4])
<map object at 0x10edcba20>
\end{codigo-python-sn}

El valor de retorno de |map| es una secuencia, pero no es una tupla ni una
lista. Por suerte, podemos convertir cualquier secuencia en una lista con la
función |list|:

\begin{codigo-python-sn}
>>> list(map((@lambda x: x * 2@), [1, 2, 3, 4, 5]))
[2, 4, 6, 8, 10]
>>> list(map((@exclamar@), ['Usando', 'la', 'función', 'map']))
['¡Usando!', '¡la!', '¡función!', '¡map!']
>>> list(map((@int@), [8.3, '42', True]))
[8, 42, 1]
\end{codigo-python-sn}

\item
La función |filter| recibe una función |f| y una secuencia, y devuelve una secuencia
que contiene los elementos de la secuencia original para los cuales el
resultado de |f(x)| es |True|~.

\begin{codigo-python-sn}
>>> list(filter((@lambda x: x % 2 == 0@), [1, 2, 3, 4, 5]))
[2, 4]
>>> list(filter((@lambda s: len(s) > 2@), ['Usando', 'la', 'función', 'filter']))
['Usando', 'función', 'filter']
\end{codigo-python-sn}

\end{itemize}
\end{subappendices}
