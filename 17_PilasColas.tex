\chapter{Pilas y colas}

En esta unidad veremos dos ejemplos de tipos abstractos de datos, de los más
clásicos: {\it pilas} y {\it colas}.

\section{Pilas}

Una {\it pila} es un TAD que tiene las siguientes operaciones (se describe también la acción que
lleva adelante cada operación):

\begin{itemize}
\item \lstinline+__init__+: Inicializa una pila nueva, vacía.

\item \lstinline!apilar!: Agrega un nuevo elemento a la pila.

\item \lstinline!desapilar!: Elimina el tope de la pila y lo devuelve.
El elemento que se devuelve es siempre el último que se agregó.

\item \lstinline!esta_vacia!: Devuelve \lstinline!True! o \lstinline!False!
según si la pila está vacía o no.

\end{itemize}

El comportamiento de una pila se puede describir mediante la frase
``Lo último que se apiló es lo primero que se usa'', que es exactamente lo que
uno hace con una pila (de platos por ejemplo): en una pila de platos uno sólo
puede ver la apariencia completa del plato de arriba, y sólo puede tomar el
plato de arriba (si se intenta tomar un plato del medio de la pila lo más
probable es que alguno de sus vecinos, o él mismo, se arruine).

Como ya se dijo, al crear un tipo abstracto de datos, es importante decidir
cuál será la representación a utilizar.  En el caso de la pila, si bien puede
haber más de una representación, por ahora veremos la más sencilla:
representaremos una pila mediante una lista de Python.

Sin embargo, para los que construyen programas que usan un TAD vale el
siguiente llamado de atención:

\begin{atencion}
Al usar esa pila dentro de un programa, deberemos ignorar que se está
trabajando sobre una lista: solamente podremos usar los métodos de pila.

Si alguien viola este principio, y usa la representación dentro del
programa usuario, termina por recibir el peor castigo imaginable para un/a
programador/a: sus programas pueden dejar de funcionar el cualquier
momento, tan pronto como quien produce del TAD decida cambiar, aunque sea
sutilmente, dicha representación.
\end{atencion}

\subsection{Pilas representadas por listas}

Definiremos una clase \lstinline!Pila! con un atributo, \lstinline!items!,
de tipo lista, que contendrá los elementos de la pila. El tope de la pila
se encontrará en la última posición de la lista, y cada vez que se apile un
nuevo elemento, se lo agregará al final.

El método \lstinline+__init__+ no recibirá parámetros adicionales, ya que
deberá crear una pila vacía (que representaremos por una lista vacía):

\begin{codigo-python-sn}
class Pila:
    """ Representa una pila con operaciones de apilar, desapilar y
        verificar si está vacía. """

    def __init__(self):
        """ Crea una pila vacía. """
        # La pila vacía se representa con una lista vacía
        self.items=[]
\end{codigo-python-sn}

El método \lstinline!apilar! se implementará agregando el nuevo elemento al
final de la lista:

\begin{codigo-python-sn}
    def apilar(self, x):
        """ Agrega el elemento x a la pila. """
        # Apilar es agregar al final de la lista.
        self.items.append(x)
\end{codigo-python-sn}

Para implementar \lstinline!desapilar!, se usará el método \lstinline!pop!
de lista que hace exactamente lo requerido: elimina el último elemento de
la lista y devuelve el valor del elemento eliminado. Si la lista está vacía
levanta una excepción, haremos lo mismo, pero cambiaremos el tipo de
excepción, para no revelar la implementación.

\begin{codigo-python-sn}
    def desapilar(self):
        """ Devuelve el elemento tope y lo elimina de la pila.
            Si la pila está vacía levanta una excepción. """
        try:
            return self.items.pop()
        except IndexError:
            raise ValueError("La pila está vacía")
\end{codigo-python-sn}

\begin{observacion}
Utilizamos los métodos \lstinline!append! y \lstinline!pop! de las listas de
Python, porque sabemos que estos métodos se ejecutan en tiempo constante.
Queremos que el tiempo de apilar o desapilar de la pila no dependa de la
cantidad de elementos contenidos.
\end{observacion}

Finalmente, el método para indicar si se trata de una pila vacía.

\begin{codigo-python-sn}
    def esta_vacia(self):
        """ Devuelve True si la lista está vacía, False si no. """
        return self.items == []
\end{codigo-python-sn}

Construimos algunas pilas y operamos con ellas:

\begin{codigo-python-sn}
>>> from clasePila import Pila
>>> p = Pila()
>>> p.esta_vacia()
True
>>> p.apilar(1)
>>> p.esta_vacia()
False
>>> p.apilar(5)
>>> p.apilar("+")
>>> p.apilar(22)
>>> p.desapilar()
22
>>> p
<clasePila.Pila instance at 0xb7523f4c>
>>> q=Pila()
>>> q.desapilar()
Traceback (most recent call last):
  File "<stdin>", line 1, in <module>
  File "clasePila.py", line 24, in desapilar
    raise ValueError("La pila está vacía")
ValueError: La pila está vacía
\end{codigo-python-sn}

\subsection{Uso de pila: calculadora científica}

La famosa calculadora portátil HP-35 (de 1972) popularizó la notación
polaca inversa (o notación prefijo) para hacer cálculos sin necesidad de
usar paréntesis. Esa notación, inventada por el lógico polaco Jan
Lukasiewicz en 1920, se basa en el principio de que un operador siempre se
escribe a continuación de sus operandos. La operación $(5-3)+8$ se
escribirá como \lstinline|5 3 - 8 +|, que se interpretará como: ``restar 3
de 5, y al resultado sumarle 8''.

Es posible implementar esta notación de manera sencilla usando una pila de
la siguiente manera, a partir de una cadena de entrada de valores separados
por blancos:

\begin{itemize}
\item Mientras se lean números, se apilan.

\item En el momento en el que se detecta una operación binaria \verb|+|,
\verb|-|, \verb|*|, \verb|/| o \verb|%| se desapilan los dos últimos
números apilados, se ejecuta la operación indicada, y el resultado de esa
operación se apila.

\item Si la expresión está bien formada, tiene que quedar al final un único
número en la pila (el resultado).

\item Los posibles errores son:

\begin{itemize}
\item Queda más de un número al final (por ejemplo si la cadena de entrada
fue \verb|"5 3"|),

\item Ingresa algún caracter que no se puede interpretar ni como número ni como
una de las cinco operaciones válidas (por ejemplo si la cadena de entrada
fue \verb|"5 3 &"|)

\item No hay suficientes operandos para realizar la operación (por ejemplo
si la cadena de entrada fue \verb|"5 3 - +"|).
\end{itemize}
\end{itemize}

La siguiente es la estrategia de resolución:

Dada una cadena con la expresión a evaluar, podemos separar sus componentes
utilizando el método \lstinline!split()!.  Recorreremos luego la lista de
componentes realizando las acciones indicadas en el párrafo anterior,
utilizando una pila auxiliar para operar. Si la expresión está bien formada
devolveremos el resultado, de lo contrario levantaremos una excepción
(devolveremos \lstinline!None!).

En el Código \ref{calculadora_polaca} está la implementación de la
calculadora descripta.

\begin{codigo}{\label{calculadora_polaca} calculadora\_polaca.py}{Una calculadora polaca inversa}
\lstinputlisting[basicstyle=\small\ttfamily]{src/17_pilas_colas/calculadora_polaca.py}
\end{codigo}

Veamos algunos casos de prueba:

\begin{itemize}
\item El caso de una expresión que es sólo un número (es correcta):

\begin{codigo-python-sn}
>>> calculadora_polaca.main()
Ingrese la expresion a evaluar: 5
DEBUG: 5
DEBUG: apila  5.0
5.0
\end{codigo-python-sn}

\item El caso en el que sobran operandos:

\begin{codigo-python-sn}
>>> calculadora_polaca.main()
Ingrese la expresion a evaluar: 4 5
DEBUG: 4
DEBUG: apila  4.0
DEBUG: 5
DEBUG: apila  5.0
DEBUG: error pila sobran operandos
Traceback (most recent call last):
  File "<stdin>", line 1, in <module>
  File "calculadora_polaca.py", line 64, in main
    print calculadora_polaca(elementos)
  File "calculadora_polaca.py", line 59, in calculadora_polaca
    raise ValueError("Sobran operandos")
ValueError: Sobran operandos
\end{codigo-python-sn}

\item El caso en el que faltan operandos:

\begin{codigo-python-sn}
>>> calculadora_polaca.main()
Ingrese la expresion a evaluar: 4 %
DEBUG: 4
DEBUG: apila  4.0
DEBUG: %
DEBUG: desapila  4.0
DEBUG: error pila faltan operandos
Traceback (most recent call last):
  File "<stdin>", line 1, in <module>
  File "calculadora_polaca.py", line 64, in main
    print calculadora_polaca(elementos)
  File "calculadora_polaca.py", line 37, in calculadora_polaca
    raise ValueError("Faltan operandos")
ValueError: Faltan operandos
\end{codigo-python-sn}

\item El caso de un operador inválido:

\begin{codigo-python-sn}
>>> calculadora_polaca.main()
Ingrese la expresion a evaluar: 4 5 &
DEBUG: 4
DEBUG: apila  4.0
DEBUG: 5
DEBUG: apila  5.0
DEBUG: &
Traceback (most recent call last):
  File "<stdin>", line 1, in <module>
  File "calculadora_polaca.py", line 64, in main
    print calculadora_polaca(elementos)
  File "calculadora_polaca.py", line 26, in calculadora_polaca
    raise ValueError("Operando inválido")
ValueError: Operando inválido
\end{codigo-python-sn}

\item \verb|4 % 5|
\begin{codigo-python-sn}
>>> calculadora_polaca.main()
Ingrese la expresion a evaluar: 4 5 %
DEBUG: 4
DEBUG: apila  4.0
DEBUG: 5
DEBUG: apila  5.0
DEBUG: %
DEBUG: desapila  5.0
DEBUG: desapila  4.0
DEBUG: apila  4.0
4.0
\end{codigo-python-sn}

\item \verb|(4 + 5) * 6|:

\begin{codigo-python-sn}
>>> calculadora_polaca.main()
Ingrese la expresion a evaluar: 4 5 + 6 *
DEBUG: 4
DEBUG: apila  4.0
DEBUG: 5
DEBUG: apila  5.0
DEBUG: +
DEBUG: desapila  5.0
DEBUG: desapila  4.0
DEBUG: apila  9.0
DEBUG: 6
DEBUG: apila  6.0
DEBUG: *
DEBUG: desapila  6.0
DEBUG: desapila  9.0
DEBUG: apila  54.0
54.0
\end{codigo-python-sn}

\item \verb|4 * (5 + 6)|:

\begin{codigo-python-sn}
>>> calculadora_polaca.main()
Ingrese la expresion a evaluar: 4 5 6 + *
DEBUG: 4
DEBUG: apila  4.0
DEBUG: 5
DEBUG: apila  5.0
DEBUG: 6
DEBUG: apila  6.0
DEBUG: +
DEBUG: desapila  6.0
DEBUG: desapila  5.0
DEBUG: apila  11.0
DEBUG: *
DEBUG: desapila  11.0
DEBUG: desapila  4.0
DEBUG: apila  44.0
44.0
\end{codigo-python-sn}

\item \verb|(4 + 5) * (3 + 8)|:

\begin{codigo-python-sn}
>>> calculadora_polaca.main()
Ingrese la expresion a evaluar: 4 5 + 3 8 + *
DEBUG: 4
DEBUG: apila  4.0
DEBUG: 5
DEBUG: apila  5.0
DEBUG: +
DEBUG: desapila  5.0
DEBUG: desapila  4.0
DEBUG: apila  9.0
DEBUG: 3
DEBUG: apila  3.0
DEBUG: 8
DEBUG: apila  8.0
DEBUG: +
DEBUG: desapila  8.0
DEBUG: desapila  3.0
DEBUG: apila  11.0
DEBUG: *
DEBUG: desapila  11.0
DEBUG: desapila  9.0
DEBUG: apila  99.0
99.0
\end{codigo-python-sn}

\end{itemize}

\ejercicioc{Si se oprime la tecla <BACKSPACE> (o <$\leftarrow$>) del
teclado, se borra el último caracter ingresado. Construir una función
\lstinline!visualizar! para modelar el tipeo de una cadena de caracteres
desde un teclado:

La función recibe una cadena de caracteres con todo lo que el usuario
ingresó por teclado (incluyendo <BACKSPACE>  que se reconoce como
\lstinline|\b|), y devuelve el texto tal como debe presentarse (por
ejemplo, \lstinline|visualizar("Holas\b chau")| debe devolver 'Hola chau').

Atención, que muchas veces la gente aprieta de más la tecla de <BACKSPACE>,
y no por eso hay que cancelar la ejecución de toda la función.
}

\subsection{¿Cuánto cuestan los métodos?}

Al elegir de una representación debemos tener en cuenta cuánto nos costarán
los métodos implementados. En nuestro caso, el tope de la pila se encuentra
en la última posición de la lista, y cada vez que se apila un nuevo
elemento, se lo agregará al final.

Por lo tanto se puede implementar el método \lstinline!apilar! mediante un
\lstinline!append!  de la lista, {\it que se ejecuta en tiempo constante}.
También el método \lstinline!desapilar!, que se implementa mediante
\lstinline!pop! de lista, {\it se ejecuta en tiempo constante}.

Vemos que la alternativa que elegimos fue barata.

Otra alternativa posible hubiera sido agregar el nuevo elemento en la
posición $0$ de la lista, es decir implementar el método \lstinline!apilar!
mediante \lstinline|self.items.insert(0,x)| y el método
\lstinline!desapilar! mediante \lstinline|del self.items[0]|. Sin embargo,
ésta no es una solución inteligente, ya que tanto insertar al comienzo de
la lista como borrar al comienzo de la lista {\it consumen tiempo
proporcional a la longitud de la lista}.

\ejercicioc{Diseñar un pequeño experimento para verificar que la
implementación elegida es mucho mejor que la implementación con listas en
la cual el elemento nuevo se inserta al principio de la lista}.

\ejercicioc{Implementar pilas mediante listas enlazadas. Analizar el costo
de los métodos a utilizar.}

\section{Colas}
Todos sabemos lo que es una cola. Más aún, ¡estamos hartos de hacer colas!

El TAD {\it cola} modela precisamente ese comportamiento: el primero que llega
es el primero en ser atendido, los demás se van {\it encolando} hasta que
les toque su turno.

Sus operaciones son:

\begin{itemize}

\item \lstinline+__init__+: Inicializa una cola nueva, vacía.

\item \lstinline!encolar!: Agrega un nuevo elemento al final de la cola.

\item \lstinline!desencolar!: Elimina el primero de la cola y lo devuelve.

\item \lstinline!esta_vacia!: Devuelve \lstinline!True! o
\lstinline!False! según si la cola está vacía o no.

\end{itemize}

\subsection{Colas implementadas sobre listas}

Al momento de realizar una implementación de una Cola, deberemos
preguntarnos ¿Cómo representamos a las colas? Veamos, en primer lugar, si
podemos implementar colas usando listas de Python, como hicimos con la Pila.

Definiremos una clase {\tt Cola} con un atributo, {\tt items}, de tipo
lista, que contendrá los elementos de la cola. El primero de la cola se
encontrará en la primera posición de la lista, y cada vez que encole un
nuevo elemento, se lo agregará al final.

El método \lstinline+__init__+ no recibirá parámetros adicionales, ya que
deberá crear una cola vacía (que representaremos por una lista vacía):

\begin{codigo-python-sn}
class Cola:
    """ Representa a una cola, con operaciones de encolar y
        desencolar.  El primero en ser encolado es también el primero
        en ser desencolado. """

    def __init__(self):
        """ Crea una cola vacía. """
        # La cola vacía se representa por una lista vacía
        self.items=[]
\end{codigo-python-sn}

El método \lstinline!encolar! se implementará agregando el nuevo elemento
al final de la lista:

\begin{codigo-python-sn}
    def encolar(self, x):
        """ Agrega el elemento x como último de la cola. """
        self.items.append(x)
\end{codigo-python-sn}

Para implementar \lstinline!desencolar!, se eliminará el primer elemento de
la lista y se devolverá el valor del elemento eliminado, utilizaremos
nuevamente el método \lstinline!pop!, pero en este caso le pasaremos la
posición $0$, para que elimine el primer elemento, no el último. Si la cola
está vacía se levantará una excepción.

\begin{codigo-python-sn}
    def desencolar(self):
        """ Elimina el primer elemento de la cola y devuelve su
            valor. Si la cola está vacía, levanta ValueError. """
        try:
          return self.items.pop(0)
        except:
          raise ValueError("La cola está vacía")
\end{codigo-python-sn}

Por último, el método \lstinline!esta_vacia!, que indicará si la cola está
o no vacía.

\begin{codigo-python-sn}
    def esta_vacia(self):
        """ Devuelve True si la cola esta vacía, False si no."""
        return self.items == []
\end{codigo-python-sn}

Veamos una ejecución de este código:

\begin{codigo-python-sn}
>>> from claseCola import Cola
>>> q = Cola()
>>> q.esta_vacia()
True
>>> q.encolar(1)
>>> q.encolar(2)
>>> q.encolar(5)
>>> q.esta_vacia()
False
>>> q.desencolar()
1
>>> q.desencolar()
2
>>> q.encolar(8)
>>> q.desencolar()
5
>>> q.desencolar()
8
>>> q.esta_vacia()
True
>>> q.desencolar()
Traceback (most recent call last):
  File "<stdin>", line 1, in <module>
  File "claseCola.py", line 24, in desencolar
    raise ValueError("La cola está vacía")
ValueError: La cola está vacía
\end{codigo-python-sn}

¿Cuánto cuesta esta implementación?  Dijimos en la sección anterior que
usar listas comunes para borrar elementos al principio da muy malos
resultados. Como en este caso necesitamos agregar elementos por un extremo
y quitar por el otro extremo, esta implementación será una buena
alternativa sólo si nuestras listas son pequeñas, ya que e medida que la
cola crece, el método \lstinline!desencolar! tardará cada vez más.

Pero si queremos hacer que tanto el \lstinline!encolar! como el
\lstinline!desencolar!  se ejecuten en tiempo constante, debemos apelar a
otra implementación.

\subsection{Colas y listas enlazadas}

En la unidad anterior vimos la clase \lstinline!ListaEnlazada!.
La clase presentada ejecutaba la inserción en la primera posición en
tiempo constante, pero el \lstinline|append| se había convertido en lineal.

Sin embargo, como ejercicio, se propuso mejorar el \lstinline|append|,
agregando un nuevo atributo que apunte al último nodo, de modo de poder
agregar elementos en tiempo constante.

Si esas mejoras estuvieran hechas, cambiar nuestra clase \lstinline!Cola!
para que utilice la \lstinline!ListaEnlazada! sería tan simple como cambiar
el constructor, para que en lugar de construir una lista de Python
construyera una lista enlazada.

\begin{codigo-python-sn}
class Cola:
    """ Cola implementada sobre lista enlazada"""
    def __init__(self):
        """ Crea una cola vacía. """
        # La cola se representa por una lista enlazada vacía.
        self.items = claseListaEnlazadaConUlt.ListaEnlazada()
\end{codigo-python-sn}

Sin embargo, una \lstinline!Cola! es bastante más sencilla que una
\lstinline!ListaEnlazadaConUlt!, por lo que también podemos
implementar una clase \lstinline!Cola! utilizando las técnicas de referencias,
que se vieron en las {\it listas enlazadas}.

Planteamos otra solución posible para obtener una cola que sea eficiente tanto al
encolar como al desencolar, utilizando los nodos de las listas enlazadas,
y solamente implementaremos insertar al final y remover al principio.

Para ello, la cola deberá tener dos atributos, \lstinline!self.primero! y
\lstinline!self.ultimo!, que en todo momento deberán apuntar al primer y
último nodo de la cola, es decir que serán los invariantes de esta cola.

En primer lugar los crearemos vacíos, ambos referenciando a
\lstinline!None!.

\begin{codigo-python-sn}
    def __init__(self):
        """ Crea una cola vacía. """
        # En el primer momento, tanto el primero como el último son None
        self.primero = None
        self.ultimo = None
\end{codigo-python-sn}

Al momento de encolar, hay dos situaciones a tener en cuenta:
\begin{itemize}

\item Si la cola está vacía (es decir, \lstinline!self.ultimo! es
\lstinline!None!), tanto \lstinline!self.primero! como
\lstinline!self.ultimo! deben pasar a referenciar al nuevo nodo, ya que
este nodo será a la vez el primero y el último.

\item Si ya había nodos en la cola, simplemente hay que agregar el nuevo a
continuación del último y actualizar la referencia de
\lstinline!self.ultimo!.

\end{itemize}

El código resultante es el siguiente.

\begin{codigo-python-sn}
    def encolar(self, x):
        """ Agrega el elemento x como último de la cola. """
        nuevo = Nodo(x)
        # Si ya hay un último, agrega el nuevo y cambia la referencia.
        if self.ultimo:
            self.ultimo.prox = nuevo
            self.ultimo = nuevo
        # Si la cola estaba vacía, el primero es también el último.
        else:
            self.primero = nuevo
            self.ultimo = nuevo
\end{codigo-python-sn}

Al momento de desencolar, será necesario verificar que la cola no esté
vacía, y de ser así levantar una excepción.  Si la cola no está vacía,
se almacena el valor del primer nodo de la cola y luego se avanza la
referencia \lstinline!self.primero! al siguiente elemento.

Nuevamente hay un caso particular a tener en cuenta y es el que sucede
cuando luego de eliminar el primer nodo de la cola, la cola queda vacía.
En este caso, además de actualizar la referencia de
\lstinline!self.primero!, también hay que actualizar la referencia de
\lstinline!self.ultimo!.

\begin{codigo-python-sn}
    def desencolar(self):
        """ Elimina el primer elemento de la cola y devuelve su
            valor. Si la cola está vacía, levanta ValueError. """
        # Si hay un nodo para desencolar
        if self.primero:
            valor = self.primero.dato
            self.primero = self.primero.prox
            # Si después de avanzar no quedó nada, también hay que
            # eliminar la referencia del último.
            if not self.primero:
                self.ultimo = None
            return valor
        else:
            raise ValueError("La cola está vacía")
\end{codigo-python-sn}

Finalmente, para saber si la cola está vacía, es posible verificar tanto si
\lstinline!self.primero! o \lstinline!self.ultimo! referencian a
\lstinline!None!.

\begin{codigo-python-sn}
    def esta_vacia(self):
        """ Devuelve True si la cola esta vacía, False si no."""
        return self.items == []
\end{codigo-python-sn}

Una vez implementada toda la interfaz de la cola, podemos probar el TAD
resultante

\begin{codigo-python-sn}
>>> from claseColaEnlazada import Cola
>>> q = Cola()
>>> q.esta_vacia()
True
>>> q.encolar("Manzanas")
>>> q.encolar("Peras")
>>> q.encolar("Bananas")
>>> q.esta_vacia()
False
>>> q.desencolar()
'Manzanas'
>>> q.desencolar()
'Peras'
>>> q.encolar("Guaraná")
>>> q.desencolar()
'Bananas'
>>> q.desencolar()
'Guaraná'
>>> q.desencolar()
Traceback (most recent call last):
  File "<stdin>", line 1, in <module>
  File "claseColaEnlazada.py", line 42, in desencolar
    raise ValueError("La cola está vacía")
ValueError: La cola está vacía
\end{codigo-python-sn}

\ejercicioc{Este ejercicio surgió (y lo hicieron ya muchas generaciones de alumnos),
haciendo cola:

Hace un montón de años había una viejísma sucursal del correo en la vereda impar
de Av. de Mayo al 800 que tenía un cartel que decía ``No se recibirán más de 5 cartas por persona''.
O sea que la gente entregaba sus cartas (hasta la cantidad permitida) y luego tenía que volver
a hacer la cola si tenía más cartas para despachar.

Modelar una cola de correo generalizada, donde en la inicialización se indica la cantidad
(no necesariamente 5) de cartas que se reciben por persona.}

\section{Resumen}

\begin{itemize}

\item Una {\bf pila} es un tipo abstracto de datos que permite agregar
elementos y sacarlos en el orden inverso al que se los colocó, de la misma
forma que una pila (de platos, libros, cartas, etc) en la vida real.

\item Las pilas son útiles en las situaciones en las que se desea operar
primero con los últimos elementos agregados, como es el caso de la notación
polaca inversa.

\item Una {\bf cola} es un tipo abstracto de datos que permite agregar
elementos y sacarlos en el mismo orden en que se los colocó, como una cola
de atención en la vida real.

\item Las colas son útiles en las situaciones en las que se desea operar
con los elementos en el orden en el que se los fue agregando, como es el
caso de un cola de atención de clientes.

\end{itemize}

\newpage
\section{Ejercicios}

\extractionlabel{guia}
\begin{ejercicio}
Escribir una clase {\it TorreDeControl} que modele el trabajo de una torre
de control de un aeropuerto, con una pista de aterrizaje.  La torre trabaja
en dos etapas: {\it reconocimiento} y {\it acción}.
\begin{partes}
    \item Escribir un método \verb!reconocimiento!, que verifique si hay algún
nuevo avión esperando para aterrizar y/o despegar, y de ser así los encole
en la cola correspondiente. Para ello, utilizar \verb!random.randrange(2)!.
    \item Escribir un método \verb!acción!, que haga aterrizar o
bien despegar, al primero de los aviones que esté esperando (los que
esperan para aterrizar tienen prioridad). Debe desencolar el avión de su
cola y devolver la información correspondiente.
    \item Escribir un método \verb!__str__! que imprima el estado actual de
ambas colas.
    \item Escribir un programa que inicialice la torre de control, y luego llame
continuamente a los métodos \verb!reconocimiento! y \verb!acción!,
imprimiendo la acción tomada y el estado de la torre de control cada vez.
\end{partes}
\end{ejercicio}


\extractionlabel{guia}
\begin{ejercicio}
Atención a los pacientes de un consultorio médico, con varios doctores.
\begin{partes}
    \item Escribir una clase {\it ColaDePacientes}, con los métodos {\it
nuevo\_paciente}, que reciba el nombre del paciente y lo encole, y un
método {\it proximo\_paciente} que devuelva el primer paciente en la cola y
lo desencole.
    \item Escribir una clase {\it Recepcion}, que contenga un diccionario con
las colas correspondientes a cada doctor o doctora, y los métodos {\it
nuevo\_paciente} que reciba el nombre del paciente y del especialista, y
{\it proximo\_paciente} que reciba el nombre de la persona liberada y
devuelva el próximo paciente en espera.
    \item Escribir un programa que permita al usuario ingresar nuevos pacientes
o indicar que un consultorio se ha liberado y en ese caso imprima el
próximo paciente en espera.
\end{partes}
\end{ejercicio}


\extractionlabel{guia}
\begin{ejercicio}
Juego de Cartas
\begin{partes}
    \item Crear una clase {\it Carta} que contenga un palo y un valor.
    \item Crear una clase {\it PilaDeCartas} que vaya apilando las cartas una
debajo de otra, pero sólo permita apilarlas si son de un número
inmediatamente inferior y de distinto palo. Si se intenta apilar una carta
incorrecta, debe lanzar una excepción.
    \item Agregar el método {\it \_\_str\_\_} a la clase PilaDeCartas, para
imprimir las cartas que se hayan apilado hasta el momento.
\end{partes}
\end{ejercicio}

\newpage
\section{Apéndice}

A continuación el código completo de la pila y las colas implementadas en
esta unidad.

\begin{codigo}{clasePila.py}{Implementación básica de una pila}
\lstinputlisting[basicstyle=\small\ttfamily]{src/17_pilas_colas/clasePila.py}
\end{codigo}

\begin{codigo}{claseCola.py}{Implementación básica de una cola}
\lstinputlisting[basicstyle=\small\ttfamily]{src/17_pilas_colas/claseCola.py}
\end{codigo}

\begin{codigo}{claseColaEnlazada.py}{Implementación de una cola enlazada}
\lstinputlisting[basicstyle=\small\ttfamily]{src/17_pilas_colas/claseColaEnlazada.py}
\end{codigo}

