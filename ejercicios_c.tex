\chapter{Lenguaje C}

\begin{ejercicio}
Escribir una función que permita calcular el área de un rectángulo dada
su base y altura.
\end{ejercicio}

\begin{ejercicio}
Escribir una función que reciba un número entero |n| y calcule el factorial de
|n|.
\begin{partes}
    \item En forma iterativa.
    \item En forma recursiva.
\end{partes}
\end{ejercicio}

\begin{ejercicio}
Escribir una función que reciba un arreglo de números y la cantidad de
elementos, y devuelva el promedio.
\end{ejercicio}

\begin{ejercicio}
Implementar la función |unsigned int strlen(const char *s)| que devuelve la
longitud de la cadena |s| (sin contar el último caracter |'\0'|).
\begin{partes}
    \item En forma iterativa.
    \item En forma recursiva.
\end{partes}
\end{ejercicio}

\begin{ejercicio}
Implementar la función |void invertir(char *s)| que invierte la cadena
|s| ({\it in-place}).
\end{ejercicio}

\begin{ejercicio}
Implementar la función |void strcpy(const char *origen, char *destino)| que
copia la cadena |origen| en la dirección de memoria apuntada por |destino|.
Asumir que |destino| apunta a un arreglo de caracteres con espacio suficiente
(|strlen(origen) + 1|).
\end{ejercicio}

\begin{ejercicio}
Implementar una función que recibe un arreglo de números y su longitud y
lo ordena mediante el algoritmo de ordenamiento por selección.
\end{ejercicio}

\begin{ejercicio}
Implementar una función que reciba un |mensaje| y dos números enteros |min| y
|max|. La función debe pedir al usuario que ingrese un número entero entre
|min| y |max| (inclusive) y devolverlo. Si el usuario ingresa un valor
inválido se le debe informar y pedir que ingrese un nuevo valor, repitiendo
hasta que ingrese un número válido.
\end{ejercicio}

\begin{ejercicio}
Implementar una función que reciba una cadena de texto y
luego imprima la cadena enmarcada entre asteriscos (|*|). Asumir que la cadena
no contiene ningún caracter |\n|. Por ejemplo, si se recibe la cadena
|"Lenguaje C"|, debe imprimir:

\begin{verbatim}
**************
* Lenguaje C *
**************
\end{verbatim}
\end{ejercicio}

\begin{ejercicio}
$\dificil$ Implementar una función que permita ejecutar cálculos matemáticos
expresados en notación polaca inversa
(\url{https://es.wikipedia.org/wiki/Notaci%C3%B3n_polaca_inversa}).

Ejemplo: |calcular_rpn("5 1 2 + 4 * + 3 -")| devuelve 14, ya que la expresión
ingresada es equivalente a $5+((1+2)*4)-3$.

Ayuda: El algoritmo de notación polaca inversa hace uso de una pila,
que se puede implementar en C mediante un arreglo con un tamaño
fijo que determinará la cantidad máxima permitida de valores apilados.
\end{ejercicio}
