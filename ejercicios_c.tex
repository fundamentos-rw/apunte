\lstset{
    language=C,
}

\chapter{Lenguaje C}

\begin{ejercicio}
Escribir una función que permita calcular el área de un rectángulo dada
su base y altura.
\end{ejercicio}

\begin{ejercicio}
Escribir una función que reciba un número entero |n| y calcule el factorial de
|n|.
\begin{partes}
    \item En forma iterativa.
    \item En forma recursiva.
\end{partes}
\end{ejercicio}

\begin{ejercicio}
Escribir una función que reciba un arreglo de números y la cantidad de
elementos, y devuelva el promedio.
\end{ejercicio}

\begin{ejercicio}
    Usando las funciones |printf| y |sizeof|, escribir un programa que imprima
    el tamaño en bytes de cada uno de los siguientes tipos: |bool|, |char|,
    |short|, |int|, |long|, |float|, |double|, |bool*|, |char*|,
    |short*|, |int*|, |long*|, |float*|, |double*|.
\end{ejercicio}

\begin{ejercicio}
Implementar la función |unsigned int strlen(const char *s)| que devuelve la
longitud de la cadena |s| (sin contar el último caracter |'\0'|).
\begin{partes}
    \item En forma iterativa.
    \item En forma recursiva.
\end{partes}
\end{ejercicio}

\begin{ejercicio}
Implementar la función |void invertir(char *s)| que invierte la cadena
|s| ({\it in-place}).
\end{ejercicio}

\begin{ejercicio}
Implementar la función |void strcpy(const char *origen, char *destino)| que
copia la cadena |origen| en la dirección de memoria apuntada por |destino|.
Asumir que |destino| apunta a un arreglo de caracteres con espacio suficiente
(|strlen(origen) + 1|).
\end{ejercicio}

\begin{ejercicio}
Implementar una función que recibe un arreglo de números y su longitud y
lo ordena mediante el algoritmo de ordenamiento por selección.
\end{ejercicio}

\begin{ejercicio}
Implementar una función que reciba un |mensaje| y dos números enteros |min| y
|max|. La función debe pedir al usuario que ingrese un número entero entre
|min| y |max| (inclusive) y devolverlo. Si el usuario ingresa un valor
inválido se le debe informar y pedir que ingrese un nuevo valor, repitiendo
hasta que ingrese un número válido.
\end{ejercicio}

\begin{ejercicio}
Implementar una función que reciba una cadena de texto y
luego imprima la cadena enmarcada entre asteriscos (|*|). Asumir que la cadena
no contiene ningún caracter |\n|. Por ejemplo, si se recibe la cadena
|"Lenguaje C"|, debe imprimir:

\begin{verbatim}
**************
* Lenguaje C *
**************
\end{verbatim}
\end{ejercicio}

\begin{ejercicio}
Implementar una función que reciba una cadena de texto |s| y un número entero |n|, y
devuelva una cadena de texto que contiene la cadena |s| repetida |n| veces.
La cadena producida debe ser alojada en el \emph{heap}, y debe terminar con un
|\0|.
\end{ejercicio}

\begin{ejercicio}
Analizar el funcionamiento de la siguiente función en lenguaje C. ¿Hay algún
problema en su implementación?

\begin{lstlisting}[numbers=none]
void f() {
    int *p = malloc(sizeof(int) * 100);
    for (int i = 0; i < 10; i++) {
        p[i] = i+1000;
    }
    for (int j = 0; j < 100; j++) {
        printf("%d\n", p[i]);
    }
}
\end{lstlisting}
\end{ejercicio}

\begin{ejercicio}
Considerar el siguiente ejemplo. ¿La invocación a |free| realmente libera la
memoria apuntada por el puntero |a|?

\begin{lstlisting}[numbers=none]
int *a = malloc(42);
int *b = a;
free(b);
\end{lstlisting}
\end{ejercicio}

\begin{ejercicio}
Considerar el siguiente ejemplo. ¿Hay algún problema con la implementación?

\begin{lstlisting}[numbers=none]
char *nombre;
fputs(“Ingresa tu nombre: “, stdout);
fgets(nombre, 100, stdin);
printf(“Hola %s!\n”, nombre);
\end{lstlisting}
\end{ejercicio}

\begin{ejercicio}
Escribir una función que se comporte como la función |input| de Python. Debe
recibir un mensaje a imprimir, y luego debe leer una línea de texto de la
consola y devolverla. La cadena producida debe ser alojada en el \emph{heap},
y debe terminar con un |\0|. Asumir que la línea de texto ingresada
no puede superar los 100 caracteres. Utilizar las funciones de la
biblioteca estándar |fputs| y |fgets|.
\end{ejercicio}
