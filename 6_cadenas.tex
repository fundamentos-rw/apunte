\chapter{Cadenas de caracteres}

Una cadena es una secuencia de caracteres. Ya las hemos usado para mostrar
mensajes, pero sus usos son mucho más amplios que sólo ése: los textos que
manipulamos mediante los editores de texto, los textos de Internet que
analizan los buscadores, los mensajes enviados mediante correo electrónico,
son todos ejemplos de cadenas de caracteres. Para poder programar este
tipo de aplicaciones debemos aprender a manipularlas. Comenzaremos a ver
ahora cómo hacer cálculos con cadenas.

\begin{sabias_que}
En Python todos los valores tienen asignado un \emph{tipo}. 	La función |type|
de Python nos permite averiguar de qué tipo es un valor. Las cadenas son de
tipo |str|:

\begin{codigo-python-sn}
>>> type(12)
<class 'int'>
>>> type(12.0)
<class 'float'>
>>> type(True)
<class 'bool'>
>>> type("Hola")
<class 'str'>
\end{codigo-python-sn}
\end{sabias_que}

\section{Operaciones con cadenas}

Ya vimos en la sección~\ref{nosolo} que es posible:

\begin{itemize}
\item Sumar cadenas entre sí (y el resultado es la concatenación
de todas las cadenas dadas):

\begin{codigo-python-sn}
>>> "Un divertido " + "programa " + "de " + "radio"
'Un divertido programa de radio'
\end{codigo-python-sn}

\item Multiplicar una cadena \lstinline+s+ por un número \lstinline+k+ (y
el resultado es la concatenación de \lstinline+s+ consigo misma,
\lstinline+k+ veces):

\begin{codigo-python-sn}
>>> 3 * "programas "
'programas programas programas '
>>> "programas " * 3
'programas programas programas '
\end{codigo-python-sn}
\end{itemize}

A continuación, otras operaciones y particularidades de las cadenas.

\subsection{Obtener la longitud de una cadena}

Se puede averiguar la cantidad de caracteres que conforman una cadena
utilizando una función provista por Python: |len|.
\begin{codigo-python-sn}
>>> len("programas ")
10
\end{codigo-python-sn}

Existe una cadena especial, que llamaremos \emph{cadena vacía}, que
es la cadena que no contiene ningún carácter (se la indica sólo con
un apóstrofe o comilla que abre, y un apóstrofe o comilla que cierra),
y que por lo tanto tiene longitud cero:

\begin{codigo-python-sn}
>>> s = ""
>>> s
''
>>> len(s)
0
\end{codigo-python-sn}

\subsection[Recorrer una cadena]{Una operación para recorrer todos los caracteres de una cadena}

Python nos permite recorrer todos los caracteres de una cadena de
manera muy sencilla, usando directamente un ciclo definido:

\begin{codigo-python-sn}
>>> for c in "programas ":
...     print(c)
...
p
r
o
g
r
a
m
a
s

>>>
\end{codigo-python-sn}

\subsection{Preguntar si una cadena contiene una subcadena}

El operador |in| nos permite preguntar si una cadena contiene una
subcadena. |a in b| es una expresión que se evalúa a |True| si la cadena |b|
contiene la subcadena |a|.

\begin{codigo-python-sn}
>>> 'qué' in 'Hola, ¿qué tal?'
True
>>> '7' in '2468'
False
\end{codigo-python-sn}

Al ser una expresión booleana, podemos utilizarlo como condición de un |if| o
un |while|:

\begin{codigo-python-sn}
if "Hola" in s:
    print("Al parecer la cadena s es un saludo")
\end{codigo-python-sn}

\subsection{Acceder a una posición de la cadena}

Queremos averiguar cuál es el carácter que está en la posición i-ésima de
una cadena.  Para ello Python nos provee de una notación con corchetes:
escribiremos \lstinline+s[i]+ para hablar de la posición i-ésima de la cadena
\lstinline!s!.

Trataremos de averiguar con qué letra empieza una cadena.

\begin{codigo-python-sn}
>>> s = "Veronica"
>>> s[1]
'e'
\end{codigo-python-sn}

\lstinline+s[1]+ nos muestra la segunda letra, no la primera. ¿Algo falló? No,
lo que sucede es que en Python las posiciones se cuentan desde 0.

\begin{codigo-python-sn}
>>> s[0]
'V'
\end{codigo-python-sn}

\begin{observacion}
Las distintas posiciones de una cadena \lstinline!s! se llaman
\emph{índices}. Los índices son números enteros que pueden tomar
valores entre \lstinline!-len(s)! y \lstinline!len(s) - 1!.

\begin{itemize}
\item Los índices positivos (entre \lstinline!0! y \lstinline!len(s) - 1!) son lo que
ya vimos: los caracteres de la cadena del primero al útimo.

\item Los índices negativos (entre |-len(s)| y |-1|) proveen una notación que hace
más fácil indicar cuál es el último
carácter de la cadena: \lstinline!s[-1]! es el último carácter de
\lstinline!s!, \lstinline!s[-2]! es el penúltimo carácter de \lstinline!s!,
\lstinline!s[-len(s)]! es el primer carácter de \lstinline!s!.
\end{itemize}
\end{observacion}

Algunos ejemplos de acceso a distintas posiciones en una cadena.

\begin{codigo-python-sn}
>>> s = "Veronica"
>>> len(s)
8
>>> s[0]
'V'
>>> s[7]
'a'
>>> s[8]
(^Traceback (most recent call last):
  File "<stdin>", line 1, in <module>
IndexError: string index out of range^)
>>> s[-1]
'a'
>>> s[-8]
'V'
>>> s[-9]
(^Traceback (most recent call last):
  File "<stdin>", line 1, in <module>
IndexError: string index out of range^)
\end{codigo-python-sn}

\ejercicioc{Escribir un ciclo que permita mostrar los caracteres de una
cadena del final al principio.}

\section{Segmentos de cadenas}

Python ofrece también una notación para identificar segmentos de una
cadena. La notación es similar a la de los rangos que vimos en los ciclos
definidos: \lstinline+s[0:2]+ se refiere a la subcadena formada por los
caracteres cuyos índices están en el rango \lstinline+[0,2)+:

\begin{codigo-python-sn}
>>> s[0:2]
'Ve'
>>> s[-4:-2]
'ni'
>>> s[0:8]
'Veronica'
\end{codigo-python-sn}

Si \lstinline!j! es un entero no negativo, se puede usar la notación
\lstinline+s[:j]+ para representar al segmento \lstinline+s[0:j]+; también
se puede usar la notación \lstinline+s[j:]+ para representar al segmento
\lstinline+s[j:len(s)]+.

\begin{codigo-python-sn}
>>> s[:3]
'Ver'
>>> s[3:]
'onica'
\end{codigo-python-sn}

Pero hay que tener cuidado con salirse del rango (en particular hay que
tener cuidado con la cadena vacía):

\begin{codigo-python-sn}
>>> s[10]
(^Traceback (most recent call last):
  File "<stdin>", line 1, in <module>
IndexError: string index out of range^)
>>> s = ""
>>> s
''
>>> len(s)
0
>>> s[0]
(^Traceback (most recent call last):
  File "<stdin>", line 1, in <module>
IndexError: string index out of range^)
\end{codigo-python-sn}

Sin embargo \lstinline+s[0:0]+ no da error. ¿Por qué?

\begin{codigo-python-sn}
>>> s[0:0]
''
\end{codigo-python-sn}

\ejercicioc{Investigar cuál es el resultado de \lstinline+s[:]+.}

\ejercicioc{Investigar cuál es el resultado de
\lstinline+s[:j]+ y  \lstinline+s[j:]+ si \lstinline!j! es un número negativo.}

\section{Las cadenas son inmutables}

Resulta que la persona sobre la que estamos hablando en realidad se llama
Veronika, con ``k''.  Como conocemos la notación de corchetes,
tratamos de corregir sólo el carácter correspondiente de la variable
\lstinline!s!:

\begin{codigo-python-sn}
>>> s[6] = "k"
(^Traceback (most recent call last):
  File "<stdin>", line 1, in <module>
TypeError: 'str' object does not support item assignment^)
\end{codigo-python-sn}

El error que se despliega nos dice que la cadena no soporta
la modificación de un carácter. Decimos que \emph{las cadenas
son inmutables}.

Si queremos corregir la ortografía de una cadena, debemos hacer
que la variable \lstinline!s! se refiera a una nueva cadena:

\begin{codigo-python-sn}
>>> s = "Veronika"
>>> s
'Veronika'
\end{codigo-python-sn}

\section{Procesamiento sencillo de cadenas}

\problemac{Nuestro primer problema es muy simple: Queremos contar
cuántas letras ``A'' hay en una cadena \lstinline!s!.}

\begin{enumerate}

\item {\bf Especificación: } Dada una cadena \lstinline!s!, la función
retorna un valor  \lstinline!contador! que representa cuántas letras ``A''
tiene \lstinline!s!.

\item {\bf Diseño: }

¿Se parece a algo que ya conocemos?

Ante todo es claro que se trata de un ciclo definido, porque lo que hay que
tratar es cada uno de los caracteres de la cadena \lstinline!s!, o sea que
estamos frente a un esquema:

\begin{codigo-nohl-sn}
para cada letra de s
    averiguar si la letra es 'A'
    y tratarla en consecuencia
\end{codigo-nohl-sn}

Nos dice la especificación que se necesita una variable
\lstinline!contador! que cuenta la cantidad de letras ``A'' que contiene
\lstinline!s!. Y por lo tanto sabemos que el tratamiento es: si la letra es
``A'' se incrementa el contador en $1$, y si la letra no es ``A'' no se lo
incrementa, o sea que nos quedamos con un esquema de la forma:

\begin{codigo-nohl-sn}
para cada letra de s
    averiguar si la letra es 'A'
    (@y si lo es, incrementar en 1 el contador@)
\end{codigo-nohl-sn}

¿Estará todo completo? Alicia Hacker nos hace notar que en el diseño no
planteamos el retorno del valor del contador. Lo completamos entonces:

\begin{codigo-nohl-sn}
para cada letra de s
    averiguar si la letra es 'A'
    y si lo es, incrementar en 1 el contador
(@retornar el valor del contador@)
\end{codigo-nohl-sn}

¿Y ahora estará todo completo? E. Lapurado, nuestro alumno impaciente nos
induce a poner manos a la obra y a programar esta solución, y el resto del
curso está de acuerdo.


\item {\bf Implementación}

Ya vimos que Python nos provee de un mecanismo para
recorrer una cadena: una instrucción \lstinline!for! que nos brinda un
carácter por vez, del primero al último.

Proponemos la siguiente solución:

\begin{codigo-python-sn}
def contarA(s):
    for letra in s:
        if letra == "A":
            contador = contador + 1
    return contador
\end{codigo-python-sn}

Y la probamos

\begin{codigo-python-sn}
>>> contarA("Ana")
(^Traceback (most recent call last):
  File "<stdin>", line 1, in <module>
  File "<stdin>", line 4, in contarA
UnboundLocalError: local variable 'contador' referenced before assignment^)
\end{codigo-python-sn}

¿Qué es lo que falló? ¡Falló el diseño! Evidentemente la variable
\lstinline+contador+ debe tomar un valor inicial antes de empezar a contar
las apariciones del caracter ``A''. Volvamos al diseño entonces.

\begin{observacion}
Es muy tentador quedarse arreglando la implementación, sin volver al diseño,
pero eso es de muy mala práctica, porque el diseño queda mal documentado,
y además podemos estar dejando de tener en cuenta otras situaciones erróneas.
\end{observacion}

\item {\bf Diseño (revisado)}
Habíamos llegado a un esquema de la forma

\begin{codigo-nohl-sn}
para cada letra de s
    averiguar si la letra es 'A'
    y si lo es, incrementar en 1 el contador
retornar el valor del contador
\end{codigo-nohl-sn}

¿Cuál es el valor inicial que debe tomar \lstinline+contador+?  Como nos
dice la especificación \lstinline+contador+ cuenta la cantidad de letras
``A'' que tiene la cadena \lstinline+s+. Pero si nos detenemos en medio de
la computación, cuando aún no se recorrió toda la cadena sino sólo los
primeros 10 caracteres, por ejemplo, el valor de \lstinline+contador+
refleja la cantidad de ``A'' que hay en los primeros 10 caracteres de
\lstinline+s+.

Si llamamos \emph{parte izquierda de } \lstinline+s+ al segmento de
\lstinline+s+ que ya se recorrió, diremos que cuando leímos los primeros 10
caracteres de \lstinline+s+, su parte izquierda es el segmento
\lstinline+s[0:10]+.

El valor inicial que debemos darle a \lstinline+contador+ debe reflejar la
cantidad de ``A'' que contiene la parte izquierda de \lstinline+s+ cuando
aún no iniciamos el recorrido, es decir cuando esta parte izquierda es
\lstinline+s[0:0]+ (o sea la cadena vacía). Pero la cantidad de caracteres
iguales a ``A'' de la cadena vacía es $0$.

Por lo tanto el diseño será:

\begin{codigo-nohl-sn}
(@inicializar el contador en 0@)
para cada letra de s
    averiguar si la letra es 'A'
    y si lo es, incrementar en 1 el contador
retornar el valor del contador
\end{codigo-nohl-sn}

Lo identificaremos como el esquema \emph{Inicialización - Ciclo de tratamiento -
Retorno de valor}. Pasamos ahora a implementar este diseño:

\item {\bf Implementación (del diseño revisado)}

\begin{codigo-python}
def contarA(s):
    """Devuelve cuántas letras "A" aparecen en la cadena s."""
    contador = 0
    for letra in s:
        if letra == "A":
            contador = contador + 1
    return contador
\end{codigo-python}

\item Prueba
\begin{codigo-python-sn}
>>> contarA("banana")
0
>>> contarA("Ana")
1
>>> contarA("lAn")
1
>>> contarA("lAAn")
2
>>> contarA("lAnA")
2
\end{codigo-python-sn}

\begin{sabias_que}
La instrucción |contador = contador + 1| puede reemplazarse por
|contador += 1|.

En general, la mayoría de los operadores tienen versiones abreviadas para
cuando la variable que queremos asignar es la misma que el primer operando:

\begin{center}
\begin{tabular}{c c}
{\bf Asignación} & {\bf Asignación abreviada} \\
\hline
\lstinline|x = x + n| & \lstinline|x += n| \\
\lstinline|x = x - n| & \lstinline|x -= n| \\
\lstinline|x = x * n| & \lstinline|x *= n| \\
\lstinline|x = x / n| & \lstinline|x /= n| \\
\end{tabular}
\end{center}
\end{sabias_que}

\item {\bf Mantenimiento:}

Esta función resulta un poco limitada. Cuando necesitemos contar
cuántas letras ``E'' hay en una cadena tendremos que hacer otra función.
Tiene sentido hacer una función más general que nos permita contar cuántas
veces aparece un carácter dado en una cadena.

\ejercicioc{Escribir una función \lstinline+contar(c, s)+ que cuente
cuántas veces aparece un carácter \lstinline!c! dado en una cadena
\lstinline!s!.}

\ejercicioc{¿Hay más letras ``A'' o más letras ``E'' en una cadena?
Escribir un programa que lo decida.}

\ejercicioc{Escribir un programa que cuente cúantas veces aparecen cada una
de las vocales en una cadena. No importa si la vocal aparece en mayúscula o
en minúscula.}

\end{enumerate}

\section{Darle formato a las cadenas}

Muchas veces es necesario darle un formato determinado a las cadenas, o dicho de
otro modo, procesar los datos de entrada para que las cadenas resultantes se
vean de una manera en particular. Además, separar el formato del texto de los
datos a mostrar nos permite enfocarnos en la presentación cuando eso es lo que
queremos.

Por ejemplo, si queremos saludar al usuario y mostrarle la
cantidad de mensajes sin leer, y en el estado tenemos |nombre| $\rightarrow$ |'Veronika'| y
|no_leidos| $\rightarrow$ |8|, podemos hacer algo como:

\begin{codigo-python-sn}
>>> "Hola, " + nombre + ", tenés " + str(no_leidos) + " mensajes sin leer"
'Hola Veronika, tenés 8 mensajes sin leer'
\end{codigo-python-sn}

Pero también podemos obtener el mismo resultado utilizando una \emph{cadena de
formato} y la función |format|:

\begin{codigo-python-sn}
>>> "Hola {}, tenés {} mensajes sin leer".format(nombre, no_leidos)
'Hola Veronika, tenés 8 mensajes sin leer'
\end{codigo-python-sn}

La función |format| devuelve la cadena resultante de reemplazar en la cadena de
formato todas las marcas |{}| por los valores indicados, convirtiendo los
valores automáticamente a cadenas de texto.

\begin{sabias_que}
La función |format| se utiliza con una sintaxis que hasta ahora no habíamos
visto: |cadena.format(v1, v2)| en lugar de la notación usual
|format(cadena, v1, v2)|.

Por ahora es suficiente con entender que en la llamada
|cadena.format(v1, v2)|,
es como si la función |format| recibiera tres parámetros: |cadena|,
|v1| y |v2|.

Esta notación es propia de la \emph{orientación a objetos}. Aprenderemos más
sobre objetos en la unidad~\ref{objetos}.
\end{sabias_que}

Al usar una cadena de formato de
esta manera, podemos ver claramente cuál es el contenido del mensaje, evitando
errores con respecto a espacios de más o de menos, interrupciones en el texto
que complican su lectura, etc.

En el caso de los valores numéricos, es posible modificar la forma en la que el
número es presentado. Por ejemplo, si se trata de un monto monetario, usualmente
queremos mostrarlo con dos dígitos decimales, para ello utilizaremos la
marca |{:.2}| para indicar dos dígitos luego del separador decimal.

\begin{codigo-python-sn}
>>> precio = 205.5
>>> 'Sin IVA: ${:.2}. Con IVA: ${:.2}'.format(precio, precio * 1.21)
'Sin IVA: $205.50. Con IVA: $248.66'
\end{codigo-python-sn}

En otras situaciones, como el caso de un valor en un estudio médico, podemos
querer mostrar el número en notación científica. En este caso utilizaremos la
marca |{:.1e}|, indicando que queremos un dígito significativo luego del separador
decimal.

\begin{codigo-python-sn}
>>> rojos = 4640000
>>> 'Glóbulos rojos: {:.1e}/uL'.format(rojos)
'Glóbulos rojos: 4.6e+06/uL'
\end{codigo-python-sn}

\begin{sabias_que}
En la versión 3.7 de Python se introdujo una sintaxis nueva para generar
cadenas de formato, anteponiendo la cadena con la letra |f|. Las tres
siguientes expresiones son equivalentes a los ejemplos mostrados con la
función |format|:

\begin{codigo-python-sn}
>>> f"Hola {nombre}, tenés {no_leidos} mensajes sin leer"
'Hola Veronika, tenés 8 mensajes sin leer'
>>> f'Sin IVA: ${precio:.2}. Con IVA: ${precio * 1.21:.2}'
'Sin IVA: $205.50. Con IVA: $248.66'
>>> f'Glóbulos rojos: {rojos:.1e}/uL'
'Glóbulos rojos: 4.6e+06/uL'
\end{codigo-python-sn}
\end{sabias_que}



\section{Nuestro primer juego}

Con todo esto ya estamos en condiciones de escribir un programa para jugar con
la computadora: el \emph{Mastermind}. El Mastermind es un juego que consiste en
deducir un código numérico de (por ejemplo) cuatro cifras.

\begin{enumerate}

\item {\bf Análisis } (explicación del juego):

Cada vez que se empieza un partido, el programa debe ``eligir'' un número de
cuatro cifras (sin cifras repetidas), que será el código que el jugador debe
adivinar en la menor cantidad de intentos posibles. Cada intento consiste en
una propuesta de un código posible que tipea el jugador, y una respuesta del
programa. Las respuestas le darán pistas al jugador para que pueda deducir el
código.

Estas pistas indican cuán cerca estuvo el número propuesto de la solución a
través de dos valores: la cantidad de \emph{aciertos} es la cantidad de
dígitos que propuso el jugador que también están en el código \emph{en la
misma posición}. La cantidad de \emph{coincidencias} es la cantidad de
digitos que propuso el jugador que también están en el código pero \emph{en
una posición distinta}.

Por ejemplo, si el código que eligió el programa es el \lstinline!2607!, y el
jugador propone el \lstinline!1406!, el programa le debe responder un acierto (el
\lstinline!0!, que está en el código original en el mismo lugar, el tercero), y
una coincidencia (el \lstinline!6!, que también está en el código original, pero
en la segunda posición, no en el cuarto como fue propuesto). Si el jugador
hubiera propuesto el \lstinline!3591!, habría obtenido como respuesta ningún
acierto y ninguna coincidencia, ya que no hay números en común con el
código original, y si se obtienen cuatro aciertos es porque el jugador
adivinó el código y ganó el juego.

\item {\bf Especificación: }
El programa, entonces, debe generar un número que el jugador no pueda predecir.
A continuación, debe pedirle al usuario que introduzca un número de cuatro
cifras distintas, y cuando éste lo ingresa, procesar la propuesta y evaluar el
número de aciertos y de coincidencias que tiene de acuerdo al código elegido. Si
es el código original, se termina el programa con un mensaje de felicitación. En
caso contrario, se informa al jugador la cantidad de aciertos y la de
coincidencias, y se le pide una nueva propuesta. Este proceso se repite hasta
que el jugador adivine el código.

\item {\bf Diseño:}
\label{str:disenno}

Lo primero que tenemos que hacer es indicarle al programa que tiene que
``elegir'' un número de cuatro cifras al azar. Esto lo hacemos a través del
módulo \lstinline!random!. Este módulo provee funciones para hacer elecciones
aleatorias\footnote{En realidad, la computadora nunca puede hacer
elecciones \emph{completamente} aleatorias. Por eso los números ``al azar''
que puede elegir se llaman \emph{pseudoaleatorios}.}.

La función del módulo que vamos a usar se llama \lstinline!choice!. Esta función
recibe una secuencia de valores, y devuelve un valor de la secuencia elegido al
azar. Una n-upla\label{mastermind} (ver sección~\ref{fun:multiple_return}) es un ejemplo de una
secuencia, por lo que podemos hacer algo como:

\begin{codigo-python-sn}
>>> import random
>>> digitos = ('0','1','2','3','4','5','6','7','8','9')
>>> random.choice(digitos)
'4'
>>> random.choice(digitos)
'1'
\end{codigo-python-sn}

Como están entre comillas, los dígitos son tratados como cadenas de caracteres
de longitud uno. Sin las comillas, habrían sido considerados números enteros. En
este caso elegimos verlos como cadenas de caracteres porque lo que nos interesa
hacer con ellos no son cuentas sino comparaciones, concatenaciones, contar
cuántas veces aparece o donde está en una cadena de mayor longitud, es decir,
las operaciones que se aplican a cadenas de texto. Entonces que sean
variables de tipo cadena de caracteres es lo que mejor se adapta a nuestro
problema.

Ahora tenemos que generar el número al azar, asegurándonos de que no haya cifras
repetidas. Esto lo podemos modelar así:

\begin{codigo-nohl-sn}
Tomar una cadena vacía
Repetir cuatro veces:
    1. Elegir un elemento al azar de la lista de dígitos
    2. Si el elemento no está en la cadena, agregarlo
    3. En caso contrario, volver al punto 1
\end{codigo-nohl-sn}

Una vez elegido el número, hay que interactuar con el usuario y pedirle su
primera propuesta. Si el número no coincide con el código, hay que buscar la
cantidad de aciertos y de coincidencias y repetir el pedido de propuestas, hasta
que el jugador adivine el código.

Para verificar la cantidad de aciertos se pueden recorrer las cuatro posiciones
de la propuesta: si alguna coincide con los dígitos en el código en esa
posición, se incrementa en uno la cantidad de aciertos. En caso contrario, se
verifica si el dígito está en alguna otra posición del código, y en ese caso se
incrementa la cantidad de coincidencias. En cualquier caso, hay que incrementar
en uno también la cantidad de intentos que lleva el jugador.

Finalmente, cuando el jugador acierta el código elegido, hay que dejar de pedir
propuestas, informar al usuario que ha ganado y terminar el programa.

\item {\bf Implementación:}
Entonces, de acuerdo a lo diseñado en \ref{str:disenno}, se muestra una
implementación en el Código~\ref{cod:mastermind}.

\begin{codigo}{\label{cod:mastermind} mastermind.py}{Juego Mastermind}
\lstinputlisting{src/mastermind.py}
\end{codigo}

\item {\bf Pruebas:}
La forma más directa de probar el programa es jugándolo, y verificando
manualmente que las respuestas que da son correctas, por ejemplo:

\begin{codigo-nohl-sn}
(~\$~) python mastermind.py
Bienvenido/a al Mastermind!
Tenes que adivinar un numero de 4 cifras distintas
Que codigo propones?: 1234
Tu propuesta (1234) tiene 0 aciertos y  1 coincidencias.
Propone otro codigo: 5678
Tu propuesta (5678) tiene 0 aciertos y  1 coincidencias.
Propone otro codigo: 1590
Tu propuesta (1590) tiene 1 aciertos y  1 coincidencias.
Propone otro codigo: 2960
Tu propuesta (2960) tiene 2 aciertos y  1 coincidencias.
Propone otro codigo: 0963
Tu propuesta (0963) tiene 1 aciertos y  2 coincidencias.
Propone otro codigo: 9460
Tu propuesta (9460) tiene 1 aciertos y  3 coincidencias.
Propone otro codigo: 6940
Felicitaciones! Adivinaste el codigo en 7 intentos.
\end{codigo-nohl-sn}

Podemos ver que para este caso el programa parece haberse comportado bien.
¿Pero cómo podemos saber que el código final era realmente el que eligió
originalmente el programa? ¿O qué habría pasado si no encontrábamos la
solución?

Para probar estas cosas recurrimos a la depuración del programa. Una forma
de hacerlo es simplemente agregar algunas líneas en el código que nos
informen lo que está sucediendo que no podemos ver. Por ejemplo, los
números que va eligiendo al azar y el código que queda al final. Así
podremos verificar si las respuestas son correctas a medida que las hacemos
y podremos elegir mejor las propuestas en las pruebas.

\begin{codigo-python-sn}
def elegir_codigo():
    """Devuelve un codigo de 4 digitos elegido al azar"""
    digitos = ('0','1','2','3','4','5','6','7','8','9')
    codigo = ''
    for i in range(4):
        candidato = random.choice(digitos)
        (@print('[DEBUG] candidato:', candidato)@)
        # Debemos asegurarnos de no repetir digitos
        while candidato in codigo:
            candidato = random.choice(digitos)
            (@print('[DEBUG] otro candidato:', candidato)@)
        codigo = codigo + candidato
        (@print('[DEBUG] el codigo va siendo:', codigo)@)
    return codigo
\end{codigo-python-sn}

De esta manera podemos monitorear cómo se va formando el código que hay que
adivinar, y los candidatos que van apareciendo pero se rechazan por estar
repetidos:

\begin{codigo-nohl-sn}
(~\$~) python master_debug.py
Bienvenido/a al Mastermind!
Tienes que adivinar un numero de cuatro cifras distintas
[DEBUG] candidato: 8
[DEBUG] el codigo va siendo: 8
[DEBUG] candidato: 0
[DEBUG] el codigo va siendo: 80
[DEBUG] candidato: 2
[DEBUG] el codigo va siendo: 802
[DEBUG] candidato: 8
[DEBUG] otro candidato: 2
[DEBUG] otro candidato: 7
[DEBUG] el codigo va siendo: 8027
Que codigo propones?:
\end{codigo-nohl-sn}

\item {\bf Mantenimiento:}
\label{str:mant}
Supongamos que queremos jugar el mismo juego, pero en lugar de hacerlo con un
número de cuatro cifras, adivinar uno de cinco. ¿Qué tendríamos que hacer para
cambiarlo?

Para empezar, habría que reemplazar el 4 en la línea 28 del programa por un
5, indicando que hay que elegir 5 dígitos al azar. Pero además, el ciclo en la
línea 40 también necesita cambiar la cantidad de veces que se va a ejecutar, 5
en lugar de 4. Y hay un lugar más, adentro del mensaje al usuario que indica las
instrucciones del juego en la línea 6.

El problema de ir cambiando estos números de a uno es que si quisiéramos volver
al programa de los 4 dígitos o quisiéramos cambiarlo por uno que juegue con 3,
tenemos que volver a hacer los reemplazos en todos lados cada vez que lo
queremos cambiar, y corremos el riesgo de olvidarnos de alguno e introducir
errores en el código.

Una forma de evitar esto es fijar la cantidad de cifras en una variable:

\begin{codigo-python-sn}
import random

(@CANT_DIGITOS = 5@)

def mastermind():
    """Funcion principal del juego Mastermind"""
    print("Bienvenido/a al Mastermind!")
    print("Tienes que adivinar un numero de (~\bfseries{\{\}}~) cifras distintas"(@.format(CANT_DIGITOS)@))
\end{codigo-python-sn}

Por convención, el nombre de la variable se escribe en mayúsculas, para indicar
que el valor asignado es constante a lo largo de la ejecución del programa.

En la función |elegir_codigo|:

\begin{codigo-python-sn}
def elegir_codigo():
    """Devuelve un codigo de (~\bfseries{CANT\_DIGITOS}~) digitos elegido al azar"""
    digitos = ('0','1','2','3','4','5','6','7','8','9')
    codigo = ''
    for i in range((@CANT_DIGITOS@)):
\end{codigo-python-sn}

Y el chequeo de aciertos y coincidencias:

\begin{codigo-python-sn}
def analizar_propuesta(propuesta, codigo):
    """Determina la cantidad de aciertos y coincidencias"""
    aciertos = 0
    coincidencias = 0
    for i in range((@CANT_DIGITOS@)):
\end{codigo-python-sn}

Con 5 dígitos, el juego se pone más difícil. Nos damos cuenta que si el jugador
no logra adivinar el código, el programa no termina: se queda preguntando
códigos y respondiendo aciertos y coincidencias para siempre. Entonces queremos
darle al usuario la posibilidad de rendirse y saber cuál era la respuesta y
terminar el programa.

Para esto agregamos en el ciclo \lstinline!while! principal una condición
extra: para seguir preguntando, la propuesta tiene que ser distinta al
código pero además tiene que ser distinta del texto \lstinline!"Me doy"!.

\begin{codigo-python-sn}
def mastermind():
    ...
    propuesta = input("Que codigo propones?: ")
    (@ME_DOY = "Me doy"@)

    while propuesta != codigo (@and propuesta != ME_DOY@):
\end{codigo-python-sn}

Entonces, ahora no sólamente sale del \lstinline!while! si acierta el
código, sino además si se rinde y quiere saber cuál era el código. Entonces
afuera del while tenemos que separar las dos posibilidades, y dar distintos
mensajes:

\begin{codigo-python-sn}
    (@if propuesta == ME_DOY:@)
        (@print("Mala suerte! El código era: {}".format(codigo))@)
    (@else:@)
        print("Felicitaciones! Adivinaste el codigo en {} intentos.".format(intentos))
\end{codigo-python-sn}

En el Código~\ref{cod:mastermind_enh} se muestra el código completo luego de
aplicar las mejoras.

\begin{codigo}{\label{cod:mastermind_enh} mastermind.py}{Juego Mastermind de 5
    dígitos y con posibilidad de rendirse}
\lstinputlisting{src/master_enh.py}
\end{codigo}
\end{enumerate}

\ejercicioc{En el punto~\ref{str:mant} (Mantenimiento) usamos una variable que
guardara el valor de la cantidad de dígitos para no tener que cambiarlo todas
las veces. ¿Cómo podemos hacer para evitar esta variable usando la función
\lstinline!len(cadena)!?}

\ejercicioc{Modificar el programa para permitir repeticiones de dígitos.
¡Cuidado con el cómputo de aciertos y coincidencias!}

% Ejercicios que no tienen mucho que ver con la unidad:
%       usar los singulares en caso de haber 1 sólo A o C
%       incluir un número máximo de intentos
%       incluir una pista si el jugador tipea "pista"
%       preguntar si se quiere jugar otra vez

\clearpage

\section{Resumen}

\begin{itemize}
\item Las cadenas de caracteres nos sirven para operar con todo tipo de
textos.  Contamos con funciones para ver su longitud, sus elementos uno a
uno, o por segmentos, comparar estos elementos con otros, etc.
\end{itemize}

\begin{referencia_python}

\begin{sintaxis}{\lstinline!len(cadena)!}
Devuelve el largo de una cadena, 0 si se trata de una cadena vacía.
\end{sintaxis}

\begin{sintaxis}{\lstinline!for caracter in cadena!}
Permite realizar una acción para cada una de las letras de una cadena.
\end{sintaxis}

\begin{sintaxis}{\lstinline!subcadena in cadena!}
Evalúa a |True| si la |cadena| contiene a la |subcadena|.
\end{sintaxis}

\begin{sintaxis}{\lstinline!cadena[i]!}
Corresponde al valor de la cadena en la posición \lstinline!i!, comenzando
desde 0.

Si se utilizan números negativos, se puede acceder a los
elementos desde el último (\lstinline!-1!) hasta el primero
(\lstinline!-len(cadena)!).
\end{sintaxis}

\begin{sintaxis}{\lstinline!cadena[i:j]!}
Permite obtener un segmento de la cadena, desde la posición \lstinline!i!
inclusive, hasta la posición \lstinline!j! exclusive.

En el caso de que se omita \lstinline!i!, se asume \lstinline!0!.  En el
caso de que se omita \lstinline!j!, se asume \lstinline!len(cadena)!.  Si
se omiten ambos, se obtiene la cadena completa.
\end{sintaxis}

\begin{sintaxis}{\lstinline!cadena.format(valores)!}
Genera otra cadena reemplazando las marcas de formato por el valor
correspondiente, luego de convertirlo de acuerdo a lo especificado en la
marca de formato.

Las marcas de formato se indican con |{}|, incluyendo opcionalmente
diversos parámetros de conversión.

A continuación una referencia de algunos tipos de conversión utilizados
frecuentemente:

\begin{center}
\begin{tabular}{l p{4cm} l}
 {\bf Tipo} & {\bf Significado} & {\bf Ejemplo} \\
\hline
\verb!{:d}! & {\footnotesize Valor entero en decimal} & \lstinline!'{:d}'.format(10.5)! $\rightarrow$ \lstinline!'10'! \\
\verb!{:o}! & {\footnotesize Valor entero en octal} & \lstinline!'{:o}'.format(8)! $\rightarrow$ \lstinline!'10'! \\
\verb!{:x}! & {\footnotesize Valor entero en hexadecimal} & \lstinline!'{:x}'.format(16)! $\rightarrow$ \lstinline!'10'! \\
\verb!{:f}! & {\footnotesize Valor de punto flotante, en decimal} &
  \lstinline!'{:f}'.format(0.1**5)! $\rightarrow$ \lstinline!'0.000010'! \\
\verb!{:.2f}! & {\footnotesize Punto flotante, con dos dígitos de precisión} &
  \lstinline!'{:.2f}'.format(0.1**5)! $\rightarrow$ \lstinline!'0.00'! \\
\verb!{:e}! & {\footnotesize Punto flotante, en notación exponencial} &
  \lstinline!'{:e}'.format(0.1**5)! $\rightarrow$ \lstinline!'1.000000e-05'! \\
\verb!{:g}! & {\footnotesize Punto flotante, lo que sea más corto} &
  \lstinline!'{:g}'.format(0.1**5)! $\rightarrow$ \lstinline!'0.00001'! \\
\verb!{:.2s}! & {\footnotesize Cadena recortada en 2 caracteres} &
  \lstinline!'{:.2s}'.format('Python')! $\rightarrow$ \lstinline!'Py'! \\
\verb!{:<6s}! & {\footnotesize Cadena alineada a la izquierda, ocupando 6 caracteres} &
  \lstinline!'{:<6s}'.format('Py')! $\rightarrow$ \lstinline!'Py    '! \\
\verb!{:^6s}! & {\footnotesize Cadena centrada} &
  \lstinline!'{:^6s}'.format('Py')! $\rightarrow$ \lstinline!'  Py  '! \\
\verb!{:>6s}! & {\footnotesize Cadena alineada a la derecha} &
  \lstinline!'{:>6s}'.format('Py')! $\rightarrow$ \lstinline!'    Py'! \\
\verb!{{! & {\footnotesize Un \verb!{!} \\
\verb!}}! & {\footnotesize Un \verb!}!} \\
\end{tabular}
\end{center}

\end{sintaxis}

\begin{sintaxis}{\lstinline!cadena.isdigit()!}
Devuelve |True| si todos los caracteres de la cadena son dígitos, |False| en
caso contrario.
\end{sintaxis}

\begin{sintaxis}{\lstinline!cadena.isalpha()!}
Devuelve |True| si todos los caracteres de la cadena son alfabéticos, |False| en
caso contrario.
\end{sintaxis}

\begin{sintaxis}{\lstinline!cadena.isalnum()!}
Devuelve |True| si todos los caracteres de la cadena son alfanuméricos, |False| en
caso contrario.
\end{sintaxis}

\begin{sintaxis}{\lstinline!cadena.capitalize()!}
Devuelve |True| si todos los caracteres de la cadena son alfanuméricos, |False| en
caso contrario.
\end{sintaxis}

\begin{sintaxis}{\lstinline!cadena.upper()!}
Devuelve una copia de la cadena convertida a mayúsculas.
\end{sintaxis}

\begin{sintaxis}{\lstinline!cadena.lower()!}
Devuelve una copia de la cadena convertida a minúsculas.
\end{sintaxis}
\end{referencia_python}


\newpage
\section{Ejercicios}

\extractionlabel{guia}
\begin{ejercicio}
Escribir funciones que dada una cadena de caracteres:
\begin{partes}
\item Imprima los dos primeros caracteres.
\item Imprima los tres últimos caracteres.
\item Imprima dicha cadena cada dos caracteres. Ej.: \texttt{'recta'} debería
imprimir \texttt{'rca'}
\item Dicha cadena en sentido inverso. Ej.: \texttt{'hola mundo!'} debe
imprimir \texttt{'!odnum aloh'}
\item Imprima la cadena en un sentido y en sentido inverso. Ej:
\texttt{'reflejo'} imprime \texttt{'reflejoojelfer'}.
\end{partes}
\end{ejercicio}


\extractionlabel{guia}
\begin{ejercicio}
Escribir funciones que dada una cadena y un caracter:
\begin{partes}
\item Inserte el caracter entre cada letra de la cadena. Ej: \texttt{'separar'}
y \texttt{','} debería devolver \texttt{'s,e,p,a,r,a,r'}
\item Reemplace todos los espacios por el caracter. Ej: \texttt{'mi archivo de
texto.txt'} y \texttt{'\_'} debería devolver
\texttt{'mi\_archivo\_de\_texto.txt'}
\item Reemplace todos los dígitos en la cadena por el caracter. Ej: \texttt{'su
clave es: 1540'} y \texttt{'X'} debería devolver \texttt{'su clave es: XXXX'}
\item Inserte el caracter cada 3 dígitos en la cadena. Ej.
\texttt{'2552552550'} y \texttt{'.'} debería devolver \texttt{'255.255.255.0'}
\end{partes}
\end{ejercicio}


\extractionlabel{guia}
\begin{ejercicio}
Modificar las funciones anteriores, para que reciban un parámetro que indique
la cantidad máxima de reemplazos o inserciones a realizar.
\end{ejercicio}


\extractionlabel{guia}
\begin{ejercicio}
Escribir una función que reciba una cadena que contiene un largo número entero y
devuelva una cadena con el número y las separaciones de miles. Por ejemplo, si
recibe \texttt{'1234567890'}, debe devolver \texttt{'1.234.567.890'}.
\end{ejercicio}


\extractionlabel{guia}
\begin{ejercicio}
Escribir una función que dada una cadena de caracteres, devuelva:
\begin{partes}
\item La primera letra de cada palabra. Por ejemplo, si recibe
\texttt{'Universal Serial Bus'} debe devolver \texttt{'USB'}.
\item Dicha cadena con la primera letra de cada palabra en mayúsculas. Por
ejemplo, si recibe \texttt{'república argentina'} debe devolver
\texttt{'República Argentina'}.
\item Las palabras que comiencen con la letra `A'. Por ejemplo, si recibe
\texttt{'Antes de ayer'} debe devolver \texttt{'Antes ayer'}
\end{partes}
\end{ejercicio}


\extractionlabel{guia}
\begin{ejercicio}
Escribir funciones que dada una cadena de caracteres:
\begin{partes}
\item Devuelva solamente las letras consonantes. Por ejemplo, si recibe
\texttt{'algoritmos'} o \texttt{'logaritmos'} debe devolver \texttt{'lgrtms'}.
\item Devuelva solamente las letras vocales. Por ejemplo, si recibe \texttt{'sin
consonantes'} debe devolver \texttt{'i ooae'}.
\item Reemplace cada vocal por su siguiente vocal. Por ejemplo, si recibe
\texttt{'vestuario'} debe devolver \texttt{'vistaerou'}.
\item Indique si se trata de un palíndromo. Por ejemplo, \texttt{'anita
lava la tina'} es un palíndromo (se lee igual de izquierda a derecha que de
derecha a izquierda).
\end{partes}
\end{ejercicio}


\extractionlabel{guia}
\begin{ejercicio}
Escribir funciones que dadas dos cadenas de caracteres:
\begin{partes}
\item Indique si la segunda cadena es una subcadena de la primera. Por ejemplo,
\texttt{'cadena'} es una subcadena de \texttt{'subcadena'}.
\item Devuelva la que sea anterior en orden alfábetico. Por ejemplo, si recibe
\texttt{'kde'} y \texttt{'gnome'} debe devolver \texttt{'gnome'}.
\end{partes}
\end{ejercicio}


\extractionlabel{guia}
\begin{ejercicio}
Escribir una función que reciba una cadena de unos y ceros (es decir, un
número en representación binaria) y devuelva el valor decimal
correspondiente.
\end{ejercicio}


\extractionlabel{guia}
\begin{ejercicio}
Implementar la función |pedir_entero(mensaje, min, max)|, que debe imprimir el
|mensaje| y luego esperar a que el usuario ingrese un valor.  Si el valor
ingresado no es un número entero, o no es un número entre |min| y |max|
(inclusive), se le debe avisar al usuario y pedir el ingreso de otro valor.
Una vez que el usuario ingresa un valor válido, la función lo debe devolver.

Ejemplo:

\begin{verbatim}
>>> edad = pedir_entero("¿Cuántos años tenés?", 0, 150)
¿Cuántos años tenés? [0-150]:
Por favor ingresa un número entre 0 y 150.
¿Cuántos años tenés? [0-150]: hola
Por favor ingresa un número entre 0 y 150.
¿Cuántos años tenés? [0-150]: -3
Por favor ingresa un número entre 0 y 150.
¿Cuántos años tenés? [0-150]: 151
Por favor ingresa un número entre 0 y 150.
¿Cuántos años tenés? [0-150]: 25
>>> edad
25
\end{verbatim}
\end{ejercicio}
